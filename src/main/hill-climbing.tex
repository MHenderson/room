The idea behind hill-climbing algorithms is to suppose there exists a \emph{neighbourhood} of feasible solutions to some problem \emph{instance}.
With each \emph{feasible} solution there is an associated \emph{cost} (or profit) and finding an optimal solution becomes a matter of finding the solution with minimum cost (or maximum profit).

A hill-climbing algorithm non-deterministically selects a solution from the neighbourhood system such that the cost is less than that of some initial solution until its procedure fails, hence finding the locally optimal solution.

