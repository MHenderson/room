\begin{example}
\label{eg:tournament}
Suppose that the Football Association proposes to host a new international tournament to be staged as a one-off event in England.
The tournament will involve eight national sides competing in a league to be staged at various stadia around the country over a period of two weeks.
The structure of the tournament is to be planned so that every team plays every other team once, the winning team being the one which accumulates most points in the manner of a normal football league (3 points for a win, 1 for a draw).

Suppose that the eight invited teams are Argentina, Brazil, Columbia, Denmark, England, France, Germany and Holland.

By writing matches as alphabetic pairs (e.g. $AB$ denoting Argentina versus Brazil) the complete list of matches (the match set, $M$) is simply the set of all unordered pairs from team set $T = \{A, B, C, D, E, F, G, H\}$.

\begin{equation*}
  \begin{split}
    M = \{
      AB, AC, AD, AE, AF, AG, AH, BC, BD, BE, BF, BG, BH, CD, \\
      CE, CF, CG, CH, DE, DF, DG, DH, EF, EG, EH, FG, FH, GH
    \}
  \end{split}
\end{equation*}

It remains to decide where and when the matches will be played.

As teams can only play at most once per day it is decided to have seven different rounds with each team competing once per round.
(Seven being the smallest possible number of rounds as each team has to play seven others).

For reasons of fairness and for the benefit of fans the F.A. also requires that each team plays once at each of the seven chosen stadia: Wembley, Emirates, Villa Park, Stadium of Light, Stamford Bridge, Old Trafford, St. James' Park.

Is it possible to arrange a tournament that satifies these conditions?

Table \ref{tab:fixtures} provides a match schedule which is suitable for such a tournament.

Looking along the rows, every team plays once per round.
Looking down columns, every stadium hosts every team exactly once.
And throughout the tournament as a whole every pair from the original match list appears exactly once, hence every team opposes every other team once.
\label{eg:football}
\end{example}

\begin{table}[h!]
  \begin{center}
    \begin{tabular}{r|ccccccc}
                       & 1  &  2 &  3 &  4 &  5 &  6 &  7 \\ \hline
               Wembley &    &    &    & AB & CD & EF & GH \\
              Emirates &    & BD & EG &    & FH & AH &    \\
            Villa Park &    & FG & AD & EH &    &    & BC \\
      Stadium of Light & DH &    & BF & CG & AE &    &    \\
       Stamford Bridge & BE &    & CH &    &    & BG & AF \\
          Old Trafford & AG & CE &    & DF &    & BH &    \\
       St. James' Park & CF & AH &    &    & BG &    & DE
    \end{tabular}
  \end{center}
  \caption{Fixtures for a new international football tournament.}
  \label{tab:fixtures}
\end{table}

Table \ref{tab:fixtures} is another Room square of side 7.
As the pairs are made from elements of an underlying set of 8 elements, we say that this is a Room square of \inlinedef{order} 8.
