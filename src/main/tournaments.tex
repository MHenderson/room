Suppose the English Football Association proposed hosting a new type of international tournament to be staged as a one-off event in England.
This tournament would involve eight national sides competing in a league that would be staged in various stadia around the country over two weeks.
The structure of the tournament would be such that every team played every other team once, with the winner being the team which accumulated most points in the manner of a normal football league (3 points for a win, 1 for a draw).

\begin{example}
Suppose the eight invited teams are Argentina, Brazil, Columbia, Denmark, England, France, Germany and Holland.

Writing matches as alphabetic pairs in the obvious way, (e.g. $AB$ denoting Argentina versus Brazil) then the complete list of matches (the match set, $M$) is simply all unordered pairs from team set $T = \{A, B, C, D, E, F, G, H\}$:
\begin{equation*}
  \begin{split}
    M = \{
      AB, AC, AD, AE, AF, AG, AH, BC, BD, BE, BF, BG, BH, CD, \\
      CE, CF, CG, CH, DE, DF, DG, DH, EF, EG, EH, FG, FH, GH
    \}
  \end{split}
\end{equation*}

It remains to be decided where and when the matches will be played.

The English F.A., for whatever reason (the financial cost of hosting eight teams, for example), has imposed a time limit of two weeks on the tournament.
Realistically the teams can only manage to play on alternate days so it is decided to have, in effect, seven different rounds with each team competing once in each round.
(Seven being the smallest number of rounds because each team has to play seven others).

For reasons of fairness the F.A. also demands the condition that each team will play once at each stadium.
Can such a tournament exist?
Suppose the stadia used are the following: Wembley, Highbury, Villa Park, Stadium of Light, Stamford Bridge, Old Trafford, St. James's Park.

Table \ref{tab:fixtures} provides a match schedule which is suitable for such a tournament.

Looking along the rows, each team plays once in each round.
Looking down columns, each stadia hosts each team exactly once.
And throughout the tournament as a whole each pair from the original match list appears exactly once, hence every team opposes every other team once.
\label{eg:football}
\end{example}

\begin{table}[h!]
  \begin{center}
    \begin{tabular}{r|ccccccc}
                       & 1  &  2 &  3 &  4 &  5 &  6 &  7 \\ \hline
               Wembley &    &    &    & AB & CD & EF & GH \\
              Highbury &    & BD & EG &    & FH & AH &    \\
            Villa Park &    & FG & AD & EH &    &    & BC \\
      Stadium of Light & DH &    & BF & CG & AE &    &    \\
       Stamford Bridge & BE &    & CH &    &    & BG & AF \\
          Old Trafford & AG & CE &    & DF &    & BH &    \\
       St. James' Park & CF & AH &    &    & BG &    & DE
    \end{tabular}
  \end{center}
  \caption{Fixtures}
  \label{tab:fixtures}
\end{table}

Table \ref{tab:fixtures} is another Room square of side 7.
Because the pairs are made from a set containing 8 elements, we say that this is a Room square of order 8.
