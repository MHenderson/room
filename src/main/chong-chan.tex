A year after Wallis had published his solution to the Fermat problem, Chong and Chan published their (independent) discovery of the strong starters which are known as the Mullin-Nemeth starters.
Also included in their paper was an alternative solution to the same problem, but their solution continued to involve the starter-adder method.
This theorem we prove instead.

\begin{theorem}
\label{thm:chong-chan}
For every Galois field of order $2^{2^m} + 1$, where $m \geq 2$, there exists a Room square of order $2^{2^m} + 2$.
\end{theorem}

\begin{proof}
The following pairs in $Z_p$ (where $p = 2^{2^d} + 1$ and $d = 2^{m - 1})$ constitute a strong starter.

\begin{enumerate}
  \item{$\{i + (r - 1)2^d, i2^d - (r - 1\}$}
  \item{$\{(2^d - i)2^d + r, (2^{d - 1} - r)2^d + 2^{d - 1} - i + 1\}$}
  \item{$\{2^{d-1} + r - 1)2^d + 2^{d - 1} + i, (2^{d - 1} + i - 1)2^d + 2^{d - 1} - (r - 1)\}$}
  \item{$\{(2^{d-1}-i)2^d+2^{d-1}+r,(2^d-r+1)2^d-i+1\}$}
\end{enumerate}

Where $1 \leq r \leq 2^{d - 2}$ and $1 \leq i \leq 2^{d-1}$, so rather than just 4 pairs there are $4 \cdot 2^{d - 2} \cdot 2^{d - 1} = 2^{2d - 1}$ pairs arranged in four different classes.
Before completing the proof we pause for an example just to illustrate the real simplicity of these apparently complicated pairs.

In order to prove that the pairs $1 \ldots 4$ are a strong starter from any $Z_p$ we need to prove the following:

\begin{enumerate}
  \item{The union of all the pairs contains each non-zero member of $Z_p$ exactly once.}
  \item{The differences are all the non-zero members of $Z_p$ exactly once.}
  \item{The sums are all distinct and non-zero.}
\end{enumerate}

This is a formidable task, one that would take many pages to prove in full detail.
So instead we sketch an outline of the proof, explicitly proving a few specific cases.

First we prove (a) completely.

The non-zero members of $Z_P$, namely $\{1, \ldots, 2^{2d}\}$, can be represented uniquely by:

\begin{equation}
C(u, v) = u2^d + v
\end{equation}

where $1 \leq v \leq 2^d$ and $0 \leq u \leq 2^d-1$.

Indeed if $u_12^d + v_1 = u_22^d + v_2$ then $(u_1 - u_2)2^d = (v_2 - v_1)$.

The RHS takes integer values in the interval $[-(2^d - 1), 2^d - 1]$, which is symmetric about the origin and smaller than $2^d$ on both sides.
Whereas the LHS takes integer multiple steps of size $2^d$, so the equality can only hold in the case when both sides equal zero.
Which implies $u_1 = u_2$, $v_1 = v_2$ and $C(u, v)$ is unique representation the non-zero members of $Z_p$.
$u$ takes $2^d$ values and $v$ takes $2^d$ values so there are $2^{2d}$ unique non-zero members of $Z_p$ represented in this way, so each member of $Z_p$ is represented.

The left and right hand members of each pair can be characterised by a range of values of $u$ and $v$ in the following manner.

Take, for instance, the left hand member of pair 1, $i + (r - 1)2^d$.
Here $v = i$ and so $1 \leq v \leq 2^{d - 1}$, while $u = (r - 1)$, so $0 \leq u \leq 2^{d - 2} - 1$.
The full list of intervals for each member of each pair is tabulated below.

\begin{table}[h!]
  \begin{center}
    \begin{tabular}{cccc}
     Pair & Member &                u                  &           V                  \\ \hline
        1 &   L    &   $[0,2^{d-2}-1]$                 &  $[1,2^{d-1}]$               \\
          &   R    &   $[0,2^{d-1}-1]$                 &  $[3 \cdot 2^{d-2}+1,2^{d}]$ \\
        2 &   L    &   $[2^{d-1},2^{d}-1]$             &  $[1,2^{d-2}]$               \\
          &   R    &   $[2^{d-2},2^{d-1}-1]$           &  $[1,2^{d-1}]$               \\
        3 &   L    &   $[2^{d-1},3 \cdot 2^{d-2}-1]$   &  $[1+2^{d-1},2^{d}]$         \\
          &   R    &   $[2^{d-1},2^{d}-1]$             &  $[1+ 2^{d-2},2^{d-1}]$      \\
        4 &   L    &   $[0,2^{d-1}-1]$, $[1+ 2^{d-1}]$ &  $3 \cdot 2^{d-2}]$          \\
          &   R    &   $[3 \cdot 2^{d-2},2^{d}-1]$     &  $[1+2^{d-1},2^{d}]$ 
    \end{tabular}
  \end{center}
  \caption{Intervals}
  \label{tab:intervals}
\end{table}

It was mentioned earlier that there were $2^{2d - 1}$ pairs, each of which has two members, so there are $2^{2d}$ elements altogether in the pairs of the starter, which is the same as the number of elements in $Z_p$.
Because $C(u, v)$ is a unique representation for each member of $Z_p$, for an element of $Z_p$ to occur more than once in the starter requires repetition of both $u$ and $v$.
This cannot happen because when two intervals overlap (as they do in the values of $v$ for 1L and 2R).

To prove (b) we need to show that the differences between two pairs of type 1 are all unique, similarly between two pairs of types 2,3 and 4.
Moreover we need to show that there can be no repetition in differences between a pair of type 1 and a pair of type 2, also type 1 with types 3 in 4.
Similarly for 2,3 and 4.
All together there are ten cases to prove, tabulated below, where a pair of numbers represents the two types of pairs from the starter.

\begin{table}[h!]
  \begin{center}
    \begin{tabular}{cccccccc}
      (i)    & 11 & (ii) & 12 & (iii) & 13 & (iv) & 14 \\
      (v)    & 22 & (vi) & 23 & (vii) & 24 &      &    \\
      (viii) & 33 & (ix) & 34 &       &    &      &    \\
      (x)    & 44 &      &    &       &    &      &    
    \end{tabular}
  \end{center}
  \caption{Cases}
  \label{tab:cases}
\end{table}

To illustrate, we prove (v), in other words that differences between two different pairs, both of type 2 are always unique.

Type 2 have the form:
$\{(2^d - i)2^d + r, (2^{d - 1} - r)2^d + 2^{d - 1} - i + 1\}$.

Therefore, a difference between the elements of a pair of type 2 has the form:

\begin{equation}
  \pm \{(2^d - i)2^d + r - (2^{d - 1} - r)2^d - 2^{d - 1} + i - 1\}
\end{equation}

If two different pairs had the same difference we could write:

\begin{equation}
(2^d - i)2^d + r - (2^{d - 1} - r)2^d - 2^{d - 1} + i - 1 \equiv \pm \{(2^d - j)2^d + s - (2^{d - 1} - s)2^d - 2^{d - 1} + j - 1\}\pmod p
\end{equation}

for some $i \neq j, r \neq s$.

There are two cases to prove, firstly consider the one involving the + sign.

\begin{equation}
(2^d - i)2^d + r - (2^{d - 1} - r)2^d - 2^{d - 1} + i - 1 \equiv (2^d - j)2^d + s -(2^{d - 1} - s)2^d - 2^{d - 1} + j -1\pmod p
\end{equation}

In this case we have been helped out with some very convenient cancelling, leaving just:

\begin{align*}
  -i2^d + r + r \cdot 2^d + i &\equiv -j \cdot 2^d + s + s \cdot 2^d + j \pmod p \\
          (-i + j)2^d + i - j &\equiv (s - r)(1 + 2^d) \pmod p \\ 
             (j - i)(2^d - 1) &\equiv (s - r)(1 + 2^d) \pmod p \\
          (j - i)(2^{2d} - 1) &\equiv (s - r)(1 + 2 \cdot 2^d + 2^{2d}) \pmod p \\
                    -2(j - i) &\equiv (s - r) 2 \cdot 2^d \pmod p \\
                      (i - j) &\equiv (s - r) 2^d \pmod p
\end{align*}

for all 1 $\leq i, j \leq 2^{d-1}$.

Therefore $(i - j)$ lies in the interval
$[-(2^{d - 1} - 1), 2^{d - 1} - 1]$,
which is symmetric about the origin, with length
$2 \cdot 2^{d - 1} - 2 = 2^d - 2 < p = 2^{2d} + 1$.
$1 \leq s,r \leq 2^{d-2}$

Therefore, $(s - r)2^d$ lies in the interval
$[-(2^{d - 2} - 1)2^d, (2^{d - 2} - 1)2^d]$,
again symmetric about the origin with length
$2^{2d} - 2^{d + 1} < p$.

So for A to hold requires that $(i - j) = (s - r)2^d$.
But the LHS has an interval with length $2^d - 2 < 2^d$ whereas the RHS is some positive or negative integer multiple of $2^d$, so the two could only be equal when $i = j, r = s$ contradicting the original hypothesis.

There is still the negative case to deal with:

\begin{align*}
(2^d - i)2^d + r - (2^{d - 1} - r)2^d - 2^{d - 1} + i - 1 &\equiv -(2^d - j)2^d - s + (2^{d - 1} - s)2^d + 2^{d - 1} - j + 1\pmod p \\
(2^d - i)2^d - (2^{d} - j)2^d + i + j &\equiv -s + (2^{d - 1}  -s)2^d + (2^{d - 1} - r)2^d - r + 2 \cdot 2^{d - 1} + 2\pmod p \\
2 \cdot 2^{2d} + (1 - 2^{d})(i + j) &\equiv -(s + r)(1 + 2^d) + 2^{2d} + 1 + 2^d + 1\pmod p
\end{align*}

Now $2^{2d} + 1 = p$.
Therefore, $2^{2d} \equiv -1\pmod p$, and so

\begin{align*}
-2 + (1 - 2^d)(i + j) &\equiv -(s + r)(1 + 2^d) + 1 + 2^d\pmod p \\
(1 - 2^d)(i + j) &\equiv -(s + r)(1 + 2^d) + 3 + 2^d\pmod p
\end{align*}

Now multiply throughout by $(1 + 2^d)$, noting that:

$(1 + 2^d)(1 - 2^d) \equiv 2\pmod p$,
$(1 + 2^d)(1 + 2^d) = 1 + 2 \cdot 2^d + 2^{2d} \equiv 2^{d + 1}\pmod p$
and
$(1 + 2^d)(3 + 2^d) = 3 + 4 \cdot 2^d + 2^{2d} \equiv 2^{d + 2} + 2 \pmod p$

\begin{align*}
  2(i + j)   &\equiv -2^{d + 1}(s + r) + 2^{d + 2} + 2 \pmod p \\
  (i + j)    &\equiv -2^{d}(s + r) + 2^{d + 1} + 1 \pmod p \\
  2^d(s + r) &\equiv 2 \cdot 2^{d} - (i + j) + 1 \pmod p
\end{align*}

for all $1 \leq s, r \leq 2^{d-2}$.

Therefore, $2 \cdot 2^d \leq 2^d(s+r) \leq 2^d2^{d-1}$.

The LHS lies in the interval $[2^{d + 1}, 2^{2d - 1}]$, which itself is located somewhere in the interval $[0, p]$.

Now $1 \leq i, j \leq 2^{d - 1}$.
Therefore $2 \leq i + j \leq 2^d$ and thus
$2^d \leq 2 \cdot 2^d -(i+j) \leq 2^{d+1} - 2$.

So the RHS lies in the interval $[2^d + 1, 2^{d + 1} - 1]$.
Again this is located with $[0, p]$.

So for B to be satisfied requires that
$2^d(s + r) = 2 \cdot 2^d - (i + j) + 1$
But the RHS and LHS intervals are disjoint so this can never happen.
So the absence of repetition in the differences of two different pairs both of type 2 is proven.
All cases involving different pairs of the same type are proven in this way (cases (i),(v),(viii),(x)).

Finally we demonstrate how the other six cases are proven, those involving pairs of different types.
Inevitably the approach is very similar.

Consider a pair of type 1 and another pair of type 4 (case iv).
If there were a repetition of differences between pairs of this type we could write:

\begin{equation*}
  i + (r - 1)2^d - i2^d + (r - 1) \equiv \pm \{(2^{d - 1} - j)2^d + 2^{d - 1} + s - (2^d - s + 1)2^d + j - 1\}\pmod p
\end{equation*}

Consider the + sign,

\begin{align*}
i-i2^d-(2^{d-1}-j)2^d-j &\equiv -(r-1)2^d-(r-1)+s-(2^d-s+1)2^d+2^{d-1}-1\pmod p \\
  (1-2^d)(i-j)-2^{2d-1} &\equiv -(2^d+1)(r-s)-2^{2d}+2^{d-1}\pmod p
\end{align*}

As $2^{2d} \equiv -1\pmod p$, it follows that

\begin{align*}
  (1-2d)(i-j) &\equiv -(2^d+1)(r-s)+2^{d-1} + 2^{2d-1} + 1\pmod p \\
  2(1-2d)(i-j) &\equiv -2(2^d+1)(r-s)+2^{d} + 2^{2d} + 2\pmod p \\
  2(1-2d)(i-j) &\equiv -2(2^d+1)(r-s)+2^{d}+1\pmod p \\
  4(i-j) &\equiv -2 \cdot 2^{d+1}(r-s)+2^{d+1}\pmod p \\
  (i-j) &\equiv -2^{d}(r-s)+2^{d-1}\pmod p \\
  2^d(r-s) &\equiv 2^{d-1}-(i-j)\pmod p
\end{align*}

In this instance the LHS has interval
$[2^d - 2^{2d - 2}, 2^{2d - 2} - 2^d]$.
while the interval of the RHS is $[1, 2^d - 1]$.
Both are smaller than $p$ in length, so for equality requires
$$2^d(r  -s) = 2^{d - 1} - (i - j)$$

But this can never be true because the left side is always either zero or an integer multiple of $2^d$, whereas the interval of the right is $[1, 2^d - 1]$.

All other cases are dealt with in a very similar manner, and the proof of (c), namely that all sums are unique, is not very different.
\end{proof}

