A year after Wallis published his solution to the Fermat problem, Chong and Chan published their (independent) discovery of the strong starters known as Mullin-Nemeth starters.

Also included in their paper was an alternative solution (Theorem \ref{thm:chong-chan}) to the Fermat problem, based on the starter-adder method.

\begin{theorem}
\label{thm:chong-chan}
For every Galois field of order $2^{2^m} + 1$, where $m \geq 2$, there exists a Room square of order $2^{2^m} + 2$.
\end{theorem}

\begin{proof}
The pairs in \eqref{eq:first} -- \eqref{eq:last} constitute a strong starter in $\integersmodp$ (where $p = 2^{2^d} + 1$ and $d = 2^{m - 1})$.

\begin{align}
  \{i + (r - 1)2^d, i2^d - (r - 1)\} & \label{eq:first} \\
  \{(2^d - i)2^d + r, (2^{d - 1} - r)2^d + 2^{d - 1} - i + 1\} & \\
  \{2^{d-1} + r - 1)2^d + 2^{d - 1} + i, (2^{d - 1} + i - 1)2^d + 2^{d - 1} - (r - 1)\} & \\
  \{(2^{d-1}-i)2^d+2^{d-1}+r,(2^d-r+1)2^d-i+1\} & \label{eq:last}
\end{align}

Where $1 \leq r \leq 2^{d - 2}$ and $1 \leq i \leq 2^{d-1}$

There are $4 \cdot 2^{d - 2} \cdot 2^{d - 1} = 2^{2d - 1}$ pairs arranged in four different classes.

To prove that the pairs \eqref{eq:first} -- \eqref{eq:last} are a strong starter based on $\integersmodp$ we need to prove the following:

\begin{enumerate}
  \item{\label{item:union} The union of all pairs contains each non-zero member of $\integersmodp$ exactly once.}
  \item{\label{item:differences} Among the differences are contained all non-zero members of $\integersmodp$ exactly once.}
  \item{\label{item:sum} The sums are all distinct and non-zero.}
\end{enumerate}

This is a formidable task, one that would take many pages to prove in full detail.
So, instead we sketch an outline of the proof, explicitly proving just a few specific cases.

First we prove \ref{item:union} completely.

The non-zero members of $\integersmodp$, namely $\{1, \ldots, 2^{2d}\}$, can be represented uniquely by:
\begin{equation}
C(u, v) = u2^d + v
\end{equation}
where $1 \leq v \leq 2^d$ and $0 \leq u \leq 2^d - 1$.

Indeed if $u_{1}2^d + v_{1} = u_{2}2^d + v_{2}$ then $(u_{1} - u_{2})2^d = (v_{2} - v_{1})$.

The right-hand side takes integer values in the interval $[-(2^d - 1), 2^d - 1]$, which is symmetric about the origin and smaller than $2^d$ on both sides, whereas the left-hand side takes integer multiple steps of size $2^d$.
So equality can only hold in the case when both sides equal zero.
This implies that $u_1 = u_2$, $v_1 = v_2$ and so $C(u, v)$ is a unique representation of the non-zero members of $\integersmodp$.
As $u$ takes $2^d$ values and $v$ takes $2^d$ values there are $2^{2d}$ unique non-zero members of $\integersmodp$ represented in this way.
So each member of $\integersmodp$ is represented.

The left and right hand members of each pair can be characterised by a range of values of $u$ and $v$ in the following manner.
For example, consider the left hand member of \eqref{eq:first}: $i + (r - 1)2^d$.
Here $v = i$ and so $1 \leq v \leq 2^{d - 1}$, while $u = (r - 1)$.
So $0 \leq u \leq 2^{d - 2} - 1$.

The full list of intervals for each member of each pair is given in Table \ref{tab:intervals}.

\begin{table}[h!]
  \begin{center}
    \begin{tabular}{cccc}
     Pair & Member &                u                  &           V                  \\ \hline
        1 &   L    &   $[0,2^{d-2}-1]$                 &  $[1,2^{d-1}]$               \\
          &   R    &   $[0,2^{d-1}-1]$                 &  $[3 \cdot 2^{d-2}+1,2^{d}]$ \\
        2 &   L    &   $[2^{d-1},2^{d}-1]$             &  $[1,2^{d-2}]$               \\
          &   R    &   $[2^{d-2},2^{d-1}-1]$           &  $[1,2^{d-1}]$               \\
        3 &   L    &   $[2^{d-1},3 \cdot 2^{d-2}-1]$   &  $[1+2^{d-1},2^{d}]$         \\
          &   R    &   $[2^{d-1},2^{d}-1]$             &  $[1+ 2^{d-2},2^{d-1}]$      \\
        4 &   L    &   $[0,2^{d-1}-1]$, $[1+ 2^{d-1}]$ &  $3 \cdot 2^{d-2}]$          \\
          &   R    &   $[3 \cdot 2^{d-2},2^{d}-1]$     &  $[1+2^{d-1},2^{d}]$ 
    \end{tabular}
  \end{center}
  \caption{Intervals}
  \label{tab:intervals}
\end{table}

Earlier, it was mentioned that there are $2^{2d - 1}$ pairs, each of which has two members.
So, in total, there are $2^{2d}$ elements among the starter pairs, which is the same as the number of elements in $\integersmodp$.
As $C(u, v)$ is a unique representation of the members of $\integersmodp$ an element of $\integersmodp$ can only occur more than once in the starter if there is a repetition of both $u$ and $v$.
This cannot happen because when two intervals overlap (as they do in the values of $v$ for 1L and 2R).

To prove \ref{item:differences} we need to show that the differences between two pairs of type 1 are all unique.
Similarly with two pairs of types 2,3 and 4.
Moreover we need to show that there can be no repetition in differences between pairs of type 1 and pairs of type 2.
The same goes for type 1 with types 3 in 4.
Similarly for types 2, 3 and 4.
In total there are ten cases to prove, shown in Table \ref{tab:cases} (where a pair of numbers represents the two types of pairs from the starter).

\begin{table}[h!]
  \begin{center}
    \begin{tabular}{cccccccc}
      (i)    & 11 & (ii) & 12 & (iii) & 13 & (iv) & 14 \\
      (v)    & 22 & (vi) & 23 & (vii) & 24 &      &    \\
      (viii) & 33 & (ix) & 34 &       &    &      &    \\
      (x)    & 44 &      &    &       &    &      &    
    \end{tabular}
  \end{center}
  \caption{Cases}
  \label{tab:cases}
\end{table}

As an example, we will prove (v).
In other words, differences between two different pairs of type 2 are always unique.

Type 2 pairs have the form:
\begin{equation*}
\{(2^d - i)2^d + r, (2^{d - 1} - r)2^d + 2^{d - 1} - i + 1\}
\end{equation*}

Therefore, a difference between the elements of a type 2 pair has the form:

\begin{equation*}
  \pm \{(2^d - i)2^d + r - (2^{d - 1} - r)2^d - 2^{d - 1} + i - 1\}
\end{equation*}

If two different pairs had the same difference we could write:
\begin{equation*}
(2^d - i)2^d + r - (2^{d - 1} - r)2^d - 2^{d - 1} + i - 1 \equiv \pm \{(2^d - j)2^d + s - (2^{d - 1} - s)2^d - 2^{d - 1} + j - 1\}\pmod p
\end{equation*}
for some $i \neq j, r \neq s$.

There are two cases to prove, firstly consider the one involving the + sign:
\begin{equation*}
(2^d - i)2^d + r - (2^{d - 1} - r)2^d - 2^{d - 1} + i - 1 \equiv (2^d - j)2^d + s -(2^{d - 1} - s)2^d - 2^{d - 1} + j -1\pmod p
\end{equation*}

In this case we have been helped out with some very convenient cancelling, leaving just:
\begin{align*}
  -i2^d + r + r \cdot 2^d + i &\equiv -j \cdot 2^d + s + s \cdot 2^d + j \pmod p \\
          (-i + j)2^d + i - j &\equiv (s - r)(1 + 2^d) \pmod p \\ 
             (j - i)(2^d - 1) &\equiv (s - r)(1 + 2^d) \pmod p \\
          (j - i)(2^{2d} - 1) &\equiv (s - r)(1 + 2 \cdot 2^d + 2^{2d}) \pmod p \\
                    -2(j - i) &\equiv (s - r) 2 \cdot 2^d \pmod p \\
                      (i - j) &\equiv (s - r) 2^d \pmod p
\end{align*}
for all 1 $\leq i, j \leq 2^{d-1}$.

Therefore $(i - j)$ lies in the interval $[-(2^{d - 1} - 1), 2^{d - 1} - 1]$, which is symmetric about the origin, with length $2 \cdot 2^{d - 1} - 2 = 2^d - 2 < p = 2^{2d} + 1$.
$1 \leq s,r \leq 2^{d-2}$

Therefore, $(s - r)2^d$ lies in the interval $[-(2^{d - 2} - 1)2^d, (2^{d - 2} - 1)2^d]$, again symmetric about the origin with length $2^{2d} - 2^{d + 1} < p$.

So for A to hold requires that $(i - j) = (s - r)2^d$.
But the left-hand side has an interval with length $2^d - 2 < 2^d$ whereas the right-hand side is a positive or negative integer multiple of $2^d$.
So the two could only be equal if $i = j$ and $r = s$, contradicting the hypothesis.

The negative case remains:
\begin{align*}
(2^d - i)2^d + r - (2^{d - 1} - r)2^d - 2^{d - 1} + i - 1 &\equiv -(2^d - j)2^d - s + (2^{d - 1} - s)2^d + 2^{d - 1} - j + 1\pmod p \\
(2^d - i)2^d - (2^{d} - j)2^d + i + j &\equiv -s + (2^{d - 1}  -s)2^d + (2^{d - 1} - r)2^d - r + 2 \cdot 2^{d - 1} + 2\pmod p \\
2 \cdot 2^{2d} + (1 - 2^{d})(i + j) &\equiv -(s + r)(1 + 2^d) + 2^{2d} + 1 + 2^d + 1\pmod p
\end{align*}

Now $2^{2d} + 1 = p$.
Therefore, $2^{2d} \equiv -1\pmod p$, and so:
\begin{align*}
-2 + (1 - 2^d)(i + j) &\equiv -(s + r)(1 + 2^d) + 1 + 2^d\pmod p \\
(1 - 2^d)(i + j) &\equiv -(s + r)(1 + 2^d) + 3 + 2^d\pmod p
\end{align*}

Multiplying throughout by $(1 + 2^d)$, noting that:
$(1 + 2^d)(1 - 2^d) \equiv 2\pmod p$,
$(1 + 2^d)(1 + 2^d) = 1 + 2 \cdot 2^d + 2^{2d} \equiv 2^{d + 1}\pmod p$
and
$(1 + 2^d)(3 + 2^d) = 3 + 4 \cdot 2^d + 2^{2d} \equiv 2^{d + 2} + 2 \pmod p$

gives:
\begin{align*}
  2(i + j)   &\equiv -2^{d + 1}(s + r) + 2^{d + 2} + 2 \pmod p \\
  (i + j)    &\equiv -2^{d}(s + r) + 2^{d + 1} + 1 \pmod p \\
  2^d(s + r) &\equiv 2 \cdot 2^{d} - (i + j) + 1 \pmod p
\end{align*}
for all $1 \leq s, r \leq 2^{d-2}$.

Therefore, $2 \cdot 2^d \leq 2^d(s+r) \leq 2^d2^{d-1}$.

The left-hand side lies in the interval $[2^{d + 1}, 2^{2d - 1}]$, which is located somewhere in the interval $[0, p]$.

Now $1 \leq i, j \leq 2^{d - 1}$.
Therefore $2 \leq i + j \leq 2^d$ and thus $2^d \leq 2 \cdot 2^d -(i+j) \leq 2^{d+1} - 2$.

So the right-hand side lies in the interval $[2^d + 1, 2^{d + 1} - 1]$.
Again this interval is located somewhere in $[0, p]$.

So for B to be satisfied requires that
$2^d(s + r) = 2 \cdot 2^d - (i + j) + 1$
But the right-hand side and left-hand side intervals are disjoint, so this can never happen.
So the absence of repetition in the differences of two different pairs both of type 2 is proven.

All cases involving different pairs of the same type ((i), (v), (viii), (x)) are proven in the same way.

Finally we demonstrate how the other six cases, those involving pairs of different types, are proved.

Consider a pair of type 1 and another pair of type 4 (case iv).

If there were a repetition of differences between pairs of this type then we could write:
\begin{equation*}
  i + (r - 1)2^d - i2^d + (r - 1) \equiv \pm \{(2^{d - 1} - j)2^d + 2^{d - 1} + s - (2^d - s + 1)2^d + j - 1\}\pmod p
\end{equation*}

First, consider the + sign,
\begin{align*}
i-i2^d-(2^{d-1}-j)2^d-j &\equiv -(r-1)2^d-(r-1)+s-(2^d-s+1)2^d+2^{d-1}-1\pmod p \\
  (1-2^d)(i-j)-2^{2d-1} &\equiv -(2^d+1)(r-s)-2^{2d}+2^{d-1}\pmod p
\end{align*}

As $2^{2d} \equiv -1\pmod p$, it follows that:
\begin{align*}
  (1-2d)(i-j) &\equiv -(2^d+1)(r-s)+2^{d-1} + 2^{2d-1} + 1\pmod p \\
  2(1-2d)(i-j) &\equiv -2(2^d+1)(r-s)+2^{d} + 2^{2d} + 2\pmod p \\
  2(1-2d)(i-j) &\equiv -2(2^d+1)(r-s)+2^{d}+1\pmod p \\
  4(i-j) &\equiv -2 \cdot 2^{d+1}(r-s)+2^{d+1}\pmod p \\
  (i-j) &\equiv -2^{d}(r-s)+2^{d-1}\pmod p \\
  2^d(r-s) &\equiv 2^{d-1}-(i-j)\pmod p
\end{align*}

In this instance the left-hand side has interval
$[2^d - 2^{2d - 2}, 2^{2d - 2} - 2^d]$.
while the interval of the right-hand side is $[1, 2^d - 1]$.
Both are smaller than $p$ in length, so equality requires $2^d(r  -s) = 2^{d - 1} - (i - j)$.

But this can never be true because the left side is always either zero or an integer multiple of $2^d$, whereas the interval of the right is $[1, 2^d - 1]$.

The remaining cases can be handled in a very similar manner.

Proving (c), namely that all sums are unique, is left as an exercise for the reader.
\end{proof}

