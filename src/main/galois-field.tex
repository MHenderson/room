Throughout this work much use will be made of a particular \inlinedef{finite field}, known as the Galois field, denoted by $GF(p^n)$.
Whenever $p^n$ is a prime (i.e. $n = 1$) the Galois field is precisely the integers under modulo $p$ arithmetic, denoted $Z_p$.
The Galois field has a number of important properties which are used in many of the proofs that follow.
We introduce some of these now.

\begin{enumerate}

\item{Every Galois field (every finite field in fact) has a \emph{primitive element}.
An element, $x$ say, is primitive in $GF(q)$ if
$\{x^0, x^1, x^2, \ldots, x^{q - 1}\}$
are all the non-zero members of $GF(q)$.}

\item{It can be shown
\cite{boseResolvableSeriesBalanced1947}
that $x^{q - 1} = 1$ is always true for any $GF(q)$ where $q$ is odd, and $x^i \neq 1$ for any $1 \leq i \leq q - 1$.}

\item{$x^{q - 1} = 1$ implies that $(x^{\frac{1}{2}(q - 1)} - 1)(x^{\frac{1}{2}(q - 1)} + 1) = 0$, therefore either $x^{\frac{1}{2}(q - 1)} = 1$ or $x^{\frac{1}{2}(q - 1)} = -1$.
Clearly because of the previous remark, only the latter can be true.}

\item{If $b$ is a non-zero residue modulo $p$, then $b$ is said to be a \inlinedef{quadratic residue} (or \inlinedef{square}) if $x^2 \equiv b \pmod p$ has solutions, otherwise $b$ is said to be a \inlinedef{quadratic non-residue} (or \inlinedef{non-square}).
So the non-zero squares are precisely the even powers of the primitive element, while the non-zero non-squares are the odd powers.}

\item{There are precisely $\frac{1}{2}(p - 1)$ squares mod $p$, and $\frac{1}{2}(p - 1)$ non-squares.}

\item{$-1$ is a square if $q \equiv 1 \pmod 4$, but not a square for $q \equiv 3 \pmod 4$.
\begin{enumerate}
  \item{If $q \equiv 1 \pmod 4$, then if $x^i$ is a square so is $-x^i$.}
  \item{If $q \equiv 3 \pmod 4$, then $x^i$ is a square $-x^i$ is a non-square.}
\end{enumerate}}

\end{enumerate}

\begin{example}
$x = 2$ is a primitive element in $GF(11)$ because, $x^0 = 1$, $x^1 = 2$, $x^2 = 4$, $x^3 = 8$, $x^4 = 5$, $x^5 = 10$, $x^6 = 9$, $x^7 = 7$, $x^8 = 3$, $x^9 = 6$
\label{eg:primitive}
\end{example}
