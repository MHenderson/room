If $(x_1, y_1), (x_2, y_2),\ldots, (x_{(q - 1)/2}, y_{(q - 1)/2})$ are the pairs of a balanced starter in $GF(q)$ then we say that starter is \inlinedef{skew} if $\pm(x_1 + y_1), \pm(x_2 + y_2),\ldots, \pm(x_{(q - 1)/2}, y_{(q - 1)/2})$ are all distinct mod $q$.

Notice that a skew starter is necessarily a strong starter (all sums are distinct).
A balanced starter is \inlinedef{symmetric} if $\{x_1, x_2,\ldots, x_{(q - 1)/2}\} = \{-x_1, -x_2,\ldots, x_{(q - 1)/2}\}$

\cite{hwangCompleteBalancedHowell1984} claimed that Schellenberg’s multiplication theorem could be applied to two $\CBHR(2n - 1)$s, both obtained from $\SSBS$, to construct a $\BRS(4n)$.
It was then shown that $\SSBS$ exist for all prime powers $2n - 1 = 8k + 5 > 5$, and so the existence of $\BRS(4n)$ for these values was believed to be established.

\begin{center}
\begin{tabular}{|c|c|c|c|c|c|c|c|c|c|c|}
\hline
  k  &  1 &  3 &  4 &   6 &   7 &  12 &  13 &  16 &  18 & 19 \\ \hline
$4n$ & 28 & 60 & 76 & 108 & 124 & 204 & 220 & 268 & 300 & 316 \\ \hline
\end{tabular}
\end{center}

\cite{andersonConstructionBalancedRoom1999} noticed a flaw in this approach and corrected it.

Suppose we wished to apply Schellenberg’s construction to two $\CBHR(2n - 1)$s, call them $P$ and $Q$.
Then it would be necessary to make a slight alteration to Theorem \ref{thm:schellenberg} to accomodate the missing diagonal pairs, i.e. $\{\infty, i\}$ from row $i$.

\begin{theorem}
Suppose we have two $\CBHR(2n - 1)$, $P$ and $Q$ based on $G = GF(2n - 1)$, with the following properties:
\begin{enumerate}
  \item{$P$ and $Q$ are a latin pair such that,}
  \item{$P \odot Q$ has a pair of disjoint
    \emph{common transversals} $T_1$ and $T_2$
    (with $T_2$ in $P$), which do not intersect that main
    diagonal, and}
  \item{the block designs obtained from $P$ and $Q$, call
    them $D(P)$ and $D(Q)$ respectively, have the property
    that if $B_i$ and $C_i$ are the blocks from $D(P)$
    obtained from row $i$, then $B_i$ and $C_i$
    are also the blocks of $D(Q)$ obtained from row $i$.}
\end{enumerate}

Then a $\BRS(4n)$ exists.
\label{thm:schellenberg-alt}
\end{theorem}

\begin{proof}
As for Theorem \ref{thm:schellenberg}.
\end{proof}

It was then claimed that the two $\CBHR(2n - 1)$s obtained from a $\SSBS$ and its transpose satisfy the three conditions of Theorem~\ref{thm:schellenberg-alt}.
Hence implying the existence of a $\BRS(4n)$.

However, a counter-example was found.

\begin{lemma}
If $p = 8k + 5$ is a prime, $p > 5$, then the pairs
\begin{equation}
(x^{4i}, -x{4i + 1}), (x^{4i + 2}, x^{4i + 1}) \qquad 0 \leq i \leq 2k
\end{equation}
form a $\SSBS$ in $GF(p)$, provided $x^2 - 1$ is a square mod
$p$.
\end{lemma}
\begin{proof}
\begin{enumerate}
\item{The elements in the starter pairs are a complete
    replication of $GF(p) \backslash \{0\}$.
    $x^{\frac{1}{2} (q - 1)} = x^{4k + 2} = -1$, therefore we
    can write the pairs as,
    $$(X^{4i}, x^{4(k + i) + 3}), (x^{4i + 2}, x^{4i + 1}) 0 \leq i \leq 2k$$
    All the members of these pairs are clearly unique and
    there are $4(2k + 1) = 8k + 4$ of them, hence they must all of
    $(GF(p) \backslash \{0\}$.}
\item{The differences are similarly a complete replication of
    $GF(p) \backslash \{0\}$. The differences are
    $\pm x^{4i}(x + 1), \pm x^{4i + 1}(x - 1)$, which can be written
    as
    $$x^{4i}(x + 1), x^{4(i + k) + 2}(x + 1), x^{4i + 1}(x - 1), x^{4(i + k) + 3}(x-1)$$
    Which are all unique provided $(x + 1)$ and $(x - 1)$ have
    the same quadratic character (meaning either both of
    neither are squares).  This is true because
    $(x + 1)(1 - x) = x^2 - 1$ is a square.}
\item{The starter is \emph{skew}, (the positive and negative sums
    are all distinct).  The sums are
    $\pm x^{4i}(1 - x), \pm x^{4i + 1}(x + 1)$
    which again are all unique, provided
    $(1 - x)$ and $(x + 1)$ have the same quadratic
    character.  True, because
    $(1 - x)(x  +1) = -(x^2 - 1) = x^{4k + 2}(x^2  -1)$, is a square.}
\item{The starter is \emph{symmetric}.
    Indeed, $-x^{4i} = x^{4(i + k) + 2}$, generates the
    same elements as $x^{4i + 2}$ shifted by $k$ places.}
\item{The associated block design is balanced.  The blocks in
    the first row are, $\{\infty\} \cup R$, due to the left
    hand members (where $R$ is the set of squares).
    $\{0\} \cup N$, due to the right hand members (where
    $N$ is the set of non-squares).
    We can show that these blocks along with their translates
    form a $\BIBD$.
    
    Using an earlier argument, we can say that
    all elements of $R$ are generated as differences of $R$
    from the same number of times (say $\lambda _1$ times).
    Also, a similar argument allows us to say that all the
    elements of $N$ are generated as differences of $R$ the
    same number of times (say $\lambda _2$). Now, of course
    $N = xR$. So we can say that to each difference of
    members of $R$ which generates a square there is a
    difference of $N$ which generates a non-square.
    Therefore every $r \in R$ occurs as a difference in
    $N, \lambda _1$ times. For the same reason every
    $n \in N$ occurs as a difference between members of
    $N, \lambda _2$ times. Therefore, each element of
    $GF(p) \backslash \{0\}$ occurs $\lambda _1 + \lambda _2$
    times as a difference in $R$ or $N$.
    
    Now, $|R| = |N| = 4k + 2$. So each of the two sets, $R$
    and $N$, gives $(4k + 2)(4k + 1)$ differences. Furthermore
    there are $8k + 4$ elements of $GF(p) \backslash \{0\}$,
    and each occurs as a difference
    $\lambda _1 + \lambda _2$ times. So,
    \begin{equation}
    2(4k + 2)(4k + 1) = (\lambda _1 + \lambda _2)(8k + 4)
    \end{equation}
    Therefore, $\lambda _1 + \lambda _2 = 4k + 1$.
    Adjoining 0 to $N$ creates each member of
    $GF(p) \backslash \{0\}$ once again, as either $0-n$ or
    $n-0$.
    Adjoining $\infty$ to $R$, makes $|R| = 4k + 2$ pairs
    involving $\infty$. Therefore, $\{\infty\} \cup R$ and
    $\{0\} \cup N$ and their translates have between them
    each pair from $\{\infty\{ \cup GF(p)$ occurring $4k + 2$
    times. Hence the block design is balanced.}
\end{enumerate}
\end{proof}

\begin{example}
Let $P$ be the $\CBHR(2n - 1)$ obtained from the $\SSBS$ in Lemma 3.2 when $p = 13$
\begin{equation}
  (1, 11), (3, 7), (4, 2), (9, 8), (10, 5), (12, 6)
\end{equation}
\end{example}

By the \inlinedef{transpose} $X^T$ of a starter $X$ we mean those pairs in the first column of the square generated by $X$.

In this case, for example, $(1, 11)$ goes in cell $(0, 12)$ therefore will be a pair in the first column $(1 + i, 11 + i)$ in position $(0 + i, 12 + i)$, such that $12 + i = 0\pmod{13}$. $(2, 12)$ goes in $(1, 0)$ hence belongs to the transpose of this starter

In general a starter-pair $(a, b)$ goes in $(0, a  +b)$, so the corresponding transpose-pair will be $(a - a - b, b - a - b) = (-b, -a)$. So the transpose of the above starter is:
$$(2, 12), (6, 10), (11, 9), (5, 4), (8, 3), (7, 1)$$

If we now apply the Schellenberg construction as suggested in
\cite{hwangCompleteBalancedHowell1984}
we obtain $P \odot Q$:
\begin{equation}
  P \odot Q = \left[\begin{array}{*{13}c}
   0,0  & 2,12  & 10,5  & 6,10  & 9,8  & 12,6 &  4,2  & 11,9  &  7,1  & 5,4  &  3,7  &  8,3  & 1,11 \\
   2,12 &  1,1  &  3,0  & 11,6  & 7,11 & 10,9 &  0,7  &  5,3  & 12,10 & 8,2  &  6,5  &  4,8  &  9,4 \\
   10,5 &  3,0  &  2,2  &  4,1  & 12,7 & 8,12 & 11,10 &  1,8  &  6,4  & 0,11 &  9,3  &  7,6  &  5,9 \\
   6,10 & 11,6  &  4,1  &  3,3  & 5,2  & 0,8  &  9,0  & 12,11 &  2,9  & 7,5  & 1,12  & 10,4  &  8,7 \\
   9,8  & 7,11  & 12,7  &  5,2  & 4,4  & 6,3  &  1,9  & 10,1  & 0,12  & 3,10 &  8,6  &  2,0  & 11,5 \\
   12,6 & 10,9  & 8,12  &  0,8  & 6,3  & 5,5  &  7,4  & 2,10  & 11,2  & 1,0  & 4,11  &  9,7  &  3,1 \\
   4,2  &  0,7  & 11,10 &  9,0  & 1,9  & 7,4  &  6,6  &  8,5  & 3,11  & 12,3 &  2,1  & 5,12  & 10,8 \\
   11,9 &  5,3  &  1,8  & 12,11 & 10,1 & 2,10 &  8,5  &  7,7  &  9,6  & 4,12 &  0,4  &  3,2  &  6,0 \\
   7,1  & 12,10 &  6,4  &  2,9  & 0,12 & 11,2 & 3,11  &  9,6  &  8,8  & 10,7 &  5,0  &  1,5  &  4,3 \\
   5,4  &  8,2  & 0,11  &  7,5  & 3,10 & 1,0  & 12,3  & 4,12  & 10,7  & 9,9  & 11,8  &  6,1  &  2,6 \\
   3,7  &  6,5  &  9,3  & 1,12  & 8,6  & 4,11 &  2,1  &  0,4  &  5,0  & 11,8 & 10,10 & 12,9  &  7,2 \\
   8,3  &  4,8  &  7,6  & 10,4  & 2,0  & 9,7  & 5,12  &  3,2  &  1,5  & 6,1  & 12,9  & 11,11 & 0,10 \\
   1,11 &  9,4  &  5,9  &  8,7  & 11,5 & 3,1  & 10,8  &  6,0  &  4,3  & 2,6  &  7,2  & 0,10  & 12,12 \\
  \end{array}\right]
\end{equation}

Although this is the join of two Latin squares, the repetition of ordered pairs means that they are not orthogonal.
An immediate solution might be to reverse the order of half the repeated pairs, but this approach ruins the join of Latin squares property.
\cite{andersonConstructionBalancedRoom1999} developed a theorem to overcome this property.
We present a slight adaptation of this theorem.
\begin{theorem}
Let $p = 8k + 5$ be a prime, $p > 5$.
Then a $\BRS$ of side $2p + 1$ exists.
\end{theorem}

\begin{proof}
Suppose we take $P \odot Q$, and label elements of the pairs from $P$ with subscript 1 and all those from $Q$ with subscript 2.
Then remove the diagonal pairs $(a, a)$, and call this new array $A$.
$A$ contains all the unordered pairs $\{a_1, b_1\},\{a_2, b_2\}$ exactly once.
The first row of $A$ contains the ordered pairs,
$(x_1^{4i}, -x_1^{4i + 1}), (x_1^{4i + 2}, -x_1^{4i + 1})$
due to $P$, and
$(x_2^{4i + 1}, -x_2^{4i}), (x_2^{4i + 1}, -x_2^{4i  +2})$
due to $Q$.

So all the squares subscript 1 appear as left hand members of pairs as do all the non-squares subscript 2.
Also, all the non-squares appear with subscript 1 in the right hand positions and all the squares with subscript 2.

If we denote the set of squares with subscript 1 by $R_1$, and with subscript 2 by $R_2$.
And similarly denote the sets of non-squares as $N_1$ and $N_2$.
Then in $A$, $R_1 \cup N_2$ occupies the left hand positions while $N_1 \cup R_2$ occupies the right.

Now define $B$ as the array of side $p$ whose first row contains the pairs:
\begin{equation}
(x_2^{4i}, -x_1^{4i + 1}), (x_1^{4i + 2}, x_2^{4i + 1}), (-x_2^{4i}, x_1^{4i + 1}), (-x_1^{4i + 2}, -x_2^{4i + 1})
\end{equation}

So in $B$ the left hand positions are occupied by $R_2 \cup R_1$, while the right are occupied by $N_1 \cup N_2$.
In order that all unordered pairs $\{a_1, b_2\}$ occur once in $B$ requires that in the first row all the members of $GF(p)$ occur once as a difference in one of two ways:

Either as $a_1 - b_2$ which are called $(1, 2)$ \inlinedef{mixed differences} or as $b_2 - a_1$, called $(2, 1)$ mixed differences.

Consider $(2, 1)$ mixed differences in the first row, they are:
\begin{align*}
       x^{4i} + x^{4i + 1} &= x^{4i}(1 + x)      \\
      -x^{4i} - x^{4i + 1} &=-x^{4i}(1 + x)      \\
   x^{4i + 1} - x^{4i + 2} &= x^{4i + 1}(1 - x)  \\
  -x^{4i + 1} + x^{4i + 2} &= -x^{4i + 1}(1 - x)
\end{align*}
i.e. $\pm x^{4i}(1 + x), \pm x^{4i + 1}(1 - x)$

Again, these are all of $GF(p)$ provided that $x^2- 1$ is a square.

Now arrange $A$ and $B$ in the following familiar manner to construct an array of side $2p + 1$:
\begin{equation}
\begin{array}{c|c|c}
\hline
      A     &   \phi   &       \theta ^T   \\ \hline
     \phi   &    B     &       \theta ^T   \\ \hline
    \theta  &  \theta  &  \infty _1, \infty _2  \\ \hline
\end{array}
\end{equation}

Next place the missing pairs in a manner similar to the original construction of Schellenberg.
Take a pair $(a_1, b_2)$ from the first row of $B$ in position $(0_2, h_2)$ and consider the cells in transversal $T_1$ of $B$, where $$T_1 = \left \{((0 + g)_2,(h + g)_2) \middle| g \in Z_p \right \}$$
If cell $(i_2, j_2)$ in $T_1$ contains $(n_1, k_2)$,
\begin{itemize}
  \item{put $(n_1, k_2)$ in cell $(i_2, \infty)$}
  \item{also put $(\infty _1, k_2)$ in $(k_1, j_2)$ and
    $(\infty _2, n_1)$ in $(n_1, j_2)$}
\end{itemize}

Now consider the transversal $T_2$ of $B$, where $(u_1, v_2)$ is a pair in $(0_2, l_2)$ and,
\begin{equation}
T_2 = \left \{((0 + g)_2, (l + g)_2) \middle| g \in Z_p \right \}
\end{equation}
If cell $(i_2, j_2)$ in $T_2$ contains $(h_1, m_2)$,
\begin{itemize}
  \item{put $(h_1, m_2)$ in cell $(\infty, j_2)$}
  \item{also put $(\infty _2, m_2)$ in $(0_2, m_1)$ and
    $(h_1, \infty _1)$ in $(0_2,h_1)$}
\end{itemize}

Finally put $(i_1, i_2)$ in $(i_2, i_2)$ for $0 \leq i \leq p - 1$, and put $(\infty _1, \infty _2)$ in cell $(\infty, \infty)$.
\end{proof}

\begin{lemma}
If $q = p^n \equiv 1\pmod 4$ is a prime power, then if we write down the non-zero elements of $GF(q)$ which are squares
\begin{equation}
 x^0, x^2, x^4, \ldots, x^{q - 3}
\end{equation}

Every non-square element of $GF(q)$ occurs exactly $(q - 1)/4$ times as a difference between these elements and every square occurs $(q - 5)/4$ times.
\end{lemma}

\begin{proof}
We can write the differences in the following way.
\begin{equation}
\begin{array}{cccc}
     (x^2-1)   &  x^{2}(x^{2}-1)  & \ldots  & x^{q-3}(x^{2}-1)   \\
     (x^4-1)   &  x^{2}(x^{4}-1)  & \ldots  & x^{q-3}(x^{4}-1)   \\
      \vdots   &     \vdots       & \ldots  &      \vdots        \\
   (x^{q-3}-1) & x^{2}(x^{q-3}-1) & \ldots  & x^{q-3}(x^{q-3}-1) \\
\end{array}
\end{equation}

Where the first column has been obtained by substracting $x^0$ from every other square.
The second column by taking $x^2$ from every other square, the third by taking $x^4$ and so on.

Because the differences in every column except the first have an even power of $x$ multiplied by some member of the first column it is clear that the quadratic nature of these terms is determined by whether or not the elements in the first column are square or not.

Therefore to complete the proof we need to show that among the $(q - 3)/2$ elements,
\begin{equation}
  x^2 - 1, x^4 - 1, \ldots, x^{q - 3} - 1
\end{equation}
every square occurs $(q - 5)/4$ times and every non-square $(q - 1)/4$ times.
We can further simplify the problem if we write,
\begin{equation}
x^{2i} - 1 = (x^i + 1)(x^i - 1) = \frac{x^i + 1}{x^i - 1}(x^i - 1)^2
\end{equation}

Then $x^{2i}-1$ is a square or non-square depending on whether:
\begin{equation}
  z_i = \frac{x^i + 1}{x^i - 1}
\end{equation}
is a square or not.
So it remains only to count the occurrences of squares/non-squares in the set:
\begin{equation}
  Z = \{z_1, z_2, \ldots, z_{(q - 3)/2}\}
\end{equation}

Suppose we consider instead the set,
\begin{equation}
  A = \{z_1, z_2, \ldots, z_{q - 2}\}
\end{equation}
which clearly has $q - 2$ members.
We can show that each member is a unique member of $GF(q)$, because if two different elements $z_i$ and $z_j$ were the same we would have:
\begin{align}
\frac{x^i + 1}{x^i - 1} &\equiv \frac{x^j + 1}{x^j - 1}\pmod q \\
         (x^j-1)(x^i+1) &\equiv (x^j+1)(x^i-1)\pmod q \\
       x^jx^i+x^j-x^i-1 &\equiv x^jx^i - x^j + x^i -1\pmod q \\
                    x^j &\equiv x^i\pmod q
\end{align}
but $1 \leq i, j \leq q - 2$, therefore the above is only satisfied if $x^j = x^i, \therefore i = j$, which contradicts the assumption that the two elements were different.

Further, if $z_i \equiv -1\pmod q$ then $x^i + 1 = -x^i + 1$, which implies $x^i = 0$.
So $-1 \notin A$.
Recall that when $q \equiv 1\pmod 4$, $-1$ is a square.

Also, if $z_i \equiv -1\pmod q$ then $x^i + 1 = x^i - 1$, so $1 \notin A$.
Therefore 1 is a square.
Finally, $x^{q - 1/2} = -1$ and so $z_{q - 1/2} = 0$.

Clearly $A = GF(q) \backslash \{1, -1\}$, and $A \backslash \{z_{q - 1/2}\} = GF(q) \backslash \{0, 1, -1\}$, so $A \backslash \{z_{q - 1/2}\}$ contains $(q - 1)/2$ non-squares and $(q - 5)/2$ squares.
\begin{equation}
A \backslash \{z_{(q - 1)/2}\} = \{z_1, z_2, \ldots, z_{(q - 3)/2}, z_{(q + 1)/2}, z_{(q + 3)/2}, \ldots, z_{q - 2}\}
\end{equation}

But,
\begin{equation}
z_{\{(q + 1)/2\} + k} = \frac{x^{\{(q + 1)/2\} + k} + 1}{x^{\{(q + 1)/2\} + k} - 1} = \frac{-x^k + 1}{-x^k - 1} = \frac{1}{z^k} k = 1, 2, \ldots, (q - 3)/2
\end{equation}

So $z_{\{(q + 1)/2\} + k}$ and $z_k$ (for $k = 1, 2, \ldots, (q - 3)/2]$) are both squares or both non-squares, hence $Z = \{z_1, z_2, \ldots, z_{(q - 3)/2}\}$ contains exactly $(q - 1)/4$ non-squares and $(q - 5)/4$ squares.
\end{proof}

\begin{corollary}
If $q = p^n \equiv 1$(mod 4) is a prime power, then if we write down the non-zero elements of $GF(q)$ which are non-squares:
\begin{equation}
x^1, x^3, x^5, \ldots, x^{q - 2}
\end{equation}
Every square element of $GF(q)$ occurs exactly $(q - 1)/4$ times as a difference between these elements and every non-square occurs $(q - 5)/4$ times.
\label{cor:non-squares}
\end{corollary}

\begin{proof}
As for the previous theorem except that each difference is multiplied by $x$, so squares become non-squares and non-squares, squares.
\end{proof}

\begin{theorem}
The blocks
\begin{equation}
\{\infty, x^0, x^2, \ldots, x^{q - 3}\}\{0, x^0, x^2, \ldots, x^{q - 3}\}
\end{equation}
along with their translates, form a $\BIBD$ with
$\lambda = (q - 1)/2$ in $GF(q)$, when
$q = p^n \equiv 1\pmod 4$
\end{theorem}

\begin{proof}
From the Lemma we have shown that in the set
\begin{equation}
\{x^0, x^2, x^4, \ldots, x^{q - 3}\}
\end{equation}
each square occurs as a difference $(q - 5)/4$ times, and each non-square $(q-1)/4$ times.
The right hand block also contributes the differences
\begin{equation}
x^0 - 0, 0 - x^0, 1 - x^2, x^2 - 1, \ldots, x^{q - 3} - 0, 0 - x^{q - 3}
\end{equation}
and
\begin{equation}
\pm x^0, \pm x^2, \ldots ,\pm x^{q - 3}
\end{equation}

And because $-1$ is a square, each square of $GF(q)$ occurs a further twice due to these differences.

\begin{center}
\begin{tabular}{c|ccc}
               & $\{\infty, x^0, x^2, \ldots, x^{q - 3}\}$ & $\{0, x^0, x^2, \ldots,x^{q - 3}\}$ &   Total     \\ \hline
     Squares   &               $(q - 5)/4$                 &         $\{(q - 5)/4\} + 2$         & $(q - 1)/2$ \\
   Non-squares &               $(q - 1)/4$                 &              $(q - 1)/4$            & $(q - 1)/2$ \\
\end{tabular}
\end{center}

Adjoining $\infty$ to a block of size $(q - 1)/2$ creates $(q - 1)/2$ pairs involving $\infty$.
Therefore in $\{\infty, x^0, x^2, \ldots, x^{q - 3}\}$ and the translates obtained from this block $\infty$ makes a pair with each member of $GF(q)$, $(q - 1)/2$ times.
Therefore the block design obtained from these two blocks is balanced, and the concurrence number is $\lambda = (q - 1)/2$.
\end{proof}

Although we don’t need the following result, it now follows straightforwardly from Corollary~\ref{cor:non-squares}.

\begin{corollary}
The blocks,
\begin{equation}
\{\infty, x^1, x^3, \ldots, x^{q - 2}\},\, \{0, x^1, x^3, \ldots, x^{q - 2}\}
\end{equation}
along with their translates, form a $\BIBD$ with
$\lambda = (q - 1)/2$ in $GF(q)$, when
$q = p^n \equiv 1\pmod 4$.
\end{corollary}

Now returning to the Anderson construction:

Let:
\begin{align}
  S &= R_1 \cup N_2 \cup \{\infty _2, \infty _1\} \\
  T &= N_1 \cup R_2 \cup \{0_1, 0_2\} \\
  U &= N_1 \cup N_2 \cup \{0 _2, \infty _1\} \\
  V &= R_2 \cup R_1 \cup \{0_1, \infty _2\}
\end{align}

Consider the pairs $\{a_1, b_1\}$ such that $a, b \in GF(q) \cup \{\infty\}$.
We know, due to the Bose result, that the blocks $\{\infty _1\} \cup R_1$ and $\{0 _1\} \cup R_1$ along with their translates contain each pair $\{a_1, b_1\}$, $(q - 1)/2$ times, also $\{\infty _1\} \cup N_1$ and $\{0 _1\} \cup N_1$ and translates contain each pair $\{a_1, b_1\}$, $(q - 1)/2$ times.
But,
\begin{equation}
\{\infty _1\} \cup R_1 \subset R_1 \cup N_2 \cup \{\infty _1, \infty _2\}
\end{equation}
and
\begin{equation}
\{\infty _1\} \cup N_1 \subset N_1 \cup R_2 \cup \{0_1, 0_2\}
\end{equation}

Therefore $S$ and $T$, along with their translates, contain each pair $\{a_1, b_1\}, 2\cdot (q - 1)/2 = q - 1$, times.

Finally, each pair $\{a_1, b_1\}$ occurs once more in the set $\{\infty _1, 0_1\} \cup R_1 \cup N_1$.
So in the entire block design each pair $\{a_1, b_1\}$ occurs $q$ times.

The same reasoning can be applied to show that the pairs $\{a_2,b_2\}$ for $a,b \in GF(q) \cup \{\infty\}$ also occur $q$ times in the block design.

It remains to show that the mixed pairs of the form $\{a_1, b_2\}$, where $a, b \in GF(q) \cup \{\infty\}$, also occur $q$ times in the block design.

Again we consider $(1, 2)$ mixed differences.
For a pair $\{a_1, a_2\}$ to occur in the block design requires a mixed difference of 0 in either $S$, $T$, $U$ or $V$.
In $T$, $0_1 - 0_2 = 0$, gives one occurrence, while in $U$ $x_{1}^{1} - x_{2}^{1}, x_{1}^{3} - x_{2}^{3}, \ldots, x_{1}^{q - 2} - x_{2}^{q - 2}$ gives $(q - 1)/2$ occurrences of 0.

Similarly, in $V$ $x_{1}^{0} - x_{2}^{0}, x_{1}^{2} - x_{2}^{2}, \ldots, x_{1}^{q - 3} - x_{2}^{q - 3}$ gives another $(q - 1)/2$ occurrences of 0 as a mixed difference.

Therefore, 0 occurs as mixed difference in $S, T, U$ and $V$ $(2\cdot (q - 1)/2) + 1 = q$ times, and so each pair of the form $\{a_1, a_2\}$, occurs $q$ times in the block design.

Next, consider pairs of the form $\{\infty _1, b_2\}$.
In $S$ and $U$, $\infty _1$ makes a pair with each of the $(q - 1)/2$ members of $N_2$.
Therefore in the blocks consisting of $S$, $U$ and their translates, $\infty _1$ makes a pair with each element $b_2$ (for all $b \in GF(q)$) $2 \cdot (q - 1)/2 = q - 1$ times.
Also in $U$, $\infty _1$ is paired with $0_2$, so $\infty _1$ makes a pair with each $b_2$ once more.
So there are $q$ pairs $\{\infty _1, b_2\}$ for each element $b_2$ in the block design.

By identical reasoning we can say that $\infty _2$ makes a pair with each $b_1$, for all $b \in GF(q)$, $(q - 1)/2$ times in $S$ and its translates.
Also $\{\infty _2, b_1\}$ (for all $b \in GF(q)$) occurs $(q + 1)/2$ times due to $V$ and its translates.
Therefore the pairs $\{\infty _2, b_1 \}$ each occur $q$ times in the block design, for all $b \in GF(q)$.

Next we confirm that all the pairs $\{a_1, b_2\}$ ($a, b \in GF(q)$) each occur $q$ times in the block design.
In order to do this we need to show that the non-zero members of $GF(q)$ each occur $q$ times as a mixed difference $a_1 - b_2$.

We have shown that for $q \equiv 1\pmod 4$
\begin{enumerate}
  \item{every square, non-square occurs as a difference of two
      squares of $GF(q)$, $s - 1$, $s$ times respectively.
      $s = (q - 1)/4$,}
  \item{every square, non-square occurs as a difference of
      non-two squares of $GF(q)$, $s$, $s - 1$ times
      respectively.}
\end{enumerate}

Consider first the mixed differences $a_1 - b_2 = c$, where $c$ is a square.

$U$ contains $N_1 \cup N_2$, so according to $B$ every square occurs as a $(1, 2)$ difference $(q - 1)/4$ times in $U$.
Also according to $B$ each non-square occurs as a $(1, 2)$ difference between members of $U$, $(q - 5)/4$ times.
Finally, $U$ contains the element $0_2$, so each non-square is generated once more.
\begin{equation}
x_1^1 - 0_2 = x^1_1, x_1^3 - 0_2 = x_1^3, \ldots, x_1^{q - 2} - 0_2 = x_1^{q - 2}
\end{equation}

Similarly, in $V$ each non-square occurs as a mixed difference $(q - 1)/4$ times while each square occurs as a difference $(q - 5)/4 + 1$ times.

Next, consider blocks $S$ and $T$.
In $T$ each square/non-square exists as a mixed difference once, in the following way:
\begin{align*}
       x_1^1 - 0_2   &= x^1      \\
       x_1^3 - 0_2   &= x^3      \\
           \vdots    &= \vdots   \\             
   x^{q-2}_1 - 0_2   &= x^{q - 2}
\end{align*}
\begin{align*}
       x_2^0 - 0_1   &= x^0      \\
       x_2^2 - 0_1   &= x^2      \\
    \vdots           &= \vdots   \\         
   x^{q-3}_2 - 0_1   &= x^{q-3}
\end{align*}

The remaining mixed differences in $S$ and $T$ all involve differences between one square and one non-square (and vice versa).
Now, for any $c \in GF(q)$, $a - b = c$ has $q$ solutions.
Two of these, $a = 0 - (-a)$ and $a = a - 0$ involve zero, and can be rejected because we have already considered all differences involving 0 in these blocks.
Further, from $A$ and $B$, we know there are $(q - 5)/4 + (q - 1)/4 = (q - 3)/2$ solutions which involve either both squares or both non-squares, and so there remain
\begin{equation}
(q - 2) - [(q - 3)/2] = (q - 1)/2
\end{equation}
solutions which involve elements of opposite quadratic character.
Hence we can say that each square, non-square occurs as a difference in $S$ and $T$ in $(q - 1)/2 + 1$ ways.

\begin{center}
\begin{tabular}{c|cc}
       & $c$ squares & $c$ non-squares \\ \hline
 S     & \multirow{2}{*}{$\frac{q - 1}{2} + 1$} & \multirow{2}{*}{$\frac{q - 1}{2} + 1$}                                  \\
 T     &                       &                       \\
 U     & $\frac{q - 1}{4}$     & $\frac{q - 5}{4} + 1$ \\
 V     & $\frac{q - 5}{4} + 1$ & $\frac{q - 1}{4}$     \\ \hline
 Total &        $q$            &      $q$
\end{tabular}
\end{center}

Which confirms that each pair $\{a_1, b_2\}$ for $a, b \in GF(a) \cup \{\infty\}$ appears in $q$ blocks of the block design.
So every pair $\{a, b\}$, $a, b \in GF(q)_1 \cup GF(q)_2 \cup \{\infty _1, \infty _2\}, a \neq b$ appears in $q$ blocks of the block design. Hence that design is a $\BIBD$, with parameters $(2(q + 1), q + 1, q)$.
