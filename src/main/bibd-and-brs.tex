In any Room square, the pairs $\{x, y\}$ are unordered.
If we replace these pairs by one of the ordered pairs $(x, y)$ or $(y, x)$ we call the resulting array an \inlinedefnoidx{ordered Room square}\index{Room square!ordered} (ORS).

Consider the following ordered Room square:
\begin{equation}
  \begin{bmatrix}
    \infty 0 &    62    &    54    &            &    31    &            &          \\
             & \infty 1 &    03    &     65     &          &     42     &          \\
             &          & \infty 2 &     14     &    06    &            &    53    \\
      64     &          &          &  \infty 3  &    25    &     10     &          \\
             &    05    &          &            & \infty 4 &     36     &    21    \\
      32     &          &          &            &          &  \infty 5  &    40    \\
      51     &    43    &    20    &     20     &          &            & \infty 6 \\
  \end{bmatrix}
  \label{eq:ors}
\end{equation}

Suppose we extract the blocks of a design (not necessarily a BIBD) from this square in one of the two following ways:

\begin{enumerate}
 \item{As \emph{half-columns}. This means taking all left hand
    members of pairs from a particular column as the members
    of one block, and all right hand members as another
    block.}
 \item{By \emph{half-rows}. Take all left members from a particular
    row as one block, and all right members as another
    block.}
\end{enumerate}

Notice that for this example, both methods generate exactly the same blocks.

\begin{equation}
\begin{split}
\{\infty,6,5,3\},\{0,2,4,1\},\{\infty,0,6,4\},\{1,3,5,2\},\{\infty,1,0,5\},\{2,4,6,3\},\{6, \infty,2,1\}, \\
\{4,3,5,0\},\{0,\infty,3,2\},\{5,4,6,1\},\{3,1,\infty,4\},\{2,6,5,0\},\{5,4,2,\infty\},\{1, 3,0,6\}
\end{split}
\end{equation}

Certainly this design seems to have the appropriate parameters to qualify as a BIBD.
The original Room square was based on eight elements, hence $v = 8$.
Also, the Room square’s seven rows each produced two blocks of four elements, hence $b = 14$ and $k = 4$.

Finally, each element occurred once in each row of the Room square, therefore in half the total number of blocks, hence $r = 7$.

\begin{equation}
bk = 4\cdot 14 = 7 \cdot 8 = vr
\end{equation}

So this design is a BIBD, provided:
\begin{equation}
\lambda = \frac{r(k - 1)}{v - 1} = \frac{7(3)}{7} = 3
\end{equation}

In other words, provided each pair of elements occurs in precisely three blocks.

By definition a \inlinedef{balanced Room square} $\BRS(n)$ is an ordered Room square based on $n$ elements whose corresponding block design, derived as above, is a BIBD.

