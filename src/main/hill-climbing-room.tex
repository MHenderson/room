To generate a Room square all that remains is to produce another one-factorisation $G$, say, which is orthogonal to $F$.
This will inevitably require slight modifications to be made to $H_1$ and $H_2$.

Now if an array $R$ is constructed in which the rows are labelled with the one-factors of $F(f_1,f_2,...,f_{n-1})$, and the columns are labelled with the partial one-factors of $G(g_1,g_2,...,g_{n-1})$.
Then $R$ will be a Room square if the $(f_i,g_j)$ cell contains $\{x,y\}$, if and only if $(f_i,\{x,y\}) \in F$ and $(g_j,\{x,y\}) \in G$ and is empty otherwise.

Again these two heuristics are due to Dinitz and Stinson and originally presented in
\cite{dinitzHillClimbingAlgorithmConstruction1987}.
Although a necessary correction has been made as will become apparent.

\begin{algorithm}[H]
  1. Choose any live point $x$.

  2. Choose any partial one-factor $g_i$ such that $x$ does
     not occur in $g_i$.

  3. Choose any $y \neq x$ such that $x$ and $y$ do not occur
     together in $G$.

  4. Let $f_j$ be the one-factor of $F$ which contains the 
     edge $\{x, y\}$.
     
  \uIf{$R(f_j, g_i)$ is not empty}{
    $OH_1$ fails \;
  }
  \uElseIf{$y$ does not occur in $g_i$}{
    Replace $G$ by $G \cup (g_i, \{x, y\})$.

    Define $R(f_j, g_i)=\{x, y\}$.
  }
  \Else{
    there is a pair in $G$ of the form
     $(g_i, \{z, y\}) \hspace{0.5cm} z \neq x$.

    Replace $G$ by
    $G \cup (g_i, \{x, y\}) \backslash (g_i, \{z, y\})$.

    Define $R(f_k, g_i)$, to be empty$^i$,
    where $(f_k, \{z, y\}) \in F$.
  }
\caption{$OH_1$}
\end{algorithm}

\begin{algorithm}[H]
  1. Choose any live partial one-factor $g_i$.

  2. Choose any $x$ and $y \neq x$ such that $x$ and $y$
     do not occur together in $g_i$.

  3. Let $f_j$ be the one-factor of $F$ which contains
     the edge $\{x, y\}$.
  
  \uIf{$R(f_j, g_i)$ is not empty}{
    $OH_2$ fails
  }
  \uElseIf{$x$ and $y$ do not occur together}{
   Replace $G$ by $G \cup (g_i, \{x, y\})$.

   Define $R(f_j, g_i) = \{x, y\}$.
  }
  \Else{
    there is a pair in $G$ of the form
    $(g_k, \{x, y\}) \hspace{0.5cm} (k \neq i$)

       Replace $G$ by
       $G \cup (g_i, \{x, y\}) \backslash (g_k, \{x, y\})$

       Define $R(f_j, g_i) = \{x, y\}$.

       Define $R(f_j, g_k)$ to be empty.
  }
\caption{$OH_2$}
\end{algorithm}

\begin{example}
Suppose the factorisation $F$ from the earlier example has been completed and is represented by the set:

\begin{equation}
  \begin{split}
    F = \{(f_1\{1,2\}),(f_1\{3,5\}),(f_1\{4,6\}),(f_2\{1,4\}),(f_2\{2,3\}),(f_2\{5,6\}), \\
    (f_3\{1,6\}),(f_3\{2,5\}),(f_3\{3,4\}),(f_4\{1,3\}),(f_4\{2,6\}),(f_4\{4,5\}),(f_5\{1,5\}),(f_5\{3,6\})\}
  \end{split}
\end{equation}

Notice that this is precisely the one-factorisation of $K_6$ shown in Figure~\ref{fig:k6-factorisation}.

Now suppose we have established the following one-factors in $G$:

\begin{equation}
  G = \{(g_1, \{1, 4\}), (g_2, \{1, 6\}), (g_3, \{3, 6\}), (g_5\{5, 6\}), (g_5\{1, 2\})\}
\end{equation}

At this state $R$ looks like 

\begin{equation}
  \begin{bmatrix}
     -  &   g_1    &    g_2    &    g_3   & g_4 &    g_5    \\
    f_1 &     -    &     -     &    -     &  -  & \{1, 2\}  \\
    f_2 & \{1, 4\} &     -     &    -     &  -  & \{5, 6\}  \\
    f_3 &     -    &  \{1, 6\} &    -     &  -  &     -     \\
    f_4 &     -    &     -     &    -     &  -  &     -     \\
    f_5 &     -    &     -     & \{3, 6\} &  -  &     -     \\
  \end{bmatrix}
\end{equation}

Now apply $OH_1$:

\begin{enumerate}
  \item{Choose $x = 5$, suitably live.}
  \item{Choose $g_3$, in which $5$ does not occur.}
  \item{$5$ does not occur together with $2$ in $G$, so we are
     free to choose $y = 2$.}
  \item{In $F$, $\{2, 5\} \in f_3$.}
  \item{
    $f_3, g_3$ is empty in $R$, also $y = 2 \notin g_3$.
    \begin{itemize}
       \item{Replace $G$ with $G \cup (g_3, \{5, 2\})$.}
       \item{Define $R(f_3, g_3) = \{5, 2\}$.}
    \end{itemize}
  }
\end{enumerate}

This decreases the cost by one, alternatively we might have chosen, at stage 3.
$y = 3$, in that case.

\begin{enumerate}
  \setcounter{enumi}{4}
  \item{$\{3, 5\} \in f_1$.}
  \item{
    $f_1, g_3$ is empty in $R$, also $y \in g_3$, occurring
      in the pair $(g_3, \{3, 6\}), z = 6$.
    \begin{itemize}
      \item{Replace $G$ with
        $G \cup (g_3, \{3, 5\}) \backslash (g_3, \{3,
        6\})$.}
      \item{Define $R(f_1, g_3) = \{3, 5\}$.}
      \item{Define $R(f_5, g_3)$ to be empty.}
    \end{itemize}
  }
\end{enumerate}

Which leaves the cost unaffected.
Suppose now that $R$ is the array after this second version of the application of $OH_1$:

\begin{equation}
  \begin{bmatrix}
     -  &   g_1    &    g_2    &    g_3   & g_4 &    g_5    \\
    f_1 &     -    &     -     & \{3, 5\} &  -  & \{1, 2\}  \\
    f_2 & \{1, 4\} &     -     &    -     &  -  & \{5, 6\}  \\
    f_3 &     -    &  \{1, 6\} &    -     &  -  &     -     \\
    f_4 &     -    &     -     &    -     &  -  &     -     \\
    f_5 &     -    &     -     &    -     &  -  &     -     \\
  \end{bmatrix}
\end{equation}

Now if we apply $OH_2$:

\begin{enumerate}
  \item{Choose $g_4$, a live partial one-factor.}
  \item{Choose $x = 1, y = 2$, neither of which occur in $g_4$.}
  \item{$(f_1, \{1, 2\}) \in F$.}
  \item{
    $f_1, g_4$ is empty in $R$, also $x$ and $y$ do occur
    together, $(g_5, \{1, 2\}) \in G$.
    \begin{itemize}
      \item{Replace $G$ with
       $G \cup (g_4, \{1, 2\}) \backslash (g_5, \{1, 2\})$}
      \item{Define $R(f_1, g_4) = \{1, 2\}$.}
      \item{Define $R(f_1, g_5)$ to be empty.}
    \end{itemize}
  }
\end{enumerate}

This procedure leaves the cost unaffected and if instead we had chosen at $2$.
$x = 3, y = 4$, then would have been required to replace $G$ with $G \cup (g_4,\{3,4\})$, and put $\{3, 4\}$ in cell $(f_3, g_4)$ of $R$, an action which reduces the cost by one.
However, we know that two orthogonal one-factorisations of $K_6$ are equivalent to a Room square of side 5, which has been shown not to exist.
Hence it would be futile to continue with this method in this particular case.
Nevertheless the example shows how the heuristics work.
\end{example}

There is no guarantee of success with repeated use of these heuristics, although Dinitz and Stinson are quick to point out that the algorithm involving $H_1$ and $H_2$ has never
\footnote{In over ten-million attempts, they claim.}
failed to produce the desired one-factorisation.
If we hope to use the $OH_1$ and $OH_2$ in a similar algorithm then the possibility of failure becomes a real possibility.

Two possibilities exist, either both heuristics fail or successive use of them leads to an infinite loop.
In order to avoid both we introduce a \emph{threshold} function which simply arrests the progress of the algorithm after a certain number of iterations of the heuristics.
Dinitz and Stinson found after experimentation that the following function is suitable.

\begin{equation}
T(n) = 100n
\end{equation}

Then the hill-climbing algorithm for finding a Room square is as follows:

\begin{algorithm}[H]
\KwIn{Use the previous hill climbing algorithm to construct $F$,
     a one-factorisation of $K_n$.}
\KwIn{Number of iterations initialised to be $0$}
\While{(number of iterations $<T(n)$) and $c(G) \neq 0$}{
   Choose $r = 1$ or $r = 2$ at random with equal probability.
   Perform $OH_r$.
   Increment number of iterations.}
\end{algorithm}
