In the next section we introduce a theorem that will allow us create Room squares of side $rn$, based on a Room square of side $r$ and a positive integer $n < r$.

As the construction used in proving that theorem is quite complicated we give an example first of the $n = 3$ case, namely tripling the size of a Room square.

\begin{example}
Suppose we wish to triplicate the following Room square.

\begin{equation}
  \begin{bmatrix}
    \infty 0 &   - &  -  &  25 &   - &  16 &  34 \\
    45 &  \infty 1 &  -  &   - &  36 &   - &  20 \\
    31 &  56 &  \infty 2 &   - &   - &  40 &   - \\
     - &  42 &  60 &  \infty 3 &   - &   - &  51 \\
    62 &   - &  53 &  01 &  \infty 4 &   - &   - \\
     - &  03 &   - &  64 &  12 &  \infty 5 &   - \\
     - &   - &  14 &   - &  05 &  23 &  \infty 6 
  \end{bmatrix}
  \label{eq:triple-room}
\end{equation}

We could start by arranging nine copies of this array into a $21 \times 21$ array $\mathcal{R}$ but such an array would only have a few properties in common with a Room square.

Consider just the first row of $\mathcal{R}$:
\begin{equation*}
  \begin{bsmallmatrix}
    \infty 0 & - & - & 25 & - & 16 & 34 & \infty 0 & - & - & 25 & - & 16 & 34 & \infty 0 & - & - & 25 & - & 16 & 34 \\
  \end{bsmallmatrix}
\end{equation*}

A partial solution is to triple the size of the set on which the square is based by adding subscripts:
\begin{equation*}
  \begin{bsmallmatrix}
    \infty_{1} 0_{1} & - & - & 2_{1}5_{1} & - & 1_{1}6_{2} & 3_{2}4_{1} & \infty_{2} 0_{2} & - & - & 2_{2}5_{2} & - & 1_{2}6_{2} & 3_{2}4_{2} & \infty_{3} 0_{3} & - & - & 2_{3}5_{3} & - & 1_{3}6_{3} & 3_{3}4_{3} \\
  \end{bsmallmatrix}
\end{equation*}

Unfortunately, this row has simply too many elements.
There should be only eleven pairs, not twelve.
Also, the new Room-ish square is based on a set of 24 elements not 22, as required.
\end{example}

Wallis's first idea was to merge the three $\infty _\mathrm{i}$ elements into a single element.
Eventually he decided instead to go back to the original square and strip out all diagonal entries.

Building a Room square is then a question of arranging arrays like \eqref{eq:triple-room-frame}, sometimes called \inlindef{frames}, into a $21 \times 21$ array, and then finding a way of filling in the missing two pairs from every row of the new square, with the aim of producing a Room square based on
\begin{equation*}
  S = \{\infty, 0_{1}, 1_{1}, \ldots, 6_{1}, 0_{2}, 1_{2}, \ldots, 6_{2}, 0_{3}, 1_{3}, \ldots, 6_{3}\}
\end{equation*}

\begin{equation}
  R_{ij} = \begin{bmatrix}
             - &           - &           - &  2_{i}5_{j} &           - &  1_{i}6_{j} &  3_{i}4_{j} \\
    4_{i}5_{j} &           - &           - &           - &  3_{i}6_{j} &           - &  2_{i}0_{j} \\
    3_{i}1_{j} &  5_{i}6_{j} &           - &           - &           - &  4_{i}0_{j} &           - \\
             - &  4_{i}2_{j} &  6_{i}0_{j} &           - &           - &           - &  5_{i}1_{j} \\
    6_{i}2_{j} &           - &  5_{i}3_{j} &  0_{i}1_{j} &           - &           - &           - \\
             - &  0_{i}3_{j} &           - &  6_{i}4_{j} &  1_{i}2_{j} &           - &           - \\
             - &           - &  1_{i}4_{j} &           - &  0_{i}5_{j} &  2_{i}3_{j} &           - 
  \end{bmatrix}
  \label{eq:triple-room-frame}
\end{equation}

Inevitably this approach leads to new problems.
Firstly consider how to arrange the frames appropriately.

\begin{example}
Suppose $R_{12}$ is placed next to $R_{13}$ in $\mathcal{R}$.
The left hand members of pairs in each row of $R_{12}$ will then be repeated in the same row of $\mathcal{R}$ due to the placing of $R_{13}$.

The same will be true for any $R_{ij}$ next to any $R_{ik}$, or $R_{ij}$ next to $R_{kj}$.
For that reason, in the array of $R_{ij}$ we require in each super-row that $i$ and $j$ take on values 1, 2 and 3.
To satisfy the column property we also require that no $R_{ij}$ occurs above or below an $R_{ik}$ or $R_{kj}$.
For that reason, for any super-column of $R_{ij}$s, $i$ and $j$ must independently take all values $1, 2, 3$.
Then every member of $S \backslash \{\infty\}$ except for the missing $\{x_i: 0 \leq x \leq 6\}$ from all columns $x_i$ occurs once in the corresponding 7 columns of $\mathcal{R}$.

Furthermore, as we are aiming for an array in which all the unordered pairs from $S$ occur once if we also insist that each value of $i$ is paired with each value of $j$ exactly once in super-array then we should obtain most of these pairs.

In fact, because $R_{ij} \cup R_{ji}$ contains unordered pairs from $\{0_{i},0_{j},1_{i},1_{j}, \ldots, 6_{i},6_{j}\}$, except those of the form $\{x_{i}, x_{j}\}$ $1 \leq i, j \leq 3$ shall obtain all the unordered pairs of $S$ except those of the form:
\begin{equation*}
\{\infty, x_1\}, \{\infty, x_2\}, \{\infty, x_3\}, \{x_1, x_2\}, \{x_1, x_3\}, \{x_2, x_3\}
\end{equation*}
for all $0 \leq x \leq 6$.

The arrangement of frames in \eqref{eq:frame-arrangement} satisfies all required conditions.

\begin{equation}
  \begin{bmatrix}
    R_{11} & R_{22} & R_{33} \\
    R_{32} & R_{13} & R_{21} \\
    R_{23} & R_{31} & R_{12} \\
  \end{bmatrix}
  \label{eq:frame-arrangement}
\end{equation}

Now when we assemble the $R_{ij}$, something quite close to a Room square of side 21, based on $S$, is constructed.
\end{example}

We could solve both problems of missing pairs as well as completing rows/columns by placing pairs of the form $\{\infty, x_{j}\}$ and $\{x_{i}, x_{j}\}$ at the intersection of rows $x_{j}$ and column $x_{j}$.
Unfortunately, this intersection is a single box, and we can't have two pairs in one box.

Wallis's solution is to permute the columns of some of the $R_{ij}$s from one super-column of the array of frames with the intention of arranging it so that the elements $x_i x_j$ are missing from some column $y \neq x_j$.
Then $\{\infty, x_j\}$ can be placed in column $x_j$ and $\{x_i,x_j\}$ can be placed in column $y$.

In creating a frame from $R$ the pair $\{\infty, 0\}$ was removed from row $0_{1}$ and column $0_{1}$.
While where is no problem in putting $\{\infty, 0_{1}\}$ in position $(0_{1}, 0_{1})$, if we want to put $\{0_{2}, 0_{3}\}$ in row $0_{1}$ it must go in some other column of block $R_{11}$, while remaining in column $1_{1}$ of blocks $R_{23}$ and $R_{32}$.
This can be achieved through a column permutation which is only applied to $R_{23}$ and $R_{32}$.

Notice that it is of little use to swap columns $0_1$ and $4_1$ of blocks $R_{23}$ and $R_{32}$ as the fourth column is already occupied in the first row of block $R_{11}$.
But we can swap $0_1$ with any of $1_1, 2_1, 3_1$ or $5_1$ because all these columns are empty in row $0_1$.
The essential property required of any column permutation, $\theta$ say, is that $(x, x\theta)$ is unoccupied in the original Room square.

\begin{lemma}
\label{lem:permute}
Given a Room square $R$ of side $r$, where $r = 2s + 1$, there are $s$ permutations $\phi_1, \phi_2, \dots, \phi_s$ of $\{1, 2, \ldots, r\}$ with the properties that $k\phi_{i} = k\phi_{j}$ never occurs unless $i = j$, and that cell $(k, k\phi_{i})$ is empty for $1 \leq k \leq r, 1 \leq i\leq s$.
\end{lemma}

\begin{proof}
Define a matrix $M$ in the following manner: If position $(k,l)$ is \inlinedef{empty} in $R$ then the $(k,l)$ position of $M$ is 1, otherwise it is 0.

As $M$ is a matrix of zeros and ones, whose every row and column sum is equal to $s$, it can be decomposed into $s$ matrices, each of which having exactly one $1$ in each row and column.

\begin{equation}
M = P_1 + P_2 + \ldots + P_s
\end{equation}

Such matrices, interpreted in the following manner, are known as \inlinedef{permutation matrices}.

Define $\phi_i$ as the permutation corresponding to matrix $P_i$ such that if $(k,l) = 1$ in $P_i$ then $k\phi _i = l$.

The definition of $M$ ensures that $(k, k\phi _i)$ is empty in $R$, while if $k\phi_{i} = k\phi_{j}$ is true for some $i \neq j$ then $P_i$ and $P_j$ both have a 1 in position $(k, k\phi_i)$, so $M$ has an entry equal to 2 or more, contradicting the definition.
\end{proof}

\begin{example}
The matrix $M$ associated with \eqref{eq:triple-room} is:

\begin{equation}
 M = \begin{bsmallmatrix}
    0 & 1 & 1 & 0 & 1 & 0 & 0 \\
    0 & 0 & 1 & 1 & 0 & 1 & 0 \\
    0 & 0 & 0 & 1 & 1 & 0 & 1 \\
    1 & 0 & 0 & 0 & 1 & 1 & 0 \\
    0 & 1 & 0 & 0 & 0 & 1 & 1 \\
    1 & 0 & 1 & 0 & 0 & 0 & 1 \\
    1 & 1 & 0 & 1 & 0 & 0 & 0 \\
  \end{bsmallmatrix}
  \label{eq:permuatation-matrix}
\end{equation}

Now \eqref{eq:permuatation-matrix} can be decomposed (not uniquely) into permutation matrices:
\begin{equation}
 M = P_1 + P_2 + P_3 = 
  \begin{bsmallmatrix}
    0 & 1 & 0 & 0 & 0 & 0 & 0 \\
    0 & 0 & 1 & 0 & 0 & 0 & 0 \\
    0 & 0 & 0 & 0 & 1 & 0 & 0 \\
    1 & 0 & 0 & 0 & 0 & 0 & 0 \\
    0 & 0 & 0 & 0 & 0 & 1 & 0 \\
    0 & 0 & 0 & 0 & 0 & 0 & 1 \\
    0 & 0 & 0 & 1 & 0 & 0 & 0 \\
  \end{bsmallmatrix}
  +
  \begin{bsmallmatrix}
    0 & 0 & 0 & 0 & 1 & 0 & 0 \\
    0 & 0 & 0 & 1 & 0 & 0 & 0 \\
    0 & 0 & 0 & 0 & 0 & 0 & 1 \\
    0 & 0 & 0 & 0 & 0 & 1 & 0 \\
    0 & 1 & 0 & 0 & 0 & 0 & 0 \\
    0 & 0 & 1 & 0 & 0 & 0 & 0 \\
    1 & 0 & 0 & 0 & 0 & 0 & 0 \\
  \end{bsmallmatrix}
  +
  \begin{bsmallmatrix}
    0 & 0 & 1 & 0 & 0 & 0 & 0 \\
    0 & 0 & 0 & 0 & 0 & 1 & 0 \\
    0 & 0 & 0 & 1 & 0 & 0 & 0 \\
    0 & 0 & 0 & 0 & 1 & 0 & 0 \\
    0 & 0 & 0 & 0 & 0 & 0 & 1 \\
    1 & 0 & 0 & 0 & 0 & 0 & 0 \\
    0 & 1 & 0 & 0 & 0 & 0 & 0 \\
  \end{bsmallmatrix}
\end{equation}

The permutations associated with these matrices are, in cycle notation:

\begin{equation}
  \phi _1 = (1235674), \phi _2 = (1524637), \phi _3 = (1345726)
\end{equation}

Applying $\phi _1$ to the columns of blocks $R_{23}$ and $R_{32}$ gives:

\begin{equation*}
  \begin{bmatrix}
         &      0_{1} &   1_{1}    &    2_{1}    &    3_{1}   &    4_{1}    &    5_{1}   &    6_{1}   \\
   0_{1} &            &            &             & 2_{1}5_{1} &             & 1_{1}6_{1} & 3_{1}4_{1} \\
   1_{1} & 4_{1}5_{1} &            &             &            & 3_{1}6_{1}  &            & 2_{1}0_{1} \\
   2_{1} & 3_{1}1_{1} & 5_{1}6_{1} &             &            &             & 4_{1}0_{1} &            \\
   3_{1} &            & 4_{1}2_{1} &  6_{1}0_{1} &            &             &            & 5_{1}1_{1} \\ 
   4_{1} & 6_{1}2_{1} &            &  5_{1}3_{1} & 0_{1}1_{1} &             &            &            \\
   5_{1} &            & 0_{1}3_{1} &             & 6_{1}4_{1} & 1_{1}2_{1}  &            &            \\
   6_{1} &            &            &  1_{1}4_{1} &            & 0_{1}5_{1}  & 2_{1}3_{1} &            \\
   0_{1} & 2_{3}5_{2} &            &             & 3_{3}4_{2} &             &            & 1_{3}6_{2} \\
   1_{1} &            & 4_{3}5_{2} &             & 2_{3}0_{2} &             & 3_{3}6_{2} &            \\
   2_{1} &            & 3_{3}1_{2} &  5_{3}6_{2} &            &             &            & 4_{3}0_{2} \\
   3_{1} &            &            &  4_{3}2_{2} & 5_{3}1_{2} & 6_{3}0_{2}  &            &            \\
   4_{1} & 0_{3}1_{2} & 6_{3}2_{2} &             &            & 5_{3}3_{2}  &            &            \\
   5_{1} & 6_{3}4_{2} &            &  0_{3}3_{2} &            &             & 1_{3}2_{2} &            \\
   6_{1} &            &            &             &            & 1_{3}4_{2}  & 0_{3}5_{2} & 2_{3}3_{2} \\
   0_{1} & 2_{2}5_{3} &            &             & 3_{2}4_{3} &             &            & 1_{2}6_{3} \\
   1_{1} &            & 4_{2}5_{3} &             & 2_{2}0_{3} &             & 3_{2}6_{3} &            \\
   2_{1} &            & 3_{2}1_{3} &  5_{2}6_{3} &            &             &            & 4_{2}0_{3} \\
   3_{1} &            &            &  4_{2}2_{3} & 5_{2}1_{3} & 6_{2}0_{3}  &            &            \\
   4_{1} & 0_{2}1_{3} & 6_{2}2_{3} &             &            & 5_{2}3_{3}  &            &            \\
   5_{1} & 6_{2}4_{3} &            &  0_{2}3_{3} &            &             & 1_{2}2_{3} &            \\
   6_{1} &            &            &             &            & 1_{2}4_{3}  & 0_{2}5_{3} & 2_{2}3_{3} \\
  \end{bmatrix}
\end{equation*}

Which leaves us free to put $\{\infty, x_1\}$ into $(x_1, x_1)$ and $\{x_2, x_3\}$ into $(x_1, (x\phi _1)_1)$ e.g. $\{1_2, 1_3\}$ can go into $(2_1, (2\phi _1)_1) = (2_1,3_1)$.
The permutation chosen ensures that cell $(2, 3)$ of the original square is empty.

Filling in the rest of block $R11$ gives:
\begin{equation*}
  \begin{bmatrix}
         &      0_{1} &   1_{1}    &    2_{1}    &    3_{1}   &    4_{1}    &    5_{1}   &    6_{1}   \\
   0_{1} & \infty     0_{1} & 0_{2}0_{3} &             & 2_{1}5_{1} &             & 1_{1}6_{1} & 3_{1}4_{1} \\
   1_{1} & 4_{1}5_{1} & \infty     1_{1} &  1_{2}1_{3} &            & 3_{1}6_{1}  &            & 2_{1}0_{1} \\
   2_{1} & 3_{1}1_{1} & 5_{1}6_{1} &  \infty     2_{1} &            & 2_{2}2_{3}  & 4_{1}0_{1} &            \\
   3_{1} & 3_{2}3_{3} & 4_{1}2_{1} &  6_{1}0_{1} & \infty     3_{1} &             &            & 5_{1}1_{1} \\ 
   4_{1} & 6_{1}2_{1} &            &  5_{1}3_{1} & 0_{1}1_{1} & \infty     4_{1}  & 4_{2}4_{3} &            \\
   5_{1} &            & 0_{1}3_{1} &             & 6_{1}4_{1} & 1_{1}2_{1}  & \infty     5_{1} & 5_{2}5_{3} \\
   6_{1} &            &            &  1_{1}4_{1} & 6_{2}6_{3} & 0_{1}5_{1}  & 2_{1}3_{1} & \infty     6_{1} \\
  \end{bmatrix}
\end{equation*}

Notice that this satisfies the row and column properties of a Room square for the first seven rows and columns.

Next we move onto the second diagonal block, because missing from row and column $x_2$ are the elements $\infty, x_1, x_2, x_3$.
However this time we try to find a home for pairs of the form $\{\infty, x_2\}$ and $\{x_1, x_3\}$.

We can put pairs of the form $\{x_1, x_3\}$ down the diagonal and permute the columns of block $R_{22}$ with a permutation from Lemma \ref{lem:permute} to ensure that column $(x\phi _2)_2$ has no $x_2$, allowing us to put $\{\infty, x_2\}$ in that column.

For instance, using $\phi _2$ we can complete block $R_{13}$, and the corresponding seven rows and columns, by putting:

$\{x_1, x_3\}$ in $x_2, x_2$ and $\{\infty, x_2\}$ in $x_2,(x \phi _2)_2$.

Taking the same approach with the third diagonal block we permute columns in $R_{33}$ using $\phi _3$ and fill-in by putting:

$\{x_1, x_2\}$ in $x_3, x_3$ and $\{\infty, x_3\}$ in $x_3, (x \phi _3)_3$

Which results in a Room square of side 21 based on:
$$S = \{\infty, 0_1, 1_1, \ldots, 6_1, 0_2, 1_2, \ldots, 6_2, 0_3, 1_3, \ldots, 6_3\}$$

This square is straightforwardly transformed to a Room square \eqref{eq:roomtwentyone} based on $\{\infty, 0, 1, \ldots, 20\}$ by using the transformation $x_{i} = x + 7(i - 1)$

\begin{equation}
  \begin{bsmallmatrix}
    \infty,0 &   7,14   &      & 2,5 &     & 1,6 & 3,4 & 10,11 &       & 8,13 & &       & 9,12 & & 15,20 & 17,18 &       & & 16,19 & &       \\
      4,5    & \infty,1 & 8,15 &     & 3,6 &     & 2,0 &  9,7  & 10,13 &      & & 11,12 &      & &       & 16,14 & 18,19 & &       & & 17,20 \\
  \end{bsmallmatrix}
  \label{eq:roomtwentyone}
\end{equation}
\end{example}

