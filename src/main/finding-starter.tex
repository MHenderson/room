Suppose we wish to construct another Room square of the same size as \eqref{eq:cyclic} based on the same symbols.
This new square will also be standardized so we need only determine the three pairs that accompany $\{0, \infty\}$ in the first row (the starter), and the positions they occupy.

The set we will use to build our starter will be ${1, 2, \ldots, 6}$.

Each member of this set must occur exactly once in the pairs of the starter – in order to satisfy the row condition for a Room square.
Because of the cyclical construction the condition is automatically true for successive rows if true
for the first.

Consider the existence in \eqref{eq:cyclic} of an arbitrary pair $\{a, b\}$.
We know one of the following must be true.

Either $\{2 + i, 5 + i\} = \{a, b\}$ or $\{1 + i, 6 + i\} = \{a, b\}$ or $\{3 + i, 4 + i\} = \{a, b\}$ for $i = 0, 1, 2, \ldots, 6$.
Say $a - b = 1$.
Then $\{2 + i, 5 + i\} = \{a, b\}$ could never be true because $(2 + i) - (5 + i) = -3\pmod 7 = 4$ and $(5 + i) - (2 + i) = 3$.
Similarly, the differences in $\{1, 6\}$ are $\pm 5$ so $\{a, b\}$ couldn't be generated from $\{1, 6\}$.

However, $(4 + i) - (3 + i) = 1$ so $\{a, b\}$ will inevitably be generated by $\{3, 4\}$ for some value of $i = 0, 1, \ldots, 6$.
e.g. $\{2, 3\} = \{3 + 6, 4 + 6\}$

Because $a$ and $b$ separately take on all values from $\{0, 1, 2, \ldots, 6\}$, their differences will similarly take on all these values (except 0 because there are no pairs of the form $\{a, a\}$) and so an essential property for the starter must be that the six differences generated by its three pairs contain all of $\{1, 2, \ldots, 6\}$.

When a starter satisfies this property, and the condition that the pairs contain in their union all of $\{1, 2, \ldots, 6\}$, it is clear that it will inevitably generate the correct pairs which populate a $7 \times 7$ Room square. 
There are three pairs in the starter, each generates seven unique pairs under cyclical construction, which along with the seven pairs generated by $\{0, \infty\}$ counts for all the 28 unordered pairs from $\{\infty, 0, 1, \ldots, 6\}$.

A starter for larger Room squares of course has to obey the same criterion.
We include a general definition based on
\cite{dinitzContemporaryDesignTheory1992}.

If $G$ is an additive Abelian group of order $g$, then a \inlinedef{starter} in $G$ is a set of unordered pairs:
\begin{equation*}
S = \{\{s_i, t_i\}:1 \leq i \leq (g - 1)/2\}
\end{equation*}
which satisfies these properties:

\begin{enumerate}
  \item{$\{s_i:1 \leq i \leq (g-1)/2\} \cup \{t_i : 1 \leq i \leq (g-1)/2\} = G \backslash \{0\}$}
  \item{$\{\pm (s_i - t_i ) : 1 \leq i \leq (g-1)/2 \} = G \backslash \{0\}$}
\end{enumerate}

Whenever we have any $t$ sets $D_1, \ldots, D_t$ each of size $k$ in which each non-zero member of an additive Abelian group can be represented as a difference between members of the $D_i \lambda$ times, we say those sets form a \inlinedef{difference system}.

Much use will be made of difference systems throughout this work.
Notice that the definition of a starter presumes standardization, and therefore that $\{\infty, i\}$ is in position $(i, i)$.
The following pairs form a starter in $G = \{0, 1, 2, \ldots, 6\}$ (an additive Abelian group with order $g = 7$.)

\begin{equation}
\{1,3\} \{2,6\} \{4,5\}
\end{equation}

Property 1 is satisfied because
$\{1,3\} \cup \{2,6\} \cup \{4,5\} = G \backslash \{0\}$

Property 2 is also satisfied because
\begin{equation}
\begin{split}
\{1 - 3 = 5, 3 - 1 = 2, 2 - 6 = 3, 6 - 2 = 4, 4 - 5 = 6, 5 - 4 = 1\} &= \{1, 2, 3, 4, 5, 6\} \\
 &= G\backslash \{0\}
\end{split}
\end{equation}

Hence
\begin{equation}
  \begin{bmatrix}
    \infty 0 &  13 &  26 &  45 \\
    \infty 1 &  24 &  30 &  56 \\
    \infty 2 &  35 &  41 &  60 \\
    \infty 3 &  46 &  52 &  01 \\
    \infty 4 &  50 &  63 &  12 \\
    \infty 5 &  61 &  04 &  23 \\
    \infty 6 &  02 &  15 &  34
  \end{bmatrix}
  \label{eq:starter}
\end{equation}

are all the unordered pairs from $\{\infty, 0, 1, \ldots, 6\}$ sorted into seven rows that contain each of $\{\infty, 0, 1, \ldots, 6\}$ exactly once.
All that remains is to determine the columns.
