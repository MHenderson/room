The Room square \eqref{eq:cyclic} has a special structure.
The pairs in any element of the array are obtained by simply adding $1 \bmod{7}$ to the pair in the element immediately above and to the left (assuming that $\infty + 1 = \infty$).

\begin{equation}
  \begin{bmatrix}
    \infty 0 &     -    &     -    &     25     &     -    &     16     &    34    \\
      45     & \infty 1 &     -    &      -     &    36    &      -     &    20    \\
      31     &    56    & \infty 2 &      -     &     -    &     40     &     -    \\
       -     &    42    &    60    &  \infty 3  &     -    &      -     &    51    \\
      62     &     -    &    53    &     01     & \infty 4 &      -     &     -    \\
       -     &    03    &     -    &     64     &    12    &  \infty 5  &     -    \\
       -     &     -    &    14    &      -     &    05    &     23     & \infty 6 \\
  \end{bmatrix}
  \label{eq:cyclic}
\end{equation}

The special structure of \eqref{eq:cyclic} means that the entire square is determined by the pairs in the first row.
Successive rows are simply developed in a cyclical manner according the simple addition rule.
Squares like \eqref{eq:cyclic} are called \inlinedef{cyclic Room squares}.

Another property of \eqref{eq:cyclic} is that the pair $\{\infty,i\}$ occurs in position $(i,i)$ for all $1 \leq i \leq 7$.
A square with this property is said to be \inlinedef{standardised}.
Any Room square can be standardised.
This is because, as mentioned previously, neither interchanging the rows or columns nor permuting the symbol-set of a Room square has any effect of the conditions that make it a Room square.

Cyclic Room squares reduce the problem of constructing a Room square to a new problem of finding an appropriate first row.
Of course these first rows cannot be chosen arbitrarily, both the pairs used and the positions in which pairs appear must satisfy certain conditions.

