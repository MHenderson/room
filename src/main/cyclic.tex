\begin{equation}
  \begin{bmatrix}
    \infty 0 &     -    &     -    &     25     &     -    &     16     &    34    \\
      45     & \infty 1 &     -    &      -     &    36    &      -     &    20    \\
      31     &    56    & \infty 2 &      -     &     -    &     40     &     -    \\
       -     &    42    &    60    &  \infty 3  &     -    &      -     &    51    \\
      62     &     -    &    53    &     01     & \infty 4 &      -     &     -    \\
       -     &    03    &     -    &     64     &    12    &  \infty 5  &     -    \\
       -     &     -    &    14    &      -     &    05    &     23     & \infty 6 \\
  \end{bmatrix}
  \label{eq:cyclic}
\end{equation}

The Room square in \eqref{eq:cyclic} has a special property.
The pairs in any element of the array are obtained by simply adding 1 (mod 7) to the pair in the element immediately above and to the left; along with the condition that

\begin{equation}
  \infty + 1 = \infty
\end{equation}

This special property means that the entire square can be determined by the pairs in the first row, with successive rows being developed in a cyclical manner according the simple addition rule.
We call squares like \eqref{eq:cyclic} \inlinedef{cyclic} Room squares.

Also notice that $\{\infty,i\}$ occurs in position $(i,i)$.
A square with this property is said to be \inlinedef{standardised}.
It is important to realise that any Room square can be standardised.
As mentioned previously neither interchanging the rows or columns nor permuting the symbol-set on which the Room square is based has any effect of the \emph{Room}-ness of that square.

The significance of cyclic Room squares is that the problem of constructing a Room square is (potentially) reduced to that of finding an appropriate first row.
These rows cannot be chosen arbitrarily, both the pairs used and the positions in which they appear need to satisfy certain criteria, but when they do exist a corresponding Room square always exists.
So proving the existence of this subclass of Room squares is a matter only of proving the existence of these special first rows.

