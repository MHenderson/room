\chapter{Closing Remarks}
\label{ch:closing-remarks}

The results we have established in the previous chapter regarding the existence of balanced Room squares represent by no means the complete story.
Du and Hwang
\cite{duExistenceSymmetricSkew1988}
have established the existence of $SSBS$ for all prime powers
\begin{equation}
q = 2^{\alpha}t + 1, \alpha \geq 2, t \geq 3
\end{equation}
where $t$ odd.

Further, Anderson has shown that consequently the construction due originally to Hwang, Kang and Yu
\cite{hwangCompleteBalancedHowell1984}
but corrected in
\cite{andersonConstructionBalancedRoom1999}
allows us to state the existence of the corresponding $BRS(2q + 2)$ in one particular case.

By far the most significant remaining result which has not been included in the previous chapter is due to B. A. Anderson who proved that $BRS(2^n)$ exist for all odd $n \geq 3$.
His construction is based upon the theory of finite geometry, an area which has also contributed constructions for Room squares (the non-balanced kind).

Other similar geometrical constructions have been used to establish the existence of $BRS(2^n)$, for $4 \leq n \leq 18$, $n$ even.
The two smallest values of $n \equiv 0 \pmod 4$ for which the existence of a $BRS(n)$ remains in doubt are 36 and 92.
The first of these, along with many others, would be established by the doubling construction if a $SSBS$ could be found in $Z_{17}$.
This remains one of the most significant open problems for $BRS$, namely to establish the existence of $SSBS$ in $Z_n$ when $n$ is a Fermat prime.

The link between graph theory and Room squares that was touched upon in the second chapter has opened many avenues of research.
Possibly the most interesting of which is the existence of perfect Room squares.

A one-factorisation of $K_n$ is said to be \inlinedef{perfect} if the union of two of its one-factors is a Hamiltonian cycle of $K_n$.
A \inlinedef{perfect Room square} is one of side $n$ in which both row and column factorisations of $K_{n + 1}$ are perfect.
Very little seems to be known about perfect one-factorisations.
Individual examples of perfect Room squares of side 11 have been constructed but no infinite classes have yet been found.
