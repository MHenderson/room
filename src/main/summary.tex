So far we have shown that all Room squares whose side can be expressed as a prime power $p^n = 2^kt + 1$ can be constructed by using the Mullin-Nemeth starters.
The Fermat primes shown to be an exception to the Mullin-Nemeth construction, but this was overcome by introducing the theorem of Chong and Chan which provides a strong starter form all Room squares of side $(2^{2^m} + 1)$, encompassing the Fermat primes.
So we have proven that all Room squares exist whose side is a prime number, other than 3 or 5.
The multiplication theorem enables us to state that all Room squares exist whose side can be factored as $p_1p_2p_3...p_n$ with $p_i \geq 7$.

The non-existence of Room squares with sides 3 and 5, prevents us from constructing those squares whose sides have a factor of 3 or 5.
Within this class of exempt Room squares the Mullin-Nemeth starters will take care of the prime power sides.
But for those whose side is not a prime power a final theorem, due to W.D. Wallis is needed to complete the proof.
