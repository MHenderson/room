A \inlinedef{balanced incomplete block design} (BIBD) is an arrangement of elements (varieties) from a set $S$ of size $v$, into $b$ subsets (blocks) each of size $k$ such that each variety occurs in $r$ blocks and any particular pair of distinct varieties occurs in $\lambda$ blocks.

\begin{example}
\begin{equation}
\label{eq:bibd}
\{a,b,c\},\{a,b,d\},\{a,c,e\},\{a,d,f\},\{a,e,f\},\{b,c,f\},\{b,d,e\},\{b,e,f\},\{c,d,e\},\{c,d,f\}
\end{equation}
\eqref{eq:bibd} is a BIBD with parameters $v = 6$, $b = 10$, $k = 3$,
$r = 5$, $\lambda = 2$.
\end{example}

In any $\BIBD(v, b, r, k, \lambda)$ there are $b$ blocks each containing $k$ elements, so $bk$ elements in total.
Also, each of the $v$ elements occurs $r$ times, so there are $rv$ elements in total.
Therefore
\begin{equation}
bk = vr
\end{equation}

Also in a particular block, any element makes a pair with $k - 1$ other elements, and each element occurs in $r$ blocks, so there are $r(k - 1)$ pairs involving that element.
Also, by definition, that element is paired with each of the other $v - 1$ members of $S$ $\lambda$ times, so:
\begin{equation}
\lambda (v - 1) = r(k - 1)
\end{equation}

The second expression rearranges to give $r = \lambda (v - 1)/(k - 1)$, which can be substituted into the first expression to give $b = v\lambda (v - 1)/k(k - 1)$.
So $b$ amd $r$ are both determined by the values of $v$, $k$ and $\lambda$.
For this reason we need only quote those three parameters when referring to a particular block design.
The example was a $\BIBD(6, 3, 2)$ (or $(6, 3, 2)$-design).
