A graph $G(V,E)$ consists of two sets.
The first $V$, is called the vertex-set, while the other $E$ consists of unordered pairs of $V$ and is called the edge set.
Usually graphs are represented with diagrams where the members of $V$ are drawn as points and the members of $E$ as lines connecting points.
Adjacency for two vertices means being connected by an edge.
The \emph{complete graph} $K_n$ is the graph on $n$ vertices in which all distinct vertices are adjacent.

\begin{figure}
  \centering
  \begin{tikzpicture}[scale=0.5]

\GraphInit[vstyle=Simple]

\begin{scope}[xshift=0 cm]
\grComplete[prefix=p]{4}
\end{scope}
\begin{scope}[xshift=9 cm]
\grComplete[prefix=q]{5}
\end{scope}

\end{tikzpicture}

  \caption{$K_{4}$ and $K_{5}$}
  \label{fig:complete}
\end{figure}

A \emph{one-factor} $f_i$ is a set of edges in which each vertex appears exactly once.

\begin{example}
Two possible one-factors of $K_4$ are:
$$f_1 = \{12,34\},\, f_2 = \{13,24\}$$
\end{example}

\begin{figure}
  \centering
  \begin{tikzpicture}[scale=0.5]

\GraphInit[vstyle=Classic]
%\draw[help lines] (0,0) grid (6,6);

\begin{scope}[xshift=0 cm]
\Vertex[x=3,y=6]{A}
\Vertex[x=6,y=0]{B}
\Vertex[x=0,y=0]{C}
\Vertex[x=3,y=3]{D}
\Edge(A)(B)
\Edge(C)(D)
\end{scope}
\begin{scope}[xshift=9 cm]
\Vertex[x=3,y=6]{A}
\Vertex[x=6,y=0]{B}
\Vertex[x=0,y=0]{C}
\Vertex[x=3,y=3]{D}
\Edge(A)(C)
\Edge(B)(D)
\end{scope}

\end{tikzpicture}

  \caption{Two one-factors of $K_{4}$}
  \label{fig:two-one-factors}
\end{figure}

A \emph{one-factorisation} of the complete graph is a set of one-factors in which all possible edges (i.e. all unordered pairs from the edge-set) appear exactly once.

\begin{figure}
  \centering
  \begin{tikzpicture}[scale=0.5]

\SetVertexMath
\GraphInit[vstyle=Classic]
%\draw[help lines] (0,0) grid (30,20);

\begin{scope}[xshift=5cm, yshift=15cm]
\grComplete[prefix=q]{6}
\end{scope}

\begin{scope}[xshift=15cm, yshift=15cm]
\grEmptyCycle{6}
\Edge(a0)(a1)
\Edge(a2)(a4)
\Edge(a3)(a5)
\end{scope}

\begin{scope}[xshift=25cm, yshift=15cm]
\grEmptyCycle{6}
\Edge(a0)(a5)
\Edge(a1)(a4)
\Edge(a2)(a3)
\end{scope}

\begin{scope}[xshift=5cm, yshift=5cm]
\grEmptyCycle{6}
\Edge(a0)(a3)
\Edge(a1)(a2)
\Edge(a4)(a5)
\end{scope}

\begin{scope}[xshift=15cm, yshift=5cm]
\grEmptyCycle{6}
\Edge(a0)(a2)
\Edge(a1)(a5)
\Edge(a3)(a4)
\end{scope}

\begin{scope}[xshift=25cm, yshift=5cm]
\grEmptyCycle{6}
\Edge(a0)(a4)
\Edge(a1)(a3)
\Edge(a2)(a5)
\end{scope}

\end{tikzpicture}

  \caption{$K_{6}$ and a one-factorisation of $K_{6}$}
  \label{fig:k6-factorisation}
\end{figure}

\begin{example}
Here
$G = K_6$
the complete graph on six vertices with
$$V = \{1, 2, 3, 4, 5, 6\}$$
$$E = \{12, 13, 14, 15, 16, 23, 24, 25, 26, 34, 35, 36, 45, 46, 56\}$$
The one-factors are

$$
f_1 = \{12, 35, 46\} \hspace{0.5cm}
f_2 = \{14, 23, 56\} \hspace{0.5cm}
f_3 = \{16, 25, 34\} \hspace{0.5cm}
f_4 = \{13, 26, 45\} \hspace{0.5cm} 
f_5 = \{15, 24, 36\}
$$
because
$f_1 \cup f_2 \cup f_3 \cup f_4 \cup f_5 = E$,
$F = \{f_1, f_2, f_3, f_4, f_5\}$
is a one-factorisation of
$G$,
shown in Figure~\ref{fig:k6-factorisation}.
\end{example}

Two one factors $f$ and $l$ are said to be \emph{orthogonal} if $f \cap l$ contains at most one edge.
Two one-factorisations $F$ and $L$ are orthogonal if every one-factor in $F$ is orthogonal to every one-factor in $L$.

Once again consider the square array in \eqref{eq:roomsquare}.
If the individual elements within the array constituted the vertex set of a graph (call it $R$) and the pairs within each box of the array were edges, we know that each row is a one-factor and each column is a one-factor (because each member of $R$ occurs precisely once in each row and once in each column).

Furthermore, because all edges from the edge-set of the complete graph (i.e. all unordered pairs from $R$) appear once within the array, we know that the rows together form a one-factorisation and the columns form another, different, one-factorisation of $K_8$.
Also, because any row factor intersects any column factor in only one pair (edge), all the row factors are orthogonal to all the column factors and hence the two one-factorisations are orthogonal.
We have demonstrated the following theorem, given in
\cite{dinitzContemporaryDesignTheory1992}
and proven in
\cite{nemethStudyRoomSquares1969}.

\begin{theorem}
The existence of a Room square of side $n$ is equivalent to the existence of two orthogonal one-factorisations of the complete graph $K_{n+1}$.
\end{theorem}

An example is given in Figure \ref{fig:kirkmans-square} based on the Room square in \eqref{eq:roomsquare}

\begin{figure}
  \centering
  \begin{tikzpicture}[scale=0.5]

\SetVertexMath
\GraphInit[vstyle=Classic]
%\draw[help lines] (0,0) grid (30,50);

% top row

\begin{scope}[xshift=5cm, yshift=45cm]
\grEmptyCycle{8}
\Edge(a0)(a1)
\Edge(a2)(a3)
\Edge(a4)(a5)
\Edge(a6)(a7)
\end{scope}

\begin{scope}[xshift=15cm, yshift=45cm]
\grEmptyCycle{8}
\Edge(a0)(a6)
\Edge(a1)(a7)
\Edge(a2)(a4)
\Edge(a3)(a5)
\end{scope}

\begin{scope}[xshift=25cm, yshift=45cm]
\grEmptyCycle{8}
\Edge(a0)(a7)
\Edge(a1)(a6)
\Edge(a2)(a5)
\Edge(a3)(a4)
\end{scope}

% second row

\begin{scope}[xshift=5cm, yshift=35cm]
\grEmptyCycle{8}
\Edge(a0)(a4)
\Edge(a1)(a5)
\Edge(a2)(a6)
\Edge(a3)(a7)
\end{scope}

\begin{scope}[xshift=15cm, yshift=35cm]
\grEmptyCycle{8}
\Edge(a0)(a5)
\Edge(a1)(a4)
\Edge(a2)(a7)
\Edge(a3)(a6)
\end{scope}

\begin{scope}[xshift=25cm, yshift=35cm]
\grEmptyCycle{8}
\Edge(a0)(a2)
\Edge(a1)(a3)
\Edge(a4)(a6)
\Edge(a5)(a7)
\end{scope}

% third row

\begin{scope}[xshift=5cm, yshift=25cm]
\grEmptyCycle{8}
\Edge(a0)(a3)
\Edge(a1)(a2)
\Edge(a4)(a7)
\Edge(a5)(a6)
\end{scope}

\begin{scope}[xshift=15cm, yshift=25cm]
\grEmptyCycle{8}
\Edge(a0)(a5)
\Edge(a1)(a3)
\Edge(a2)(a6)
\Edge(a4)(a7)
\end{scope}

\begin{scope}[xshift=25cm, yshift=25cm]
\grEmptyCycle{8}
\Edge(a0)(a6)
\Edge(a1)(a2)
\Edge(a3)(a4)
\Edge(a5)(a7)
\end{scope}

% fourth row

\begin{scope}[xshift=5cm, yshift=15cm]
\grEmptyCycle{8}
\Edge(a0)(a4)
\Edge(a1)(a6)
\Edge(a2)(a7)
\Edge(a3)(a5)
\end{scope}

\begin{scope}[xshift=15cm, yshift=15cm]
\grEmptyCycle{8}
\Edge(a0)(a1)
\Edge(a2)(a5)
\Edge(a3)(a7)
\Edge(a4)(a6)
\end{scope}

\begin{scope}[xshift=25cm, yshift=15cm]
\grEmptyCycle{8}
\Edge(a0)(a3)
\Edge(a1)(a5)
\Edge(a2)(a4)
\Edge(a6)(a7)
\end{scope}

% bottom row

\begin{scope}[xshift=5cm, yshift=5cm]
\grEmptyCycle{8}
\Edge(a0)(a2)
\Edge(a1)(a7)
\Edge(a3)(a6)
\Edge(a4)(a5)
\end{scope}

\begin{scope}[xshift=15cm, yshift=5cm]
\grEmptyCycle{8}
\Edge(a0)(a7)
\Edge(a1)(a4)
\Edge(a2)(a3)
\Edge(a5)(a6)
\end{scope}

\begin{scope}[xshift=25cm, yshift=5cm]
\grComplete[prefix=a]{8}
\end{scope}

\end{tikzpicture}

  \caption{Two orthogonal one-factorisations of $K_{8}$ based on Kirkman's square of 1850.}
  \label{fig:kirkmans-square}
\end{figure}
