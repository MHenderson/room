The \inlinedef{starter-adder} method employed in the above example was introduced\footnote{Both Howell and Whitfield had previously found starters and adders, but the method described here due to Stanton and Mullin.}
by Stanton and Mullin in
\cite{stantonConstructionRoomSquares1968}.
They used it to construct Room squares of side 11.
They also went on to apply the method to larger squares, possibly providing the first evidence of infinite numbers of Room squares.

Two lemmas from
\cite{stantonConstructionRoomSquares1968}
establish the existence of starters for larger orders

\begin{lemma}
In an additive Abelian group $G$ of order $g = 2n - 1$ the pairs
\begin{equation*}
  \{\{n - 1, n\}, \{n - 2, n + 1\}, \{n - 3, n + 2\}, \{n - 4, n + 3\}, \ldots\{1, 2n - 2\}\}
\end{equation*}
are a starter for a Room square of side $2n - 1$.
\end{lemma}

\begin{example}
A Room square of side $2n - 1 = 19$.

\begin{equation*}
S_{19} = \{\{9, 10\}, \{8, 11\}, \{7, 12\}, \{6, 13\}, \{5, 14\}, \{4, 15\}, \{3, 16\}, \{2, 17\}, \{1, 18\}\}
\end{equation*}
is a starter.

Indeed, the differences are
\begin{equation*}
\begin{split}
  & \{\pm(10 - 9), \pm(11 - 8), \pm(12 - 7), \pm(13 - 6), \pm(14 - 5), \pm(3 - 16), \pm(2 - 17), \pm(18 - 1)\} \\ 
  &= \{1, 18, 3, 16, 5, 14, 7, 12, 9, 10, 8, 11, 6, 13, 4, 15, 2, 17\} \\
  &= G \backslash \{0\}
\end{split}
\end{equation*}
\end{example}

\begin{lemma}
In the Galois field of order $k - 1$, with primitive root $a$, the following pairs form a starter for a Room square of side $k$.
\begin{equation}
  \{\{a, a^n\}, \{a^2, a^{n + 1}\}, \{a^3, a^{n + 2}\}, \ldots, \{a^{n - 1}, a^{2n - 2}\}\}
\end{equation}
\end{lemma}

\begin{example}
A Room square of side $2n - 1 = 23$. $n = 12$. $a = 5$.
The set of pairs
\begin{equation*}
\begin{split}
S_{23} &= \{\{5, 5^{12}\}, \{5^2, 5^{13}\}, \{5^3, 5^{14}\}, \ldots, \{5^{11}, 5^{22}\}\} \\
       &= \{\{5, 18\}, \{2, 21\}, \{10, 13\}, \ldots, \{22, 1\}\}
\end{split}
\end{equation*}
is a starter.
\end{example}

On closer inspection the two types of starters are identical.
The starter in the first lemma has pairs whose elements always sum to $2n - 1$, while in the second lemma, because $a^{n - 1} = -1$, pairs can be written in the form $(a^{x}, -a^{x})$, with a general element having the form $\{j, -j\}$.

Starters of this form are called \inlinedef{patterned} starters.

As starters are certain to exist the only remaining difficulty lies in finding a corresponding adder.
Stanton and Mullin found adders corresponding to patterned starters for $k = 7, 11, 13, 15, 17$.
They had problems with 9 (but were able to construct one using a different method) and finding it too laborious to construct adders for $k > 19$ they developed an algorithm which, when implemented in Fortran, was able to find patterned starters with adders for all odd $k$ up to 49, with no further gaps.
Based on this evidence, they conjectured that Room squares exist for all odd side greater than 5.

They also found an interesting result regarding the number of Room Squares obtained from patterned starters, summarised in Table \ref{tab:patterned}. 

\begin{table}[h!]
  \begin{center}
    \begin{tabular}{c|c}
    Value of $k$ & Number of PRS \\ \hline
               7 &      2        \\
               9 &      0        \\
              11 &      4        \\
              13 &      8        \\
              15 &     44        \\
              17 &    416        \\
              19 &  > 976
    \end{tabular}
  \end{center}
  \caption{Number of patterned Room squares (PRS)}
  \label{tab:patterned}
\end{table}

Table \ref{tab:patterned} suggests that the number of patterned Room squares increases very rapidly.
Considering that patterned Room squares are a special case of cyclic Room squares which in turn are a special case of Room squares it would seem that the number of Room squares potentially grows very fast.

Before introducing a class of starters for which a corresponding adder is guaranteed we quickly confirm that when a starter and adder exist then a Room square will always result.
This seems obvious from the method outlined in the previous section but here we prove it explicitly.

\begin{example}
A square is constructed on the set $G \cup \{\infty\}$, where $G$ is an additive Abelian group of order $2n-1$.

\begin{equation*}
G = \{g_0 = 0, g_1, g_2, \ldots, g_{2n-2}\}
\end{equation*}

The columns and rows of the square are labelled as follows:

\begin{equation*}
  \begin{bmatrix}
        -      & g_0  &  g_1  &  g_2  & \ldots &  g_{2n - 1} \\
       g_0     &   -  &   -   &   -   &    -   &     -       \\
       g_1     &   -  &   -   &   -   &    -   &     -       \\
       g_2     &   -  &   -   &   -   &    -   &     -       \\
     \vdots    &   -  &   -   &   -   &    -   &     -       \\
    g_{2n - 1} &   -  &   -   &   -   &    -   &     -       \\
  \end{bmatrix}
\end{equation*}

If a starter
$\{\{s_1, t_1\}, \{s_2, t_2\} \ldots \{s_{n - 1}, t_{n - 1}\}\}$
and an adder
$\{a_1, a_2, \ldots, a_{n - 1}\}$
can be obtained from $G$ then by populating the square with pairs of elements from $G$ such that for all $g_i \in G$

\begin{enumerate}
  \item{$\{\infty, g_{i}\}$ goes in $(g_{i}, g_{i})$}
  \item{$\{s_{i} + g_{i}, t_{i} + g_{i}\}$ goes in $(g_{i}, g_{i} - a_{i})$}
\end{enumerate}
\label{eg:starter-adder}
\end{example}

\begin{theorem}
\label{thm:starter-adder}
If an Abelian group $G$ of odd order $2n - 1$ admits a starter and an adder, then there exists a Room square of order $2n$.
\end{theorem}

\begin{proof}

The square in Example \ref{eg:starter-adder} is a Room square on $G \cup \{\infty\}$.

Row $g_0 = 0$, contains the pairs $\{s_i, t_i\}: 1 \leq i \leq n - 1$, which are the elements of the starter, hence all of $G \backslash \{0\}$.
These pairs are accompanied by $\{\infty, 0\}$, so row 0 contains all of $G \cup \{\infty\}$.
Subsequent rows simple contain a permutation of the same elements, hence the row property of Room squares is satisfied for all rows.

As mentioned before, the starter forms a difference system in $G \backslash \{0\}$, so all unordered pairs of this set occur along with all unordered pairs of the form $\{\infty, g_i\}$ for all $1 \leq i \leq n - 1$.
Hence all unordered pairs from $G \cup \{\infty\}$ occur in the square exactly once.

All pairs of the form $\{s_i + a_i, t_i + a_i\}$ go in $(a_i, 0)$, i.e. column 0.
According to the definition of a starter these pairs are all of $G \backslash \{0\}$, and we know that $\{\infty,0\}$ is also in column 0.
So the first column, and hence all others, contains all of $G \cup \{\infty\}$, thus satisfying the column property of Room squares.
\end{proof}
