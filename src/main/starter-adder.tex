The \inlinedef{starter-adder} method employed in the above example was introduced\footnote{Both Howell and Whitfield had previously found starters and adders, but the precise method used here due to Stanton and Mullin.}
by Stanton and Mullin
\cite{stantonConstructionRoomSquares1968},
who used it to construct Room squares of side 11.
They also went on to apply the method to larger squares and gave the first real suggestions that the number of Room squares is infinite.

Two simple lemmas given in
\cite{stantonConstructionRoomSquares1968}
demonstrated that the problem of finding starters for larger Room squares was straightforward.
In fact they can be guaranteed always to exist, and the only difficulty comes from finding a corresponding adder, which is not guaranteed to exist.

\begin{lemma}
In an additive Abelian group $G$ of order $g = 2n-1$, then pairs
\begin{equation*}
  \{n - 1, n\}, \{n - 2, n + 1\}, \{n - 3, n + 2\}, \{n - 4, n + 3\}, \ldots\{1, 2n - 2\}
\end{equation*}
are a starter for a Room square of side $2n - 1$.
\end{lemma}

\begin{example}
A Room square of side $2n - 1 = 19$.

\begin{equation*}
S_{19} = \{\{9, 10\}, \{8, 11\}, \{7, 12\}, \{6, 13\}, \{5, 14\}, \{4, 15\}, \{3, 16\}, \{2, 17\}, \{1, 18\}\}
\end{equation*}
is a starter.

Indeed, the differences are
\begin{equation*}
\begin{split}
  & \{\pm(10 - 9), \pm(11 - 8), \pm(12 - 7), \pm(13 - 6), \pm(14 - 5), \pm(3 - 16), \pm(2 - 17), \pm(18 - 1)\} \\ 
  &= \{1, 18, 3, 16, 5, 14, 7, 12, 9, 10, 8, 11, 6, 13, 4, 15, 2, 17\} \\
  &= G \backslash \{0\}
\end{split}
\end{equation*}
\end{example}

\begin{lemma}
In the Galois field of order $k - 1$, with primitive root $a$ the following pairs form a starter for a Room square of side $k$.
\begin{equation}
  \{a, a^n\}, \{a^2, a^{n + 1}\}, \{a^3, a^{n + 2}\}, \ldots, \{a^{n - 1}, a^{2n - 2}\}
\end{equation}
\end{lemma}

\begin{example}
A Room square of side $2n - 1 = 23$. $n = 12$. $a = 5$.
The set of pairs
\begin{equation*}
\begin{split}
S_{23} &= \{\{5, 5^{12}\}, \{5^2, 5^{13}\}, \{5^3, 5^{14}\}, \ldots, \{5^{11}, 5^{22}\}\} \\
       &= \{\{5, 18\}, \{2, 21\}, \{10, 13\}, \ldots, \{22, 1\}\}
\end{split}
\end{equation*}
is a starter.
\end{example}

On closer inspection the two types of starters are identical\footnote{The starter in then first lemma has pairs whose elements always sum to $2n-1$, while the second lemma, because $a^{n-1}=-1$, has pairs which can be written in the form $(a^x,-a^x)$.}, with a general element being of the form $\{j, -j\}$.

Starters of this form are called \inlinedef{patterned} starters.

Stanton and Mullin went on to they could find adders corresponding to the patterned starters for $k = 7, 11, 13, 15, 17$.
They had problems with 9 (but were able to construct one using a different method) and finding it too laborious for $k > 19$ they developed an algorithm which, when implemented in Fortran, was able to find patterned starters with adders for all odd $k$ up to 49, with no further gaps.
Suggesting the possibility (which they conjectured) that there are Room squares for all odd side greater than 5.

They also found an interesting result regarding the number of Room Squares which could be obtained from patterned starters, summarised in Table \ref{tab:patterned}. 

\begin{table}[h!]
  \begin{center}
    \begin{tabular}{c|c}
    Value of $k$ & Number of PRS \\ \hline
    7 & 2 \\
    9 & 0 \\
    11 & 4 \\
    13 & 8 \\
    15 & 44 \\
    17 & 416 \\
    19 & The program was turned off after the production of 967 PRS
    \end{tabular}
  \end{center}
  \caption{Number of patterned Room squares (PRS)}
  \label{tab:patterned}
\end{table}

Stanton and Mullin's results suggest that the number of PRS (patterned Room squares) increases very rapidly.
Which, bearing in mind that the PRS are a sub-class of CRS (cyclic Room squares), which are in turn a sub-class of Room squares, implies that there are vast numbers of Room squares of large order.

Before introducing a class of starters for which the existence of a corresponding adder is guaranteed we quickly confirm that when a starter and adder exist then a Room square will always result.
This seems obvious from the method outlined in the previous section, but now we prove it explicitly.

\begin{theorem}
\label{thm:starter-adder}
If an Abelian group $G$ of odd order $2n - 1$ admits a starter and an adder, then there exists a Room square of order $2n$.
\end{theorem}

\begin{proof}
A square is constructed on the set $G \cup \{\infty\}$, where $G$ is an additive Abelian group of order $2n-1$.

\begin{equation}
G = \{g_0 = 0, g_1, g_2, \ldots, g_{2n-2}\}
\end{equation}

The columns and rows of the square are labelled as follows:

\begin{equation}
  \begin{bmatrix}
        -      & g_0  &  g_1  &  g_2  & \ldots &  g_{2n - 1} \\
       g_0     &   -  &   -   &   -   &    -   &     -       \\
       g_1     &   -  &   -   &   -   &    -   &     -       \\
       g_2     &   -  &   -   &   -   &    -   &     -       \\
     \vdots    &   -  &   -   &   -   &    -   &     -       \\
    g_{2n - 1} &   -  &   -   &   -   &    -   &     -       \\
  \end{bmatrix}
\end{equation}

If a starter
$\{\{s_1, t_1\}, \{s_2, t_2\} \ldots \{s_{n - 1}, t_{n - 1}\}\}$
and an adder
$\{a_1, a_2, \ldots, a_{n - 1}\}$
can be obtained from $G$ and if the square is populated by pairs of elements from $G$ according to the following rules:

\begin{enumerate}
  \item{$\{\infty, g_i\}$ goes in $(g_i, g_i)$}
  \item{While $\{s_i + g_i, t_i + g_i\}$ goes in $(g_i, g_i - a_i)$}
\end{enumerate}

for all $g_i \in G$.
The resulting square will be a Room square on $G \cup \{\infty\}$.

\begin{enumerate}
  \item{Row $g_0 = 0$, contains the pairs
      $\{s_i, t_i\}:1 \leq i \leq n-1$,
      which are the elements of the starter,
      hence all of $G \backslash \{0\}$. These pairs are
      accompanied by $\{\infty, 0\}$, so row 0 contains all
      of $G \cup \{\infty\}$. Subsequent rows simple
      contain a permutation of the same elements, hence
      the \emph{row property} of Room squares is satisfied for
      all rows.}
  \item{As mentioned before, the starter forms a difference
      system in $G \backslash \{0\}$, so all unordered
      pairs of this set occur along with all unordered
      pairs of the form
      $\{\infty, g_i\}: 1 \leq i \leq n - 1$,
      hence \emph{all unordered pairs from}
      $G \cup \{\infty\}$ \emph{occur} in the square exactly once.}
  \item{All pairs of the form $\{s_i + a_i, t_i + a_i\}$ go in
      $(a_i, 0)$, i.e. column 0. According to the
      definition of a starter these pairs are all of $G
      \backslash \{0\}$, and we know that $\{\infty,0\}$
      is also in column 0. So the first column, and hence
      all others, contains all of $G \cup \{\infty\}$,
      thus satisfying the \emph{column property} of Room
      squares.}
\end{enumerate}
\end{proof}
