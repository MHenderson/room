\begin{equation}
  \begin{bmatrix}
    \infty 0 &     -    &     -    &     25     &     -    &     16     &    34    \\
      45     & \infty 1 &     -    &      -     &    36    &      -     &    20    \\
      31     &    56    & \infty 2 &      -     &     -    &     40     &     -    \\
       -     &    42    &    60    &  \infty 3  &     -    &      -     &    51    \\
      62     &     -    &    53    &     01     & \infty 4 &      -     &     -    \\
       -     &    03    &     -    &     64     &    12    &  \infty 5  &     -    \\
       -     &     -    &    14    &      -     &    05    &     23     & \infty 6 \\
  \end{bmatrix}
  \label{eq:cyclic}
\end{equation}

The Room square in \eqref{eq:cyclic} has a special property.
The pairs in any element of the array are obtained by simply adding 1 (mod 7) to the pair in the element immediately above and to the left; along with the condition that

\begin{equation}
  \infty + 1 = \infty
\end{equation}

This special property means that the entire square can be determined by the pairs in the first row, with successive rows being developed in a cyclical manner according the simple addition rule.
We call squares like \eqref{eq:cyclic} \inlinedef{cyclic} Room squares.

Also notice that $\{\infty,i\}$ occurs in position $(i,i)$.
A square with this property is said to be \inlinedef{standardised}.
It is important to realise that any Room square can be standardised.
As mentioned previously neither interchanging the rows or columns nor permuting the symbol-set on which the Room square is based has any effect of the \emph{Room}-ness of that square.

The significance of cyclic Room squares is that the problem of constructing a Room square is (potentially) reduced to that of finding an appropriate first row.
These rows cannot be chosen arbitrarily, both the pairs used and the positions in which they appear need to satisfy certain criteria, but when they do exist a corresponding Room square always exists.
So proving the existence of this subclass of Room squares is a matter only of proving the existence of these special first rows.

\subsection{Finding a starter}

Suppose we wish to construct another Room square of the same size as \eqref{eq:cyclic} based on the same symbols.
This new square will also be standardized so we need only determine the three pairs that accompany $\{0, \infty\}$ in the first row (the starter), and the positions they occupy.

The set we will use to build our starter will be ${1, 2, \ldots, 6}$.

Each member of this set must occur exactly once in the pairs of the starter – in order to satisfy the row condition for a Room square.
Because of the cyclical construction the condition is automatically true for successive rows if true
for the first.

Consider the existence in \eqref{eq:cyclic} of an arbitrary pair $\{a, b\}$.
We know one of the following must be true.

Either $\{2 + i, 5 + i\} = \{a, b\}$ or $\{1 + i, 6 + i\} = \{a, b\}$ or $\{3 + i, 4 + i\} = \{a, b\}$ for $i = 0, 1, 2, \ldots, 6$.
Say $a - b = 1$.
Then $\{2 + i, 5 + i\} = \{a, b\}$ could never be true because $(2 + i) - (5 + i) = -3\pmod 7 = 4$ and $(5 + i) - (2 + i) = 3$.
Similarly, the differences in $\{1, 6\}$ are $\pm 5$ so $\{a, b\}$ couldn't be generated from $\{1, 6\}$.

However, $(4 + i) - (3 + i) = 1$ so $\{a, b\}$ will inevitably be generated by $\{3, 4\}$ for some value of $i = 0, 1, \ldots, 6$.
e.g. $\{2, 3\} = \{3 + 6, 4 + 6\}$

Because $a$ and $b$ separately take on all values from $\{0, 1, 2, \ldots, 6\}$, their differences will similarly take on all these values (except 0 because there are no pairs of the form $\{a, a\}$) and so an essential property for the starter must be that the six differences generated by its three pairs contain all of $\{1, 2, \ldots, 6\}$.

When a starter satisfies this property, and the condition that the pairs contain in their union all of $\{1, 2, \ldots, 6\}$, it is clear that it will inevitably generate the correct pairs which populate a $7 \times 7$ Room square. 
There are three pairs in the starter, each generates seven unique pairs under cyclical construction, which along with the seven pairs generated by $\{0, \infty\}$ counts for all the 28 unordered pairs from $\{\infty, 0, 1, \ldots, 6\}$.

A starter for larger Room squares of course has to obey the same criterion.
We include a general definition based on
\cite{dinitzContemporaryDesignTheory1992}.

If $G$ is an additive Abelian group of order $g$, then a \inlinedef{starter} in $G$ is a set of unordered pairs:
\begin{equation*}
S = \{\{s_i, t_i\}:1 \leq i \leq (g - 1)/2\}
\end{equation*}
which satisfies these properties:

\begin{enumerate}
  \item{$\{s_i:1 \leq i \leq (g-1)/2\} \cup \{t_i : 1 \leq i \leq (g-1)/2\} = G \backslash \{0\}$}
  \item{$\{\pm (s_i - t_i ) : 1 \leq i \leq (g-1)/2 \} = G \backslash \{0\}$}
\end{enumerate}

Whenever we have any $t$ sets $D_1, \ldots, D_t$ each of size $k$ in which each non-zero member of an additive Abelian group can be represented as a difference between members of the $D_i \lambda$ times, we say those sets form a \inlinedef{difference system}.

Much use will be made of difference systems throughout this work.
Notice that the definition of a starter presumes standardization, and therefore that $\{\infty, i\}$ is in position $(i, i)$.
The following pairs form a starter in $G = \{0, 1, 2, \ldots, 6\}$ (an additive Abelian group with order $g = 7$.)

\begin{equation}
\{1,3\} \{2,6\} \{4,5\}
\end{equation}

Property 1 is satisfied because
$\{1,3\} \cup \{2,6\} \cup \{4,5\} = G \backslash \{0\}$

Property 2 is also satisfied because
\begin{equation}
\begin{split}
\{1 - 3 = 5, 3 - 1 = 2, 2 - 6 = 3, 6 - 2 = 4, 4 - 5 = 6, 5 - 4 = 1\} &= \{1, 2, 3, 4, 5, 6\} \\
 &= G\backslash \{0\}
\end{split}
\end{equation}

Hence
\begin{equation}
  \begin{bmatrix}
    \infty 0 &  13 &  26 &  45 \\
    \infty 1 &  24 &  30 &  56 \\
    \infty 2 &  35 &  41 &  60 \\
    \infty 3 &  46 &  52 &  01 \\
    \infty 4 &  50 &  63 &  12 \\
    \infty 5 &  61 &  04 &  23 \\
    \infty 6 &  02 &  15 &  34
  \end{bmatrix}
  \label{eq:starter}
\end{equation}

are all the unordered pairs from $\{\infty, 0, 1, \ldots, 6\}$ sorted into seven rows that contain each of $\{\infty, 0, 1, \ldots, 6\}$ exactly once.
All that remains is to determine the columns.

\subsection{Finding an adder}

In constructing the starter we made use of the fact that each row has to contain each symbol exactly once and all unordered pairs from the symbol set have to occur exactly once in the whole array.
The remaining condition – namely, that each symbol must occur once in each column – is now employed to finish the construction.

Again, because of the cyclical nature of Room squares generated from starters we can be sure that if one column contains each member of the symbol set, all columns will.

Also, because we have decided to construct a standardized Room Square we know that column $i$ contains $\{\infty,i\}$.
So the final column (column 6) contains $\{\infty,6\}$, and depending on where we place the starter pairs it will also include:
\begin{equation*}
\{1,3\} + x \hspace{1cm} \{2,6\} + y \hspace{1cm}\{4,5\} + z
\end{equation*}
For some distinct values of $x,y$ and $z$ (only one pair allowed per box).
Considering that the new pairs to form column 6 must contain in their union each of
$\{0,1,2,...,5\}$
we build the following table.

\begin{equation}
  \begin{bmatrix}
    x &  13 + x & 26 + y & z & 45 + z \\
    0 &    13   &   26   & 0 &   45   \\
    1 &    24   &   30   & 1 &   56   \\
    2 &    35   &   41   & 2 &   60   \\
    3 &    46   &   52   & 3 &   01   \\
    4 &    50   &   63   & 4 &   12   \\
    5 &    61   &   04   & 5 &   23   \\
  \end{bmatrix}
  \label{eq:adder}
\end{equation}

Our task is simply to determine three unique values for $x, y$ and $z$ such that $13 + x, 26 + y$ and $45 + z$ contain in their union each of $\{0, 1, 2, \ldots, 5\}$.
These values will then determine the positions to place 13, 26 and 45 in row 1.

Choosing 4 from the first column corresponds to having 50 appear in the final column of the Room Square and forces the selection of $y = 2$ from the next column of the table, (41 being the only pair not containing any of the already used 5, 6 or 0).
23 is the only possible choice from the final column, accompanied by a value of $z = 5$.
These three numbers are known as an \inlinedef{adder} corresponding to the starter 13, 26, 45.
This is not necessarily the only adder.

If 50 is to be generated in the final column of the Room square by the pair 13 in the first row, then 13 must go in column $7 - 4 = 3$.
Similarly 26 has to be put in column $7 - 2 = 5$ and 45 in $7 - 5 = 2$.
We can now construct our cyclic room square.

\begin{equation}
  \begin{bmatrix}
    \infty 0 &  45 &  13 &   - &  26 &   - &   - \\
     - &  \infty 1 &  56 &  24 &   - &  30 &   - \\
     - &   - &  \infty 2 &  60 &  35 &   - &  41 \\
    52 &   - &   - &  \infty 3 &  01 &  46 &   - \\
     - &  63 &   - &   - &  \infty 4 &  12 &  50 \\
    61 &   - &  04 &   - &   - &  \infty 5 &  23 \\
    34 &  02 &   - &  15 &   - &   - &  \infty 6 
  \end{bmatrix}
  \label{eq:cyclic-room}
\end{equation}

In general, we define an adder by considering the elements which must accompany $\{\infty, 0\}$ in column 0.
Therefore an adder is defined in the following way:

An \inlinedef{adder} for a starter
$S = \{\{s_i, t_i\}: 1 \leq i \leq (g - 1)/2 \}$
is a set of $(g - 1)/2$ distinct non-zero elements
$a_1, a_2, ..., a_{(g - 1)/2}$ of $G$ such that:
$s_1 + a_1, t_1 + a_1, s_2 + a_2, \ldots, s_{(g - 1)/2} + a_{(g - 1)/2}, t_{(g - 1)/2} + a_{(g - 1)/2}$
are precisely all the non-zero elements of $G$.

The \inlinedef{starter-adder} method employed in the above example was introduced\footnote{Both Howell and Whitfield had previously found starters and adders, but the precise method used here due to Stanton and Mullin.}
by Stanton and Mullin
\cite{stantonConstructionRoomSquares1968},
who used it to construct Room squares of side 11.
They also went on to apply the method to larger squares and gave the first real suggestions that the number of Room squares is infinite.

Two simple lemmas given in
\cite{stantonConstructionRoomSquares1968}
demonstrated that the problem of finding starters for larger Room squares was straightforward.
In fact they can be guaranteed always to exist, and the only difficulty comes from finding a corresponding adder, which is not guaranteed to exist.

\begin{lemma}
In an additive Abelian group $G$ of order $g = 2n-1$, then pairs
\begin{equation*}
  \{n - 1, n\}, \{n - 2, n + 1\}, \{n - 3, n + 2\}, \{n - 4, n + 3\}, \ldots\{1, 2n - 2\}
\end{equation*}
are a starter for a Room square of side $2n - 1$.
\end{lemma}

\begin{example}
A Room square of side $2n - 1 = 19$.

\begin{equation*}
S_{19} = \{\{9, 10\}, \{8, 11\}, \{7, 12\}, \{6, 13\}, \{5, 14\}, \{4, 15\}, \{3, 16\}, \{2, 17\}, \{1, 18\}\}
\end{equation*}
is a starter.

Indeed, the differences are
\begin{equation*}
\begin{split}
  & \{\pm(10 - 9), \pm(11 - 8), \pm(12 - 7), \pm(13 - 6), \pm(14 - 5), \pm(3 - 16), \pm(2 - 17), \pm(18 - 1)\} \\ 
  &= \{1, 18, 3, 16, 5, 14, 7, 12, 9, 10, 8, 11, 6, 13, 4, 15, 2, 17\} \\
  &= G \backslash \{0\}
\end{split}
\end{equation*}
\end{example}

\begin{lemma}
In the Galois field of order $k - 1$, with primitive root $a$ the following pairs form a starter for a Room square of side $k$.
\begin{equation}
  \{a, a^n\}, \{a^2, a^{n + 1}\}, \{a^3, a^{n + 2}\}, \ldots, \{a^{n - 1}, a^{2n - 2}\}
\end{equation}
\end{lemma}

\begin{example}
A Room square of side $2n - 1 = 23$. $n = 12$. $a = 5$.
The set of pairs
\begin{equation*}
\begin{split}
S_{23} &= \{\{5, 5^{12}\}, \{5^2, 5^{13}\}, \{5^3, 5^{14}\}, \ldots, \{5^{11}, 5^{22}\}\} \\
       &= \{\{5, 18\}, \{2, 21\}, \{10, 13\}, \ldots, \{22, 1\}\}
\end{split}
\end{equation*}
is a starter.
\end{example}

On closer inspection the two types of starters are identical\footnote{The starter in then first lemma has pairs whose elements always sum to $2n-1$, while the second lemma, because $a^{n-1}=-1$, has pairs which can be written in the form $(a^x,-a^x)$.}, with a general element being of the form $\{j, -j\}$.

Starters of this form are called \inlinedef{patterned} starters.

Stanton and Mullin went on to they could find adders corresponding to the patterned starters for $k = 7, 11, 13, 15, 17$.
They had problems with 9 (but were able to construct one using a different method) and finding it too laborious for $k > 19$ they developed an algorithm which, when implemented in Fortran, was able to find patterned starters with adders for all odd $k$ up to 49, with no further gaps.
Suggesting the possibility (which they conjectured) that there are Room squares for all odd side greater than 5.

They also found an interesting result regarding the number of Room Squares which could be obtained from patterned starters, summarised in Table \ref{tab:patterned}. 

\begin{table}[h!]
  \begin{center}
    \begin{tabular}{c|c}
    Value of $k$ & Number of PRS \\ \hline
    7 & 2 \\
    9 & 0 \\
    11 & 4 \\
    13 & 8 \\
    15 & 44 \\
    17 & 416 \\
    19 & The program was turned off after the production of 967 PRS
    \end{tabular}
  \end{center}
  \caption{Number of patterned Room squares (PRS)}
  \label{tab:patterned}
\end{table}

Stanton and Mullin's results suggest that the number of PRS (patterned Room squares) increases very rapidly.
Which, bearing in mind that the PRS are a sub-class of CRS (cyclic Room squares), which are in turn a sub-class of Room squares, implies that there are vast numbers of Room squares of large order.

Before introducing a class of starters for which the existence of a corresponding adder is guaranteed we quickly confirm that when a starter and adder exist then a Room square will always result.
This seems obvious from the method outlined in the previous section, but now we prove it explicitly.

\begin{theorem}
\label{thm:starter-adder}
If an Abelian group $G$ of odd order $2n - 1$ admits a starter and an adder, then there exists a Room square of order $2n$.
\end{theorem}

\begin{proof}
A square is constructed on the set $G \cup \{\infty\}$, where $G$ is an additive Abelian group of order $2n-1$.

\begin{equation}
G = \{g_0 = 0, g_1, g_2, \ldots, g_{2n-2}\}
\end{equation}

The columns and rows of the square are labelled as follows:

\begin{equation}
  \begin{bmatrix}
        -      & g_0  &  g_1  &  g_2  & \ldots &  g_{2n - 1} \\
       g_0     &   -  &   -   &   -   &    -   &     -       \\
       g_1     &   -  &   -   &   -   &    -   &     -       \\
       g_2     &   -  &   -   &   -   &    -   &     -       \\
     \vdots    &   -  &   -   &   -   &    -   &     -       \\
    g_{2n - 1} &   -  &   -   &   -   &    -   &     -       \\
  \end{bmatrix}
\end{equation}

If a starter
$\{\{s_1, t_1\}, \{s_2, t_2\} \ldots \{s_{n - 1}, t_{n - 1}\}\}$
and an adder
$\{a_1, a_2, \ldots, a_{n - 1}\}$
can be obtained from $G$ and if the square is populated by pairs of elements from $G$ according to the following rules:

\begin{enumerate}
  \item{$\{\infty, g_i\}$ goes in $(g_i, g_i)$}
  \item{While $\{s_i + g_i, t_i + g_i\}$ goes in $(g_i, g_i - a_i)$}
\end{enumerate}

for all $g_i \in G$.
The resulting square will be a Room square on $G \cup \{\infty\}$.

\begin{enumerate}
  \item{Row $g_0 = 0$, contains the pairs
      $\{s_i, t_i\}:1 \leq i \leq n-1$,
      which are the elements of the starter,
      hence all of $G \backslash \{0\}$. These pairs are
      accompanied by $\{\infty, 0\}$, so row 0 contains all
      of $G \cup \{\infty\}$. Subsequent rows simple
      contain a permutation of the same elements, hence
      the \emph{row property} of Room squares is satisfied for
      all rows.}
  \item{As mentioned before, the starter forms a difference
      system in $G \backslash \{0\}$, so all unordered
      pairs of this set occur along with all unordered
      pairs of the form
      $\{\infty, g_i\}: 1 \leq i \leq n - 1$,
      hence \emph{all unordered pairs from}
      $G \cup \{\infty\}$ \emph{occur} in the square exactly once.}
  \item{All pairs of the form $\{s_i + a_i, t_i + a_i\}$ go in
      $(a_i, 0)$, i.e. column 0. According to the
      definition of a starter these pairs are all of $G
      \backslash \{0\}$, and we know that $\{\infty,0\}$
      is also in column 0. So the first column, and hence
      all others, contains all of $G \cup \{\infty\}$,
      thus satisfying the \emph{column property} of Room
      squares.}
\end{enumerate}
\end{proof}

