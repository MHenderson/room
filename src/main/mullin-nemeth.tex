If $x$ is a primitive element in $G = GF(p^n)$, then the elements $\{x^1, x^2, \ldots, x^{p^n - 1} = 1\}$ are, by definition, all of $G \backslash \{0\}$.
Also, $G \backslash \{0\} = \{x^0 = 1, x^1, \ldots, x^{p^n - 2}\}$.

\begin{example}[label=eg:mullin-nemeth]
$x = 5$ is a primitive root for the field $GF(23)$. Indeed $5^0 = 1$, $5^1 = 5$, $5^2 = 2$, $5^3 = 10$, $5^4 = 4$, $5^5 = 20$, $5^6 = 8$, $5^7 = 17$, $5^8 = 16$, $5^9 = 11$, $5^{10} = 9$ $5^{11} = 22$, $5^{12} = 18$, $5^{13} = 21$, $5^{14} = 13$, $5^{15} = 19$, $5^{16} = 3$, $5^{17} = 15$, $5^{18} = 6$, $5^{19} = 7$, $5^{20} = 12$, $5^{21} = 14$ are all non-zero elements of $GF(23)$.
\end{example}

Mullin and Nemeth in
\cite{mullinFurnishingRoomSquares1969}
use the theory of primitive elements to create strong starters in the additive group of (nearly) any Galois field of prime-power order.

Thanks to Theorems
\ref{thm:starter-adder}
and
\ref{thm:furnishing}
this proves the existence of Room squares for (nearly) all orders $p^n + 1$.

Before introducing the general construction for these starters in Theorem \ref{thm:strong-starter-3} we illustrate the basic method with examples of specific cases in Example \ref{ex:strong-starter}, Theorem \ref{thm:strong-starter} and Theorem \ref{thm:strong-starter-2}.

\begin{example}[label=ex:strong-starter]
A strong starter from Example \ref{eg:mullin-nemeth} is obtained by simply pairing the elements in the order in which they are generated.
\begin{equation}
S = \{\{1, 5\}, \{2, 10\}, \{4, 20\}, \{8, 17\}, \{16, 11\}, \{9, 22\}, \{18, 21\}, \{13, 19\}, \{3, 15\}, \{6, 7\}, \{12, 14\}\}
\end{equation}

Each member of $GF(23)$ occurs once.

Furthermore, the differences
\begin{equation}
  \{\pm 4, \pm 8, \pm 7, \pm 9, \pm 10, \pm 5, \pm 3, \pm 6, \pm 11, \pm 1, \pm 2\}
\end{equation}
are similarly all of $GF(23)$.
So $S$ is a starter.

Additionally, the sums
\begin{equation}
  \{6, 12, 1, 2, 4, 8, 16, 9, 18, 13, 3\}
\end{equation}
are all unique.
Therefore $S$ is strong and
\begin{equation}
  A = \{17, 11, 22, 21, 19, 15, 7, 14, 5, 10, 20\}
\end{equation}
is an adder for $S$.

So, the following row generates a Room square of order 24.

\begin{equation*}
  \begin{bsmallmatrix}
    \infty, 0 & 4,20 & 8,17 & 12,14 & 16,11 & - & 1,5 & - & 9,22 & 13,19 & - & - & 2,10 & 6,7 & - & - & 18,21 & - & 3,15 & - & - & - & - \\
  \end{bsmallmatrix}
\end{equation*}

\end{example}

Example \ref{ex:strong-starter} illustrates the simplest case of a general theorem of Mullin and Nemeth, where the Galois field is $\integersmodp$ (the integers mod $p$), with $p = 23 = 3\pmod 4$ a prime.

\begin{theorem}
\label{thm:strong-starter}
If $p = 4m + 3$ is prime, $m \geq 1$, then
\begin{equation}
S = \{\{x^0, x^1\}, \{x^2, x^3\}, \ldots, \{x^{4m}, x^{4m + 1}\}\}
\end{equation}
is a strong starter in $\integersmodp$, and hence a Room square of order $p + 1$ exists.
\end{theorem}

Example \ref{ex:strong-starter} comes from Theorem \ref{thm:strong-starter} with $m = 5$ and $x = 5$.

A slightly more general version of Theorem \ref{thm:strong-starter}, proved here instead, involves fields of prime power order where $p^{n} = 2t + 1$, with $t > 1$ and odd.
When $p^{n}$ is not prime (i.e $n > 1$), the field in question is no longer the integers modulo $p$.
Instead the primitive element is an irreducible polynomial whose coefficients belong to $\integersmodp$.

\begin{theorem}
\label{thm:strong-starter-2}
If $p^n = 2t + 1 = 3\pmod 4$ then
\begin{equation*}
S = \{\{x^0, x^1\}, \{x^2, x^3\}, \ldots, \{x^{2t - 2}, x^{2t - 1}\}\}
\end{equation*}
is a strong starter in $GF(p^n), (p^n \neq 3)$
\end{theorem}

\begin{proof}
As $x$ is a primitive element the elements in the starter are all non-zero members of $GF(P^n)$.
The differences are, respectively
\begin{equation*}
  D = \{\pm x^0(1 - x), \pm x^2 (1 - x), \ldots, \pm x^{2t - 2}(1 - x)\}
\end{equation*}

As $(1 - x)$ is a non-zero member of $GF(p^n)$, to show that these differences are all $2t$ non-zero members of $GF(p^n)$ we to show that the $2t$ differences are all distinct and non-zero.

First, observe that
\begin{equation*}
D = \{\pm x^{2i}(1 - x) : 0 \leq i \leq t - 1\}
\end{equation*}

Now, as $(1 - x) \neq 0$ it follows that $x^{2i}(1 - x) = x^{2j}(1 - x)$ and therefore that
$x^{2i} = x^{2j}$ which implies that $i = j$\footnote{because $0 \leq 2i, 2j \leq 2t - 2 < p^{n - 1}$, and the primitive element, by definition, produces each element of $GF(p^n)$ exactly once as the indices range from 0 to $p^{n - 1}$.}.
Similarly
$-x^{2i}(1 - x) = -x^{2j}(1 - x)$ can only be true when $i = j$.
So all positive differences are unique as are all negative differences.

A possibility for repetition remains when the signs are opposite.
For example, if
\begin{equation}
\label{eq:signs}
x^{2i}(1 - x) = -x^{2j}(1 - x)
\end{equation}

In this case, either $i = j$ or $i \neq j$.

Suppose that $i = j$.
Then \eqref{eq:signs} becomes $x^{2i} + x^{2i} = 0$ which implies that $2x^{2i} = 0$.
However, because $i$ takes values $0 \ldots t - 1$ it follows that $x^2 = 0$ when $i = 1$, contradicting the order of the primitive element.

Now suppose that $i \neq j$.
Without loss of generality, assume that $i < j$ and write $x^{2i} = -x^{2j}$.
Now, as $x^{2i}(1 + x^{2j - 2i}) = 0$ it follows that $x^{2j - 2i} = -1$.
In $GF(2t + 1)$, as $x^{\frac{1}{2}(q - 1)} = x^{t} = -1$ it follows that
\begin{equation*}
2j - 2i = t
\end{equation*}
This is a contradiction as $t$ is odd.

So $S$ is a starter.

To prove that $S$ is strong simply note that the sums are:
\begin{equation*}
  S = \{x^0(1 + x), x^2(1 + x), \ldots, x^{2t - 2}(1 + x)\}
\end{equation*}

Now, $1 + x = 0$ implies that $x = -1$ is only true when $p^{n} = 3$.
So $(1 + x) \neq 0$.
Therefore $x^{2i}(1 + x) = x^{2j}(1 + x)$ which implies that $x^{2i} = x^{2j}$.
We have already seen that $x^{2i} = x^{2j}$ is only true for $i = j$.

So all the sums are unique and the starter is strong.
\end{proof}

Theorems
\ref{thm:strong-starter}
and
\ref{thm:strong-starter-2},
show that Room squares exist for all $p^n\equiv 3\pmod 4$.
Furthermore, when $p^n\equiv 3\pmod 4$ is prime, these Room squares are based on $\integersmodp$.

The most generalised case of Mullin and Nemeth's theorem establishes the existence of Room squares for all prime powers $p^n = 2^{k}t + 1$ where $k > 1$ and $t > 1$ is odd.

\begin{theorem}
\label{thm:strong-starter-3}
A strong starter exists in $GF(p^n)$, where $p^n = 2^{k}t + 1$ (with $k > 1$ and $t > 1$ odd).
\end{theorem}

\begin{proof}
Let $d = 2^{k - 1}$.

Let
\begin{equation*}
S = \left\{
  \begin{array}{cccc}
    (x^0,x^d) & (x^{2d},x^{3d}) & \ldots & (x^{(2t - 2)d},x^{(2t - 1)d}) \\
    (x^1,x^{d + 1}) & (x^{2d + 1},x^{3d + 1}) & \ldots & (x^{(2t - 2){d + 1}},x^{(2t - 1){d + 1}}) \\
    \vdots & \vdots & \ldots & \vdots \\
    (x^{d - 1},x^{2d - 1}) & (x^{3d - 1},x^{4d - 1}) & \ldots & (x^{(2t - 2){d - 1}},x^{2td - 1})
  \end{array}
\right\}
\end{equation*}

$S$ is a strong starter.

The pairs have been placed in an array to emphasise that, when read vertically, this is an exhaustive list of all the non-zero elements of $GH(p^{n})$, ordered by exponents.
In the $k = 1, d = 1$ case this starter reduces to the one given in Theorem \ref{thm:strong-starter-2}.

To prove that $S$ is a starter we need to show that the differences between pairs are all of $GF(p^n)$.
To prove that the starter is strong we need to show that the sums of pairs are all distinct and non-zero.

The differences are:
\begin{equation*}
  D = \left\{
  \begin{array}{cccc}
    x^0(1 - x^d), & x^{2d}(1 - x^d), & \ldots, & x^{(2t - 2)d}(1 - x^d) \\
    x^1(1 - x^d), & x^{2d + 1}(1 - x^d), & \ldots, & x^{(2t - 2)d + 1}(1 - x^d) \\
    \vdots & \vdots & \ldots & \vdots \\
    x^{d - 1}(1 - x^d), & x^{3d - 1}(1 - x^d), & \ldots, & x^{(2t - 1)d - 1}(1 - x^d)
  \end{array}
  \right\}
\end{equation*}

The order of $x$ is $p^{n}-1 = 2^{k}t = 2^{k - 1}2t = 2td > d$, (meaning $x^{2td} = 1$ and $x^\alpha \neq 1$ when $1 \leq \alpha < 2dt$) and so $x^d \neq 1$, so $(1 - x^d) \neq 0$.

So the differences can all be written in the form:
\begin{equation*}
\pm x^{2id + j}(1 - x^d), 0 \leq i \leq t-1, 0 \leq j \leq d-1
\end{equation*}

If there is repetition, either of the form $y = y$ or $-y = -y$, where $y = x^{2id + j}(1 - x^d)$, then the following holds:
\begin{equation*}
\pm x^{2id + j}(1 - x^d) = \pm x^{2Id + J}(1 - x^d)
\end{equation*}

As $(1 - x^d) \neq 0$ we can cancelling by $(1 - x^d)$, giving:
\begin{equation}
x^{2id + j} = -x^{2ID + J}
\end{equation}

Dividing through by $x^{2Id + j}$ leaves:
\begin{align*}
  x^{2id - 2Id} &= x^{J = j} \\
  x^{2d(i - I)} &= x^{J - j}
\end{align*}

If $i \neq I$, the left hand side has an index which is an integer multiple of $d$.
The index on the right hand side, however, can never be an integer multiple of $d$ because $J$ and $j$ range over the integers $0 \ldots d - 1$.
So the only possibility for equality is when both indices are zero, i.e. $i = I$ and $j = J$.

As in an earlier proof we must deal with the possibility of repetition for differences of opposite sign.
A repeated difference requires:
\begin{align*}
  x^{2id + j} &= -x^{2ID + J} \\
  x^{2id + j} + x^{2Id + J} &= 0
\end{align*}

Assume that $2id + j < 2Id + J$ and rewrite this expression as:
\begin{equation*}
x^{2id + j}(1 + x^{(2I - 2i)d + (J - j)}) = 0
\end{equation*}

This implies that $x^{(2I - 2i)d + (J - j)} = -1$.

But $x^{\frac{1}{2}(q - 1)} = -1$ in $GF(q)$.
So in this case $q - 1 = 2^kt$ or $\frac{1}{2}(q - 1) = 2^{k - 1}t = dt$.
Therefore, $x^{dt} = -1$ and so $(2I - 2i)d + (J - j) = dt$.

So either $(J - j)$ is an integer multiple of $d$ or zero.
As $J$ and $j$ both take only the values in $\{0, \ldots, d - 1\}$, it follows that $(J - j)$ is in the interval $[1 - d, d - 1]$ and hence must be zero.

Thus,
\begin{eqnarray*}
  (2I - 2i)d &= dt \\
     2I - 2i &= t
\end{eqnarray*}

As $t$ is strictly odd, this is a contradiction.
Hence differences are all unique, belong to $GF(p^n)$ and there are $2td$ of them.
So each member of $GF(p^n)$ occurs exactly once as a difference.
Thus $S$ is a starter.

To prove that $S$ is strong, write sums as:
\begin{equation*}
  \left\{
  \begin{array}{cccc}
    x^0(1 + x^d), & x^{2d}(1 + x^d), & \ldots, & x^{(2t - 2)d}(1 + x^d) \\
    x^1(1 + x^d), & x^{2d + 1}(1 + x^d), & \ldots, & x^{(2t - 2)d + 1}(1 + x^d) \\
    \vdots & \vdots & \ldots & \vdots \\
    x^{d - 1}(1 + x^d), & x^{3d - 1}(1 + x^d), & \ldots, & x^{(2t - 1)d - 1}(1 + x^d)
  \end{array}
  \right\}
\end{equation*}

Observe that $x^d = -1$ implies that $d = dt$ ($x^{dt} = -1$ implies that $t = 1$, contradicting the assumption that $t > 1$).
So $(1 + x^d) \neq 0$ and the earlier argument involving $(1 - x^d)$ can be invoked, replacing $(1 - x^d)$ with $(1 + x^d)$.

So $S$ is a strong starter.
\end{proof}
