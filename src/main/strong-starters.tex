The next stage in proving the existence of Room squares came about, not by continuing to try to find adders for starters that were already known (the patterned starters, for example), but when Mullin and Nemeth in
\cite{mullinFurnishingRoomSquares1969},
discovered a class of starters that generated their own adders.

\begin{theorem}
\label{thm:furnishing}
Suppose a starter
$\{\{s_1, t_1\}, \{s_2, t_2\}, \ldots, \{s_{(g - 1)/2}, t_{(g - 1)/2}\}\}$
exists, such that the sums of each pair
$(s_1 + t_1, s_2 + t_2, etc...)$
are all distinct and non-zero, then that starter is said to be strong, and
\begin{equation}
 A(S) = \{a_i = -(s_i + t_i):1 \leq i \leq (g - 1)/2\}
\end{equation}
is an adder for a starter.
\end{theorem}

% https://tex.stackexchange.com/questions/59573/is-it-possible-to-use-the-enumerate-itemize-environment-within-the-proof-remark
\begin{proof}\leavevmode
\begin{enumerate}
  \item{The $a_i$ are all distinct and non-zero:
    All the $(s_i + t_i)$ are, by definition, distinct
    and non-zero. Therefore all the $a_i = -(s_i + t_i)$ are
    distinct and non-zero.}
  \item{$s_1 + a_1, t_1 + a_1, s_2 + a_2, \ldots, s_{(g - 1)/2}, t_{(g - 1)/2} + a_{(g - 1)/2}$
    are precisely all the non-zero elements of $G$.
    \begin{align*}
      s_1 + a_1 = s_1 - (s_1 + t_1) = -t_1 &= t_{(g - 1)/2}  \\
      t_1 + a_1 = t_1 - (s_1 + t_1) = -s_1 &= s_{(g - 1)/2}  \\
      s_{(g - 1)/2} + a_{(g - 1)/2} = -t_{(g - 1)/2} &= t_1  \\
      t_{(g - 1)/2} + a_{(g - 1)/2} = -s_{(g - 1)/2} &= s_1 
    \end{align*}
    Are all the non-zero elements of $G$ in reverse order.}
\end{enumerate}
(Notice that the patterned starter is not strong, on the contrary, the sums of its pairs are all identical.)
\end{proof}

\begin{example}
The pairs
$(5, 7)(11, 6)(2, 8)(9, 12)(10, 1)(3, 4)$,
constitute a strong starter for a Room square of side 13, based on $G = Z_{13}$.

Firstly, the pairs satisfy the conditions for being a starter, as the union of all pairs is equal to $G \backslash \{0\}$, and similarly the differences are all of $G\backslash \{0\}$.
Secondly the sums of the pairs, respectively $12, 4, 10, 8, 11, 7$ are all distinct and non-zero.
Therefore an adder is
$\{-12, -4, -10, -8, -11, -7\} = \{1, 9, 3, 5, 2, 6\}$
So the following is a legitimate first row for a cyclic Room square of order 14.

\begin{equation}
  \begin{bmatrix}
    \infty, 0 & - & - & - & 11,6 & - & - & 3,4 & 9,12 & - & 2,8 & 1,10 & 5,7 \\
  \end{bmatrix}
\end{equation}

\end{example}

Mullin and Nemeth originally discovered strong starters for Room squares embedded within another type of combinatorial design, known as a Steiner triple system.
With these they were able to prove that Room squares exist for all sides
$v = 1\pmod 6$.
Rather than examine this approach we move on to a type of starter which provides its own adder.
