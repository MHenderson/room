In
\cite{mullinFurnishingRoomSquares1969},
Mullin and Nemeth introduce a class of starters that generate their own adders.

A starter
$S = \{\{s_1, t_1\}, \{s_2, t_2\}, \ldots, \{s_{(g - 1)/2}, t_{(g - 1)/2}\}\}$
is \inlinedef{strong} if
the pairwise sums
$(s_1 + t_1, s_2 + t_2, \ldots)$
are all distinct and non-zero.

\begin{theorem}
\label{thm:furnishing}
Suppose
$S = \{\{s_1, t_1\}, \{s_2, t_2\}, \ldots, \{s_{(g - 1)/2}, t_{(g - 1)/2}\}\}$
is a strong starter.

Then
\begin{equation}
 A(S) = \{a_i = -(s_i + t_i):1 \leq i \leq (g - 1)/2\}
\end{equation}
is an adder for $S$.
\end{theorem}

\begin{proof}
All the $(s_i + t_i)$ are, by definition, distinct and non-zero.
Therefore all the $a_i = -(s_i + t_i)$ are distinct and non-zero.

The sequence $s_1 + a_1, t_1 + a_1, s_2 + a_2, \ldots, s_{(g - 1)/2}, t_{(g - 1)/2} + a_{(g - 1)/2}$ lists all of the non-zero elements of $G$ in reverse order. 

\begin{align*}
      s_1 + a_1 = s_1 - (s_1 + t_1) = -t_1 &= t_{(g - 1)/2}  \\
      t_1 + a_1 = t_1 - (s_1 + t_1) = -s_1 &= s_{(g - 1)/2}  \\
      s_{(g - 1)/2} + a_{(g - 1)/2} = -t_{(g - 1)/2} &= t_1  \\
      t_{(g - 1)/2} + a_{(g - 1)/2} = -s_{(g - 1)/2} &= s_1 
\end{align*}

\end{proof}

Notice that a patterned starter is not strong.
On the contrary, the sums of a patterned starter's pairs are all identical.

\begin{example}

$S = \{(5, 7), (11, 6), (2, 8), (9, 12), (10, 1), (3, 4)\}$
is a strong starter for a Room square of side 13.

Firstly, the pairs satisfy the conditions for being a starter, as the union of all pairs is equal to $G \backslash \{0\}$, and similarly the differences are all of $G\backslash \{0\}$.

Secondly the sums of the pairs, $\{12, 4, 10, 8, 11, 7\}$ are all distinct and non-zero.

Therefore 
$\{1, 9, 3, 5, 2, 6\} = \{-12, -4, -10, -8, -11, -7\}$
is an adder fo $S$.

So \eqref{eq:room14} is the first row of a cyclic Room square of order 14.

\begin{equation}
  \label{eq:room14}
  \begin{bmatrix}
    \infty, 0 & - & - & - & 11,6 & - & - & 3,4 & 9,12 & - & 2,8 & 1,10 & 5,7 \\
  \end{bmatrix}
\end{equation}

\end{example}

Mullin and Nemeth originally discovered strong starters for Room squares embedded within another type of combinatorial design, known as a Steiner triple system.
With this approach they proved that Room squares exist for all sides $v \equiv 1 \pmod 6$.

