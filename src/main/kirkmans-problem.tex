In 1850 Thomas Penyngton Kirkman, an English mathematician from Bolton, published the following problem in \emph{The Lady’s and Gentleman’s Diary}.

\begin{quotation}
Fifteen young ladies of a school walk out three abreast for seven days in succession: it is required to arrange them daily so that no two shall walk abreast more than once.
\end{quotation}

In solving this problem Kirkman discovered the following square array, which he observed was a ``very curious arrangement''.

\begin{equation}
  \begin{bmatrix}
       &    &    & hi & kl & mn & op \\
       & il & mo &    & np & hk &    \\
       & no & hl & mp &    &    & ik \\
    lp &    & in & ko & hm &    &    \\
    im &    & kp &    &    & lo & hn \\
    ho & km &    & ln &    & ip &    \\
    kn & hp &    &    & io &    & lm 
  \end{bmatrix}
  \label{eq:roomsquare}
\end{equation}

The curiosity of this square is that each of the letters
$h, i, k, l, m, n, o, p$
occurs precisely once in every column and row, while in the entire square every letter is paired with every other letter exactly once.

Kirkman used this square to solve his schoolgirl problem.
To each pair in the first column he added the element 1, to each pair in the second column 2 and so on.
Additionally, to every row, he introduced a triple of missing numbers.
For example, the first row has no elements in any of the first three columns so the numbers 1, 2 and 3 do not appear in any triples generated from this row.
So Kirkman added the triple $(1, 2, 3)$ to the other triples generated by the first row giving $\{(1, 2, 3), (h, i, 4), (k, l, 5), (m, n, 6), (o, p, 7)\}$.
He repeated the same process for every row.

\begin{table}[h!]
  \begin{center}
    \label{tab:kirkman-solution}
    \begin{tabular}{c|ccccc}
      Day 1 & $123$ & $hi4$ & $kl5$ & $mn6$ & $op7$ \\
      Day 2 & $147$ & $il2$ & $mo3$ & $np5$ & $hk6$ \\
      Day 3 & $156$ & $no2$ & $hl3$ & $mp4$ & $ik7$ \\
      Day 4 & $267$ & $lo2$ & $in3$ & $ko4$ & $hm5$ \\
      Day 5 & $245$ & $io2$ & $kp3$ & $lo6$ & $hn7$ \\
      Day 6 & $357$ & $ho2$ & $km2$ & $ln4$ & $ip6$ \\
      Day 7 & $346$ & $ko2$ & $hp2$ & $io5$ & $lm7$
    \end{tabular}
  \end{center}
  \caption{Kirkman's solution to the schoolgirl problem.}
\end{table}

The seven rows of unique triples then correspond to seven days in which the elements, corresponding to schoolgirls, are paired together exactly once.
Kirkman's solution is shown in Table \ref{tab:kirkman-solution}.

Kirkman is often regarded as the originator of the object in \eqref{eq:roomsquare}, which has subsequently come to be known as a \inlinedef{Room square} (after T.G. Room).
