The triplication example in Section \ref{sec:triplication} was somewhat contrived because the frame arrangement was chosen to satisfy a property yet to be explained.
The property in question is that frames $R_{ij}$ and $R_{ji}$ occur in the same super-column of the super-array $\mathcal{R}$.
This is so that permutations applied to both or neither of $R_{ij}$ and $R_{ji}$ preserve the contents of the columns as far as is required.

That a frame arrangement having this property exists for any odd integer is fundamental to Theorem \ref{thm:wallis}.

\begin{lemma}
For all odd $n$ there exists an array with these properties:
\begin{enumerate}
  \item{the entries of the array consist of all the ordered pairs of the set $N = \{1, 2, \ldots, n\}$ once each,}
  \item{the entries of a given row or column contain between them every member of $N$ once as a left member and once as a right member,}
  \item{if $(x,y)$ occurs in a given column of the array $(y,x)$ also occurs in that column.}
\end{enumerate}
\end{lemma}

\begin{proof}
Let $A_{n}$ be an $n \times n$ array whose $(i, j)$ entry is the ordered pair $(j - i + 1, i + j - 1)$ with both elements being reduced modulo $n$ to lie on the interval $[1, n]$.
\begin{enumerate}
  \item{
    There are clearly $n^{2}$ ordered pairs obtainable from $N$.
    $A_{n}$ has $n^{2}$ cells, so it is only necessary to show that each cell contains a unique pair.
    For that reason consider any two pairs from different cells, $(x_1, y_1)$ and $(x_2, y_2)$, for these to be equal requires both $x_1 = x_2$ and $y_1 = y_2$.

    Now, $x_1 = x_2$ implies $j_1-i_1 + 1 = j_2-i_2+1$.
    While, $y_1 = y_2$ implies $i_1 + j_1 - 1 = i_2 + j_2 - 1$.

     Together these imply,
     \begin{align}
       j_1 - i_1 &= j_2 - i_2 \label{eq:eq1} \\
       i_1 + j_1 &= i_2 + j_2 \label{eq:eq2}
     \end{align}

     \eqref{eq:eq2} gives, $j_2 = i_1 + j_1 - i_2$, which on substitution in \eqref{eq:eq1} gives, $j_1 - i_1 = i_1 + j_1 - 2i_2$ which implies $2i_1 - 2i_2 = 0$.
     
     Therefore $i_1 = i_2$.
     Substituting this into either expression gives $j_1 = j_2$.
     Thereby contradicting the assumption that the pairs occurred in different cells.
     Hence every cell contains a unique pair, so all the ordered pairs from $N$ occur exactly once in $A_n$.
  }
  \item{
    Consider row $i$ of $A_n$:

    \begin{tabular}{ccccc}
          $j = 1$   &     $j = 2$      & \ldots & $j = - 1$     &     $j = 0$      \\ \hline
       $(2 - i, i)$ & $(3 - i, i + 1)$ & \ldots & $(-i, i - 2)$ & $(1 - i, i - 1)$ 
    \end{tabular}

    Left hand members are $\{2-i, 3-i, \ldots, 1-i\}$, while the right hand members are $\{i, i + 1, \ldots, i - 1\}$.
    Both sets contain $n$ unique integers on the interval $[1,n]$ and hence both sets must be $N$.
    Similarly, consider column $j$, the left hand positions are occupied by $\{j, j - 1, \ldots, j + 2, j + 1\}$, while the right contain $\{j, j + 1, \ldots, j - 2, j - 1\}$.
    For the same reasons both these sets are equal to $N$.
  }
  \item{
    Consider a pair $(x,y)$.
    From the definition of $A_n$, $x = j-i + 1$, $y = i + j - 1$.
    So $x + y = j - i + 1 + i + j - 1 = 2j$.
    Therefore $j = \frac{1}{2}(x + y)$.
    So if $(x, y)$ is in column $j$ of $A_n$ then so is $(y, x)$, since $\frac{1}{2}(x + y) = \frac{1}{2}(y + x)$.
  }
\end{enumerate}
\end{proof}

\begin{example}
Looking back at the previous example of triplication.

\begin{equation}
 A = \begin{bmatrix}
  1,1 & 2,2 & 3,3 \\
  3,2 & 1,3 & 2,1 \\
  2,3 & 3,1 & 1,2 \\
 \end{bmatrix}
\end{equation}

Pairs $(3, 2)$ and $(2, 3)$ both appear in the first column, so $\phi _{13} = \phi _{12}$.
While pairs $(1, 3)$ and $(3, 1)$ both appear in the second column, so $\phi _{21} = \phi _{23}$.
Furthermore, both pairs $(2, 1)$ and $(1, 2)$ appear in the third column, so $\phi _{32} = \phi _{31}$.

The diagonal pairs are $(1, 1)$, $(1, 3)$, $(1, 2)$ so
\begin{equation*}
  \phi _{11} = \phi _{21} = \phi _{23} = \phi _{31} = \phi _{32} = id
\end{equation*}

The remaining permutations, $\phi _{12}, \phi _{13}, \phi _{22}, \phi _{33}$ are chosen according to the Lemma \ref{lem:permute} and the following array is the Room square in \ref{eq:roomtwentyone} after the missing pairs have been placed and the transformation to $\{\infty, 0, 1, ..., 20\}$ made.

\begin{equation*}
  \begin{bmatrix}
   R_{11}\phi_{11} & R_{22}\phi_{22} & R_{33}\phi_{33} \\
   R_{32}\phi_{13} & R_{13}\phi_{21} & R_{21}\phi_{32} \\
   R_{23}\phi_{12} & R_{31}\phi_{23} & R_{12}\phi_{31} \\
  \end{bmatrix}
  =
  \begin{bmatrix}
      R_{11}id    & R_{22}\phi_{2} & R_{33}\phi_{3} \\
   R_{32}\phi_{1} &   R_{13}id     &    R_{21}id    \\
   R_{23}\phi_{1} &   R_{31}id     &    R_{12}id    \\
  \end{bmatrix}
\end{equation*}

\end{example}

\begin{theorem}
\label{thm:wallis}
If $r$ and $n$ are odd integers such that $r \geq n$, and if there is a Room square $R$ of side $r$, then there is a Room square of side $rn$.
\end{theorem}

\begin{proof}
Let $r = 2d + 1$ and $n = 2t + 1$.

For a given $i$ select $n$ permutations as follows:
\begin{enumerate}
  \item{$\phi _{jk} = \phi _{jl}$ if, and only if, $(k, l)$, and $(l, k)$ appear in column $j$ of $A_n$.}
  \item{If cell $(j, j)$ of $A_n$ contains $(x, y)$ then $\phi _{jx} = \phi _{jy} = id$ (the identity permutation).}
  \item{All the $\phi _{jk} (\neq id)$ are selected from the permutations associated with $R$, according to Lemma \ref{lem:permute}.}
\end{enumerate}

Construct a Room square of side $rn$ by replacing every entry $(k, l)$ of $A_n$ by $R_{kl} \phi _{jk}$ (the array $R_{kl}$ under the column permutation $\phi _{jk}$), where $(k, l)$ is in column $j$ of $A_n$.

In the resulting array each element of
\begin{equation*}
  S = \{0_1, 1_1, \ldots, (r - 1)_1, 0_2, 1_2, \ldots, (r - 1)_2, 0_n, 1_n, \ldots, (r - 1)_n\}
\end{equation*}
appears exactly once in every row and column, except that $x_{j}$ is missing from row and column $x_{j}$, $1 \leq j \leq n 0 \leq x \leq (r - 1)$, and both $x_{k}$ and $x_{l}$ are missing from column $(x \phi _{jk})_j$ for every entry $(k, l)$ in column $j$ of $A_{n}$.

The array also contains every unordered pair from $S$ exactly once, except those of the form $\{x_{k}, x_{l}\}$.
Now, for each $k$, if $(k, l)$ is an entry of column $j$ of $A_{n}$ put $\{x_{k}, x_{l}\}$ in $(x_{j}, (x \phi _{jk})_j)$, using $\{\infty, x_{j}\}$ instead of $\{x_{j}, x_{j}\}$ in every case.

The completed array contains each of $\{\infty\} \cup S$ exactly once per row and column and every unordered pair from the same set exactly once.

Finally, map $\{\infty\} \cup S$ onto $\{\infty \} \cup \integersmod{rn}$ by replacing every $x_{i}$ with $x + r(i - 1)$.

The final array is a Room square of side $rn$.
\end{proof}

