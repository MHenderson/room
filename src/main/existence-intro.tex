The theorem which will ultimately be established in Chapter \ref{ch:existence-theorem} relies upon a fundamental theorem in number theory – in fact \emph{the} fundamental theorem.
The Fundamental Theorem of Arithmetic states that every positive integer, except 1, can be expressed uniquely as a product of primes.
Proof of this theorem can be found in
\cite{hardyIntroductionTheoryNumbers1979}.

The proof which established the existence of Room squares will rely upon various other theorems which collectively establish the existence of all Room squares with prime side, except 3 and 5.
Then multiplication theorems will be developed to establish the existence of composite Room squares (those whose side is the product of two or more primes).
Clearly if the prime Room squares can be proven to exist, and hence composite Room squares, the fundamental theorem will allow us to state that all Room squares exist with odd positive integer side.
Apart from a few exceptional cases, this is basically what we will be able to do.
