Having a theorem which enables new Room squares to be composed from old Room squares is of vital importance to the proof of the existence of Room squares.
With such a theorem, in conjunction with the Mullin-Nemeth starters, we will be able to construct Room squares of almost any order.
The exceptions will be due to the non-existence of orders 4 and 6.
The multiplication theorem that will be proven is:

This theorem was proposed initially in
\cite{bruckWhatLoop1963}
but later a counter-example to this method was found
\cite{mullinCounterexampleDirectProduct1969}
The proof here is based upon
\cite{andersonCombinatorialDesignsConstruction1990},
which in turn is based upon the proof in
\cite{stantonMultiplicationTheoremRoom1972a}.

\begin{theorem}
\label{thm:multiply}
	If Room squares of side m and side n exist then a Room square of side $mn$ also exists.
\end{theorem}

\begin{proof}
$M$ and $N$ are two Room squares.
$M$ is of side $m$ and based on $\{0, 1, 2, \ldots, m\}$, while $N$ is of side $n$ and based on $\{0, 1, 2, \ldots, n\}$.

The join of two Latin squares $A$ and $B$ is the array whose $(i, j)$ entry contains the ordered pair formed from the $(i, j)$ entry of $A$ taking the left position and the $(i, j)$ entry of $B$ taking the right.
If the join of two Latin squares contains $n^2$ unique ordered pairs, the two Latin squares $A$ and $B$ are said to be orthogonal.

$L_1$ and $L_2$ are two mutually Latin squares (MOLS) based on $\{1, 2, 3, \ldots, n\}$.

We construct the new Room square $R = MN$ by replacing each element of $M$ by an $n \times n$ array according to the following flow diagram where $(i, j)$ is a pair from $M$.

This procedure has replaced each pair in $m$ by an $n \times n$ array, resulting in an $mn \times mn$ array.
This array is based upon
$\{0, n + 1, n + 2, \ldots, n + mn\}$,
and we now prove that it has the properties of a Room square, namely:

\begin{enumerate}
  \item{Each element of the array is either empty or contains an unordered pair.}
  \item{Each row and column contains each of $\{0, n + 1, n + 2, \ldots, n + mn\}$ exactly once.}
  \item{Each pair from $\{0, n + 1, n + 2, \ldots, n + mn\}$ occurs exactly once in the array.}
\end{enumerate}

The first property is easily satisfied.
The procedure followed did nothing but replace empty elements and unordered pairs with arrays containing nothing more than empty elements or pairs.

The second is similarly straightforward.
Consider an arbitrary row of the new square $R$, call it $i$.
This row arose from applying prescriptions (i),(ii), and (iii) to some row of $M$.
This row of $M$ contained the elements $0 \ldots m$ exactly once.
One of these elements, call it $a$, was paired with 0.
So in $i$ from (ii) occur the numbers $(0;1 + an, \ldots, m + an)$ exactly once.

In the join of two MOLS the numbers $1, \ldots, n$ occur twice per row, once in $L_1$ once in $L_2$.
These are replaced by $(1 + un, \ldots, n + un)$ and $(1 + vn, \ldots, n + vn)$ as $u$ and $v$ take on all values $1, 2, \ldots, m$ excluding a.

Together these two prescriptions produce the elements $\{0;1 + n, 2 + n, \ldots, n + mn\}$ exactly once per row and column.

To prove condition 3 is true we show that $R$ contains the correct number of pairs and that these pairs are distinct.
Because we have shown 2 to be correct these pairs must be the right ones.

Any Room square, of side $n$, contains $\frac{1}{2}(n + 1)$ pairs per row, therefore $\frac{1}{2}n(n + 1)$ pairs over Room square of side $mn$ ought to contain $\frac{1}{2}mn(mn + 1)$ pairs.

In $M$ there were $m$ instances of $\{0, k\}$, each of these was replaced by a Room square of side $\frac{1}{2}n(n + 1) \cdot m$ pairs were contributed by (ii) to $R$.

In $M$ there were
$\frac{1}{2}(m + 1) - 1 = \frac{1}{2} (m  -1)$
pairs per row of the form $\{u, v\}$, therefore $\frac{1}{2}m(m - 1)$ of these pairs throughout $M$.
These were replaced by MOLS of side $n$, containing $n^2$ pairs each.
So $\frac{1}{2}m(m - 1) \cdot n^2$ pairs were contributed to $R$ from (iii). 

(i) contributed no pairs to $R$.

\begin{align*}
  \frac{1}{2} mn(n + 1) + \frac{1}{2}m(m - 1)n^2 &= \frac{1}{2}mn\{(n + 1) + n(m - 1)\} \\
  &= \frac{1}{2}mn(mn + 1)
\end{align*}

So the number of pairs in $R$ is correct.

To show that all the pairs are distinct consider $P(i,j)$ which represents those pairs generated from the element $(i,j)$ of $M$.
The pairs within $P(i,j)$ are always distinct because they are the pairs in a Room square or a join of two MOLS.

However we also need to show that $P(i,j)$ and $P(h,k)$ have no pairs in common when $i,j \neq h,k$. There are three cases to consider.
Both sets of pairs are chosen from the join of two MOLS, both sets are from Room squares, or one set from each.

If both sets of pairs were generated from the join of two MOLS then the pairs have the form $\{un + l_1, vn + l_2\}$.
If this pair occurs in both $P(i, j)$ and $P(h, k)$ then $(l_1, l_2)$ occurs in two different places in joins of two MOLS which is a contradiction.

The case where both sets of pairs are generated by Room squares is easily dealt with, because the Room squares used to construct $R$ are based on different sets, so two could never contain same pair.

In the case where one set of pairs belongs to a Room square and the other to a join of two MOLS, consider differences.
The greatest difference in pairs from a Room square of side $n$ based on $\{1, \ldots, n\}$ is $n - 1$ and (ii) maintains differences.
While the smallest difference in a pair from the join of two MOLS occurs when $l_1 = l_2$.
Then,
\begin{equation}
  (un + l_1) - (vn + l_2) = (u - v)n
\end{equation}
and so the smallest difference is $n$.
So these pairs can be equal.
\end{proof}

