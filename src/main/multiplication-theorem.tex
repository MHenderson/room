Theorem \ref{thm:multiply} allows new Room squares to be created from old Room squares, something which is very important the proof of the existence of Room squares.
Using Theorem \ref{thm:multiply} in conjunction with Mullin-Nemeth starters allows Room squares of nearly any order to be constructed.
The exceptions are due to the non-existence of Room squares of order 4 and 6.

The proof of Theorem \ref{thm:multiply} makes use of \inlinedef{mutually orthogonal latin squares} (MOLS), which can be defined in terms of the join of two latin squares.

The \inlinedef{join} of two latin squares $A$ and $B$ is the array whose $(i, j)$ entry contains the ordered pair formed from the $(i, j)$ entry of $A$ taking the left position and the $(i, j)$ entry of $B$ taking the right.

If the join of two latin squares $A$ and $B$ contains $n^{2}$ unique ordered pairs then $A$ and $B$ are said to be \inlinedef{orthogonal}.

\begin{theorem}
\label{thm:multiply}
If Room squares of side $m$ and side $n$ exist then a Room square of side $mn$ also exists.
\end{theorem}

\begin{proof}
Let $M$ and $N$ be two Room squares: $M$, of side $m$, is based on $\{0, 1, 2, \ldots, m\}$ while $N$, of side $n$, is based on $\{0, 1, 2, \ldots, n\}$.

Let $L_1$ and $L_2$ be mutually orthogonal latin squares based on $\{1, 2, 3, \ldots, n\}$.

Construct a new Room square $R = MN$ by replacing every entry $(i, j)$ of $M$ by an $n \times n$ array according to the flow diagram in Figure \ref{fig:flow}. 

\begin{figure}
  \centering
  \tikzstyle{process} = [rectangle, minimum width=3cm, minimum height=1cm, text centered, text width=3cm, draw=black, fill=orange!30]
\tikzstyle{decision} = [diamond, minimum width=3cm, minimum height=1cm, text centered, draw=black, fill=green!30]

\tikzstyle{arrow} = [thick,->,>=stealth]

\begin{tikzpicture}[node distance=2cm]
\node (empty) [decision] {Is $(i, j)$ empty?};
\node (contains) [decision, below right of = empty, xshift = 3cm, yshift = -2cm] {Does $(i, j)$ contain $\{0, k\}$?};
\node (out1) [process, below left of = empty, xshift = -2cm, yshift = -1cm] {Replace $(i, j)$ by an empty array of side $n$.};
\node (out2) [process, below left of = contains, xshift = -2cm, yshift = -2cm] {Replace $(i, j)$ with a Room square $N + kn$, obtained by adding $kn$ to every non-zero entry of $N$};
\node (out3) [process, below right of = contains, xshift = 3cm, yshift = -2cm] {Add $un$ to every entry of $L_{1}$ and $vn$ to every entry of $L_{2}$ and place the join of the resulting MOLS in $(i, j)$.};
\draw [arrow] (empty) -| node[anchor=east] {Yes} (out1);
\draw [arrow] (empty) -| node[anchor=west] {No} (contains);
\draw [arrow] (contains) -| node[anchor=east] {Yes} (out2);
\draw [arrow] (contains) -| node[anchor=west] {No} (out3);
\end{tikzpicture}

  \caption{A flow chart for the multiplication of Room squares.}
  \label{fig:flow}
\end{figure}

This procedure replaces every pair in $M$ by an $n \times n$ array, resulting in an $mn \times mn$ array, $R$.

$R$ is based on $\{0, n + 1, n + 2, \ldots, n + mn\}$.
Next we show that that $R$ is a Room square.

As the procedure did nothing but replace empty elements and unordered pairs with arrays containing nothing more than empty elements or pairs.

Next we show that every row and column contains each member of $\{0, n + 1, n + 2, \ldots, n + mn\}$ exactly once.

Consider an arbitrary row $r$ of the new square $R$.
$r$ is constructed by applying prescriptions (i), (ii), and (iii) to some row $m$ of $M$ containing the elements $\{0, \ldots, m\}$ exactly once.
One of these elements, call it $a$, is paired with 0 in $r$.
By (ii) the numbers $\{0, 1 + an, \ldots, m + an\}$ occur exactly once in $r$.

In the join of two MOLS the numbers $\{1, \ldots, n\}$ occur twice per row, once in $L_{1}$ once in $L_{2}$.
These are replaced by $\{1 + un, \ldots, n + un\}$ and $\{1 + vn, \ldots, n + vn\}$ as $u$ and $v$ take on all values $\{1, 2, \ldots, m\}$ excluding $a$.

Together these two prescriptions produce the elements $\{0, 1 + n, 2 + n, \ldots, n + mn\}$ exactly once per row and column.

Finally we show that every pair from $\{0, n + 1, n + 2, \ldots, n + mn\}$ occurs exactly once in $R$ by showing that $R$ contains the correct number of pairs and that these pairs are distinct.

Any Room square, of side $n$, contains $\frac{1}{2}(n + 1)$ pairs per row, therefore $\frac{1}{2}n(n + 1)$ pairs over Room square of side $mn$ ought to contain $\frac{1}{2}mn(mn + 1)$ pairs.

In $M$ there are $m$ instances of $\{0, k\}$, each of which is replaced by a Room square of side $\frac{1}{2}n(n + 1) \cdot m$ pairs were contributed by (ii) to $R$.

In $M$ there are
$\frac{1}{2}(m + 1) - 1 = \frac{1}{2} (m  -1)$
pairs per row of the form $\{u, v\}$, therefore $\frac{1}{2}m(m - 1)$ of these pairs throughout $M$.
These are replaced by MOLS of side $n$, each containing $n^{2}$ pairs.
So $\frac{1}{2}m(m - 1) \cdot n^{2}$ pairs are contributed to $R$ by (iii). 

(i) contributes no pairs to $R$.

The total number of pairs in $R$ is therefore:
\begin{align*}
  \frac{1}{2} mn(n + 1) + \frac{1}{2}m(m - 1)n^2 &= \frac{1}{2}mn\{(n + 1) + n(m - 1)\} \\
  &= \frac{1}{2}mn(mn + 1)
\end{align*}

To show that all the pairs are distinct let $P(i, j)$ denote those pairs generated from the element $(i, j)$ of $M$.

Pairs within $P(i,j)$ are always distinct because they are the pairs in a Room square or a join of two MOLS.

We also need to show that $P(i, j)$ and $P(h, k)$ have no pairs in common when $i, j \neq h, k$.

There are three cases to consider: both sets of pairs are chosen from the join of two MOLS, both sets are from Room squares, or one set from each.

If both sets of pairs are generated from the join of two MOLS then the pairs have the form $\{un + l_{1}, vn + l_{2}\}$.
If this pair occurs in both $P(i, j)$ and $P(h, k)$ then $(l_{1}, l_{2})$ occurs in two different places in joins of two MOLS which is a contradiction.

The case where both sets of pairs are generated by Room squares is easily dealt with, because the Room squares used to construct $R$ are based on different sets, so two could never contain the same pair.

In the case where one set of pairs belongs to a Room square and the other to a join of two MOLS, consider differences.

The greatest difference in pairs from a Room square of side $n$ based on $\{1, \ldots, n\}$ is $n - 1$ and (ii) maintains differences.
While the smallest difference in a pair from the join of two MOLS occurs when $l_{1} = l_{2}$.

Now,
\begin{equation*}
  (un + l_1) - (vn + l_2) = (u - v)n
\end{equation*}
so the smallest difference is $n$.

So these pairs cannot be equal.
\end{proof}

Theorem \ref{thm:multiply} was originally proposed in
\cite{bruckWhatLoop1963}
but later a counter-example showing the flaw in this approach was published in
\cite{mullinCounterexampleDirectProduct1969}.

The above proof is based on
\cite{andersonCombinatorialDesignsConstruction1990},
which is based on the proof in
\cite{stantonMultiplicationTheoremRoom1972a}.

