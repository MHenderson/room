A graph $G(V,E)$ consists of two sets.
The first $V$, is called the vertex-set, while the other $E$ consists of unordered pairs of $V$ and is called the edge set.
Usually graphs are represented with diagrams where the members of $V$ are drawn as points and the members of $E$ as lines connecting points.
Adjacency for two vertices means being connected by an edge.
The \emph{complete graph} $K_n$ is the graph on $n$ vertices in which all distinct vertices are adjacent.

\begin{figure}
  \centering
  \begin{tikzpicture}[scale=0.5]

\GraphInit[vstyle=Simple]

\begin{scope}[xshift=0 cm]
\grComplete[prefix=p]{4}
\end{scope}
\begin{scope}[xshift=9 cm]
\grComplete[prefix=q]{5}
\end{scope}

\end{tikzpicture}

  \caption{$K_{4}$ and $K_{5}$}
  \label{fig:complete}
\end{figure}
