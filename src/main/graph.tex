\begin{figure}[h]
  \centering
  \begin{tikzpicture}[scale=0.5]

\GraphInit[vstyle=Simple]

\begin{scope}[xshift=0 cm]
\grComplete[prefix=p]{4}
\end{scope}
\begin{scope}[xshift=9 cm]
\grComplete[prefix=q]{5}
\end{scope}

\end{tikzpicture}

  \caption{$K_{4}$ and $K_{5}$}
  \label{fig:complete}
\end{figure}

A \inlinedef{graph} $G = G(V, E)$ consists of two sets.
The first $V$, is called the \inlinedef{vertex-set}, while the other $E$ consists of unordered pairs of $V$ and is called the \inlinedef{edge set}.
Graphs are often represented by diagrams where the members of $V$ are drawn as points and the members of $E$ as lines connecting points.
Examples of such diagrams are shown in Figure \ref{fig:complete}.
Vertices $u, v$ are said to be \inlinedef{adjacent} if they are connected by an edge $e = uv$.
The \inlinedef{complete graph} $K_n$ is the graph on $n$ vertices in which all distinct vertices are adjacent.
Figure \ref{fig:complete} shows the complete graphs $K_{4}$ and $K_{5}$
