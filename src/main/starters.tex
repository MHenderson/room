Suppose we wish to construct another Room square of the same size as \eqref{eq:cyclic} based on the same symbols.
This new square will also be standardised so we need only determine the three pairs that accompany $\{0, \infty\}$ in the first row (the starter), and the positions they occupy.

In order to satisfy the row condition for Room squares each member of $\{1, 2, \ldots, 6\}$ must appear exactly once among the pairs of the starter.
The cyclical construction and the nature of finite fields ensures that if this condition holds for the first row then it also holds for successive rows.

So consider the existence in \eqref{eq:cyclic} of an arbitrary pair $\{a, b\}$ somewhere in the finished Room square, not on the main diagonal.
One of the following statements must be true.

Either
\begin{itemize}
  \item $\{2 + i, 5 + i\} = \{a, b\}$ or,
  \item $\{1 + i, 6 + i\} = \{a, b\}$ or,
  \item $\{3 + i, 4 + i\} = \{a, b\}$
\end{itemize}
for some value $i \in \{0, 1, 2, \ldots, 6\}$.

Suppose that $a - b = 1$.
Then $\{2 + i, 5 + i\} \neq \{a, b\}$ because $(2 + i) - (5 + i) = -3\pmod 7 = 4$ and $(5 + i) - (2 + i) = 3$.
Similarly, the differences in $\{1, 6\}$ are $\pm 5$ so $\{a, b\}$ cannot be generated from $\{1, 6\}$.

However, $(4 + i) - (3 + i) = 1$.
So $\{a, b\}$ is generated by $\{3, 4\}$ for some value of $i \in \{0, 1, \ldots, 6\}$.
In fact, $\{2, 3\} = \{3 + 6, 4 + 6\}$.

Because $a$ and $b$ separately take on all values from $\{0, 1, 2, \ldots, 6\}$ their differences similarly take on all these values (except 0 as there are no pairs of the form $\{a, a\}$).
So an essential property for the starter must be that the six differences generated by its three pairs contain all of $\{1, 2, \ldots, 6\}$.

If a starter satisfies this property, and all the pairs contain in their union all of $\{1, 2, \ldots, 6\}$ it will generate all of the correct pairs to populate a $7 \times 7$ Room square. 
There are three pairs in the starter, each of which generates seven unique pairs under cyclical construction, giving 21 pairs.
Adding the seven pairs generated by $\{0, \infty\}$ accounts for all 28 unordered pairs from $\{\infty, 0, 1, \ldots, 6\}$.

Starters for larger Room squares of course must obey the same criteria.
The definition below comes from
\cite{dinitzContemporaryDesignTheory1992}.

If $G$ is an additive Abelian group of order $g$, then a \inlinedef{starter} in $G$ is a set of unordered pairs:
\begin{equation*}
S = \{\{s_i, t_i\}:1 \leq i \leq (g - 1)/2\}
\end{equation*}
which satisfies these properties:

\begin{enumerate}
  \item{$\{s_i:1 \leq i \leq (g-1)/2\} \cup \{t_i : 1 \leq i \leq (g-1)/2\} = G \backslash \{0\}$}
  \item{$\{\pm (s_i - t_i ) : 1 \leq i \leq (g-1)/2 \} = G \backslash \{0\}$}
\end{enumerate}

Notice that the definition of a starter presumes standardisation.

Whenever we have any $t$ sets $D_1, \ldots, D_t$ each of size $k$ in which each non-zero member of an additive Abelian group can be represented as a difference between members of the $D_i \lambda$ times, we say those sets form a \inlinedef{difference system}.
Much use of difference systems is made in this monograph.

\begin{example}
\label{eg:starter}
The following pairs form a starter in $G = \{0, 1, 2, \ldots, 6\}$ (an additive Abelian group with order $g = 7$.)

\begin{equation*}
  \{\{1,3\}, \{2,6\}, \{4,5\}\}
\end{equation*}

Property 1 is satisfied because
$\{1,3\} \cup \{2,6\} \cup \{4,5\} = G \backslash \{0\}$.

Property 2 is also satisfied because
\begin{equation*}
\begin{split}
\{1 - 3 = 5, 3 - 1 = 2, 2 - 6 = 3, 6 - 2 = 4, 4 - 5 = 6, 5 - 4 = 1\} &= \{1, 2, 3, 4, 5, 6\} \\
 &= G\backslash \{0\}
\end{split}
\end{equation*}

Hence
\begin{equation*}
  \begin{bmatrix}
    \infty 0 &  13 &  26 &  45 \\
    \infty 1 &  24 &  30 &  56 \\
    \infty 2 &  35 &  41 &  60 \\
    \infty 3 &  46 &  52 &  01 \\
    \infty 4 &  50 &  63 &  12 \\
    \infty 5 &  61 &  04 &  23 \\
    \infty 6 &  02 &  15 &  34
  \end{bmatrix}
  \label{eq:starter}
\end{equation*}

are all the unordered pairs made from elements of $\{\infty, 0, 1, \ldots, 6\}$ sorted into seven rows each containing all elements of $\{\infty, 0, 1, \ldots, 6\}$ exactly once.
\end{example}
