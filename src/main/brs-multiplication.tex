Although the Hwang starter-adder construction establishes the existence of an infinite class of $\BRS$, like the Mullin-Nemeth construction for Room squares, exceptions still remain.

Particularly all those $\BRS$ of order $q + 1$ where $q$ is not a prime power, yet is congruent 3 modulo 4.
For example, the existence of $\BRS(16)$ is not established yet, because 15 is not a prime power.

One approach to resolving this problem would be to establish a multiplication theorem like those established for Room squares.
The following doubling construction for $\BRS$ (due to Schellenberg), for example, enables to construction of a $\BRS(16)$ from two $\BRS(8)$.

Significantly, used along with Hwang’s starter-adder construction, it establishes the existence of another infinite class of $\BRS$, those of order $2(q + 1)$, where $q$ is subject to the conditions of Theorem \ref{thm:hwang}.

Before presenting the doubling construction a few definitions need to be made, and another result regarding self-complementary block designs established.

The \inlinedef{reduced} Room square $\hat{R}$, is obtained by taking a standardised Room square $R$ and removing the pairs involving $\infty$, i.e. the diagonal pairs.

Two $ORS$, $R$ and $S$ (both of side $a$) are said to be a \inlinedef{Latin pair} if on forming the join of $\hat{R}$ and $\hat{S}$, and placing the pair $(i, i)$ in cell $(i, i)$ for $0 \leq i \leq q - 1$ the join of two $MOLS$ is obtained, denoted by $R \odot S$.

A \inlinedef{common traversal} of $R \odot S$ is a set of $q$ cells, one from each row and each column, whose $n$ ordered pairs have $n$ distinct first elements and $n$ distinct second elements.

Suppose we have a set of varieties $V = \{1, 2, 3, \ldots\}$, then by a set $V'$, we mean $V'=\{1', 2', 3', \ldots\}$.

\begin{lemma}
If
\begin{equation}
  \bigcup\limits_{i=1}^{2k-1} \left \{C_i,\bar{C_i} \right \} 
  \qquad \bigcup\limits_{i=1}^{2k-1} \left \{D_i,\bar{D_i} \right \}
\end{equation}
are the blocks of two self-complementary $\BIBD$s, both defined on a set of elements $V$, where $\bar{C}$ denotes the complement of $C$, with parameters $(2k, 2(2k - 1), 2k - 1, k, k-1)$ then.

\begin{equation}
\bigcup\limits_{i=1}^{2k-1} \left \{C_i \cup D'_i, \bar{C_i} \cup \bar{D'_i}, C_i \cup \bar{D'_i}, \bar{C_i} \cup D'_i  \right \} \cup \{V,V'\}
\end{equation}
is the set of blocks of a self-complementary $\BIBD$, defined on the set of elements $V \cup V'$, with parameters $(4k, 2(4k - 1), 4k - 1, 2k, 2k - 1)$.
\end{lemma}

\begin{proof}
Consider an arbitrary pair $\{a, b\}$ in the new $\BIBD$ where $a, b \in V$.
The concurrence number of the original $\BIBD$ was $k - 1$, so this pair occurred $k - 1$ times in the blocks
\begin{equation}
\bigcup\limits_{i = 1}^{2k - 1} \left\{C_i, \bar{C_i} \right\}
\end{equation}
but these blocks clearly appear twice each in the new $\BIBD$.
Also the pair $\{a, b\}$ occurs once in the block $V$, so the pair $\{a, b\}$ occurs $2(k - 1) + 1 = 2k - 1$ times.
The same would be true for any pair $\{a, b\}$ when $a, b \in V'$.

Now consider an arbitrary pair $\{a, b\}$ when $a \in V$, and $b \in V'$.
From the definition of complementary blocks $a$ occurs either $C$ or $\bar{C}$ and $b$ in either $D'$ or $\bar{D'}$, for some value of $i$, and because all pair combinations between $C$-blocks and $D'$-blocks are formed in the new design and because $i$ takes $2k - 1$ values in the new design so the pair makes $2k - 1$ appearances.

Hence the new block design is a $\BIBD$ with $\lambda = 2k - 1$.
\end{proof}

\begin{theorem}
\label{thm:schellenberg}
Suppose we have two $\BRS(q+1)$, $R$ and $S$ based on $G = GF(q)$, with the following properties:
\begin{enumerate}
  \item{$R$ and $S$ are a \emph{latin pair} such that}
  \item{$R \odot S$ has a pair of disjoint \emph{common transversals}
    $T_1$ and $T_2$ (with $T_2$ in $R$), which do not
    intersect the main diagonal, and}
  \item{the block designs obtained from $R$ and $S$, call them
    $D(R)$ and $D(S)$ respectively, have the property that
    if $\{\infty\} \cup B_i$ and $\{i\} \cup C_i$ are the
    blocks of $D(R)$ obtained from row $i$, then $\{i\} \cup
    B_i$ and $\{\infty\} \cup C_i$ are the blocks of $D(S)$
    obtained from row $i$.}
\end{enumerate}

Then a $\BRS(2(q + 1))$ exists.
\end{theorem}

\begin{proof}
Define $A$ to be the array obtained from the superposition of $\hat{R}$ and $\hat{S'}$.
Where $\hat{S'}$ is obtained by replacing the elements of $G$ in $\hat{S}$ (the reduced RS) by the corresponding elements of $G'$ and replacing $\infty$ by $\infty'$.
$R$ and $S$ were $\BRS$ so each column and row of $A$ contains each member of $G \cup G' \cup \{\infty, \infty'\}$, except those elements $i,i',\infty$ and $\infty'$ which are missing from row and column $i$ due to the reduction process.
Further, $A$ contains every pair $\{i,j\},\{i',j'\} i,j \in G,i \neq j$ exactly once.
Define $B$ as the array obtained from $R \odot S$ when each pair $(i,j)$ is replaced by $(i, j')$.

Now, $D$ is defined as the array obtained from $B$ according to this procedure.

If cell $(m, n), m\neq n$ of $R$ is not empty, replace $(i, j')$ in cell $(m, n)$ of $B$ by $(j', i)$.
Because $R \odot S$ is a pair of superposed orthogonal latin squares $D$ contains each member of $G \cup G'$ exactly once in each row and column.
Also, the way in which we have repalced elements ensures that every unordered pair $\{i, j'\}$ for all $i, j \in G$ occurs exactly once in $D$.

Next, construct $C$ by arranging the arrays $A$ and $D$ according to the following layout, in which $\phi$ is the $q \times q$ array with every cell empty, $\theta$ the $1 \times q$ array of empty cells and $\theta ^T$ the transpose of $\theta$
\begin{equation}
  C =
  \begin{bmatrix}
    A & \phi & \theta^T \\
    \phi & D & \theta^T \\
    \theta & \theta & (\infty',\infty)
  \end{bmatrix}
\end{equation}

The rows and columns of this new array are labelled $0, 1, 2, \ldots, q - 1, 0', 1', 2', \ldots, (q - 1)',l$.
So that the pair $(\infty ', \infty)$ is in cell $(l, l)$.

Let
\begin{equation}
 T'_p = \{ (i'_m,j'_m) | (i_m,j_m) \text{ is a cell transversal } T_p \}
\end{equation}
be the set of cells of $C$, corresponding to transversal
$T_p (p=1,2)$ of $R \odot S$.

Construct a new array $F$, based on $C$ according to the following prescription:

Consider cell $(i'_m, j'_m) \in T'_1$. If cell $(i_m, j_m)$ is not empty in $R$ then $(i'_m, j'_m)$ of $C$ contains a pair $(k', n)$, otherwise it contains a pair $(n, k')$. [In either case $k, nm \in G k \neq n$ (the transversal does not intersect the diagonal)].
Remove whichever pair appears in cell $(i'_m, j'_m)$ and put it in cell $(i'_m, l)$.
Also place $(\infty', k')$, $(\infty, n)$ in cells $(k, j'_m), (n, j'_m)$ respectively.

If $(i'_m, j'_m) \in T'_2$ then a pair $(k', n)$ appears in cell $(i'_m, j'_m)$.
Again remove it, but this time put it in $(l, j'_m)$.
In addition pairs $(k', \infty), (\infty ', n)$ go in cells $(i'_m, k), (i'_m, n)$ respectively.

$F$ is a $\BRS$ because,
\begin{itemize}
  \item{$A$ and $D$ between them intially contain all the
    unordered pairs $\{i, j\}, \{i', j'\}$
    $i, j \in G, i\neq j$ and
    $\{i,j'\}$ for all $i,j \in G$. So $F$ has an
    ordered pair corresponding to each of these.}
  \item{The procedure in the preceding paragraph contributes
    the remaining ordered pairs
    $(\infty', k'),(\infty,n),(k',\infty),(\infty ',n)$ for all $k,n \in G k \neq n$
    in such a way that the
    elements missing from the first $q$ rows and columns are
    suitably placed, and $\infty$ and $\infty '$ are placed
    in each row and column of $F$.  Hence each row and
    column of $F$ contains every member of
    $\{\infty, \infty '\} \cup G \cup G'$
    exactly once.}
\end{itemize}

The block design obtained from the rows of $F$ has blocks
\begin{equation}
\left.
\begin{split}
\{\infty\} \cup B_i \cup \{\infty'\} \cup C'_i \\
\{i\} \cup C_i \cup \{i'\} \cup B'_i
\end{split}
\right \}
0 \leq i \leq q-1
\end{equation}
from rows $0, 1, \ldots, q - 1$ blocks
\begin{equation}
\left.
\begin{split}
\{\infty\} \cup B_i \cup \{i'\} \cup B'_i \\
\{i\} \cup C_i \cup \{\infty\} \cup C'_i
\end{split}
\right \}
0 \leq i \leq q-1
\end{equation}
from rows $0', 1', \ldots, (q - 1)'$, and the blocks
\begin{equation}
\{\infty\} \cup G, \{\infty '\} \cup G'
\end{equation}
obtained from row $l$.

According to condition 3 of Theorem \ref{thm:schellenberg}, these blocks for a $\BIBD$.
\end{proof}

Schellenberg applied his multiplicative construction to the $\BRS$ generated by Hwang’s starter-adder construction, and in doing so was able to establish the following.

\begin{theorem}
For $q \equiv 3\pmod 4$, with $q$ a prime power strictly greater than 3, there exists a $\BRS(2(q+1))$.
\end{theorem}

\begin{proof}
Construct $R$ from the following balanced starter,
\begin{equation}
X = \left \{ (x^{2i},x^{2i+1}):0 \leq i \leq \frac{q-3}{2} \right \}
\end{equation}
with adder,
\begin{equation}
A(X) = \left \{ -x^{2i}(1+x):0 \leq i \leq \frac{q-3}{2} \right \}
\end{equation}

Clearly this is just the Hwang starter-adder of Theorem \ref{thm:hwang}, although the elements in the starter pairs have swapped order.

Now, construct $S$ from the balanced starter,
\begin{equation}
Y = \left \{ (x^{2i-1},x^{2i}):0 \leq i \leq \frac{q-3}{2} \right \}
\end{equation}
\begin{equation}
A(Y) = \left \{ -x^{2i-1}(1+x):0 \leq i \leq \frac{q-3}{2} \right \}
\end{equation}
$Y$ has been obtained simply by swapping the pairs of $X$ and multiplying them by $x^{-1}$.
Therefore, that $Y$ is a balanced starter follows from the fact that $X$ is balanced, as was established in Theorem \ref{thm:hwang}.

It remains to show that $R$ and $S$ satisfy conditions 1, 2 and 3 of Theorem \ref{thm:schellenberg}.
\begin{enumerate}
  \item{$R$ and $S$ are a Latin pair. The positions of the
      starter pairs in the first row of each square are
      determined by the adder, so
      $A(X) \cap A(Y) = \varnothing$ ensures that each
      cell of the first row, hence subsequent rows, of
      $R \odot S$ contains only one pair.
      
      That $R \odot S$ is a pair of superposed Latin squares
      requires that for the pairs in any row, the left hand
      members take all values $0,\ldots, q - 1$, and the right
      members also take values $0,\ldots, q - 1$.  Clearly this
      property holds for all rows if it holds for the first.
      The pairs in the first row of $R \odot S$ are
      \begin{equation}
        \{(0, 0)\} \cup \left \{ (x^{2i - 1}, x^{2i}), (x^{2i}, x^{2i + 1}) \,|\, 0 \leq i \leq \frac{q - 3}{2} \right \}
      \end{equation}
      
      So all $(q - 1)/2$ non-squares and all $(q - 1)/2$
      squares occur as left hand members of pairs, and
      similarly as right hand members. So $R \odot S$ is a
      pair of superposed Latin squares.
      
      For these Latin
      squares to be orthogonal requires that there is no
      repetition of ordered pairs.  For a repetition to
      occur would require some repetition of ordered
      differences among the starter pairs.  Either:
      \begin{equation}
        x^{2i - 1} - x^{2i} = x^{2i} - x^{2i + 1}
      \end{equation}
      or
      \begin{equation}
        x^{2i} - x^{2i-1} = x^{2i+1} - x^{2i}
      \end{equation}
      In other words,
      \begin{equation}
       \pm (x^{2i - 1} + x^{2i + 1}) = \pm (x^{2i} + x^{2i})
      \end{equation}
      Therefore,
      \begin{equation}
        x^{-1} + x = 2
      \end{equation}
      and so $x = 1$.

      Clearly false, so ordered differences are unique.}
  \item{$R \odot S$ has a pair of disjoint \emph{common transversals}.
      For any fixed non-zero $j$ the cells
      $(i, i + j)$, $0 \leq i \leq q - 1$ form a common
      transversal which does not intersect the diagonal.}
  \item{The block designs for $R$ and $S$ are balanced. The
      starters $X$ and $Y$ are balanced starters hence the
      block designs for $R$ and $S$ are $\BIBD$s.}
  \end{enumerate}
\end{proof}

So the Schellenberg multiplication construction establishes the existence of some new orders $q + 1$ where $q$ is not necessarily a prime power. e.g. the orders in bold are now established which had not been under Hwang’s construction:

\begin{center}
  \begin{tabular}{|c|c|c|c|c|c|c|c|c|c|c|c|}
  \hline
     Hwang     &  8 & 12 &    & 20 & 24 & 28 & 32 &    &    & 44 & 48 \\ \hline
  Schellenberg & \textbf{16} & 24 & 32 & \textbf{40} & 48 & \textbf{56} & \textbf{64} & 72 & 80 & \textbf{88} & \textbf{96} \\ \hline
  \end{tabular}
\end{center}

Even with both these theorems, there are obviously missing orders.
The first exception, 36 occurs, for example, because 35 is not a prime power and because 17 is not congruent to 3 modulo 4.

Schellenberg applied the multiplicative construction to the case when $p^r \equiv 1\pmod 4$ [then, of course, $2(p^r + 1) \equiv 0\pmod 4$ as is required] and was able to obtain further results.
Although, as an interesting parallel to the Mullin-Nemeth construction, he also had problems with the Fermat primes and again they were treated separately.
Rather than look at his approach we consider the later approach by Du, Yu, Hwang and Kang amongst others.

