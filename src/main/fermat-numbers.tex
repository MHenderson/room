Unfortunately, in establishing the Mullin-Nemeth starters we were forced to exclude a similarly vast, potentially infinite, class of Room squares by insisting that $t$ be strictly greater than one.
These exceptional Room squares have side $2^k + 1$.

Rectifying this problem is essential if we are to prove the existence of Room squares.
As mentioned previously, the proof relies on a multiplication theorem, so proving that all the prime Room squares exist is vital.
Although the theorem of Mullin-Nemeth will take care of all squares with prime power side, the multiplication theorem is necessary for proving the existence of those whose side can be decomposed into prime factors different from each other.
In fact, the multiplication theorem means that we can ignore the Mullin-Nemeth construction except in the prime case, resorting to multiplication to recover the prime power squares.
Similarly we are only concerned with recovering the exceptional squares with side $2^k + 1$, when $2^k + 1$ is prime.

Primes of this form are known as \inlinedef{Fermat numbers} or \inlinedef{Fermat primes}, after Pierre de Fermat who, 360 years ago conjectured that numbers of the form $2^k + 1$ are always prime when $k$ is a power of two.

\begin{align*}
  F_0 = 2^1 + 1 &= 3 \\
  F_1 = 2^2 + 1 &= 5 \\
  F_2 = 2^4 + 1 &= 17 \\
  F_3 = 2^8 + 1 &= 257 \\
  F_4 = 2^{16} + 1 &= 65537
\end{align*}

After the first four of Fermat’s numbers, all of which were known to him to be prime.
Nearly one hundred years later Euler calculated the following,

\begin{equation}
F_5 = 2^{32}+1 = 4294967297 = 641\times 6700417
\end{equation}

and in doing so disproved Fermat’s conjecture.

Since Euler’s time, $F_6$, $F_7$ and $F_8$ have all been factorised.\footnote{In 1880 F.Landry showed $F_6=2^{64}+1=274177 \times 67280421310721$.
In 1975 Brillhart and Morrison showed $F_7=2^{128}+1=59649589127497217 \times 5704689200685129054721$.
In 1981, Brent and Pollard found that $2^{256}+1=1238926361552897 \times 93461639715357977769163558199606896584051237541638188580280321$.}
It is also known, although most of the factorisations remain unknown, that $F_m$ is composite for $m = [9...23]$.
$F_{24}$, a number with over 5 million digits, remains in doubt.

Whether there be an infinite number of Fermat primes or whether, as empirically seems to be the case, there are only finitely many (possibly just five) such primes, in order for the proof of the existence of Room squares for all odd side greater than 7 to be complete these Fermat prime Room squares must be included.

When the problem of Fermat Room squares was tackled first in the early 1970s, W. D. Wallis used a theorem of J. D. Horton which adapted a famous result of E. H. Moore from the theory of Steiner triple systems.

Moore, in 1893, was able to prove that if Steiner triple systems of orders $v_1$, $v_2$ and $v_3$ exist, where the $v_2$ system is a sub-system of the $v_3$ system, then an STS of order $v_1(v_2 - v_3) + v_3$ also exists.
Horton
\cite{hortonVariationsThemeMoore1970}
adapted this result to other combinatorial objects including Room squares and Wallis
\cite{wallisCombinatoricsRoomSquares2006}
was able to use this Moore-type construction method to include all of the Fermat primes, except $F_3 = 257$.\footnote{Wallis presented a Room square of side 257 a conference in 1973, completing his proof (different from the one presented here) of the existence of Room squares.}

\begin{example}
If Room squares with side $v_1$, $v_2$ and $v_3$ exist, where the square of side $v_2$ is a sub-square of the square with side $v_3$, then a Room square of side $F_4 = 65537$ exists.
Room squares of side 7 and 11 exist, according to the theory of Mullin-Nemeth.
Applying Horton’s theorem once, with $v_3 = 0$ gives a new square of side $v_1v_2 = 77$ (note that Horton’s theorem reduces to the multiplication theorem when $v_3 = 0$).

The trivial Room square of side one exists, and the Mullin-Nemeth starters will provide a Room square of size 13.
So we can apply Horton’s theorem once again to gain a Room square of side 989 because:

\begin{equation}
989 = 13(77 - 1) + 1
\end{equation}

Finally we can use Mullin-Nemeth to produce a Room square of side 67, and a final application of Horton’s theorem gives:

\begin{equation}
65537 = 67(989 - 11) + 11
\end{equation}
\end{example}

The proof of Horton's theorem and also an explanation of Wallis's application of that theorem to solving the Fermat prime problem is excluded because another solution was subsequently found.
A year after Wallis had published his solution to the Fermat problem, Chong and Chan published their (independent) discovery of the strong starters which are known as the Mullin-Nemeth starters.
Also included in their paper was an alternative solution to the same problem, but their solution continued to involve the starter-adder method.
This theorem we prove instead.

\begin{theorem}
\label{thm:chong-chan}
For every Galois field of order $2^{2^m} + 1$, where $m \geq 2$, there exists a Room square of order $2^{2^m} + 2$.
\end{theorem}

\begin{proof}
The following pairs in $Z_p$ (where $p = 2^{2^d} + 1$ and $d = 2^{m - 1})$ constitute a strong starter.

\begin{enumerate}
  \item{$\{i + (r - 1)2^d, i2^d - (r - 1\}$}
  \item{$\{(2^d - i)2^d + r, (2^{d - 1} - r)2^d + 2^{d - 1} - i + 1\}$}
  \item{$\{2^{d-1} + r - 1)2^d + 2^{d - 1} + i, (2^{d - 1} + i - 1)2^d + 2^{d - 1} - (r - 1)\}$}
  \item{$\{(2^{d-1}-i)2^d+2^{d-1}+r,(2^d-r+1)2^d-i+1\}$}
\end{enumerate}

Where $1 \leq r \leq 2^{d - 2}$ and $1 \leq i \leq 2^{d-1}$, so rather than just 4 pairs there are $4 \cdot 2^{d - 2} \cdot 2^{d - 1} = 2^{2d - 1}$ pairs arranged in four different classes.
Before completing the proof we pause for an example just to illustrate the real simplicity of these apparently complicated pairs.
\end{proof}

\begin{example}
Suppose $p = 2^{2\cdot 2} + 1 = 17 = F_2$, then $d = 2^1 \Rightarrow m = 2$ $r = 1, 1 \leq i \leq 2$ and the following pairs should be a strong starter.

\begin{tabular}{ll}
     & $i = 1, r = 1$                                                                   \\
  1  & $\{1 + 0 \cdot 2^2,1 \cdot 2^2 - 0\}                              = \{1, 4\}$    \\
  2  & $\{(2^2 - 1)2^2 + 1, (2^1 - 1)2^2 + 2^1 - 1 + 1\}                 = \{13, 6\}$   \\
  3  & $\{(2^1 + 1 - 1)2^2 + 2^1 + 1, (2^1 + 1 - 1)2^2 + 2^1 - (1 - 1)\} = \{11, 10\}$  \\
  4  & $\{(2^1 - 1)2^2 + 2^1 + 1,(2^2 - 1 + 1)2^2 - 1 + 1\}              = \{7, 16\}$
\end{tabular}

\begin{tabular}{ll}
     & $i = 2, r = 1$                                                                   \\
  1  & $\{2 + 0 \cdot 2^2, 2 \cdot 2^2 - 0\}                             = \{2, 8\}$    \\
  2  & $\{(2^2 - 2)2^2 + 1, (2^1 - 1)2^2 + 2^1 - 2 + 1\}                 = \{9, 5\}$    \\
  3  & $\{(2^1 + 1 - 1)2^2 + 2^1 + 2, (2^1 + 2 - 1)2^2 + 2^1 - (1 - 1)\} = \{12, 14\}$  \\
  4  & $\{(2^1 - 2)2^2 + 2^1 + 1, (2^2 - 1 + 1)2^2 - 2 + 1\}             = \{3, 15\}$
\end{tabular}

The pairs generated by this method contain each non-zero member of $Z_{17}$ exactly once in their union satisfying the first property of a starter.

The differences are
$\{\pm 3, \pm 7, \pm 1, \pm 8, \pm 6, \pm 4, \pm 2, \pm 5\} = Z_{17} \backslash \{0\}$,
satisfying the other necessary property of a starter.
The sums $5, 2, 4, 6, 10, 14, 9, 1$ are all unique, hence the starter is strong and the set $\{-5, -2, -4, -6, -10, -14, -9, -1\} = \{12, 15, 13, 11, 3, 8, 16\}$ is an adder.
So the following first row will generate a Room square under cyclic construction:

\begin{equation*}
  \begin{bmatrix}
    \infty, 0 & 3,15 & 13,6 & - & 11,10 & 1,4 & 7,16 & - & - & 12,14 & 2,8 & - & - & - & 9,5 & - & -
  \end{bmatrix}
\end{equation*}
\end{example}

\begin{proof}
In order to prove that the pairs $1 \ldots 4$ are a strong starter from any $Z_p$ we need to prove the following:

\begin{enumerate}
  \item{The union of all the pairs contains each non-zero member of $Z_p$ exactly once.}
  \item{The differences are all the non-zero members of $Z_p$ exactly once.}
  \item{The sums are all distinct and non-zero.}
\end{enumerate}

This is a formidable task, one that would take many pages to prove in full detail.
So instead we sketch an outline of the proof, explicitly proving a few specific cases.

First we prove (a) completely.

The non-zero members of $Z_P$, namely $\{1, \ldots, 2^{2d}\}$, can be represented uniquely by:

\begin{equation}
C(u, v) = u2^d + v
\end{equation}

where $1 \leq v \leq 2^d$ and $0 \leq u \leq 2^d-1$.

Indeed if $u_12^d + v_1 = u_22^d + v_2$ then $(u_1 - u_2)2^d = (v_2 - v_1)$.

The RHS takes integer values in the interval $[-(2^d - 1), 2^d - 1]$, which is symmetric about the origin and smaller than $2^d$ on both sides.
Whereas the LHS takes integer multiple steps of size $2^d$, so the equality can only hold in the case when both sides equal zero.
Which implies $u_1 = u_2$, $v_1 = v_2$ and $C(u, v)$ is unique representation the non-zero members of $Z_p$.
$u$ takes $2^d$ values and $v$ takes $2^d$ values so there are $2^{2d}$ unique non-zero members of $Z_p$ represented in this way, so each member of $Z_p$ is represented.

The left and right hand members of each pair can be characterised by a range of values of $u$ and $v$ in the following manner.

Take, for instance, the left hand member of pair 1, $i + (r - 1)2^d$.
Here $v = i$ and so $1 \leq v \leq 2^{d - 1}$, while $u = (r - 1)$, so $0 \leq u \leq 2^{d - 2} - 1$.
The full list of intervals for each member of each pair is tabulated below.

\begin{table}[h!]
  \begin{center}
    \begin{tabular}{cccc}
     Pair & Member &                u                  &           V                  \\ \hline
        1 &   L    &   $[0,2^{d-2}-1]$                 &  $[1,2^{d-1}]$               \\
          &   R    &   $[0,2^{d-1}-1]$                 &  $[3 \cdot 2^{d-2}+1,2^{d}]$ \\
        2 &   L    &   $[2^{d-1},2^{d}-1]$             &  $[1,2^{d-2}]$               \\
          &   R    &   $[2^{d-2},2^{d-1}-1]$           &  $[1,2^{d-1}]$               \\
        3 &   L    &   $[2^{d-1},3 \cdot 2^{d-2}-1]$   &  $[1+2^{d-1},2^{d}]$         \\
          &   R    &   $[2^{d-1},2^{d}-1]$             &  $[1+ 2^{d-2},2^{d-1}]$      \\
        4 &   L    &   $[0,2^{d-1}-1]$, $[1+ 2^{d-1}]$ &  $3 \cdot 2^{d-2}]$          \\
          &   R    &   $[3 \cdot 2^{d-2},2^{d}-1]$     &  $[1+2^{d-1},2^{d}]$ 
    \end{tabular}
  \end{center}
  \caption{Intervals}
  \label{tab:intervals}
\end{table}

It was mentioned earlier that there were $2^{2d - 1}$ pairs, each of which has two members, so there are $2^{2d}$ elements altogether in the pairs of the starter, which is the same as the number of elements in $Z_p$.
Because $C(u, v)$ is a unique representation for each member of $Z_p$, for an element of $Z_p$ to occur more than once in the starter requires repetition of both $u$ and $v$.
This cannot happen because when two intervals overlap (as they do in the values of $v$ for 1L and 2R).

To prove (b) we need to show that the differences between two pairs of type 1 are all unique, similarly between two pairs of types 2,3 and 4.
Moreover we need to show that there can be no repetition in differences between a pair of type 1 and a pair of type 2, also type 1 with types 3 in 4.
Similarly for 2,3 and 4.
All together there are ten cases to prove, tabulated below, where a pair of numbers represents the two types of pairs from the starter.

\begin{table}[h!]
  \begin{center}
    \begin{tabular}{cccccccc}
      (i)    & 11 & (ii) & 12 & (iii) & 13 & (iv) & 14 \\
      (v)    & 22 & (vi) & 23 & (vii) & 24 &      &    \\
      (viii) & 33 & (ix) & 34 &       &    &      &    \\
      (x)    & 44 &      &    &       &    &      &    
    \end{tabular}
  \end{center}
  \caption{Cases}
  \label{tab:cases}
\end{table}

To illustrate, we prove (v), in other words that differences between two different pairs, both of type 2 are always unique.

Type 2 have the form:
$\{(2^d - i)2^d + r, (2^{d - 1} - r)2^d + 2^{d - 1} - i + 1\}$.

Therefore, a difference between the elements of a pair of type 2 has the form:

\begin{equation}
  \pm \{(2^d - i)2^d + r - (2^{d - 1} - r)2^d - 2^{d - 1} + i - 1\}
\end{equation}

If two different pairs had the same difference we could write:

\begin{equation}
(2^d - i)2^d + r - (2^{d - 1} - r)2^d - 2^{d - 1} + i - 1 \equiv \pm \{(2^d - j)2^d + s - (2^{d - 1} - s)2^d - 2^{d - 1} + j - 1\}\pmod p
\end{equation}

for some $i \neq j, r \neq s$.

There are two cases to prove, firstly consider the one involving the + sign.

\begin{equation}
(2^d - i)2^d + r - (2^{d - 1} - r)2^d - 2^{d - 1} + i - 1 \equiv (2^d - j)2^d + s -(2^{d - 1} - s)2^d - 2^{d - 1} + j -1\pmod p
\end{equation}

In this case we have been helped out with some very convenient cancelling, leaving just:

\begin{align*}
  -i2^d + r + r \cdot 2^d + i &\equiv -j \cdot 2^d + s + s \cdot 2^d + j \pmod p \\
          (-i + j)2^d + i - j &\equiv (s - r)(1 + 2^d) \pmod p \\ 
             (j - i)(2^d - 1) &\equiv (s - r)(1 + 2^d) \pmod p \\
          (j - i)(2^{2d} - 1) &\equiv (s - r)(1 + 2 \cdot 2^d + 2^{2d}) \pmod p \\
                    -2(j - i) &\equiv (s - r) 2 \cdot 2^d \pmod p \\
                      (i - j) &\equiv (s - r) 2^d \pmod p
\end{align*}

for all 1 $\leq i, j \leq 2^{d-1}$.

Therefore $(i - j)$ lies in the interval
$[-(2^{d - 1} - 1), 2^{d - 1} - 1]$,
which is symmetric about the origin, with length
$2 \cdot 2^{d - 1} - 2 = 2^d - 2 < p = 2^{2d} + 1$.
$1 \leq s,r \leq 2^{d-2}$

Therefore, $(s - r)2^d$ lies in the interval
$[-(2^{d - 2} - 1)2^d, (2^{d - 2} - 1)2^d]$,
again symmetric about the origin with length
$2^{2d} - 2^{d + 1} < p$.

So for A to hold requires that $(i - j) = (s - r)2^d$.
But the LHS has an interval with length $2^d - 2 < 2^d$ whereas the RHS is some positive or negative integer multiple of $2^d$, so the two could only be equal when $i = j, r = s$ contradicting the original hypothesis.

There is still the negative case to deal with:

\begin{align*}
(2^d - i)2^d + r - (2^{d - 1} - r)2^d - 2^{d - 1} + i - 1 &\equiv -(2^d - j)2^d - s + (2^{d - 1} - s)2^d + 2^{d - 1} - j + 1\pmod p \\
(2^d - i)2^d - (2^{d} - j)2^d + i + j &\equiv -s + (2^{d - 1}  -s)2^d + (2^{d - 1} - r)2^d - r + 2 \cdot 2^{d - 1} + 2\pmod p \\
2 \cdot 2^{2d} + (1 - 2^{d})(i + j) &\equiv -(s + r)(1 + 2^d) + 2^{2d} + 1 + 2^d + 1\pmod p
\end{align*}

Now $2^{2d} + 1 = p$.
Therefore, $2^{2d} \equiv -1\pmod p$, and so

\begin{align*}
-2 + (1 - 2^d)(i + j) &\equiv -(s + r)(1 + 2^d) + 1 + 2^d\pmod p \\
(1 - 2^d)(i + j) &\equiv -(s + r)(1 + 2^d) + 3 + 2^d\pmod p
\end{align*}

Now multiply throughout by $(1 + 2^d)$, noting that:

$(1 + 2^d)(1 - 2^d) \equiv 2\pmod p$,
$(1 + 2^d)(1 + 2^d) = 1 + 2 \cdot 2^d + 2^{2d} \equiv 2^{d + 1}\pmod p$
and
$(1 + 2^d)(3 + 2^d) = 3 + 4 \cdot 2^d + 2^{2d} \equiv 2^{d + 2} + 2 \pmod p$

\begin{align*}
  2(i + j)   &\equiv -2^{d + 1}(s + r) + 2^{d + 2} + 2 \pmod p \\
  (i + j)    &\equiv -2^{d}(s + r) + 2^{d + 1} + 1 \pmod p \\
  2^d(s + r) &\equiv 2 \cdot 2^{d} - (i + j) + 1 \pmod p
\end{align*}

for all $1 \leq s, r \leq 2^{d-2}$.

Therefore, $2 \cdot 2^d \leq 2^d(s+r) \leq 2^d2^{d-1}$.

The LHS lies in the interval $[2^{d + 1}, 2^{2d - 1}]$, which itself is located somewhere in the interval $[0, p]$.

Now $1 \leq i, j \leq 2^{d - 1}$.
Therefore $2 \leq i + j \leq 2^d$ and thus
$2^d \leq 2 \cdot 2^d -(i+j) \leq 2^{d+1} - 2$.

So the RHS lies in the interval $[2^d + 1, 2^{d + 1} - 1]$.
Again this is located with $[0, p]$.

So for B to be satisfied requires that
$2^d(s + r) = 2 \cdot 2^d - (i + j) + 1$
But the RHS and LHS intervals are disjoint so this can never happen.
So the absence of repetition in the differences of two different pairs both of type 2 is proven.
All cases involving different pairs of the same type are proven in this way (cases (i),(v),(viii),(x)).

Finally we demonstrate how the other six cases are proven, those involving pairs of different types.
Inevitably the approach is very similar.

Consider a pair of type 1 and another pair of type 4 (case iv).
If there were a repetition of differences between pairs of this type we could write:

\begin{equation*}
  i + (r - 1)2^d - i2^d + (r - 1) \equiv \pm \{(2^{d - 1} - j)2^d + 2^{d - 1} + s - (2^d - s + 1)2^d + j - 1\}\pmod p
\end{equation*}

Consider the + sign,

\begin{align*}
i-i2^d-(2^{d-1}-j)2^d-j &\equiv -(r-1)2^d-(r-1)+s-(2^d-s+1)2^d+2^{d-1}-1\pmod p \\
  (1-2^d)(i-j)-2^{2d-1} &\equiv -(2^d+1)(r-s)-2^{2d}+2^{d-1}\pmod p
\end{align*}

As $2^{2d} \equiv -1\pmod p$, it follows that

\begin{align*}
  (1-2d)(i-j) &\equiv -(2^d+1)(r-s)+2^{d-1} + 2^{2d-1} + 1\pmod p \\
  2(1-2d)(i-j) &\equiv -2(2^d+1)(r-s)+2^{d} + 2^{2d} + 2\pmod p \\
  2(1-2d)(i-j) &\equiv -2(2^d+1)(r-s)+2^{d}+1\pmod p \\
  4(i-j) &\equiv -2 \cdot 2^{d+1}(r-s)+2^{d+1}\pmod p \\
  (i-j) &\equiv -2^{d}(r-s)+2^{d-1}\pmod p \\
  2^d(r-s) &\equiv 2^{d-1}-(i-j)\pmod p
\end{align*}

In this instance the LHS has interval
$[2^d - 2^{2d - 2}, 2^{2d - 2} - 2^d]$.
while the interval of the RHS is $[1, 2^d - 1]$.
Both are smaller than $p$ in length, so for equality requires
$$2^d(r  -s) = 2^{d - 1} - (i - j)$$

But this can never be true because the left side is always either zero or an integer multiple of $2^d$, whereas the interval of the right is $[1, 2^d - 1]$.

All other cases are dealt with in a very similar manner, and the proof of (c), namely that all sums are unique, is not very different.
\end{proof}

