\chapter{An Existence Theorem for Room Squares}
\label{ch:existence-theorem}

\begin{theorem}
There exists a Room square of side $v$, where $v$ is a positive integer different from 3 or 5.
\end{theorem}

\begin{proof}
A Room square of side one exists trivially.

Any positive integer can be rewritten in the form:

\begin{equation}
v = 3^{a}5^{b}n
\end{equation}

where $n$ is relatively prime to fifteen, i.e.  $(15, n) = 1$.

Suppose that $a + b = 0$, then a Room square exists due to Theorems
\ref{thm:starter-adder},
\ref{thm:furnishing},
\ref{thm:strong-starter-3},
\ref{thm:chong-chan},
\ref{thm:multiply},
and
\ref{thm:wallis}.

For $a + b = 1$, then either $v = 3n$ or $v = 5n$.

In both cases, provided $v > 5$, Theorem \ref{thm:wallis} guarantees the existence of a Room square of side $v$.

Now consider $a + b = 2$.
There are three cases.

\begin{enumerate}
  \item{$v = 3\cdot 5\cdot n$}
  \item{$v = 3^{2}n$}
  \item{$v = 5^{2}n$}
\end{enumerate}

In all cases, a Room square exists by Theorem \ref{thm:wallis}, provided that Room squares of side nine, fifteen and twenty-five all exist.

A Room square of side twenty-five exists by Theorem \ref{thm:strong-starter-3} and Room squares of side nine and fifteen were constructed in
\cite{mullinFurnishingRoomSquares1969}

For examples of Room squares of orders ten and sixteen see, respectively, XXXX and XXX.

Greater values of $a + b$ follow by induction since if $a + b \geq 3$, where $t = 3^{c}5^{d}n$ and $c + d \geq 2$.
In both cases applying Theorem \ref{thm:wallis} will provide a square of side $v$ given one of side $t$.

\end{proof}
