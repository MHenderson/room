\chapter{An Existence Theorem for Room Squares}
\label{ch:existence-theorem}

\begin{theorem}
There exists a Room square of side $v$, where $v$ is a positive integer different from 3 or 5.
\end{theorem}

\begin{proof}
A Room square of side one exists trivially.

Any positive integer can be rewritten in the form:

\begin{equation}
v = 3^{a}5^{b}n
\end{equation}

where $n$ is relatively prime to fifteen, i.e.  $(15, n) = 1$.

Suppose that $a + b = 0$, then a Room square exists due to Theorems
\ref{thm:starter-adder},
\ref{thm:furnishing},
\ref{thm:strong-starter-3},
\ref{thm:chong-chan},
\ref{thm:multiply},
and
\ref{thm:wallis}.

For $a + b = 1$, then either $v = 3n$ or $v = 5n$.

In both cases, provided $v > 5$, Theorem \ref{thm:wallis} guarantees the existence of a Room square of side $v$.

Now consider $a + b = 2$.
There are three cases.

\begin{enumerate}
  \item{$v = 3\cdot 5\cdot n$}
  \item{$v = 3^{2}n$}
  \item{$v = 5^{2}n$}
\end{enumerate}

In all cases, a Room square exists by Theorem \ref{thm:wallis}, provided that Room squares of side nine, fifteen and twenty-five all exist.

A Room square of side twenty-five exists by Theorem \ref{thm:strong-starter-3} and Room squares of side nine and fifteen were constructed in
\cite{mullinFurnishingRoomSquares1969}

For examples of Room squares of orders ten and sixteen see, respectively, \eqref{eq:order-ten} and \eqref{eq:order-sixteen}.

\begin{equation}
  \begin{bmatrix}
    -  & 5,0 &  -  &  -  &  -  & 3,1 & 6,7 & 2,8 & 4,9 \\ 
   8,7 & 3,4 &  -  & 5,9 & 6,1 &  -  &  -  &  -  & 2,0 \\ 
   4,6 & 7,9 & 2,3 & 1,0 & 5,8 &  -  &  -  &  -  &  -  \\ 
    -  &  -  &  -  & 8,4 & 0,9 & 2,6 & 5,3 &  -  & 7,1 \\ 
   5,2 & 8,6 & 7,0 &  -  &  -  &  -  & 4,1 & 9,3 &  -  \\ 
   3,0 &  -  & 8,1 &  -  &  -  &  -  & 2,9 & 7,4 & 5,6 \\ 
    -  & 1,2 & 5,4 &  -  & 7,3 & 9,8 &  -  & 6,0 &  -  \\ 
    -  &  -  & 6,9 & 2,7 &  -  & 4,0 &  -  & 5,1 & 8,3 \\ 
   1,9 &  -  &  -  & 3,6 & 4,2 & 5,7 & 0,8 &  -  &  -  \\ 
  \end{bmatrix}
  \label{eq:order-ten}
\end{equation}

\begin{equation}
  \begin{bsmallmatrix}
    7,11 &   -   &   -   &   -   & 12,14 &   -   &   -   &  8,9  &  2,3  &   -   & 13,1  &  0,6  &   -   & 15,4  & 10,5  \\ 
     -   &  8,1  &   -   &   -   &   -   &  2,15 &  7,12 &  6,14 &  5,11 & 10,4  &   -   &  3,13 &   -   &   -   &  9,0  \\ 
     -   & 12,4  &  6,13 &   -   &   -   &  5,0  &   -   &   -   &  1,15 &  2,8  &  7,9  &  0,11 &   -   &   -   & 14,3  \\ 
    1,12 &   -   &  5,15 &  6,8  &   -   &   -   &   -   &   ,   &  9,13 & 14,0  &   -   &   -   &   -   & 10,7  & 11,2  \\ 
     -   &   -   &  9,11 &   -   &  8,15 &   -   &   -   &   ,   &   -   & 12,5  &  3,6  &  2,14 &   -   &  0,13 &  4,7  \\ 
   10,9  & 15,7  & 12,3  &   -   &   -   & 11,1  &  8,0  &   ,   &   -   &   -   &  4,14 &   -   &   -   &  6,5  &   -   \\ 
    2,6  &   -   &   -   &   -   &  3,0  &  9,14 &  1,5  &   ,   &   -   &   -   & 15,10 &  7,8  &  4,13 &   -   &   -   \\ 
    8,13 &  9,3  &   -   &   -   &  5,4  &   -   &   -   &   -   &  7,0  &   -   &   -   & 12,15 &  2,10 & 14,11 &  1,6  \\ 
     -   &  5,13 &   -   &  0,10 &   -   &  3,7  &  9,2  &   -   &  4,6  & 11,15 &   -   &   -   &  1,14 &   -   &  8,12 \\ 
     -   &   -   &  2,4  &  1,7  &   -   & 13,10 & 15,6  &   -   &  8,14 &   -   & 11,0  &   -   &  5,3  & 12,9  &   -   \\ 
     -   & 14,10 &   -   &  5,2  &  9,1  & 12,6  &  4,11 &   ,   &   -   & 13,7  &   -   &   -   &   -   &  8,3  &   -   \\ 
     -   &  2,0  &   -   & 14,15 & 13,11 &  4,8  &   -   &   -   & 12,10 &  3,1  &   -   &  5,9  &  7,6  &   -   &   -   \\ 
    5,14 &   -   &  8,10 & 11,3  &  7,2  &   -   &   -   &   -   &   -   &  6,9  &   -   &  1,4  & 12,0  &   -   & 13,15 \\ 
    3,15 &   -   &  0,1  &  9,4  &  6,10 &   -   &  3,14 &   ,   &   -   &   -   & 12,2  &   -   &  8,11 &   -   &   -   \\ 
    4,0  &  6,11 &  7,14 & 13,12 &   -   &   -   & 10,3  &   -   &   -   &   -   &  8,5  &   -   &  9,15 &  1,2  &   -   \\ 
  \end{bsmallmatrix}
  \label{eq:order-sixteen}
\end{equation}

Greater values of $a + b$ follow by induction since if $a + b \geq 3$, where $t = 3^{c}5^{d}n$ and $c + d \geq 2$.
In both cases applying Theorem \ref{thm:wallis} will provide a square of side $v$ given one of side $t$.

\end{proof}
