\chapter{Proving the Existence of Room Squares}
\label{ch:existence-proof}

\section{Introduction}

The theorem which will ultimately be established in Chapter \ref{ch:existence-theorem} relies upon a fundamental theorem in number theory – in fact \emph{the} fundamental theorem.
The Fundamental Theorem of Arithmetic states that every positive integer, except 1, can be expressed uniquely as a product of primes.
Proof of this theorem can be found in
\cite{hardyIntroductionTheoryNumbers1979}.

The proof which established the existence of Room squares will rely upon various other theorems which collectively establish the existence of all Room squares with prime side, except 3 and 5.
Then multiplication theorems will be developed to establish the existence of composite Room squares (those whose side is the product of two or more primes).
Clearly if the prime Room squares can be proven to exist, and hence composite Room squares, the fundamental theorem will allow us to state that all Room squares exist with odd positive integer side.
Apart from a few exceptional cases, this is basically what we will be able to do.

\section{Starters, adders and cyclic Room squares}

\begin{equation}
  \begin{bmatrix}
    \infty 0 &     -    &     -    &     25     &     -    &     16     &    34    \\
      45     & \infty 1 &     -    &      -     &    36    &      -     &    20    \\
      31     &    56    & \infty 2 &      -     &     -    &     40     &     -    \\
       -     &    42    &    60    &  \infty 3  &     -    &      -     &    51    \\
      62     &     -    &    53    &     01     & \infty 4 &      -     &     -    \\
       -     &    03    &     -    &     64     &    12    &  \infty 5  &     -    \\
       -     &     -    &    14    &      -     &    05    &     23     & \infty 6 \\
  \end{bmatrix}
  \label{eq:cyclic}
\end{equation}

The Room square in \eqref{eq:cyclic} has a special property.
The pairs in any element of the array are obtained by simply adding 1 (mod 7) to the pair in the element immediately above and to the left; along with the condition that

\begin{equation}
  \infty + 1 = \infty
\end{equation}

This special property means that the entire square can be determined by the pairs in the first row, with successive rows being developed in a cyclical manner according the simple addition rule.
We call squares like \eqref{eq:cyclic} \inlinedef{cyclic} Room squares.

Also notice that $\{\infty,i\}$ occurs in position $(i,i)$.
A square with this property is said to be \inlinedef{standardised}.
It is important to realise that any Room square can be standardised.
As mentioned previously neither interchanging the rows or columns nor permuting the symbol-set on which the Room square is based has any effect of the \emph{Room}-ness of that square.

The significance of cyclic Room squares is that the problem of constructing a Room square is (potentially) reduced to that of finding an appropriate first row.
These rows cannot be chosen arbitrarily, both the pairs used and the positions in which they appear need to satisfy certain criteria, but when they do exist a corresponding Room square always exists.
So proving the existence of this subclass of Room squares is a matter only of proving the existence of these special first rows.

\subsection{Finding a starter}

Suppose we wish to construct another Room square of the same size as \eqref{eq:cyclic} based on the same symbols.
This new square will also be standardized so we need only determine the three pairs that accompany $\{0, \infty\}$ in the first row (the starter), and the positions they occupy.

The set we will use to build our starter will be ${1, 2, \ldots, 6}$.

Each member of this set must occur exactly once in the pairs of the starter – in order to satisfy the row condition for a Room square.
Because of the cyclical construction the condition is automatically true for successive rows if true
for the first.

Consider the existence in \eqref{eq:cyclic} of an arbitrary pair $\{a, b\}$.
We know one of the following must be true.

Either $\{2 + i, 5 + i\} = \{a, b\}$ or $\{1 + i, 6 + i\} = \{a, b\}$ or $\{3 + i, 4 + i\} = \{a, b\}$ for $i = 0, 1, 2, \ldots, 6$.
Say $a - b = 1$.
Then $\{2 + i, 5 + i\} = \{a, b\}$ could never be true because $(2 + i) - (5 + i) = -3\pmod 7 = 4$ and $(5 + i) - (2 + i) = 3$.
Similarly, the differences in $\{1, 6\}$ are $\pm 5$ so $\{a, b\}$ couldn't be generated from $\{1, 6\}$.

However, $(4 + i) - (3 + i) = 1$ so $\{a, b\}$ will inevitably be generated by $\{3, 4\}$ for some value of $i = 0, 1, \ldots, 6$.
e.g. $\{2, 3\} = \{3 + 6, 4 + 6\}$

Because $a$ and $b$ separately take on all values from $\{0, 1, 2, \ldots, 6\}$, their differences will similarly take on all these values (except 0 because there are no pairs of the form $\{a, a\}$) and so an essential property for the starter must be that the six differences generated by its three pairs contain all of $\{1, 2, \ldots, 6\}$.

When a starter satisfies this property, and the condition that the pairs contain in their union all of $\{1, 2, \ldots, 6\}$, it is clear that it will inevitably generate the correct pairs which populate a $7 \times 7$ Room square. 
There are three pairs in the starter, each generates seven unique pairs under cyclical construction, which along with the seven pairs generated by $\{0, \infty\}$ counts for all the 28 unordered pairs from $\{\infty, 0, 1, \ldots, 6\}$.

A starter for larger Room squares of course has to obey the same criterion.
We include a general definition based on
\cite{dinitzContemporaryDesignTheory1992}.

If $G$ is an additive Abelian group of order $g$, then a \inlinedef{starter} in $G$ is a set of unordered pairs:
\begin{equation*}
S = \{\{s_i, t_i\}:1 \leq i \leq (g - 1)/2\}
\end{equation*}
which satisfies these properties:

\begin{enumerate}
  \item{$\{s_i:1 \leq i \leq (g-1)/2\} \cup \{t_i : 1 \leq i \leq (g-1)/2\} = G \backslash \{0\}$}
  \item{$\{\pm (s_i - t_i ) : 1 \leq i \leq (g-1)/2 \} = G \backslash \{0\}$}
\end{enumerate}

Whenever we have any $t$ sets $D_1, \ldots, D_t$ each of size $k$ in which each non-zero member of an additive Abelian group can be represented as a difference between members of the $D_i \lambda$ times, we say those sets form a \inlinedef{difference system}.

Much use will be made of difference systems throughout this work.
Notice that the definition of a starter presumes standardization, and therefore that $\{\infty, i\}$ is in position $(i, i)$.
The following pairs form a starter in $G = \{0, 1, 2, \ldots, 6\}$ (an additive Abelian group with order $g = 7$.)

\begin{equation}
\{1,3\} \{2,6\} \{4,5\}
\end{equation}

Property 1 is satisfied because
$\{1,3\} \cup \{2,6\} \cup \{4,5\} = G \backslash \{0\}$

Property 2 is also satisfied because
\begin{equation}
\begin{split}
\{1 - 3 = 5, 3 - 1 = 2, 2 - 6 = 3, 6 - 2 = 4, 4 - 5 = 6, 5 - 4 = 1\} &= \{1, 2, 3, 4, 5, 6\} \\
 &= G\backslash \{0\}
\end{split}
\end{equation}

Hence
\begin{equation}
  \begin{bmatrix}
    \infty 0 &  13 &  26 &  45 \\
    \infty 1 &  24 &  30 &  56 \\
    \infty 2 &  35 &  41 &  60 \\
    \infty 3 &  46 &  52 &  01 \\
    \infty 4 &  50 &  63 &  12 \\
    \infty 5 &  61 &  04 &  23 \\
    \infty 6 &  02 &  15 &  34
  \end{bmatrix}
  \label{eq:starter}
\end{equation}

are all the unordered pairs from $\{\infty, 0, 1, \ldots, 6\}$ sorted into seven rows that contain each of $\{\infty, 0, 1, \ldots, 6\}$ exactly once.
All that remains is to determine the columns.

\subsection{Finding an adder}

In constructing the starter we made use of the fact that each row has to contain each symbol exactly once and all unordered pairs from the symbol set have to occur exactly once in the whole array.
The remaining condition – namely, that each symbol must occur once in each column – is now employed to finish the construction.

Again, because of the cyclical nature of Room squares generated from starters we can be sure that if one column contains each member of the symbol set, all columns will.

Also, because we have decided to construct a standardized Room Square we know that column $i$ contains $\{\infty,i\}$.
So the final column (column 6) contains $\{\infty,6\}$, and depending on where we place the starter pairs it will also include:
\begin{equation*}
\{1,3\} + x \hspace{1cm} \{2,6\} + y \hspace{1cm}\{4,5\} + z
\end{equation*}
For some distinct values of $x,y$ and $z$ (only one pair allowed per box).
Considering that the new pairs to form column 6 must contain in their union each of
$\{0,1,2,...,5\}$
we build the following table.

\begin{equation}
  \begin{bmatrix}
    x &  13 + x & 26 + y & z & 45 + z \\
    0 &    13   &   26   & 0 &   45   \\
    1 &    24   &   30   & 1 &   56   \\
    2 &    35   &   41   & 2 &   60   \\
    3 &    46   &   52   & 3 &   01   \\
    4 &    50   &   63   & 4 &   12   \\
    5 &    61   &   04   & 5 &   23   \\
  \end{bmatrix}
  \label{eq:adder}
\end{equation}

Our task is simply to determine three unique values for $x, y$ and $z$ such that $13 + x, 26 + y$ and $45 + z$ contain in their union each of $\{0, 1, 2, \ldots, 5\}$.
These values will then determine the positions to place 13, 26 and 45 in row 1.

Choosing 4 from the first column corresponds to having 50 appear in the final column of the Room Square and forces the selection of $y = 2$ from the next column of the table, (41 being the only pair not containing any of the already used 5, 6 or 0).
23 is the only possible choice from the final column, accompanied by a value of $z = 5$.
These three numbers are known as an \inlinedef{adder} corresponding to the starter 13, 26, 45.
This is not necessarily the only adder.

If 50 is to be generated in the final column of the Room square by the pair 13 in the first row, then 13 must go in column $7 - 4 = 3$.
Similarly 26 has to be put in column $7 - 2 = 5$ and 45 in $7 - 5 = 2$.
We can now construct our cyclic room square.

\begin{equation}
  \begin{bmatrix}
    \infty 0 &  45 &  13 &   - &  26 &   - &   - \\
     - &  \infty 1 &  56 &  24 &   - &  30 &   - \\
     - &   - &  \infty 2 &  60 &  35 &   - &  41 \\
    52 &   - &   - &  \infty 3 &  01 &  46 &   - \\
     - &  63 &   - &   - &  \infty 4 &  12 &  50 \\
    61 &   - &  04 &   - &   - &  \infty 5 &  23 \\
    34 &  02 &   - &  15 &   - &   - &  \infty 6 
  \end{bmatrix}
  \label{eq:cyclic-room}
\end{equation}

In general, we define an adder by considering the elements which must accompany $\{\infty, 0\}$ in column 0.
Therefore an adder is defined in the following way:

An \inlinedef{adder} for a starter
$S = \{\{s_i, t_i\}: 1 \leq i \leq (g - 1)/2 \}$
is a set of $(g - 1)/2$ distinct non-zero elements
$a_1, a_2, ..., a_{(g - 1)/2}$ of $G$ such that:
$s_1 + a_1, t_1 + a_1, s_2 + a_2, \ldots, s_{(g - 1)/2} + a_{(g - 1)/2}, t_{(g - 1)/2} + a_{(g - 1)/2}$
are precisely all the non-zero elements of $G$.

The \inlinedef{starter-adder} method employed in the above example was introduced\footnote{Both Howell and Whitfield had previously found starters and adders, but the precise method used here due to Stanton and Mullin.}
by Stanton and Mullin
\cite{stantonConstructionRoomSquares1968},
who used it to construct Room squares of side 11.
They also went on to apply the method to larger squares and gave the first real suggestions that the number of Room squares is infinite.

Two simple lemmas given in
\cite{stantonConstructionRoomSquares1968}
demonstrated that the problem of finding starters for larger Room squares was straightforward.
In fact they can be guaranteed always to exist, and the only difficulty comes from finding a corresponding adder, which is not guaranteed to exist.

\begin{lemma}
In an additive Abelian group $G$ of order $g = 2n-1$, then pairs
\begin{equation*}
  \{n - 1, n\}, \{n - 2, n + 1\}, \{n - 3, n + 2\}, \{n - 4, n + 3\}, \ldots\{1, 2n - 2\}
\end{equation*}
are a starter for a Room square of side $2n - 1$.
\end{lemma}

\begin{example}
A Room square of side $2n - 1 = 19$.

\begin{equation*}
S_{19} = \{\{9, 10\}, \{8, 11\}, \{7, 12\}, \{6, 13\}, \{5, 14\}, \{4, 15\}, \{3, 16\}, \{2, 17\}, \{1, 18\}\}
\end{equation*}
is a starter.

Indeed, the differences are
\begin{equation*}
\begin{split}
  & \{\pm(10 - 9), \pm(11 - 8), \pm(12 - 7), \pm(13 - 6), \pm(14 - 5), \pm(3 - 16), \pm(2 - 17), \pm(18 - 1)\} \\ 
  &= \{1, 18, 3, 16, 5, 14, 7, 12, 9, 10, 8, 11, 6, 13, 4, 15, 2, 17\} \\
  &= G \backslash \{0\}
\end{split}
\end{equation*}
\end{example}

\begin{lemma}
In the Galois field of order $k - 1$, with primitive root $a$ the following pairs form a starter for a Room square of side $k$.
\begin{equation}
  \{a, a^n\}, \{a^2, a^{n + 1}\}, \{a^3, a^{n + 2}\}, \ldots, \{a^{n - 1}, a^{2n - 2}\}
\end{equation}
\end{lemma}

\begin{example}
A Room square of side $2n - 1 = 23$. $n = 12$. $a = 5$.
The set of pairs
\begin{equation*}
\begin{split}
S_{23} &= \{\{5, 5^{12}\}, \{5^2, 5^{13}\}, \{5^3, 5^{14}\}, \ldots, \{5^{11}, 5^{22}\}\} \\
       &= \{\{5, 18\}, \{2, 21\}, \{10, 13\}, \ldots, \{22, 1\}\}
\end{split}
\end{equation*}
is a starter.
\end{example}

On closer inspection the two types of starters are identical\footnote{The starter in then first Lemma has pairs whose elements always sum to $2n-1$, while the second Lemma, because $a^{n-1}=-1$, has pairs which can be written in the form $(a^x,-a^x)$.}, with a general element being of the form $\{j, -j\}$.

Starters of this form are called \inlinedef{patterned} starters.

Stanton and Mullin went on to they could find adders corresponding to the patterned starters for $k = 7, 11, 13, 15, 17$.
They had problems with 9 (but were able to construct one using a different method) and finding it too laborious for $k > 19$ they developed an algorithm which, when implemented in Fortran, was able to find patterned starters with adders for all odd $k$ up to 49, with no further gaps.
Suggesting the possibility (which they conjectured) that there are Room squares for all odd side greater than 5.

They also found an interesting result regarding the number of Room Squares which could be obtained from patterned starters, summarised in table \ref{tab:patterned}. 

\begin{table}[h!]
  \begin{center}
    \begin{tabular}{c|c}
    Value of $k$ & Number of PRS \\ \hline
    7 & 2 \\
    9 & 0 \\
    11 & 4 \\
    13 & 8 \\
    15 & 44 \\
    17 & 416 \\
    19 & The program was turned off after the production of 967 PRS
    \end{tabular}
  \end{center}
  \caption{Number of patterned Room squares (PRS)}
  \label{tab:patterned}
\end{table}

Stanton and Mullin's results suggest that the number of PRS (patterned Room squares) increases very rapidly.
Which, bearing in mind that the PRS are a sub-class of CRS (cyclic Room squares), which are in turn a sub-class of Room squares, implies that there are vast numbers of Room squares of large order.

Before introducing a class of starters for which the existence of a corresponding adder is guaranteed we quickly confirm that when a starter and adder exist then a Room square will always result.
This seems obvious from the method outlined in the previous section, but now we prove it explicitly.

\begin{theorem}
\label{thm:starter-adder}
If an Abelian group $G$ of odd order $2n - 1$ admits a starter and an adder, then there exists a Room square of order $2n$.
\end{theorem}

\begin{proof}
A square is constructed on the set $G \cup \{\infty\}$, where $G$ is an additive Abelian group of order $2n-1$.

\begin{equation}
G = \{g_0 = 0, g_1, g_2, \ldots, g_{2n-2}\}
\end{equation}

The columns and rows of the square are labelled as follows:

\begin{equation}
  \begin{bmatrix}
        -      & g_0  &  g_1  &  g_2  & \ldots &  g_{2n - 1} \\
       g_0     &   -  &   -   &   -   &    -   &     -       \\
       g_1     &   -  &   -   &   -   &    -   &     -       \\
       g_2     &   -  &   -   &   -   &    -   &     -       \\
     \vdots    &   -  &   -   &   -   &    -   &     -       \\
    g_{2n - 1} &   -  &   -   &   -   &    -   &     -       \\
  \end{bmatrix}
\end{equation}

If a starter
$\{\{s_1, t_1\}, \{s_2, t_2\} \ldots \{s_{n - 1}, t_{n - 1}\}\}$
and an adder
$\{a_1, a_2, \ldots, a_{n - 1}\}$
can be obtained from $G$ and if the square is populated by pairs of elements from $G$ according to the following rules:

\begin{enumerate}
  \item{$\{\infty, g_i\}$ goes in $(g_i, g_i)$}
  \item{While $\{s_i + g_i, t_i + g_i\}$ goes in $(g_i, g_i - a_i)$}
\end{enumerate}

for all $g_i \in G$.
The resulting square will be a Room square on $G \cup \{\infty\}$.

\begin{enumerate}
  \item{Row $g_0 = 0$, contains the pairs
      $\{s_i, t_i\}:1 \leq i \leq n-1$,
      which are the elements of the starter,
      hence all of $G \backslash \{0\}$. These pairs are
      accompanied by $\{\infty, 0\}$, so row 0 contains all
      of $G \cup \{\infty\}$. Subsequent rows simple
      contain a permutation of the same elements, hence
      the \emph{row property} of Room squares is satisfied for
      all rows.}
  \item{As mentioned before, the starter forms a difference
      system in $G \backslash \{0\}$, so all unordered
      pairs of this set occur along with all unordered
      pairs of the form
      $\{\infty, g_i\}: 1 \leq i \leq n - 1$,
      hence \emph{all unordered pairs from}
      $G \cup \{\infty\}$ \emph{occur} in the square exactly once.}
  \item{All pairs of the form $\{s_i + a_i, t_i + a_i\}$ go in
      $(a_i, 0)$, i.e. column 0. According to the
      definition of a starter these pairs are all of $G
      \backslash \{0\}$, and we know that $\{\infty,0\}$
      is also in column 0. So the first column, and hence
      all others, contains all of $G \cup \{\infty\}$,
      thus satisfying the \emph{column property} of Room
      squares.}
\end{enumerate}
\end{proof}

\section{Strong Starters}
\label{sec:strong-starters}

The next stage in proving the existence of Room squares came about, not by continuing to try to find adders for starters that were already known (the patterned starters, for example), but when Mullin and Nemeth in
\cite{mullinFurnishingRoomSquares1969},
discovered a class of starters that generated their own adders.

\begin{theorem}
\label{thm:furnishing}
Suppose a starter
$\{\{s_1, t_1\}, \{s_2, t_2\}, \ldots, \{s_{(g - 1)/2}, t_{(g - 1)/2}\}\}$
exists, such that the sums of each pair
$(s_1 + t_1, s_2 + t_2, etc...)$
are all distinct and non-zero, then that starter is said to be strong, and
\begin{equation}
 A(S) = \{a_i = -(s_i + t_i):1 \leq i \leq (g - 1)/2\}
\end{equation}
is an adder for a starter.
\end{theorem}

% https://tex.stackexchange.com/questions/59573/is-it-possible-to-use-the-enumerate-itemize-environment-within-the-proof-remark
\begin{proof}\leavevmode
\begin{enumerate}
  \item{The $a_i$ are all distinct and non-zero:
    All the $(s_i + t_i)$ are, by definition, distinct
    and non-zero. Therefore all the $a_i = -(s_i + t_i)$ are
    distinct and non-zero.}
  \item{$s_1 + a_1, t_1 + a_1, s_2 + a_2, \ldots, s_{(g - 1)/2}, t_{(g - 1)/2} + a_{(g - 1)/2}$
    are precisely all the non-zero elements of $G$.
    \begin{align*}
      s_1 + a_1 = s_1 - (s_1 + t_1) = -t_1 &= t_{(g - 1)/2}  \\
      t_1 + a_1 = t_1 - (s_1 + t_1) = -s_1 &= s_{(g - 1)/2}  \\
      s_{(g - 1)/2} + a_{(g - 1)/2} = -t_{(g - 1)/2} &= t_1  \\
      t_{(g - 1)/2} + a_{(g - 1)/2} = -s_{(g - 1)/2} &= s_1 
    \end{align*}
    Are all the non-zero elements of $G$ in reverse order.}
\end{enumerate}
(Notice that the patterned starter is not strong, on the contrary, the sums of its pairs are all identical.)
\end{proof}

\begin{example}
The pairs
$(5, 7)(11, 6)(2, 8)(9, 12)(10, 1)(3, 4)$,
constitute a strong starter for a Room square of side 13, based on $G = Z_{13}$.

Firstly, the pairs satisfy the conditions for being a starter, as the union of all pairs is equal to $G \backslash \{0\}$, and similarly the differences are all of $G\backslash \{0\}$.
Secondly the sums of the pairs, respectively $12, 4, 10, 8, 11, 7$ are all distinct and non-zero.
Therefore an adder is
$\{-12, -4, -10, -8, -11, -7\} = \{1, 9, 3, 5, 2, 6\}$
So the following is a legitimate first row for a cyclic Room square of order 14.

\begin{equation}
  \begin{bmatrix}
    \infty, 0 & - & - & - & 11,6 & - & - & 3,4 & 9,12 & - & 2,8 & 1,10 & 5,7 \\
  \end{bmatrix}
\end{equation}

\end{example}

Mullin and Nemeth originally discovered strong starters for Room squares embedded within another type of combinatorial design, known as a Steiner triple system.
With these they were able to prove that Room squares exist for all sides
$v = 1\pmod 6$.
Rather than examine this approach we move on to a type of starter which provides its own adder.

\section{Mullin-Nemeth Starters}
\label{sec:mullin-nemeth}

If $x$ is a primitive element in $G = GF(p^n)$, then the elements $x^1, x^2, \ldots, x^{p^n - 1} = 1$ are, by definition, all of $G \backslash \{0\}$.
Alternatively, we can write $G \backslash \{0\} = \{x^0 = 1, x^1, \ldots, x^{p^n - 2}\}$.

\begin{example}[label=eg:mullin-nemeth]
The field $GF(23)$ has a primitive root $x = 5$, because $5^0 = 1$, $5^1 = 5$, $5^2 = 2$, $5^3 = 10$, $5^4 = 4$, $5^5 = 20$, $5^6 = 8$, $5^7 = 17$, $5^8 = 16$, $5^9 = 11$, $5^{10} = 9$ $5^{11} = 22$, $5^{12} = 18$, $5^{13} = 21$, $5^{14} = 13$, $5^{15} = 19$, $5^{16} = 3$, $5^{17} = 15$, $5^{18} = 6$, $5^{19} = 7$, $5^{20} = 12$, $5^{21} = 14$ are all the non-zero elements of $GF(23)$.
\end{example}

Mullin and Nemeth in
\cite{mullinFurnishingRoomSquares1969}
used the theory of primitive elements to create strong starters in the additive group of (nearly) any Galois field of prime-power order.
Which, because theorems
\ref{thm:starter-adder}
and
\ref{thm:furnishing}
were already known, was equivalent to proving the existence of Room squares for (nearly) all orders $p^n + 1$.
Before introducing the general construction for these starters, we illustrate the basic method with a couple of examples of particular cases.

\begin{example}[label=ex:strong-starter]
We can create a strong starter from Example \ref{eg:mullin-nemeth} simply by pairing the elements in the order in which they were generated.
\begin{equation}
S = \{\{1, 5\}, \{2, 10\}, \{4, 20\}, \{8, 17\}, \{16, 11\}, \{9, 22\}, \{18, 21\}, \{13, 19\}, \{3, 15\}, \{6, 7\}, \{12, 14\}\}
\end{equation}
is a strong starter.

Obviously each member of $GF(23)$ occurs once, because of the definition of a primitive root.
The differences
\begin{equation}
  \{\pm 4, \pm 8, \pm 7, \pm 9, \pm 10, \pm 5, \pm 3, \pm 6, \pm 11, \pm 1, \pm 2\}
\end{equation}
are similarly all of $GF(23)$, so $S$ is a starter.
The sums
\begin{equation}
  \{6, 12, 1, 2, 4, 8, 16, 9, 18, 13, 3\}
\end{equation}
are all unique, and therefore $S$ is strong and
\begin{equation}
  A = \{17, 11, 22, 21, 19, 15, 7, 14, 5, 10, 20\}
\end{equation}
is an adder for $S$.
So, the following row will generate a Room square of order 24 under cyclic construction.

\begin{equation}
  \begin{smallmatrix}
    \infty, 0 & 4,20 & 8,17 & 12,14 & 16,11 & - & 1,5 & - & 9,22 & 13,19 & - & - & 2,10 & 6,7 & - & - & 18,21 & - & 3,15 & - & - & - & - \\
  \end{smallmatrix}
\end{equation}

\end{example}

This is an example of the simplest case of the general theorem of Mullin and Nemeth, where the Galois field is $Z_p$ (the integers mod $p$), with $p = 23 = 3\pmod 4$ a prime.

\begin{theorem}
\label{thm:strong-starter}
If $p = 4m + 3$ is prime, $m \geq 1$, then
\begin{equation}
S = \{\{x^0, x^1\}, \{x^2, x^3\}, \ldots, \{x^{4m}, x^{4m+1}\}\}
\end{equation}
is a strong starter in $Z_p$, and hence a Room square of order $p + 1$ exists.
\end{theorem}

Example
\ref{ex:strong-starter}
took $m = 5$ and $x = 5$.

A slightly more general version of Theorem
\ref{thm:strong-starter},
which we prove instead, involves any field of prime power order where $p^n = 2t + 1$, with $t > 1$ and odd.
Of course, when $p^n$ is not prime, the field will no longer be the integers modulo $p$, instead the primitive element will be an irreducible polynomial whose coefficients belong to $Z_p$.

\begin{theorem}
\label{thm:strong-starter-2}
If $p^n = 2t + 1 = 3\pmod 4$ then
\begin{equation*}
S = \{\{x^0, x^1\}, \{x^2, x^3\}, \ldots, \{x^{2t - 2}, x^{2t - 1}\}\}
\end{equation*}
is a strong starter in $GF(p^n), (p^n \neq 3)$
\end{theorem}

\begin{proof}
$x$ is a primitive element, so the elements in the starter are all the non-zero members of $GF(P^n)$.
The differences are, respectively
\begin{equation*}
\pm x^0(1 - x), \pm x^2 (1 - x), \ldots, \pm x^{2t - 2}(1 - x)
\end{equation*}
$(1 - x)$ is a non-zero ($x = 1$ is not primitive) member of $GF(p^n)$.

So in order to show that these differences are all the $2t$ non-zero members of $GF(p^n)$ we merely need to prove that the $2t$ differences are all distinct and non-zero.

All the differences can be written
$\pm x^{2i}(1 - x)$, $0 \leq i \leq t - 1$

$(1 - x) \neq 0$

$x^{2i}(1 - x) = x^{2j}(1 - x)$

$\Rightarrow x^{2i} = x^{2j}$
$\Rightarrow i = j$,
because $0 \leq 2i, 2j \leq 2t - 2 < p^{n - 1}$, and the primitive element, by definition, produces each element of $GF(p^n)$ exactly once as the indices range from 0 to $p^{n - 1}$.
Similarly
$-x^{2i}(1 - x) = -x^{2j}(1 - x)$ only when $i = j$.
So all the positive differences are unique, similarly the negative.
However, there remains a possibility for repetition when the signs are opposite:

\begin{equation}
\label{eq:signs}
x^{2i}(1 - x) = -x^{2j}(1 - x)
\end{equation}

Either $i = j$ or $i \neq j$.

Let $i = j$,

\eqref{eq:signs} becomes
$x^{2i} + x^{2i} = 0, \Rightarrow 2x^{2i} = 0$,
but $i$ takes values $0 \ldots t - 1$, so $x^2 = 0$ when $i = 1$, contradicting the order of the primitive element.
In the $i \neq j$ case, we assume (without loss of generality) that $i < j$ and write $x^{2i} = -x^{2j}$.
As $x^{2i}(1 + x^{2j - 2i}) = 0$ it follows that $x^{2j - 2i} = -1$

In $GF(2t + 1)$, $x^{\frac{1}{2}(q - 1)} = x^t = -1$

\begin{equation*}
2j - 2i = t
\end{equation*}

but this is a contradiction as we insisted that $t$ be odd.

So $S$ is a starter.

To prove that $S$ is strong we simply note that the sums can
be written:

\begin{equation*}
x^0(1 + x), x^2(1 + x), \ldots, x^{2t - 2}(1 + x)
\end{equation*}

$1 + x = 0 \Rightarrow x = -1$ is only true when $p^n = 3$.

So $(1 + x) \neq 0$.

So $x^{2i}(1 + x) = x^{2j}(1 + x) \Rightarrow x^{2i} = x^{2j}$.

We have already shown that $x^{2i} = x^{2j}$ is only true for $i = j$.

So all the sums are unique, and the starter is strong.

Hence, by Theorems
\ref{thm:strong-starter}
and
\ref{thm:strong-starter-2},
Room squares exist for all $p^n\equiv 3\pmod 4$, and in the case when $p^n\equiv 3\pmod 4$ is
prime, these Room squares are based on $Z_p$.
\end{proof}

The most generalised case of Mullin and Nemeth’s theorem proves the existence of Room squares for all prime powers $p^n = 2^kt + 1$ where $k > 1$ and $t > 1$ is odd ($k$ and $t$, both positive integers), and reduces to Theorem \ref{thm:strong-starter} when $k = 1$.

\begin{theorem}
\label{thm:strong-starter-3}
A strong starter exists in $GF(p^n)$, where $p^n = 2^kt + 1$
(with $k > 1$ and $t > 1$ is odd).
\end{theorem}

\begin{proof}
Let $d = 2^{k-1}$.

Then the strong starter in question looks like this:

\begin{equation*}
S = \left\{
  \begin{array}{cccc}
    (x^0,x^d) & (x^{2d},x^{3d}) & \ldots & (x^{(2t - 2)d},x^{(2t - 1)d}) \\
    (x^1,x^{d + 1}) & (x^{2d + 1},x^{3d + 1}) & \ldots & (x^{(2t - 2){d + 1}},x^{(2t - 1){d + 1}}) \\
    \vdots & \vdots & \ldots & \vdots \\
    (x^{d - 1},x^{2d - 1}) & (x^{3d - 1},x^{4d - 1}) & \ldots & (x^{(2t - 2){d - 1}},x^{2td - 1})
  \end{array}
\right\}
\end{equation*}

Where the pairs have been placed in an array to emphasise that, when read vertically, this is an exhaustive list of all the non-zero elements of $GH(p^n)$, ordered according to powers.
Of course, in the $k = 1, d = 1$ case this starter reduces to the one quoted in Theorem
\ref{thm:strong-starter-2}.

To prove that $S$ is a starter we need also to show, as usual, that the differences between pairs are all of $GF(p^n)$, and to show that the starter is strong we need to show that the sums of pairs are all distinct and non-zero.

The differences can be written in the following scheme:

\begin{equation*}
  \begin{array}{cccc}
    x^0(1 - x^d), & x^{2d}(1 - x^d), & \ldots, & x^{(2t - 2)d}(1 - x^d) \\
    x^1(1 - x^d), & x^{2d + 1}(1 - x^d), & \ldots, & x^{(2t - 2)d + 1}(1 - x^d) \\
    \vdots & \vdots & \ldots & \vdots \\
    x^{d - 1}(1 - x^d), & x^{3d - 1}(1 - x^d), & \ldots, & x^{(2t - 1)d - 1}(1 - x^d)
  \end{array}
\end{equation*}

The order of $x$ is $p^n-1 = 2^kt = 2^{k-1}2t = 2td > d$, (meaning $x^{2td} = 1$ and $x^\alpha \neq 1$ when $1 \leq \alpha < 2dt$) and so $x^d \neq 1$, so $(1 - x^d) \neq 0$.
We can write the differences in a general form:

\begin{equation}
\pm x^{2id + j}(1 - x^d), 0 \leq i \leq t-1, 0 \leq j \leq d-1
\end{equation}

If there were repetition, either of the form $D = D$ or $-D = -D$, where $D = x^{2id + j}(1 - x^d)$,
then the following must hold:

\begin{equation}
\pm x^{2id + j}(1 - x^d) = \pm x^{2Id + J}(1 - x^d)
\end{equation}

Cancelling by $(1 - x^d)$, legitimate because $(1 - x^d) \neq 0$, gives:

\begin{equation}
x^{2id + j} = -x^{2ID + J}
\end{equation}

dividing through by $x^{2Id + j}$ leaves

\begin{align*}
  x^{2id - 2Id} &= x^{J = j} \\
  x^{2d(i - I)} &= x^{J - j}
\end{align*}

But if $i \neq I$, then the LHS has an index which is an integer multiple of $d$.
The index in the RHS, however, can never be an integer multiple of $d$ because $J$ and $j$ range over the integers $0 \ldots d - 1$.
So the only possibility for equality is when both indices are zero, i.e. $i = I$ and $j = J$.

As in the previous proof we have to deal with the possibility of repetition for differences of opposite sign.

For coincidence we require:

\begin{align*}
  x^{2id + j} &= -x^{2ID + J} \\
  x^{2id + j} + x^{2Id + J} &= 0
\end{align*}

We assume that $2id + j < 2Id + J$ and rewrite this expression as:

\begin{equation}
x^{2id + j}(1 + x^{(2I - 2i)d + (J - j)}) = 0
\end{equation}

Which implies that $x^{(2I - 2i)d + (J - j)} = -1$.

But in $GF(q)$, $x^{\frac{1}{2}(q - 1)} = -1$.
Where, in this case $q - 1 = 2^kt$, so $\frac{1}{2}(q - 1) = 2^{k - 1}t = dt$.

Therefore, $x^{dt} = -1$ and so $(2I - 2i)d + (J - j) = dt$.

Thus, $(J - j)$ is an integer multiple of $d$ or zero.

But $J$ and $j$ both take only the values $0 \ldots d-1$, so $(J - j)$ is in the interval $[1 - d, d - 1]$ and hence must be zero, leaving

\begin{eqnarray*}
  (2I - 2i)d &= dt \\
  2I - 2i &= t
\end{eqnarray*}

But $t$ is strictly odd, and so we have reached a contradiction, hence the differences are all unique, belong to $GF(p^n)$ and there are $2td$ of them, hence each member of $GF(p^n)$ occurs exactly once as a difference. So $S$ is a starter.
To prove that the starter is strong we write the sums as

\begin{equation*}
  \begin{array}{cccc}
    x^0(1 + x^d), & x^{2d}(1 + x^d), & \ldots, & x^{(2t - 2)d}(1 + x^d) \\
    x^1(1 + x^d), & x^{2d + 1}(1 + x^d), & \ldots, & x^{(2t - 2)d + 1}(1 + x^d) \\
    \vdots & \vdots & \ldots & \vdots \\
    x^{d - 1}(1 + x^d), & x^{3d - 1}(1 + x^d), & \ldots, & x^{(2t - 1)d - 1}(1 + x^d)
  \end{array}
\end{equation*}

and notice that $x^d = -1 \Rightarrow d = dt$ (because $x^{dt} = -1) \Rightarrow t = 1$, but instead we insisted that $t$ be strictly greater than one (this being the reason why).
So $(1 + x^d) \neq 0$ and the above argument involving $(1 - x^d)$ can be invoked, replacing $(1 - x^d)$ by $(1 + x^d)$.

So $S$ is a strong starter, and the general theorem of Mullin and Nemeth is proven, guaranteeing the existence of a vast class of Room Squares.
\end{proof}

\section{The Trouble with Fermat Numbers}
\label{sec:trouble-fermat}

Unfortunately, in establishing the Mullin-Nemeth starters we were forced to exclude a similarly vast, potentially infinite, class of Room squares by insisting that $t$ be strictly greater than one.
These exceptional Room squares have side $2^k + 1$.

Rectifying this problem is essential if we are to prove the existence of Room squares.
As mentioned previously, the proof relies on a multiplication theorem, so proving that all the prime Room squares exist is vital.
Although the theorem of Mullin-Nemeth will take care of all squares with prime power side, the multiplication theorem is necessary for proving the existence of those whose side can be decomposed into prime factors different from each other.
In fact, the multiplication theorem means that we can ignore the Mullin-Nemeth construction except in the prime case, resorting to multiplication to recover the prime power squares.
Similarly we are only concerned with recovering the exceptional squares with side $2^k + 1$, when $2^k + 1$ is prime.

Primes of this form are known as \inlinedef{Fermat numbers} or \inlinedef{Fermat primes}, after Pierre de Fermat who, 360 years ago conjectured that numbers of the form $2^k + 1$ are always prime when $k$ is a power of two.

\begin{align*}
  F_0 = 2^1 + 1 &= 3 \\
  F_1 = 2^2 + 1 &= 5 \\
  F_2 = 2^4 + 1 &= 17 \\
  F_3 = 2^8 + 1 &= 257 \\
  F_4 = 2^{16} + 1 &= 65537
\end{align*}

After the first four of Fermat’s numbers, all of which were known to him to be prime.
Nearly one hundred years later Euler calculated the following,

\begin{equation}
F_5 = 2^{32}+1 = 4294967297 = 641\times 6700417
\end{equation}

and in doing so disproved Fermat’s conjecture.

Since Euler’s time, $F_6$, $F_7$ and $F_8$ have all been factorised.\footnote{In 1880 F.Landry showed $F_6=2^{64}+1=274177 \times 67280421310721$.
In 1975 Brillhart and Morrison showed $F_7=2^{128}+1=59649589127497217 \times 5704689200685129054721$.
In 1981, Brent and Pollard found that $2^{256}+1=1238926361552897 \times 93461639715357977769163558199606896584051237541638188580280321$.}
It is also known, although most of the factorisations remain unknown, that $F_m$ is composite for $m = [9...23]$.
$F_{24}$, a number with over 5 million digits, remains in doubt.

Whether there be an infinite number of Fermat primes or whether, as empirically seems to be the case, there are only finitely many (possibly just five) such primes, in order for the proof of the existence of Room squares for all odd side greater than 7 to be complete these Fermat prime Room squares must be included.

When the problem of Fermat Room squares was tackled first in the early 1970s, W. D. Wallis used a Theorem of J. D. Horton which adapted a famous result of E. H. Moore from the theory of Steiner triple systems.

Moore, in 1893, was able to prove that if Steiner triple systems of orders $v_1$, $v_2$ and $v_3$ exist, where the $v_2$ system is a sub-system of the $v_3$ system, then an STS of order $v_1(v_2 - v_3) + v_3$ also exists.
Horton
\cite{hortonVariationsThemeMoore1970}
adapted this result to other combinatorial objects including Room squares and Wallis
\cite{wallisCombinatoricsRoomSquares2006}
was able to use this Moore-type construction method to include all of the Fermat primes, except $F_3 = 257$.\footnote{Wallis presented a Room square of side 257 a conference in 1973, completing his proof (different from the one presented here) of the existence of Room squares.}

\begin{example}
If Room squares with side $v_1$, $v_2$ and $v_3$ exist, where the square of side $v_2$ is a sub-square of the square with side $v_3$, then a Room square of side $F_4 = 65537$ exists.
Room squares of side 7 and 11 exist, according to the theory of Mullin-Nemeth.
Applying Horton’s theorem once, with $v_3 = 0$ gives a new square of side $v_1v_2 = 77$ (note that Horton’s theorem reduces to the multiplication theorem when $v_3 = 0$).

The trivial Room square of side one exists, and the Mullin-Nemeth starters will provide a Room square of size 13.
So we can apply Horton’s theorem once again to gain a Room square of side 989 because:

\begin{equation}
989 = 13(77 - 1) + 1
\end{equation}

Finally we can use Mullin-Nemeth to produce a Room square of side 67, and a final application of Horton’s theorem gives:

\begin{equation}
65537 = 67(989 - 11) + 11
\end{equation}
\end{example}

The proof of Horton's theorem and also an explanation of Wallis's application of that theorem to solving the Fermat prime problem is excluded because another solution was subsequently found.
A year after Wallis had published his solution to the Fermat problem, Chong and Chan published their (independent) discovery of the strong starters which are known as the Mullin-Nemeth starters.
Also included in their paper was an alternative solution to the same problem, but their solution continued to involve the starter-adder method.
This theorem we prove instead.

\begin{theorem}
\label{thm:chong-chan}
For every Galois field of order $2^{2^m} + 1$, where $m \geq 2$, there exists a Room square of order $2^{2^m} + 2$.
\end{theorem}

\begin{proof}
The following pairs in $Z_p$ (where $p = 2^{2^d} + 1$ and $d = 2^{m - 1})$ constitute a strong starter.

\begin{enumerate}
  \item{$\{i + (r - 1)2^d, i2^d - (r - 1\}$}
  \item{$\{(2^d - i)2^d + r, (2^{d - 1} - r)2^d + 2^{d - 1} - i + 1\}$}
  \item{$\{2^{d-1} + r - 1)2^d + 2^{d - 1} + i, (2^{d - 1} + i - 1)2^d + 2^{d - 1} - (r - 1)\}$}
  \item{$\{(2^{d-1}-i)2^d+2^{d-1}+r,(2^d-r+1)2^d-i+1\}$}
\end{enumerate}

Where $1 \leq r \leq 2^{d - 2}$ and $1 \leq i \leq 2^{d-1}$, so rather than just 4 pairs there are $4 \cdot 2^{d - 2} \cdot 2^{d - 1} = 2^{2d - 1}$ pairs arranged in four different classes.
Before completing the proof we pause for an example just to illustrate the real simplicity of these apparently complicated pairs.
\end{proof}

\begin{example}
Suppose $p = 2^{2\cdot 2} + 1 = 17 = F_2$, then $d = 2^1 \Rightarrow m = 2$ $r = 1, 1 \leq i \leq 2$ and the following pairs should be a strong starter.

\begin{tabular}{ll}
     & $i = 1, r = 1$                                                                   \\
  1  & $\{1 + 0 \cdot 2^2,1 \cdot 2^2 - 0\}                              = \{1, 4\}$    \\
  2  & $\{(2^2 - 1)2^2 + 1, (2^1 - 1)2^2 + 2^1 - 1 + 1\}                 = \{13, 6\}$   \\
  3  & $\{(2^1 + 1 - 1)2^2 + 2^1 + 1, (2^1 + 1 - 1)2^2 + 2^1 - (1 - 1)\} = \{11, 10\}$  \\
  4  & $\{(2^1 - 1)2^2 + 2^1 + 1,(2^2 - 1 + 1)2^2 - 1 + 1\}              = \{7, 16\}$
\end{tabular}

\begin{tabular}{ll}
     & $i = 2, r = 1$                                                                   \\
  1  & $\{2 + 0 \cdot 2^2, 2 \cdot 2^2 - 0\}                             = \{2, 8\}$    \\
  2  & $\{(2^2 - 2)2^2 + 1, (2^1 - 1)2^2 + 2^1 - 2 + 1\}                 = \{9, 5\}$    \\
  3  & $\{(2^1 + 1 - 1)2^2 + 2^1 + 2, (2^1 + 2 - 1)2^2 + 2^1 - (1 - 1)\} = \{12, 14\}$  \\
  4  & $\{(2^1 - 2)2^2 + 2^1 + 1, (2^2 - 1 + 1)2^2 - 2 + 1\}             = \{3, 15\}$
\end{tabular}

The pairs generated by this method contain each non-zero member of $Z_{17}$ exactly once in their union satisfying the first property of a starter.

The differences are
$\{\pm 3, \pm 7, \pm 1, \pm 8, \pm 6, \pm 4, \pm 2, \pm 5\} = Z_{17} \backslash \{0\}$,
satisfying the other necessary property of a starter.
The sums $5, 2, 4, 6, 10, 14, 9, 1$ are all unique, hence the starter is strong and the set $\{-5, -2, -4, -6, -10, -14, -9, -1\} = \{12, 15, 13, 11, 3, 8, 16\}$ is an adder.
So the following first row will generate a Room square under cyclic construction:

\begin{equation*}
  \begin{bmatrix}
    \infty, 0 & 3,15 & 13,6 & - & 11,10 & 1,4 & 7,16 & - & - & 12,14 & 2,8 & - & - & - & 9,5 & - & -
  \end{bmatrix}
\end{equation*}
\end{example}

\begin{proof}
In order to prove that the pairs $1 \ldots 4$ are a strong starter from any $Z_p$ we need to prove the following:

\begin{enumerate}
  \item{The union of all the pairs contains each non-zero member of $Z_p$ exactly once.}
  \item{The differences are all the non-zero members of $Z_p$ exactly once.}
  \item{The sums are all distinct and non-zero.}
\end{enumerate}

This is a formidable task, one that would take many pages to prove in full detail.
So instead we sketch an outline of the proof, explicitly proving a few specific cases.

First we prove (a) completely.

The non-zero members of $Z_P$, namely $\{1, \ldots, 2^{2d}\}$, can be represented uniquely by:

\begin{equation}
C(u, v) = u2^d + v
\end{equation}

where $1 \leq v \leq 2^d$ and $0 \leq u \leq 2^d-1$.

Indeed if $u_12^d + v_1 = u_22^d + v_2$ then $(u_1 - u_2)2^d = (v_2 - v_1)$.

The RHS takes integer values in the interval $[-(2^d - 1), 2^d - 1]$, which is symmetric about the origin and smaller than $2^d$ on both sides.
Whereas the LHS takes integer multiple steps of size $2^d$, so the equality can only hold in the case when both sides equal zero.
Which implies $u_1 = u_2$, $v_1 = v_2$ and $C(u, v)$ is unique representation the non-zero members of $Z_p$.
$u$ takes $2^d$ values and $v$ takes $2^d$ values so there are $2^{2d}$ unique non-zero members of $Z_p$ represented in this way, so each member of $Z_p$ is represented.

The left and right hand members of each pair can be characterised by a range of values of $u$ and $v$ in the following manner.

Take, for instance, the left hand member of pair 1, $i + (r - 1)2^d$.
Here $v = i$ and so $1 \leq v \leq 2^{d - 1}$, while $u = (r - 1)$, so $0 \leq u \leq 2^{d - 2} - 1$.
The full list of intervals for each member of each pair is tabulated below.

\begin{table}[h!]
  \begin{center}
    \begin{tabular}{cccc}
     Pair & Member &                u                  &           V                  \\ \hline
        1 &   L    &   $[0,2^{d-2}-1]$                 &  $[1,2^{d-1}]$               \\
          &   R    &   $[0,2^{d-1}-1]$                 &  $[3 \cdot 2^{d-2}+1,2^{d}]$ \\
        2 &   L    &   $[2^{d-1},2^{d}-1]$             &  $[1,2^{d-2}]$               \\
          &   R    &   $[2^{d-2},2^{d-1}-1]$           &  $[1,2^{d-1}]$               \\
        3 &   L    &   $[2^{d-1},3 \cdot 2^{d-2}-1]$   &  $[1+2^{d-1},2^{d}]$         \\
          &   R    &   $[2^{d-1},2^{d}-1]$             &  $[1+ 2^{d-2},2^{d-1}]$      \\
        4 &   L    &   $[0,2^{d-1}-1]$, $[1+ 2^{d-1}]$ &  $3 \cdot 2^{d-2}]$          \\
          &   R    &   $[3 \cdot 2^{d-2},2^{d}-1]$     &  $[1+2^{d-1},2^{d}]$ 
    \end{tabular}
  \end{center}
  \caption{Intervals}
  \label{tab:intervals}
\end{table}

It was mentioned earlier that there were $2^{2d - 1}$ pairs, each of which has two members, so there are $2^{2d}$ elements altogether in the pairs of the starter, which is the same as the number of elements in $Z_p$.
Because $C(u, v)$ is a unique representation for each member of $Z_p$, for an element of $Z_p$ to occur more than once in the starter requires repetition of both $u$ and $v$.
This cannot happen because when two intervals overlap (as they do in the values of $v$ for 1L and 2R).

To prove (b) we need to show that the differences between two pairs of type 1 are all unique, similarly between two pairs of types 2,3 and 4.
Moreover we need to show that there can be no repetition in differences between a pair of type 1 and a pair of type 2, also type 1 with types 3 in 4.
Similarly for 2,3 and 4.
All together there are ten cases to prove, tabulated below, where a pair of numbers represents the two types of pairs from the starter.

\begin{table}[h!]
  \begin{center}
    \begin{tabular}{cccccccc}
      (i)    & 11 & (ii) & 12 & (iii) & 13 & (iv) & 14 \\
      (v)    & 22 & (vi) & 23 & (vii) & 24 &      &    \\
      (viii) & 33 & (ix) & 34 &       &    &      &    \\
      (x)    & 44 &      &    &       &    &      &    
    \end{tabular}
  \end{center}
  \caption{Cases}
  \label{tab:cases}
\end{table}

To illustrate, we prove (v), in other words that differences between two different pairs, both of type 2 are always unique.

Type 2 have the form:
$\{(2^d - i)2^d + r, (2^{d - 1} - r)2^d + 2^{d - 1} - i + 1\}$.

Therefore, a difference between the elements of a pair of type 2 has the form:

\begin{equation}
  \pm \{(2^d - i)2^d + r - (2^{d - 1} - r)2^d - 2^{d - 1} + i - 1\}
\end{equation}

If two different pairs had the same difference we could write:

\begin{equation}
(2^d - i)2^d + r - (2^{d - 1} - r)2^d - 2^{d - 1} + i - 1 \equiv \pm \{(2^d - j)2^d + s - (2^{d - 1} - s)2^d - 2^{d - 1} + j - 1\}\pmod p
\end{equation}

for some $i \neq j, r \neq s$.

There are two cases to prove, firstly consider the one involving the + sign.

\begin{equation}
(2^d - i)2^d + r - (2^{d - 1} - r)2^d - 2^{d - 1} + i - 1 \equiv (2^d - j)2^d + s -(2^{d - 1} - s)2^d - 2^{d - 1} + j -1\pmod p
\end{equation}

In this case we have been helped out with some very convenient cancelling, leaving just:

\begin{align*}
  -i2^d + r + r \cdot 2^d + i &\equiv -j \cdot 2^d + s + s \cdot 2^d + j \pmod p \\
          (-i + j)2^d + i - j &\equiv (s - r)(1 + 2^d) \pmod p \\ 
             (j - i)(2^d - 1) &\equiv (s - r)(1 + 2^d) \pmod p \\
          (j - i)(2^{2d} - 1) &\equiv (s - r)(1 + 2 \cdot 2^d + 2^{2d}) \pmod p \\
                    -2(j - i) &\equiv (s - r) 2 \cdot 2^d \pmod p \\
                      (i - j) &\equiv (s - r) 2^d \pmod p
\end{align*}

for all 1 $\leq i, j \leq 2^{d-1}$.

Therefore $(i - j)$ lies in the interval
$[-(2^{d - 1} - 1), 2^{d - 1} - 1]$,
which is symmetric about the origin, with length
$2 \cdot 2^{d - 1} - 2 = 2^d - 2 < p = 2^{2d} + 1$.
$1 \leq s,r \leq 2^{d-2}$

Therefore, $(s - r)2^d$ lies in the interval
$[-(2^{d - 2} - 1)2^d, (2^{d - 2} - 1)2^d]$,
again symmetric about the origin with length
$2^{2d} - 2^{d + 1} < p$.

So for A to hold requires that $(i - j) = (s - r)2^d$.
But the LHS has an interval with length $2^d - 2 < 2^d$ whereas the RHS is some positive or negative integer multiple of $2^d$, so the two could only be equal when $i = j, r = s$ contradicting the original hypothesis.

There is still the negative case to deal with:

\begin{align*}
(2^d - i)2^d + r - (2^{d - 1} - r)2^d - 2^{d - 1} + i - 1 &\equiv -(2^d - j)2^d - s + (2^{d - 1} - s)2^d + 2^{d - 1} - j + 1\pmod p \\
(2^d - i)2^d - (2^{d} - j)2^d + i + j &\equiv -s + (2^{d - 1}  -s)2^d + (2^{d - 1} - r)2^d - r + 2 \cdot 2^{d - 1} + 2\pmod p \\
2 \cdot 2^{2d} + (1 - 2^{d})(i + j) &\equiv -(s + r)(1 + 2^d) + 2^{2d} + 1 + 2^d + 1\pmod p
\end{align*}

Now $2^{2d} + 1 = p$.
Therefore, $2^{2d} \equiv -1\pmod p$, and so

\begin{align*}
-2 + (1 - 2^d)(i + j) &\equiv -(s + r)(1 + 2^d) + 1 + 2^d\pmod p \\
(1 - 2^d)(i + j) &\equiv -(s + r)(1 + 2^d) + 3 + 2^d\pmod p
\end{align*}

Now multiply throughout by $(1 + 2^d)$, noting that:

$(1 + 2^d)(1 - 2^d) \equiv 2\pmod p$,
$(1 + 2^d)(1 + 2^d) = 1 + 2 \cdot 2^d + 2^{2d} \equiv 2^{d + 1}\pmod p$
and
$(1 + 2^d)(3 + 2^d) = 3 + 4 \cdot 2^d + 2^{2d} \equiv 2^{d + 2} + 2 \pmod p$

\begin{align*}
  2(i + j)   &\equiv -2^{d + 1}(s + r) + 2^{d + 2} + 2 \pmod p \\
  (i + j)    &\equiv -2^{d}(s + r) + 2^{d + 1} + 1 \pmod p \\
  2^d(s + r) &\equiv 2 \cdot 2^{d} - (i + j) + 1 \pmod p
\end{align*}

for all $1 \leq s, r \leq 2^{d-2}$.

Therefore, $2 \cdot 2^d \leq 2^d(s+r) \leq 2^d2^{d-1}$.

The LHS lies in the interval $[2^{d + 1}, 2^{2d - 1}]$, which itself is located somewhere in the interval $[0, p]$.

Now $1 \leq i, j \leq 2^{d - 1}$.
Therefore $2 \leq i + j \leq 2^d$ and thus
$2^d \leq 2 \cdot 2^d -(i+j) \leq 2^{d+1} - 2$.

So the RHS lies in the interval $[2^d + 1, 2^{d + 1} - 1]$.
Again this is located with $[0, p]$.

So for B to be satisfied requires that
$2^d(s + r) = 2 \cdot 2^d - (i + j) + 1$
But the RHS and LHS intervals are disjoint so this can never happen.
So the absence of repetition in the differences of two different pairs both of type 2 is proven.
All cases involving different pairs of the same type are proven in this way (cases (i),(v),(viii),(x)).

Finally we demonstrate how the other six cases are proven, those involving pairs of different types.
Inevitably the approach is very similar.

Consider a pair of type 1 and another pair of type 4 (case iv).
If there were a repetition of differences between pairs of this type we could write:

\begin{equation*}
  i + (r - 1)2^d - i2^d + (r - 1) \equiv \pm \{(2^{d - 1} - j)2^d + 2^{d - 1} + s - (2^d - s + 1)2^d + j - 1\}\pmod p
\end{equation*}

Consider the + sign,

\begin{align*}
i-i2^d-(2^{d-1}-j)2^d-j &\equiv -(r-1)2^d-(r-1)+s-(2^d-s+1)2^d+2^{d-1}-1\pmod p \\
  (1-2^d)(i-j)-2^{2d-1} &\equiv -(2^d+1)(r-s)-2^{2d}+2^{d-1}\pmod p
\end{align*}

As $2^{2d} \equiv -1\pmod p$, it follows that

\begin{align*}
  (1-2d)(i-j) &\equiv -(2^d+1)(r-s)+2^{d-1} + 2^{2d-1} + 1\pmod p \\
  2(1-2d)(i-j) &\equiv -2(2^d+1)(r-s)+2^{d} + 2^{2d} + 2\pmod p \\
  2(1-2d)(i-j) &\equiv -2(2^d+1)(r-s)+2^{d}+1\pmod p \\
  4(i-j) &\equiv -2 \cdot 2^{d+1}(r-s)+2^{d+1}\pmod p \\
  (i-j) &\equiv -2^{d}(r-s)+2^{d-1}\pmod p \\
  2^d(r-s) &\equiv 2^{d-1}-(i-j)\pmod p
\end{align*}

In this instance the LHS has interval
$[2^d - 2^{2d - 2}, 2^{2d - 2} - 2^d]$.
while the interval of the RHS is $[1, 2^d - 1]$.
Both are smaller than $p$ in length, so for equality requires
$$2^d(r  -s) = 2^{d - 1} - (i - j)$$

But this can never be true because the left side is always either zero or an integer multiple of $2^d$, whereas the interval of the right is $[1, 2^d - 1]$.

All other cases are dealt with in a very similar manner, and the proof of (c), namely that all sums are unique, is not very different.
\end{proof}

\section{A Multiplication Theorem}

Having a theorem which enables new Room squares to be composed from old Room squares is of vital importance to the proof of the existence of Room squares.
With such a theorem, in conjunction with the Mullin-Nemeth starters, we will be able to construct Room squares of almost any order.
The exceptions will be due to the non-existence of orders 4 and 6.
The multiplication theorem that will be proven is:

This theorem was proposed initially in
\cite{bruckWhatLoop1963}
but later a counter-example to this method was found
\cite{mullinCounterexampleDirectProduct1969}
The proof here is based upon
\cite{andersonCombinatorialDesignsConstruction1990},
which in turn is based upon the proof in
\cite{stantonMultiplicationTheoremRoom1972a}.

\begin{theorem}
\label{thm:multiply}
	If Room squares of side m and side n exist then a Room square of side $mn$ also exists.
\end{theorem}

\begin{proof}
$M$ and $N$ are two Room squares.
$M$ is of side $m$ and based on $\{0, 1, 2, \ldots, m\}$, while $N$ is of side $n$ and based on $\{0, 1, 2, \ldots, n\}$.

The join of two Latin squares $A$ and $B$ is the array whose $(i, j)$ entry contains the ordered pair formed from the $(i, j)$ entry of $A$ taking the left position and the $(i, j)$ entry of $B$ taking the right.
If the join of two Latin squares contains $n^2$ unique ordered pairs, the two Latin squares $A$ and $B$ are said to be orthogonal.

$L_1$ and $L_2$ are two mutually Latin squares (MOLS) based on $\{1, 2, 3, \ldots, n\}$.

We construct the new Room square $R = MN$ by replacing each element of $M$ by an $n \times n$ array according to the following flow diagram where $(i, j)$ is a pair from $M$.

This procedure has replaced each pair in $m$ by an $n \times n$ array, resulting in an $mn \times mn$ array.
This array is based upon
$\{0, n + 1, n + 2, \ldots, n + mn\}$,
and we now prove that it has the properties of a Room square, namely:

\begin{enumerate}
  \item{Each element of the array is either empty or contains an unordered pair.}
  \item{Each row and column contains each of $\{0, n + 1, n + 2, \ldots, n + mn\}$ exactly once.}
  \item{Each pair from $\{0, n + 1, n + 2, \ldots, n + mn\}$ occurs exactly once in the array.}
\end{enumerate}

The first property is easily satisfied.
The procedure followed did nothing but replace empty elements and unordered pairs with arrays containing nothing more than empty elements or pairs.

The second is similarly straightforward.
Consider an arbitrary row of the new square $R$, call it $i$.
This row arose from applying prescriptions (i),(ii), and (iii) to some row of $M$.
This row of $M$ contained the elements $0 \ldots m$ exactly once.
One of these elements, call it $a$, was paired with 0.
So in $i$ from (ii) occur the numbers $(0;1 + an, \ldots, m + an)$ exactly once.

In the join of two MOLS the numbers $1, \ldots, n$ occur twice per row, once in $L_1$ once in $L_2$.
These are replaced by $(1 + un, \ldots, n + un)$ and $(1 + vn, \ldots, n + vn)$ as $u$ and $v$ take on all values $1, 2, \ldots, m$ excluding a.

Together these two prescriptions produce the elements $\{0;1 + n, 2 + n, \ldots, n + mn\}$ exactly once per row and column.

To prove condition 3 is true we show that $R$ contains the correct number of pairs and that these pairs are distinct.
Because we have shown 2 to be correct these pairs must be the right ones.

Any Room square, of side $n$, contains $\frac{1}{2}(n + 1)$ pairs per row, therefore $\frac{1}{2}n(n + 1)$ pairs over Room square of side $mn$ ought to contain $\frac{1}{2}mn(mn + 1)$ pairs.

In $M$ there were $m$ instances of $\{0, k\}$, each of these was replaced by a Room square of side $\frac{1}{2}n(n + 1) \cdot m$ pairs were contributed by (ii) to $R$.

In $M$ there were
$\frac{1}{2}(m + 1) - 1 = \frac{1}{2} (m  -1)$
pairs per row of the form $\{u, v\}$, therefore $\frac{1}{2}m(m - 1)$ of these pairs throughout $M$.
These were replaced by MOLS of side $n$, containing $n^2$ pairs each.
So $\frac{1}{2}m(m - 1) \cdot n^2$ pairs were contributed to $R$ from (iii). 

(i) contributed no pairs to $R$.

\begin{align*}
  \frac{1}{2} mn(n + 1) + \frac{1}{2}m(m - 1)n^2 &= \frac{1}{2}mn\{(n + 1) + n(m - 1)\} \\
  &= \frac{1}{2}mn(mn + 1)
\end{align*}

So the number of pairs in $R$ is correct.

To show that all the pairs are distinct consider $P(i,j)$ which represents those pairs generated from the element $(i,j)$ of $M$.
The pairs within $P(i,j)$ are always distinct because they are the pairs in a Room square or a join of two MOLS.

However we also need to show that $P(i,j)$ and $P(h,k)$ have no pairs in common when $i,j \neq h,k$. There are three cases to consider.
Both sets of pairs are chosen from the join of two MOLS, both sets are from Room squares, or one set from each.

If both sets of pairs were generated from the join of two MOLS then the pairs have the form $\{un + l_1, vn + l_2\}$.
If this pair occurs in both $P(i, j)$ and $P(h, k)$ then $(l_1, l_2)$ occurs in two different places in joins of two MOLS which is a contradiction.

The case where both sets of pairs are generated by Room squares is easily dealt with, because the Room squares used to construct $R$ are based on different sets, so two could never contain same pair.

In the case where one set of pairs belongs to a Room square and the other to a join of two MOLS, consider differences.
The greatest difference in pairs from a Room square of side $n$ based on $\{1, \ldots, n\}$ is $n - 1$ and (ii) maintains differences.
While the smallest difference in a pair from the join of two MOLS occurs when $l_1 = l_2$.
Then,
\begin{equation}
  (un + l_1) - (vn + l_2) = (u - v)n
\end{equation}
and so the smallest difference is $n$.
So these pairs can be equal.
\end{proof}

\section{Summary}

So far we have shown that all Room squares whose side can be expressed as a prime power $p^n = 2^kt + 1$ can be constructed by using the Mullin-Nemeth starters.
The Fermat primes shown to be an exception to the Mullin-Nemeth construction, but this was overcome by introducing the theorem of Chong and Chan which provides a strong starter form all Room squares of side $(2^{2^m} + 1)$, encompassing the Fermat primes.
So we have proven that all Room squares exist whose side is a prime number, other than 3 or 5.
The multiplication theorem enables us to state that all Room squares exist whose side can be factored as $p_1p_2p_3...p_n$ with $p_i \geq 7$.

The non-existence of Room squares with sides 3 and 5, prevents us from constructing those squares whose sides have a factor of 3 or 5.
Within this class of exempt Room squares the Mullin-Nemeth starters will take care of the prime power sides.
But for those whose side is not a prime power a final theorem, due to W.D. Wallis is needed to complete the proof.

\section{$n$-tuplication of Room Squares}

Before proving the main theorem, we look at an example of triplication in order to introduce this fairly complicated construction.

The approach taken is to take a Room square of side 7, create 9 arrays very similar in structure to this Room square, and then arrange these 9 arrays into a $21 \times 21$ side array which is very Room square.  

\begin{example}
Suppose we wished to triplicate the following Room square.

\begin{equation}
  \begin{bmatrix}
    \infty 0 &   - &  -  &  25 &   - &  16 &  34 \\
    45 &  \infty 1 &  -  &   - &  36 &   - &  20 \\
    31 &  56 &  \infty 2 &   - &   - &  40 &   - \\
     - &  42 &  60 &  \infty 3 &   - &   - &  51 \\
    62 &   - &  53 &  01 &  \infty 4 &   - &   - \\
     - &  03 &   - &  64 &  12 &  \infty 5 &   - \\
     - &   - &  14 &   - &  05 &  23 &  \infty 6 
  \end{bmatrix}
  \label{eq:triple-room}
\end{equation}

Of course, assembling nine of these into an array is pointless, because the new square will have little in common with a $21 \times 21$ Room square.

Suppose we just triplicate the first row.
We have,

\begin{equation*}
  \begin{bsmallmatrix}
    \infty 0 & - & - & 25 & - & 16 & 34 & \infty 0 & - & - & 25 & - & 16 & 34 & \infty 0 & - & - & 25 & - & 16 & 34 \\
  \end{bsmallmatrix}
\end{equation*}

An obvious step to correct this is to triple the size of the set on which the square is based.
Suppose we do that in the following way.

\begin{equation*}
  \begin{bsmallmatrix}
    \infty_{1} 0_{1} & - & - & 2_{1}5_{1} & - & 1_{1}6_{2} & 3_{2}4_{1} & \infty_{2} 0_{2} & - & - & 2_{2}5_{2} & - & 1_{2}6_{2} & 3_{2}4_{2} & \infty_{3} 0_{3} & - & - & 2_{3}5_{3} & - & 1_{3}6_{3} & 3_{3}4_{3} \\
  \end{bsmallmatrix}
\end{equation*}

Unfortunately, this row has simply too many elements.
There should be only eleven pairs, not twelve, and the new Room-ish square would be based on a set of 24 elements not 22 as we require.
Wallis’s original idea had been to somehow merge the three $\infty _\mathrm{i}$ into one element, but he eventually decided instead to go back to the original square and strip out the diagonal elements.
Building a Room square is then a matter of arranging the following arrays, sometimes called frames.

\begin{equation}
  R_{ij} = \begin{bmatrix}
             - &           - &           - &  2_{1}5_{1} &           - &  1_{1}6_{1} &  3_{1}4_{1} \\
    4_{1}5_{1} &           - &           - &           - &  3_{1}6_{1} &           - &  2_{1}0_{1} \\
    3_{1}1_{1} &  5_{1}6_{1} &           - &           - &           - &  4_{1}0_{1} &   - \\
             - &  4_{1}2_{1} &  6_{1}0_{1} &           - &           - &           - &  5_{1}1_{1} \\
    6_{1}2_{1} &           - &  5_{1}3_{1} &  0_{1}1_{1} &           - &           - &   - \\
             - &  0_{1}3_{1} &           - &  6_{1}4_{1} &  1_{1}2_{1} &           - &   - \\
             - &           - &  1_{1}4_{1} &           - &  0_{1}5_{1} &  2_{1}3_{1} &   - 
  \end{bmatrix}
  \label{eq:triple-room-frame}
\end{equation}

into a $21 \times 21$ array, and subsequently finding some way to fill in the missing two pairs from row of the new square, with the aim of producing a Room square based on

\begin{equation}
S = \{\infty, 0_1, 1_1, \ldots, 6_1, 0_2, 1_2, \ldots, 6_2, 0_3, 1_3, \ldots, 6_3\}
\end{equation}

Inevitably this approach leads to new problems.

Firstly consider how to arrange the frames appropriately.
Suppose we put $R_{12}$ next to $R_{13}$.
The left hand members of pairs in each row of $R_{12}$ will be repeated in the same row of the $21 \times 21$ square due to the placing of $R_{13}$.
The same would be true for any $R_{ij}$ next to any $R_{ik}$, next to $R_{kj}$.
So for that reason, in the super-array of $R_{ij}$s we require in each super-row that each value 1,2 and 3 and similarly that $j$ takes on all these values.
To satisfy the column for our new Room square we also require that no $R_{ij}$ occurs above or below an $R_{ik}$ or $R_{kj}$, and that reason we must also insist that for any super-column of $R_{ij}$s, $i$ and $j$ independently values $1, 2, 3$, so that each member of $S \backslash \{\infty\}$ occurs once in the corresponding 7 columns of finished Room square – (except for the missing $\{x_i: 0 \leq x \leq 6\}$ from all columns $x_i$).

Furthermore, as we are aiming for an array in which all the unordered pairs from $S$ occur once if we also insist that each value of $i$ is paired with each value of $j$ exactly once in super-array then we should obtain most of these pairs.

In fact, because $R_{ij} \cup R_{ji}$ contains unordered pairs from $\{0_i,0_j,1_i,1_j,...,6_i,6_j\}$, except those of the form $\{x_i,x_j\}$ $1 \leq i,j \leq 3$ shall obtain all the unordered pairs of $S$ except those of the form

\begin{equation}
\{\infty, x_1\}, \{\infty, x_2\}, \{\infty, x_3\}, \{x_1, x_2\}, \{x_1, x_3\}, \{x_2, x_3\}
\end{equation}

for all $0 \leq x \leq 6$.

The following arrangement of frames satisfies all the conditions we require, as well as a further condition which shall be explained later.

\begin{equation}
  \begin{bmatrix}
    R_{11} & R_{22} & R_{33} \\
    R_{32} & R_{13} & R_{21} \\
    R_{23} & R_{31} & R_{12} \\
  \end{bmatrix}
\end{equation}

Now when we assemble the $R_{ij}$, something quite close to a Room square of side 21, based on $S$, is constructed.

\end{example}

The ideal solution to this problem (because it solves the problem of missing pairs as well as completing rows/columns) would be to place pairs of the form $\{\infty, x_j\}$ and $\{x_i, x_j\}$ at the intersection of rows $x_j$ and column $x_j$, but of course this intersection is a single box, and we don’t want two pairs in one box.

Wallis’s solution to this problem was to permute the columns of some of the $R_{ij}$s from one super-column of the array of frames with the intention of arranging it so that the elements $x_i x_j$ would be vacant from some column $y \neq x_j$.
This enables us to put the $\{\infty, x_j\}$ in column $x_j$ and the $\{x_i,x_j\}$ in column $y$.

The elements missing from row $0_1$ ($\infty, 0_1, 0_2, 0_3$), are also missing from column $0_1$ (due to the removal of the pair $\{\infty,0\}$ from the original Room square to create the frame).
There is no problem in putting $\{\infty,0_1\}$ in position $(0_1,0_1)$, but if we want to put $\{0_2,0_3\}$ in row $0_1$ it must go in some other column of block $R_{11}$, while remaining in column $1_1$ of blocks $R_{23}$ and $R_{32}$. This can be achieved through a column permutation applied only to $R_{23}$ and $R_{32}$.

Notice that it would be of little use to swap columns $0_1$ and $4_1$ of blocks $R_{23}$ and $R_{32}$, as the fourth column is occupied already in the first row of block $R_{11}$.
But we could swap $0_1$ with any of $1_1,2_1,3_1$ or $5_1$ because all these columns are empty in row $0_1$.
Clearly the essential property we require of any column permutation that we decide to use, call it $\theta$, is that $(x,x\theta)$ is unoccupied in the original Room square.

\begin{lemma}
\label{lem:permute}
Given a Room square $R$ of side $r$, where $r=2s+1$, there are $s$ permutations $\phi_1,\phi_2,...,\phi_s$ of $\{1,2,...,r\}$ with the properties that $k\phi_i=k\phi_j$ never occurs unless $i=j$, and that cell $(k,k\phi_i)$ is empty for $1 \leq k \leq r, 1 \leq i\leq s$.
\end{lemma}

\begin{proof}
We define a matrix $M$ in the following manner: If position $(k,l)$ is \inlinedef{empty} in $R$ then the $(k,l)$ position of $M$ is 1, otherwise it is 0.

Because $M$ is a matrix of zeros and ones, whose every row and column sum is equal to $s$, it can be decomposed into $s$ matrices, each of which having exactly one $1$ in each row and column.

\begin{equation}
M = P_1 + P_2 + \ldots + P_s
\end{equation}
\end{proof}

These matrices, when interpreted in the following or similar manner, are known as \inlinedef{permutation matrices}.

Define $\phi_i$ as the permutation corresponding to matrix $P_i$ such that if $(k,l) = 1$ in $P_i$ then $k\phi _i = l$.

The definition of $M$ ensures that the $(k, k\phi _i)$ position is empty in $R$, while if $k\phi_{i} = k\phi_{j}$ was true for some $i \neq j$ then $P_i$ and $P_j$ both have a 1 in position $(k, k\phi_i)$, so $M$ would have an entry equal to 2 or more, contradicting the definition.

\begin{example}
The matrix $M$ associated with the square from Figure 19 is:

\begin{equation}
 M = \begin{bsmallmatrix}
    0 & 1 & 1 & 0 & 1 & 0 & 0 \\
    0 & 0 & 1 & 1 & 0 & 1 & 0 \\
    0 & 0 & 0 & 1 & 1 & 0 & 1 \\
    1 & 0 & 0 & 0 & 1 & 1 & 0 \\
    0 & 1 & 0 & 0 & 0 & 1 & 1 \\
    1 & 0 & 1 & 0 & 0 & 0 & 1 \\
    1 & 1 & 0 & 1 & 0 & 0 & 0 \\
  \end{bsmallmatrix}
\end{equation}

Which can be decomposed (not uniquely) into these permutation matrices:

\begin{equation}
 M = P_1 + P_2 + P_3 = 
  \begin{bsmallmatrix}
    0 & 1 & 0 & 0 & 0 & 0 & 0 \\
    0 & 0 & 1 & 0 & 0 & 0 & 0 \\
    0 & 0 & 0 & 0 & 1 & 0 & 0 \\
    1 & 0 & 0 & 0 & 0 & 0 & 0 \\
    0 & 0 & 0 & 0 & 0 & 1 & 0 \\
    0 & 0 & 0 & 0 & 0 & 0 & 1 \\
    0 & 0 & 0 & 1 & 0 & 0 & 0 \\
  \end{bsmallmatrix}
  +
  \begin{bsmallmatrix}
    0 & 0 & 0 & 0 & 1 & 0 & 0 \\
    0 & 0 & 0 & 1 & 0 & 0 & 0 \\
    0 & 0 & 0 & 0 & 0 & 0 & 1 \\
    0 & 0 & 0 & 0 & 0 & 1 & 0 \\
    0 & 1 & 0 & 0 & 0 & 0 & 0 \\
    0 & 0 & 1 & 0 & 0 & 0 & 0 \\
    1 & 0 & 0 & 0 & 0 & 0 & 0 \\
  \end{bsmallmatrix}
  +
  \begin{bsmallmatrix}
    0 & 0 & 1 & 0 & 0 & 0 & 0 \\
    0 & 0 & 0 & 0 & 0 & 1 & 0 \\
    0 & 0 & 0 & 1 & 0 & 0 & 0 \\
    0 & 0 & 0 & 0 & 1 & 0 & 0 \\
    0 & 0 & 0 & 0 & 0 & 0 & 1 \\
    1 & 0 & 0 & 0 & 0 & 0 & 0 \\
    0 & 1 & 0 & 0 & 0 & 0 & 0 \\
  \end{bsmallmatrix}
\end{equation}

The permutations associated with these matrices are, in cycle notation:

\begin{equation}
\phi _1 = (1235674), \phi _2 = (1524637), \phi _3 = (1345726)
\end{equation}

If we choose to apply $\phi _1$ to the columns of blocks $R_{23}$ and $R{32}$ we get:

\begin{equation}
  \begin{bmatrix}
         &      0_{1} &   1_{1}    &    2_{1}    &    3_{1}   &    4_{1}    &    5_{1}   &    6_{1}   \\
   0_{1} &            &            &             & 2_{1}5_{1} &             & 1_{1}6_{1} & 3_{1}4_{1} \\
   1_{1} & 4_{1}5_{1} &            &             &            & 3_{1}6_{1}  &            & 2_{1}0_{1} \\
   2_{1} & 3_{1}1_{1} & 5_{1}6_{1} &             &            &             & 4_{1}0_{1} &            \\
   3_{1} &            & 4_{1}2_{1} &  6_{1}0_{1} &            &             &            & 5_{1}1_{1} \\ 
   4_{1} & 6_{1}2_{1} &            &  5_{1}3_{1} & 0_{1}1_{1} &             &            &            \\
   5_{1} &            & 0_{1}3_{1} &             & 6_{1}4_{1} & 1_{1}2_{1}  &            &            \\
   6_{1} &            &            &  1_{1}4_{1} &            & 0_{1}5_{1}  & 2_{1}3_{1} &            \\
   0_{1} & 2_{3}5_{2} &            &             & 3_{3}4_{2} &             &            & 1_{3}6_{2} \\
   1_{1} &            & 4_{3}5_{2} &             & 2_{3}0_{2} &             & 3_{3}6_{2} &            \\
   2_{1} &            & 3_{3}1_{2} &  5_{3}6_{2} &            &             &            & 4_{3}0_{2} \\
   3_{1} &            &            &  4_{3}2_{2} & 5_{3}1_{2} & 6_{3}0_{2}  &            &            \\
   4_{1} & 0_{3}1_{2} & 6_{3}2_{2} &             &            & 5_{3}3_{2}  &            &            \\
   5_{1} & 6_{3}4_{2} &            &  0_{3}3_{2} &            &             & 1_{3}2_{2} &            \\
   6_{1} &            &            &             &            & 1_{3}4_{2}  & 0_{3}5_{2} & 2_{3}3_{2} \\
   0_{1} & 2_{2}5_{3} &            &             & 3_{2}4_{3} &             &            & 1_{2}6_{3} \\
   1_{1} &            & 4_{2}5_{3} &             & 2_{2}0_{3} &             & 3_{2}6_{3} &            \\
   2_{1} &            & 3_{2}1_{3} &  5_{2}6_{3} &            &             &            & 4_{2}0_{3} \\
   3_{1} &            &            &  4_{2}2_{3} & 5_{2}1_{3} & 6_{2}0_{3}  &            &            \\
   4_{1} & 0_{2}1_{3} & 6_{2}2_{3} &             &            & 5_{2}3_{3}  &            &            \\
   5_{1} & 6_{2}4_{3} &            &  0_{2}3_{3} &            &             & 1_{2}2_{3} &            \\
   6_{1} &            &            &             &            & 1_{2}4_{3}  & 0_{2}5_{3} & 2_{2}3_{3} \\
  \end{bmatrix}
\end{equation}

Which leaves us free to put $\{\infty, x_1\}$ into $(x_1, x_1)$ and $\{x_2, x_3\}$ into $(x_1, (x\phi _1)_1)$ e.g. $\{1_2, 1_3\}$ can go into $(2_1, (2\phi _1)_1) = (2_1,3_1)$.
The permutation chosen ensures that cell $(2, 3)$ of the original square is empty.

Filling in the rest of block $R11$ gives:

\begin{equation}
  \begin{bmatrix}
         &      0_{1} &   1_{1}    &    2_{1}    &    3_{1}   &    4_{1}    &    5_{1}   &    6_{1}   \\
   0_{1} & \infty     0_{1} & 0_{2}0_{3} &             & 2_{1}5_{1} &             & 1_{1}6_{1} & 3_{1}4_{1} \\
   1_{1} & 4_{1}5_{1} & \infty     1_{1} &  1_{2}1_{3} &            & 3_{1}6_{1}  &            & 2_{1}0_{1} \\
   2_{1} & 3_{1}1_{1} & 5_{1}6_{1} &  \infty     2_{1} &            & 2_{2}2_{3}  & 4_{1}0_{1} &            \\
   3_{1} & 3_{2}3_{3} & 4_{1}2_{1} &  6_{1}0_{1} & \infty     3_{1} &             &            & 5_{1}1_{1} \\ 
   4_{1} & 6_{1}2_{1} &            &  5_{1}3_{1} & 0_{1}1_{1} & \infty     4_{1}  & 4_{2}4_{3} &            \\
   5_{1} &            & 0_{1}3_{1} &             & 6_{1}4_{1} & 1_{1}2_{1}  & \infty     5_{1} & 5_{2}5_{3} \\
   6_{1} &            &            &  1_{1}4_{1} & 6_{2}6_{3} & 0_{1}5_{1}  & 2_{1}3_{1} & \infty     6_{1} \\
  \end{bmatrix}
\end{equation}

Notice that this satisfies the row and column properties of a Room square for the first seven rows and columns.

Next we move onto the second diagonal block, because missing from row and column $x_2$ are the elements $\infty, x_1, x_2, x_3$.
However this time we try to find a home for pairs of the form $\{\infty, x_2\}$ and $\{x_1, x_3\}$.

We can put pairs of the form $\{x_1, x_3\}$ down the diagonal and permute the columns of block $R_{22}$ with a permutation from Lemma \ref{lem:permute} to ensure that column $(x\phi _2)_2$ has no $x_2$, allowing us to put $\{\infty, x_2\}$ in that column.

For instance, using $\phi _2$ we can complete block $R_{13}$, and the corresponding seven rows and columns, by putting:

$\{x_1, x_3\}$ in $x_2, x_2$ and $\{\infty, x_2\}$ in $x_2,(x \phi _2)_2$.

Taking the same approach with the third diagonal block we permute columns in $R_{33}$ using $\phi _3$ and fill-in by putting:

$\{x_1, x_2\}$ in $x_3, x_3$ and $\{\infty, x_3\}$ in $x_3, (x \phi _3)_3$

Which results in a Room square of side 21 based on:
$$S = \{\infty, 0_1, 1_1, \ldots, 6_1, 0_2, 1_2, \ldots, 6_2, 0_3, 1_3, \ldots, 6_3\}$$

This square is straightforwardly transformed to a Room square
\eqref{eq:roomtwentyone}
based on $\{\infty, 0, 1, \ldots, 20\}$ by using: $x_i = x + 7(i - 1)$

\begin{equation}
  \label{eq:roomtwentyone}
  \begin{bsmallmatrix}
    \infty,0 &   7,14   &      & 2,5 &     & 1,6 & 3,4 & 10,11 &       & 8,13 & &       & 9,12 & & 15,20 & 17,18 &       & & 16,19 & &       \\
      4,5    & \infty,1 & 8,15 &     & 3,6 &     & 2,0 &  9,7  & 10,13 &      & & 11,12 &      & &       & 16,14 & 18,19 & &       & & 17,20 \\
  \end{bsmallmatrix}
\end{equation}

We could have done this from the beginning, but it is perhaps simpler to keep track of the missing elements by maintaining the subscript notation.
\end{example}

The preceding triplication was slightly contrived because the arrangement of frames at the beginning was so chosen because it satisfied a property as yet unexplained.
This property being that the $R_{ij}$ and $R_{ji}$ occur in the same super-column of the array as frames.
Clearly this must be so in order that permutations, when applied to both or neither of $R_{ij}$ and $R_{ji}$, preserve the contents of the columns as far as is required.

That an arrangement of frames with this property can be guaranteed to exist for any odd integer is fundamental to the generalised theorem.

\begin{lemma}
For all odd $n$ there exists an array with these properties:
\begin{enumerate}
  \item{the entries of the array consist of all the ordered pairs of the set $N = \{1, 2, \ldots, n\}$ once each.}
  \item{the entries of a given row or column contain between them every member of $N$ once as a left member and once as a right member.}
  \item{if $(x,y)$ occurs in a given column of the array $(y,x)$ also occurs in that column.}
\end{enumerate}
\end{lemma}

\begin{proof}
$A_n$ is an $n \times n$ array whose $(i, j)$ entry is the
ordered pair $(j - i + 1, i + j - 1)$ with both elements being
reduced modulo $n$ to lie on the interval $[1, n]$.
\begin{enumerate}
  \item{There are clearly $n^2$ ordered pairs obtainable from
     $N$. $A_n$ has $n^2$ cells, so it is only necessary to
     show that each cell contains a unique pair. For that
     reason consider any two pairs from different cells,
     $(x_1, y_1)$ and $(x_2, y_2)$, for these to be equal
     requires both $x_1 = x_2$ and $y_1 = y_2$.

     Now, $x_1 = x_2$ implies $j_1-i_1 + 1 = j_2-i_2+1$
     While, $y_1 = y_2$ implies $i_1 + j_1 - 1 = i_2 + j_2 - 1$.

     Together these imply,
     \begin{align}
       j_1 - i_1 &= j_2 - i_2 \label{eq:eq1} \\
       i_1 + j_1 &= i_2 + j_2 \label{eq:eq2}
     \end{align}

     \eqref{eq:eq2} gives, $j_2 = i_1 + j_1 - i_2$,
     which on substitution in \eqref{eq:eq1} gives,
     $j_1 - i_1 = i_1 + j_1 - 2i_2$ which implies
     $2i_1 - 2i_2 = 0$.
     
     Therefore $i_1 = i_2$.
     Substituting this into either expression gives
     $j_1 = j_2$.
     Thereby
     contradicting the assumption that the pairs occurred
     in different cells. Hence every cell contains a unique
     pair, so all the ordered pairs from $N$ occur exactly
     once in $A_n$.}
  \item{Consider row $i$ of $A_n$:

    \begin{tabular}{ccccc}
          $j = 1$   &     $j = 2$      & \ldots & $j = - 1$     &     $j = 0$      \\ \hline
       $(2 - i, i)$ & $(3 - i, i + 1)$ & \ldots & $(-i, i - 2)$ & $(1 - i, i - 1)$ 
    \end{tabular}

     Left hand members are $\{2-i, 3-i, \ldots, 1-i\}$, while
     the right hand members are $\{i, i + 1, \ldots, i - 1\}$.
     Both sets
     contain $n$ unique integers on the interval $[1,n]$
     and hence both sets must be $N$, Similarly, consider
     column $j$, the left hand positions are occupied by
     $\{j, j - 1, \ldots, j + 2, j + 1\}$, while the right
     contain $\{j, j + 1, \ldots, j - 2, j - 1\}$.
     For the same reasons both these sets are equal to $N$.}
  \item{Consider a pair $(x,y)$. From the definition of $A_n$,
     $x = j-i + 1$, $y = i + j - 1$. So
     $x + y = j - i + 1 + i + j - 1 = 2j$.
     Therefore $j = \frac{1}{2}(x + y)$. So if $(x, y)$ is in
     column $j$ of $A_n$ then so is $(y, x)$, since
     $\frac{1}{2}(x + y) = \frac{1}{2}(y + x)$.}
\end{enumerate}
\end{proof}

We can now present the general result.

\begin{theorem}
\label{thm:wallis}
If $r$ and $n$ are odd integers such that $r \geq n$, and if there is a Room square $R$ of side $r$, then there is a Room square of side $rn$.
\end{theorem}

\begin{proof}
Let $r = 2d + 1$ and $n = 2t + 1$.

For a given $i$ select $n$ permutations as follows:

\begin{enumerate}
  \item{$\phi _{jk} = \phi _{jl}$ if, and only if, $(k, l)$, and $(l, k)$ appear in column $j$ of $A_n$.}
  \item{If cell $(j, j)$ of $A_n$ contains $(x, y)$ then $\phi _{jx} = \phi _{jy} = id$ (the identity permutation).}
  \item{All the $\phi _{jk} (\neq id)$ are selected from the permutations associated with $R$, according to Lemma \ref{lem:permute}.}
\end{enumerate}

Now a Room square of side $rn$ is constructed by replacing each entry $(k, l)$ of $A_n$ by $R_{kl} \phi _{jk}$ (the array $R_{kl}$ under the column permutation $\phi _{jk}$), where $(k, l)$ is in column $j$ of $A_n$.

The resulting array has each element of
\begin{equation}
S = \{0_1,1_1,...,(r-1)_1,0_2,1_2,...,(r-1)_2,0_n,1_n,...,(r-1)_n\}
\end{equation}
appearing exactly once in each row and column, except that $x_j$ is missing from row and column
$x_j,$ $1 \leq j \leq n 0 \leq x \leq (r-1)$, and $x_k$ and $x_l$ are missing from column $(x \phi _{jk})_j$ for each entry $(k,l)$ in column $j$ of $A_n$.

The array also contains every unordered pair from $S$ exactly once, except those of the form $\{x_k,x_l\}$.
Now, for each $k$, if $(k, l)$ is an entry of column $j$ of $A_n$ put $\{x_k, x_l\}$ in $(x_j, (x \phi _{jk})_j)$, using $\{\infty, x_j\}$ instead of $\{x_j, x_j\}$ in every case.

The completed array contains each of $\{\infty\} \cup S$ exactly once per row and column and every unordered pair from the same set exactly once.

Finally, map $\{\infty\} \cup S$ onto $\{\infty \} \cup Z_{rn}$ by replacing every $x_i$ by $x + r(i - 1)$.
The finished array is a Room square of side $rn$.
\end{proof}

\begin{example}
Looking back at the previous example of triplication.

\begin{equation}
 A = \begin{bmatrix}
  1,1 & 2,2 & 3,3 \\
  3,2 & 1,3 & 2,1 \\
  2,3 & 3,1 & 1,2 \\
 \end{bmatrix}
\end{equation}

Now, in column 1 $(3, 2)$ and $(2, 3)$ both appear, so $\phi _{13} = \phi _{12}$.
While in column 2 occur both (1,3) and (3,1), so $\phi _{21} = \phi _{23}$.
Furthermore, both $(2, 1)$ and $(1, 2)$ appear in the third column, so $\phi _{32} = \phi _{31}$.
The diagonal pairs are $(1, 1)$, $(1, 3)$, $(1, 2)$ so
\begin{equation}
\phi _{11} = \phi _{21} = \phi _{23} = \phi _{31} = \phi _{32} = id
\end{equation}

The remaining permutations, $\phi _{12}, \phi _{13}, \phi _{22}, \phi _{33}$ are chosen according to the Lemma \ref{lem:permute}
and the following array is the Room square in \ref{eq:roomtwentyone} once the missing pairs have been placed and the transformation to $\{\infty, 0, 1, ..., 20\}$ made.

\begin{equation}
  \begin{bmatrix}
   R_{11}\phi_{11} & R_{22}\phi_{22} & R_{33}\phi_{33} \\
   R_{32}\phi_{13} & R_{13}\phi_{21} & R_{21}\phi_{32} \\
   R_{23}\phi_{12} & R_{31}\phi_{23} & R_{12}\phi_{31} \\
  \end{bmatrix}
  =
  \begin{bmatrix}
      R_{11}id    & R_{22}\phi_{2} & R_{33}\phi_{3} \\
   R_{32}\phi_{1} &   R_{13}id     &    R_{21}id    \\
   R_{23}\phi_{1} &   R_{31}id     &    R_{12}id    \\
  \end{bmatrix}
\end{equation}

\end{example}
