% Options for packages loaded elsewhere
\PassOptionsToPackage{unicode}{hyperref}
\PassOptionsToPackage{hyphens}{url}
%
\documentclass[
  12pt,
  a4paper]{book}
\usepackage{lmodern}
\usepackage{amssymb,amsmath}
\usepackage{ifxetex,ifluatex}
\ifnum 0\ifxetex 1\fi\ifluatex 1\fi=0 % if pdftex
  \usepackage[T1]{fontenc}
  \usepackage[utf8]{inputenc}
  \usepackage{textcomp} % provide euro and other symbols
\else % if luatex or xetex
  \usepackage{unicode-math}
  \defaultfontfeatures{Scale=MatchLowercase}
  \defaultfontfeatures[\rmfamily]{Ligatures=TeX,Scale=1}
\fi
% Use upquote if available, for straight quotes in verbatim environments
\IfFileExists{upquote.sty}{\usepackage{upquote}}{}
\IfFileExists{microtype.sty}{% use microtype if available
  \usepackage[]{microtype}
  \UseMicrotypeSet[protrusion]{basicmath} % disable protrusion for tt fonts
}{}
\makeatletter
\@ifundefined{KOMAClassName}{% if non-KOMA class
  \IfFileExists{parskip.sty}{%
    \usepackage{parskip}
  }{% else
    \setlength{\parindent}{0pt}
    \setlength{\parskip}{6pt plus 2pt minus 1pt}}
}{% if KOMA class
  \KOMAoptions{parskip=half}}
\makeatother
\usepackage{xcolor}
\IfFileExists{xurl.sty}{\usepackage{xurl}}{} % add URL line breaks if available
\IfFileExists{bookmark.sty}{\usepackage{bookmark}}{\usepackage{hyperref}}
\hypersetup{
  pdftitle={Room Squares},
  pdfauthor={Matthew Henderson},
  hidelinks,
  pdfcreator={LaTeX via pandoc}}
\urlstyle{same} % disable monospaced font for URLs
\usepackage{longtable,booktabs}
% Correct order of tables after \paragraph or \subparagraph
\usepackage{etoolbox}
\makeatletter
\patchcmd\longtable{\par}{\if@noskipsec\mbox{}\fi\par}{}{}
\makeatother
% Allow footnotes in longtable head/foot
\IfFileExists{footnotehyper.sty}{\usepackage{footnotehyper}}{\usepackage{footnote}}
\makesavenoteenv{longtable}
\usepackage{graphicx,grffile}
\makeatletter
\def\maxwidth{\ifdim\Gin@nat@width>\linewidth\linewidth\else\Gin@nat@width\fi}
\def\maxheight{\ifdim\Gin@nat@height>\textheight\textheight\else\Gin@nat@height\fi}
\makeatother
% Scale images if necessary, so that they will not overflow the page
% margins by default, and it is still possible to overwrite the defaults
% using explicit options in \includegraphics[width, height, ...]{}
\setkeys{Gin}{width=\maxwidth,height=\maxheight,keepaspectratio}
% Set default figure placement to htbp
\makeatletter
\def\fps@figure{htbp}
\makeatother
\setlength{\emergencystretch}{3em} % prevent overfull lines
\providecommand{\tightlist}{%
  \setlength{\itemsep}{0pt}\setlength{\parskip}{0pt}}
\setcounter{secnumdepth}{5}
\newtheorem{theorem}{Theorem}
\newtheorem{example}{Example}
\usepackage[a4paper,margin=2cm]{geometry}
\usepackage{amsmath}

\title{Room Squares}
\author{Matthew Henderson}
\date{September 12, 2020}

\begin{document}
\maketitle

{
\setcounter{tocdepth}{1}
\tableofcontents
}
\hypertarget{introduction}{%
\chapter{Introduction}\label{introduction}}

\hypertarget{kirkmans-schoolgirl-problem}{%
\section{Kirkman's Schoolgirl Problem}\label{kirkmans-schoolgirl-problem}}

In 1850 Thomas Penyngton Kirkman, an English mathematician from Bolton,
published the following problem in the \emph{Lady's and Gentleman's Diary.}

\begin{quote}
Fifteen young ladies of a school walk out three abreast for seven
days in succession: it is required to arrange them daily so that no
two shall walk abreast more than once.
\end{quote}

In solving this problem Kirkman discovered the following square array,
which he observed was a very ``curious arrangement''.

\begin{longtable}[]{@{}lllllll@{}}
\caption{Kirkman's curious arrangement}\tabularnewline
\toprule
\endhead
\begin{minipage}[t]{0.06\columnwidth}\raggedright
\strut
\end{minipage} & \begin{minipage}[t]{0.06\columnwidth}\raggedright
\strut
\end{minipage} & \begin{minipage}[t]{0.06\columnwidth}\raggedright
\strut
\end{minipage} & \begin{minipage}[t]{0.06\columnwidth}\raggedright
hi\strut
\end{minipage} & \begin{minipage}[t]{0.06\columnwidth}\raggedright
kl\strut
\end{minipage} & \begin{minipage}[t]{0.06\columnwidth}\raggedright
mn\strut
\end{minipage} & \begin{minipage}[t]{0.06\columnwidth}\raggedright
op\strut
\end{minipage}\tabularnewline
\begin{minipage}[t]{0.06\columnwidth}\raggedright
\strut
\end{minipage} & \begin{minipage}[t]{0.06\columnwidth}\raggedright
il\strut
\end{minipage} & \begin{minipage}[t]{0.06\columnwidth}\raggedright
mo\strut
\end{minipage} & \begin{minipage}[t]{0.06\columnwidth}\raggedright
\strut
\end{minipage} & \begin{minipage}[t]{0.06\columnwidth}\raggedright
np\strut
\end{minipage} & \begin{minipage}[t]{0.06\columnwidth}\raggedright
hk\strut
\end{minipage} & \begin{minipage}[t]{0.06\columnwidth}\raggedright
\strut
\end{minipage}\tabularnewline
\begin{minipage}[t]{0.06\columnwidth}\raggedright
\strut
\end{minipage} & \begin{minipage}[t]{0.06\columnwidth}\raggedright
no\strut
\end{minipage} & \begin{minipage}[t]{0.06\columnwidth}\raggedright
hl\strut
\end{minipage} & \begin{minipage}[t]{0.06\columnwidth}\raggedright
mp\strut
\end{minipage} & \begin{minipage}[t]{0.06\columnwidth}\raggedright
\strut
\end{minipage} & \begin{minipage}[t]{0.06\columnwidth}\raggedright
\strut
\end{minipage} & \begin{minipage}[t]{0.06\columnwidth}\raggedright
ik\strut
\end{minipage}\tabularnewline
\begin{minipage}[t]{0.06\columnwidth}\raggedright
lp\strut
\end{minipage} & \begin{minipage}[t]{0.06\columnwidth}\raggedright
\strut
\end{minipage} & \begin{minipage}[t]{0.06\columnwidth}\raggedright
in\strut
\end{minipage} & \begin{minipage}[t]{0.06\columnwidth}\raggedright
ko\strut
\end{minipage} & \begin{minipage}[t]{0.06\columnwidth}\raggedright
hm\strut
\end{minipage} & \begin{minipage}[t]{0.06\columnwidth}\raggedright
\strut
\end{minipage} & \begin{minipage}[t]{0.06\columnwidth}\raggedright
\strut
\end{minipage}\tabularnewline
\begin{minipage}[t]{0.06\columnwidth}\raggedright
im\strut
\end{minipage} & \begin{minipage}[t]{0.06\columnwidth}\raggedright
\strut
\end{minipage} & \begin{minipage}[t]{0.06\columnwidth}\raggedright
kp\strut
\end{minipage} & \begin{minipage}[t]{0.06\columnwidth}\raggedright
\strut
\end{minipage} & \begin{minipage}[t]{0.06\columnwidth}\raggedright
\strut
\end{minipage} & \begin{minipage}[t]{0.06\columnwidth}\raggedright
lo\strut
\end{minipage} & \begin{minipage}[t]{0.06\columnwidth}\raggedright
hn\strut
\end{minipage}\tabularnewline
\begin{minipage}[t]{0.06\columnwidth}\raggedright
ho\strut
\end{minipage} & \begin{minipage}[t]{0.06\columnwidth}\raggedright
km\strut
\end{minipage} & \begin{minipage}[t]{0.06\columnwidth}\raggedright
\strut
\end{minipage} & \begin{minipage}[t]{0.06\columnwidth}\raggedright
ln\strut
\end{minipage} & \begin{minipage}[t]{0.06\columnwidth}\raggedright
\strut
\end{minipage} & \begin{minipage}[t]{0.06\columnwidth}\raggedright
ip\strut
\end{minipage} & \begin{minipage}[t]{0.06\columnwidth}\raggedright
\strut
\end{minipage}\tabularnewline
\begin{minipage}[t]{0.06\columnwidth}\raggedright
kn\strut
\end{minipage} & \begin{minipage}[t]{0.06\columnwidth}\raggedright
hp\strut
\end{minipage} & \begin{minipage}[t]{0.06\columnwidth}\raggedright
\strut
\end{minipage} & \begin{minipage}[t]{0.06\columnwidth}\raggedright
\strut
\end{minipage} & \begin{minipage}[t]{0.06\columnwidth}\raggedright
io\strut
\end{minipage} & \begin{minipage}[t]{0.06\columnwidth}\raggedright
\strut
\end{minipage} & \begin{minipage}[t]{0.06\columnwidth}\raggedright
lm\strut
\end{minipage}\tabularnewline
\bottomrule
\end{longtable}

The curiosity of this square is that each of the letters h, i, k, l, m, n, o, p
occurs precisely once in every column and row, while in the entire
square each of the letters makes a pair with every other letter exactly
once. Kirkman was able to employ this square to solve his Schoolgirl
Problem. To each pair in the first column he added the element 1, to
each pair in the second column 2 and so on. In addition he introduced
the missing triple of numbers to each row. (e.g.~row one has no elements
in any of the first three columns so the numbers 1,2 and 3 would not
appear hence he would add the triple 123). The seven rows of unique
triples then corresponded to seven days in which the elements,
corresponding to schoolgirls, were paired together exactly once
throughout the arrangement. Thereby solving the problem.

\begin{longtable}[]{@{}cccccc@{}}
\caption{\label{tab:kirkman-solution} Kirkman's Schoolgirl's Solution}\tabularnewline
\toprule
\endhead
\begin{minipage}[t]{0.09\columnwidth}\centering
Day 1\strut
\end{minipage} & \begin{minipage}[t]{0.07\columnwidth}\centering
123\strut
\end{minipage} & \begin{minipage}[t]{0.07\columnwidth}\centering
hi4\strut
\end{minipage} & \begin{minipage}[t]{0.07\columnwidth}\centering
kl5\strut
\end{minipage} & \begin{minipage}[t]{0.07\columnwidth}\centering
mn6\strut
\end{minipage} & \begin{minipage}[t]{0.07\columnwidth}\centering
op7\strut
\end{minipage}\tabularnewline
\begin{minipage}[t]{0.09\columnwidth}\centering
Day 2\strut
\end{minipage} & \begin{minipage}[t]{0.07\columnwidth}\centering
147\strut
\end{minipage} & \begin{minipage}[t]{0.07\columnwidth}\centering
il2\strut
\end{minipage} & \begin{minipage}[t]{0.07\columnwidth}\centering
mo3\strut
\end{minipage} & \begin{minipage}[t]{0.07\columnwidth}\centering
np5\strut
\end{minipage} & \begin{minipage}[t]{0.07\columnwidth}\centering
hk6\strut
\end{minipage}\tabularnewline
\begin{minipage}[t]{0.09\columnwidth}\centering
Day 3\strut
\end{minipage} & \begin{minipage}[t]{0.07\columnwidth}\centering
156\strut
\end{minipage} & \begin{minipage}[t]{0.07\columnwidth}\centering
no2\strut
\end{minipage} & \begin{minipage}[t]{0.07\columnwidth}\centering
hl3\strut
\end{minipage} & \begin{minipage}[t]{0.07\columnwidth}\centering
mp4\strut
\end{minipage} & \begin{minipage}[t]{0.07\columnwidth}\centering
ik7\strut
\end{minipage}\tabularnewline
\begin{minipage}[t]{0.09\columnwidth}\centering
Day 4\strut
\end{minipage} & \begin{minipage}[t]{0.07\columnwidth}\centering
267\strut
\end{minipage} & \begin{minipage}[t]{0.07\columnwidth}\centering
lo2\strut
\end{minipage} & \begin{minipage}[t]{0.07\columnwidth}\centering
in3\strut
\end{minipage} & \begin{minipage}[t]{0.07\columnwidth}\centering
ko4\strut
\end{minipage} & \begin{minipage}[t]{0.07\columnwidth}\centering
hm5\strut
\end{minipage}\tabularnewline
\begin{minipage}[t]{0.09\columnwidth}\centering
Day 5\strut
\end{minipage} & \begin{minipage}[t]{0.07\columnwidth}\centering
245\strut
\end{minipage} & \begin{minipage}[t]{0.07\columnwidth}\centering
io2\strut
\end{minipage} & \begin{minipage}[t]{0.07\columnwidth}\centering
kp3\strut
\end{minipage} & \begin{minipage}[t]{0.07\columnwidth}\centering
lo6\strut
\end{minipage} & \begin{minipage}[t]{0.07\columnwidth}\centering
hn7\strut
\end{minipage}\tabularnewline
\begin{minipage}[t]{0.09\columnwidth}\centering
Day 6\strut
\end{minipage} & \begin{minipage}[t]{0.07\columnwidth}\centering
357\strut
\end{minipage} & \begin{minipage}[t]{0.07\columnwidth}\centering
ho2\strut
\end{minipage} & \begin{minipage}[t]{0.07\columnwidth}\centering
km2\strut
\end{minipage} & \begin{minipage}[t]{0.07\columnwidth}\centering
ln4\strut
\end{minipage} & \begin{minipage}[t]{0.07\columnwidth}\centering
ip6\strut
\end{minipage}\tabularnewline
\begin{minipage}[t]{0.09\columnwidth}\centering
Day 7\strut
\end{minipage} & \begin{minipage}[t]{0.07\columnwidth}\centering
346\strut
\end{minipage} & \begin{minipage}[t]{0.07\columnwidth}\centering
ko2\strut
\end{minipage} & \begin{minipage}[t]{0.07\columnwidth}\centering
hp2\strut
\end{minipage} & \begin{minipage}[t]{0.07\columnwidth}\centering
io5\strut
\end{minipage} & \begin{minipage}[t]{0.07\columnwidth}\centering
lm7\strut
\end{minipage}\tabularnewline
\bottomrule
\end{longtable}

Kirkman was a notable mathematician who is often regarded as the
originator of the object in Table
\ref{tab:kirkman-solution},
which has subsequently become known as a Room square (after T.G. Room).

\hypertarget{tournaments}{%
\section{Tournaments}\label{tournaments}}

Suppose the English Football Association proposed hosting a new type of
international tournament to be staged as a one-off event in England.
This tournament would involve eight national sides competing in a league
that would be staged in various stadia around the country over two
weeks. The structure of the tournament would be such that every team
played every other team once, with the winner being the team which
accumulated most points in the manner of a normal football league (3
points for a win, 1 for a draw).

To know which matches need playing is simple. Suppose the eight invited
teams are:

\begin{longtable}[]{@{}cc@{}}
\caption{Teams}\tabularnewline
\toprule
\endhead
\begin{minipage}[t]{0.16\columnwidth}\centering
Argentina\strut
\end{minipage} & \begin{minipage}[t]{0.16\columnwidth}\centering
England\strut
\end{minipage}\tabularnewline
\begin{minipage}[t]{0.16\columnwidth}\centering
Brazil\strut
\end{minipage} & \begin{minipage}[t]{0.16\columnwidth}\centering
France\strut
\end{minipage}\tabularnewline
\begin{minipage}[t]{0.16\columnwidth}\centering
Columbia\strut
\end{minipage} & \begin{minipage}[t]{0.16\columnwidth}\centering
Germany\strut
\end{minipage}\tabularnewline
\begin{minipage}[t]{0.16\columnwidth}\centering
Denmark\strut
\end{minipage} & \begin{minipage}[t]{0.16\columnwidth}\centering
Holland\strut
\end{minipage}\tabularnewline
\bottomrule
\end{longtable}

If we write matches as alphabetic pairs in the obvious way, (e.g.~ab
denoting Argentina versus Brazil). The complete list of matches (the
match set, M) is simply all unordered pairs from team set, T:
\[T = \{a, b, c, d, e, f, g, h\}.\] i.e.~

\begin{equation*}
\begin{split}
M = \{
  ab,ac,ad,ae,af,ag,ah,bc,bd,be,bf,bg,bh,cd,\\
  ce,cf,cg,ch,de,df,dg,dh,ef,eg,eh,fg,fh,gh
\}
\end{split}
\end{equation*}

It remains to be decided where and when the matches will be played.

The English F.A., for whatever reason (the financial cost of hosting
eight teams, for example), has imposed a time limit of two weeks on the
tournament. Realistically the treams can only manage to play on
alternate days so it is decided to have, in effect, seven different
``rounds'' with each team competing once in each round. (Seven being
the smallest number of rounds because each team has to play seven
others).

For reasons of fairness the F.A. also demands the condition that each
team will play once at each stadium.
Can such a tournament exist?
Suppose the stadia used are the following:

\begin{longtable}[]{@{}cc@{}}
\caption{\label{tab:stadium}Stadium}\tabularnewline
\toprule
\endhead
\begin{minipage}[t]{0.05\columnwidth}\centering
1\strut
\end{minipage} & \begin{minipage}[t]{0.25\columnwidth}\centering
Wembley\strut
\end{minipage}\tabularnewline
\begin{minipage}[t]{0.05\columnwidth}\centering
2\strut
\end{minipage} & \begin{minipage}[t]{0.25\columnwidth}\centering
Highbury\strut
\end{minipage}\tabularnewline
\begin{minipage}[t]{0.05\columnwidth}\centering
3\strut
\end{minipage} & \begin{minipage}[t]{0.25\columnwidth}\centering
Villa Park\strut
\end{minipage}\tabularnewline
\begin{minipage}[t]{0.05\columnwidth}\centering
4\strut
\end{minipage} & \begin{minipage}[t]{0.25\columnwidth}\centering
Stadium of Light\strut
\end{minipage}\tabularnewline
\begin{minipage}[t]{0.05\columnwidth}\centering
5\strut
\end{minipage} & \begin{minipage}[t]{0.25\columnwidth}\centering
Stamford Bridge\strut
\end{minipage}\tabularnewline
\begin{minipage}[t]{0.05\columnwidth}\centering
6\strut
\end{minipage} & \begin{minipage}[t]{0.25\columnwidth}\centering
Old Trafford\strut
\end{minipage}\tabularnewline
\begin{minipage}[t]{0.05\columnwidth}\centering
7\strut
\end{minipage} & \begin{minipage}[t]{0.25\columnwidth}\centering
St.~James Park\strut
\end{minipage}\tabularnewline
\bottomrule
\end{longtable}

Table
\ref{tab:fixtures}
provides a match schedule which is suitable for
such a tournament.

\begin{longtable}[]{@{}cccccccc@{}}
\caption{\label{tab:fixtures} Fixture List for an International Soccer League}\tabularnewline
\toprule
\begin{minipage}[b]{0.10\columnwidth}\centering
~\strut
\end{minipage} & \begin{minipage}[b]{0.05\columnwidth}\centering
1\strut
\end{minipage} & \begin{minipage}[b]{0.05\columnwidth}\centering
2\strut
\end{minipage} & \begin{minipage}[b]{0.05\columnwidth}\centering
3\strut
\end{minipage} & \begin{minipage}[b]{0.05\columnwidth}\centering
4\strut
\end{minipage} & \begin{minipage}[b]{0.05\columnwidth}\centering
5\strut
\end{minipage} & \begin{minipage}[b]{0.05\columnwidth}\centering
6\strut
\end{minipage} & \begin{minipage}[b]{0.05\columnwidth}\centering
7\strut
\end{minipage}\tabularnewline
\midrule
\endfirsthead
\toprule
\begin{minipage}[b]{0.10\columnwidth}\centering
~\strut
\end{minipage} & \begin{minipage}[b]{0.05\columnwidth}\centering
1\strut
\end{minipage} & \begin{minipage}[b]{0.05\columnwidth}\centering
2\strut
\end{minipage} & \begin{minipage}[b]{0.05\columnwidth}\centering
3\strut
\end{minipage} & \begin{minipage}[b]{0.05\columnwidth}\centering
4\strut
\end{minipage} & \begin{minipage}[b]{0.05\columnwidth}\centering
5\strut
\end{minipage} & \begin{minipage}[b]{0.05\columnwidth}\centering
6\strut
\end{minipage} & \begin{minipage}[b]{0.05\columnwidth}\centering
7\strut
\end{minipage}\tabularnewline
\midrule
\endhead
\begin{minipage}[t]{0.10\columnwidth}\centering
\textbf{1}\strut
\end{minipage} & \begin{minipage}[t]{0.05\columnwidth}\centering
\strut
\end{minipage} & \begin{minipage}[t]{0.05\columnwidth}\centering
\strut
\end{minipage} & \begin{minipage}[t]{0.05\columnwidth}\centering
\strut
\end{minipage} & \begin{minipage}[t]{0.05\columnwidth}\centering
ab\strut
\end{minipage} & \begin{minipage}[t]{0.05\columnwidth}\centering
cd\strut
\end{minipage} & \begin{minipage}[t]{0.05\columnwidth}\centering
ef\strut
\end{minipage} & \begin{minipage}[t]{0.05\columnwidth}\centering
gh\strut
\end{minipage}\tabularnewline
\begin{minipage}[t]{0.10\columnwidth}\centering
\textbf{2}\strut
\end{minipage} & \begin{minipage}[t]{0.05\columnwidth}\centering
\strut
\end{minipage} & \begin{minipage}[t]{0.05\columnwidth}\centering
bd\strut
\end{minipage} & \begin{minipage}[t]{0.05\columnwidth}\centering
eg\strut
\end{minipage} & \begin{minipage}[t]{0.05\columnwidth}\centering
\strut
\end{minipage} & \begin{minipage}[t]{0.05\columnwidth}\centering
fh\strut
\end{minipage} & \begin{minipage}[t]{0.05\columnwidth}\centering
ah\strut
\end{minipage} & \begin{minipage}[t]{0.05\columnwidth}\centering
\strut
\end{minipage}\tabularnewline
\begin{minipage}[t]{0.10\columnwidth}\centering
\textbf{3}\strut
\end{minipage} & \begin{minipage}[t]{0.05\columnwidth}\centering
\strut
\end{minipage} & \begin{minipage}[t]{0.05\columnwidth}\centering
fg\strut
\end{minipage} & \begin{minipage}[t]{0.05\columnwidth}\centering
ad\strut
\end{minipage} & \begin{minipage}[t]{0.05\columnwidth}\centering
eh\strut
\end{minipage} & \begin{minipage}[t]{0.05\columnwidth}\centering
\strut
\end{minipage} & \begin{minipage}[t]{0.05\columnwidth}\centering
\strut
\end{minipage} & \begin{minipage}[t]{0.05\columnwidth}\centering
bc\strut
\end{minipage}\tabularnewline
\begin{minipage}[t]{0.10\columnwidth}\centering
\textbf{4}\strut
\end{minipage} & \begin{minipage}[t]{0.05\columnwidth}\centering
dh\strut
\end{minipage} & \begin{minipage}[t]{0.05\columnwidth}\centering
\strut
\end{minipage} & \begin{minipage}[t]{0.05\columnwidth}\centering
bf\strut
\end{minipage} & \begin{minipage}[t]{0.05\columnwidth}\centering
cg\strut
\end{minipage} & \begin{minipage}[t]{0.05\columnwidth}\centering
ae\strut
\end{minipage} & \begin{minipage}[t]{0.05\columnwidth}\centering
\strut
\end{minipage} & \begin{minipage}[t]{0.05\columnwidth}\centering
\strut
\end{minipage}\tabularnewline
\begin{minipage}[t]{0.10\columnwidth}\centering
\textbf{5}\strut
\end{minipage} & \begin{minipage}[t]{0.05\columnwidth}\centering
be\strut
\end{minipage} & \begin{minipage}[t]{0.05\columnwidth}\centering
\strut
\end{minipage} & \begin{minipage}[t]{0.05\columnwidth}\centering
ch\strut
\end{minipage} & \begin{minipage}[t]{0.05\columnwidth}\centering
\strut
\end{minipage} & \begin{minipage}[t]{0.05\columnwidth}\centering
\strut
\end{minipage} & \begin{minipage}[t]{0.05\columnwidth}\centering
bg\strut
\end{minipage} & \begin{minipage}[t]{0.05\columnwidth}\centering
af\strut
\end{minipage}\tabularnewline
\begin{minipage}[t]{0.10\columnwidth}\centering
\textbf{6}\strut
\end{minipage} & \begin{minipage}[t]{0.05\columnwidth}\centering
ag\strut
\end{minipage} & \begin{minipage}[t]{0.05\columnwidth}\centering
ce\strut
\end{minipage} & \begin{minipage}[t]{0.05\columnwidth}\centering
\strut
\end{minipage} & \begin{minipage}[t]{0.05\columnwidth}\centering
df\strut
\end{minipage} & \begin{minipage}[t]{0.05\columnwidth}\centering
\strut
\end{minipage} & \begin{minipage}[t]{0.05\columnwidth}\centering
bh\strut
\end{minipage} & \begin{minipage}[t]{0.05\columnwidth}\centering
\strut
\end{minipage}\tabularnewline
\begin{minipage}[t]{0.10\columnwidth}\centering
\textbf{7}\strut
\end{minipage} & \begin{minipage}[t]{0.05\columnwidth}\centering
cf\strut
\end{minipage} & \begin{minipage}[t]{0.05\columnwidth}\centering
ah\strut
\end{minipage} & \begin{minipage}[t]{0.05\columnwidth}\centering
\strut
\end{minipage} & \begin{minipage}[t]{0.05\columnwidth}\centering
\strut
\end{minipage} & \begin{minipage}[t]{0.05\columnwidth}\centering
bg\strut
\end{minipage} & \begin{minipage}[t]{0.05\columnwidth}\centering
\strut
\end{minipage} & \begin{minipage}[t]{0.05\columnwidth}\centering
de\strut
\end{minipage}\tabularnewline
\bottomrule
\end{longtable}

Looking along the rows, each team plays once in each round. Looking down
columns, each stadia hosts each team exactly once. And throughout the
tournament as a whole each pair from the original match list appears
exactly once, hence every team opposes every other team once. Table
\ref{tab:fixtures}
is another Room square of side 7. Alternatively, because the pairs are
made from a set containing 8 elements, we say that this is a Room square
of order 8.

\hypertarget{t.g.-room-1902-86}{%
\section{T.G. Room, (1902-86)}\label{t.g.-room-1902-86}}

In 1955, Thomas Gerald Room, then Professor of Mathematics at the
University of Sydney, published a brief note in the Mathematical Gazette
entitled \emph{A new type of magic square}
(Room 1955).
In it he presented another
example of a square array with the same properties as Kirkman's. This
square, Room explained, had been discovered as ``a by-product of
another investigation''. It was preceded in the note by a particularly
efficient statement of the properties of these squares, which have
subsequently been known by his name.

\begin{quote}
The problem is to arrange the \(n(2n-1)\) symbols \(rs\) (which is the
same as \(sr\)) formed from all pairs of \(2n\) digits such that in each
row and each column there appear \(n\) symbols (and \(n-1\) blanks) which
among them contain all \(2n\) digits.
\end{quote}

Room's note went on to explain that while the trivial \(n = 1\) Room square
exists\footnote{The Room squre of side 1 is just the single array element
  containing the pair \(\{0,1\}\).} , the non-existence of those with \(n = 2\) (side 2) and \(n = 3\)
(side 5) is easily proven. Room considered the \(n = 2\) proof so
straightforward that it was omitted from this note, while for the \(n = 3\)
case he made reference to a graph-theoretic proof.

Consider the
\(n = 2\)
case.
We are required to place
all pairs from a set of
four digits into a
\(3 \times 3\)
array.
If we choose to use
the set of non-negative integers
\(\{0, 1, 2, 3\}\),
then we need to
find somewhere to put
each of the pairs
\(\{01, 02, 03, 12, 13, 23\}\).
That we can swap
rows and columns
of a Room square
without damaging that square's Room-ness
is self-evident.
Therefore,
there is no loss of generality
in assuming that a
\(3 \times 3\)
Room square has the pair
\(\{0, 1\}\)
in cell
\((1, 1)\).

\begin{longtable}[]{@{}ccc@{}}
\toprule
\endhead
\begin{minipage}[t]{0.06\columnwidth}\centering
01\strut
\end{minipage} & \begin{minipage}[t]{0.05\columnwidth}\centering
-\strut
\end{minipage} & \begin{minipage}[t]{0.05\columnwidth}\centering
-\strut
\end{minipage}\tabularnewline
\bottomrule
\end{longtable}

\begin{center}\rule{0.5\linewidth}{0.5pt}\end{center}

\begin{center}\rule{0.5\linewidth}{0.5pt}\end{center}

Table: \label{tab:threebythree}

If we hope to make this array a Room square we must place the pair
\(\{2, 3\}\) in the first row, while to complete the first column we must
also place the same pair in either position \((1, 2)\) or \((1, 3)\), but each
pair is only allowed to appear once. So there is no Room square of side
3, order 4.

For the \(n = 3\), a Room square of side 5, case we consider the following
array:

\begin{longtable}[]{@{}ccccc@{}}
\toprule
\endhead
\begin{minipage}[t]{0.06\columnwidth}\centering
01\strut
\end{minipage} & \begin{minipage}[t]{0.06\columnwidth}\centering
23\strut
\end{minipage} & \begin{minipage}[t]{0.06\columnwidth}\centering
45\strut
\end{minipage} & \begin{minipage}[t]{0.05\columnwidth}\centering
-\strut
\end{minipage} & \begin{minipage}[t]{0.05\columnwidth}\centering
-\strut
\end{minipage}\tabularnewline
\begin{minipage}[t]{0.06\columnwidth}\centering
24\strut
\end{minipage} & \begin{minipage}[t]{0.06\columnwidth}\centering
\strut
\end{minipage} & \begin{minipage}[t]{0.06\columnwidth}\centering
\strut
\end{minipage} & \begin{minipage}[t]{0.05\columnwidth}\centering
\strut
\end{minipage} & \begin{minipage}[t]{0.05\columnwidth}\centering
\strut
\end{minipage}\tabularnewline
\begin{minipage}[t]{0.06\columnwidth}\centering
35\strut
\end{minipage} & \begin{minipage}[t]{0.06\columnwidth}\centering
\strut
\end{minipage} & \begin{minipage}[t]{0.06\columnwidth}\centering
\strut
\end{minipage} & \begin{minipage}[t]{0.05\columnwidth}\centering
\strut
\end{minipage} & \begin{minipage}[t]{0.05\columnwidth}\centering
\strut
\end{minipage}\tabularnewline
\bottomrule
\end{longtable}

\begin{itemize}
\item
  \begin{center}\rule{0.5\linewidth}{\linethickness}\end{center}
\end{itemize}

Table: \label{tab:fivebyfive}

There is no loss in generality in using this array because we can
reorder rows and columns to obtain the first row in this form and then
the first column must contain either the pairs \(\{01, 24, 35\}\) or
\(\{01, 25, 34\}\) and the latter can be converted into the former by the
permutation \((45)\)\footnote{This is cycle notation, and stands for the permutation
  \(4 \rightarrow 5, 5 \rightarrow 4\).} which leaves the first row unchanged. We now show
that completion of this square is impossible. The pairs
\(\{2,5\},\{3,4\}\) must appear somewhere in the array other than the
first three rows or columns. Also they must appear in separate
rows/columns to prevent a forced recurrence of \(\{0, 1\}\). Suppose we put
\(\{2, 5\}\) in \((4, 4)\) and \(\{3, 4\}\) in \((5, 5)\), then we know that cells
\((4, 5)\) and \((5, 4)\) are empty, as the only pair which could legally go
in either would be \(\{0, 1\}\).

Hence we know that cells \((4, 2), (4, 3), (5, 2), (5, 3)\) each contain pairs.
Take cell \((5, 2)\), it could only contain \(\{0, 5\}\) or \(\{1, 5\}\), and the
latter becomes the former under \((01)\), so assume it contains \(\{0, 5\}\).
We are now forced to fill in the other cells to give the array in Figure
8.

\begin{longtable}[]{@{}ccccc@{}}
\caption{\label{tab:nofivebyfive}}\tabularnewline
\toprule
\endhead
\begin{minipage}[t]{0.06\columnwidth}\centering
01\strut
\end{minipage} & \begin{minipage}[t]{0.06\columnwidth}\centering
23\strut
\end{minipage} & \begin{minipage}[t]{0.06\columnwidth}\centering
45\strut
\end{minipage} & \begin{minipage}[t]{0.06\columnwidth}\centering
-\strut
\end{minipage} & \begin{minipage}[t]{0.06\columnwidth}\centering
-\strut
\end{minipage}\tabularnewline
\begin{minipage}[t]{0.06\columnwidth}\centering
24\strut
\end{minipage} & \begin{minipage}[t]{0.06\columnwidth}\centering
-\strut
\end{minipage} & \begin{minipage}[t]{0.06\columnwidth}\centering
-\strut
\end{minipage} & \begin{minipage}[t]{0.06\columnwidth}\centering
-\strut
\end{minipage} & \begin{minipage}[t]{0.06\columnwidth}\centering
-\strut
\end{minipage}\tabularnewline
\begin{minipage}[t]{0.06\columnwidth}\centering
35\strut
\end{minipage} & \begin{minipage}[t]{0.06\columnwidth}\centering
-\strut
\end{minipage} & \begin{minipage}[t]{0.06\columnwidth}\centering
-\strut
\end{minipage} & \begin{minipage}[t]{0.06\columnwidth}\centering
-\strut
\end{minipage} & \begin{minipage}[t]{0.06\columnwidth}\centering
-\strut
\end{minipage}\tabularnewline
\begin{minipage}[t]{0.06\columnwidth}\centering
-\strut
\end{minipage} & \begin{minipage}[t]{0.06\columnwidth}\centering
14\strut
\end{minipage} & \begin{minipage}[t]{0.06\columnwidth}\centering
03\strut
\end{minipage} & \begin{minipage}[t]{0.06\columnwidth}\centering
25\strut
\end{minipage} & \begin{minipage}[t]{0.06\columnwidth}\centering
-\strut
\end{minipage}\tabularnewline
\begin{minipage}[t]{0.06\columnwidth}\centering
-\strut
\end{minipage} & \begin{minipage}[t]{0.06\columnwidth}\centering
05\strut
\end{minipage} & \begin{minipage}[t]{0.06\columnwidth}\centering
12\strut
\end{minipage} & \begin{minipage}[t]{0.06\columnwidth}\centering
-\strut
\end{minipage} & \begin{minipage}[t]{0.06\columnwidth}\centering
34\strut
\end{minipage}\tabularnewline
\bottomrule
\end{longtable}

We still need to place the pairs \(\{0, 2\}\) and \(\{0, 4\}\), which cannot
be done because neither can appear in the second row and they cannot
both appear in the third row. Hence there is no Room square of side 5,
order 6.

The real significance of Room's note was that mathematicians soon took
on the task of determining the spectra of Room squares (those values of
\(n\) for which Room squares exist). Research which cumulated 19 years
later in the complete statement of the existence of Room squares, made
by W.D. Wallis, that:

\begin{quote}
\emph{``Room squares exist for all odd positive integer sides except 3
and 5''} Wallis (1974)
\end{quote}

Proving this statement, which was suspected to be true from an early
stage, turned out to be protracted and difficult.

The most significant breakthrough came in 1968 when Stanton and Mullin
introduced the starter-adder method for constructing Room squares. This
method reduces the problem of constructing Room squares to the problem
of finding a certain type of initial row from which a Room square can be
developed straightforwardly.

In this work much emphasis will be placed upon the proof of the
existence of Room squares.

\hypertarget{the-galois-field}{%
\section{The Galois Field}\label{the-galois-field}}

Throughout this work much use will be made of a particular \emph{finite
field}, known as the Galois field, denoted by \(GF(p^n)\). Whenever \(p^n\)
is a prime (i.e.~\(n=1\)) the Galois field is precisely the integers under
modulo p arithmetic, denoted \(Z_p\). The Galois field has a number of
important properties which are used in many of the proofs that follow,
we introduce some of these now.

\begin{itemize}
\tightlist
\item
  Every Galois field (every finite field in fact) has a \emph{primitive
  element}. An element, \(x\) say, is primitive in \(GF(q)\) if
  \(x^0,x^1,x^2,...,x^{q-1}\) are all the non-zero members of \(GF(q)\).
\end{itemize}

\hypertarget{example}{%
\subsection{Example}\label{example}}

\(x=2\) is a primitive element in \(GF(11)\) because,
\[x^0 = 1 \hspace{0.5cm} x^1 = 2 \hspace{0.5cm} x^2 = 4 \hspace{0.5cm} x^3 = 8 \hspace{0.5cm} x^4 = 5\]
\[x^5 = 10 \hspace{0.5cm} x^6 = 9 \hspace{0.5cm} x^7 = 7 \hspace{0.5cm} x^8 = 3 \hspace{0.5cm} x^9 = 6\]

\begin{itemize}
\item
  It can be shown {[}3{]} that \(x^{q-1}=1\) is always true for any \(GF(q)\)
  where \(q\) is odd, and \(x^i \neq 1\) for any \(1 \leq i \leq q-1\)
\item
  \(x^{q-1}=1\) implies that
  \((x^{\frac{1}{2}(q-1)}-1)(x^{\frac{1}{2}(q-1)}+1)=0\), therefore
  either \(x^{\frac{1}{2}(q-1)}=1\) or \(x^{\frac{1}{2}(q-1)}=-1\).
  Clearly because of the previous remark, only the latter can be true.
\item
  If \(b\) is a non-zero residue modulo \(p\), then \(b\) is a quadratic
  residue (or square) if \(x^2 \equiv b(\textrm{mod } p)\) has
  solutions, otherwise \(b\) is a quadratic non-residue (or non-square).
  So the non-zero squares are precisely the even powers of the
  primitive element, while the non-zero non-squares are the odd
  powers.
\item
  There are precisely \(\frac{1}{2}(p-1)\) squares mod \(p\), and
  \(\frac{1}{2}(p-1)\) non-squares.
\item
  \(-1\) is a square if \(q \equiv 1(\textrm{mod } 4)\), but not a square
  for \(q \equiv 3(\textrm{mod } 4)\)

  \begin{itemize}
  \item
    \(q \equiv 1(\textrm{mod } 4)\), then if \(x^i\) is a square so is
    \(-x^i\).
  \item
    \(q \equiv 3(\textrm{mod } 4)\), then \(x^i\) is a square \(-x^i\) is
    a non-square.
  \end{itemize}
\end{itemize}

\hypertarget{a-graph-theoretic-approach-to-constructing-room-squares}{%
\chapter{A Graph-Theoretic Approach to Constructing Room Squares}\label{a-graph-theoretic-approach-to-constructing-room-squares}}

\hypertarget{graph-factorisations}{%
\section{Graph factorisations}\label{graph-factorisations}}

A graph \(G(V,E)\) consists of two sets. The first \(V\), is called the
vertex-set, while the other \(E\) consists of unordered pairs of \(V\) and
is called the edge set. Usually graphs are represented with diagrams
where the members of \(V\) are drawn as points and the members of \(E\) as
lines connecting points. Adjacency for two vertices means being
connected by an edge. The \emph{complete graph} \(K_n\) is the graph on \(n\)
vertices in which all distinct vertices are adjacent.

\begin{figure}
\centering
\includegraphics{figure/complete-graph-1.pdf}
\caption{\label{fig:complete-graph}\(K_4\) and \(K_5\)}
\end{figure}

A \emph{one-factor} \(f_i\) is a set of edges in which each vertex appears
exactly once.

\begin{example}
Two possible one-factors of
$K_4$
are:
$$f_1 = \{12,34\},\, f_2 = \{13,24\}$$
\end{example}

A \emph{one-factorisation} of the complete graph is a set of one-factors
in which all possible edges (i.e.~all unordered pairs from the edge-set)
appear exactly once.

\begin{figure}
\centering
\includegraphics{figure/K6-1.pdf}
\caption{\label{fig:K6}\(K_6\)}
\end{figure}

\begin{example}
Here
$G = K_6$
the complete graph on six vertices with
$$V = \{1, 2, 3, 4, 5, 6\}$$
$$E = \{12, 13, 14, 15, 16, 23, 24, 25, 26, 34, 35, 36, 45, 46, 56\}$$
The one-factors are

$$
f_1 = \{12, 35, 46\} \hspace{0.5cm}
f_2 = \{14, 23, 56\} \hspace{0.5cm}
f_3 = \{16, 25, 34\} \hspace{0.5cm}
f_4 = \{13, 26, 45\} \hspace{0.5cm} 
f_5 = \{15, 24, 36\}
$$
because
$f_1 \cup f_2 \cup f_3 \cup f_4 \cup f_5 = E$,
$F = \{f_1, f_2, f_3, f_4, f_5\}$
is a one-factorisation of
$G$,
shown in Figure~\ref{fig:one-factorisation}.
\end{example}

\begin{figure}
\centering
\includegraphics{figure/one-factorisation-1.pdf}
\caption{\label{fig:one-factorisation}One-factorisation of \(K_6\)}
\end{figure}

Two one factors \(f\) and \(l\) are said to be \emph{orthogonal} if
\(f \cap l\) contains at most one edge. Two one-factorisations \(F\) and \(L\)
are orthogonal if every one-factor in \(F\) is orthogonal to every
one-factor in \(L\).

Once again consider the square array in Figure 2. If the individual
elements within the array constituted the vertex set of a graph (call it
\(R\)) and the pairs within each box of the array were edges, we know that
each row is a one-factor and each column is a one-factor (because each
member of \(R\) occurs precisely once in each row and once in each
column). Further more, because all edges from the edge-set of the
complete graph (i.e.~all unordered pairs from \(R\)) appear once within
the array, we know that the rows together form a one-factorisation and
the columns form another, different, one-factorisation of \(K_8\). Also,
because any row factor intersects any column factor in only one pair
(edge), all the row factors are orthogonal to all the column factors and
hence the two one-factorisations are orthogonal. We have demonstrated
the following theorem, given in {[}8{]} and proven in {[}19{]}.

\begin{theorem}
The existence of a Room square of side
$n$
is equivalent to the existence of two orthogonal
one-factorisations of the complete graph
$K_{n+1}$.
\end{theorem}

An example is given in figure 12 based on the Room square in Figure 2.

\textbf{Figure 12 Two orthogonal one-factorisations of \(K_8\) based on
Kirkman's square of 1850.}

\hypertarget{hill-climbing-algorithm-for-room-squares}{%
\section{Hill-climbing algorithm for Room squares}\label{hill-climbing-algorithm-for-room-squares}}

The idea behind Hill-climbing algorithms is to suppose there exists a
\textbf{neighbourhood} of feasible solutions to some problem \textbf{instance}. With
each \textbf{feasible} solution there is an associated \textbf{cost} (or profit) and
finding an optimal solution becomes a matter of finding the solution with
minimum cost (or maximum profit).

A hill-climbing algorithm non-deterministically selects a solution from
the neighbourhood system such that the cost is less than that of some
initial solution until its procedure fails, hence finding the locally
optimal solution.

\hypertarget{an-algorithm-for-one-factorisations}{%
\subsection{An algorithm for One-Factorisations}\label{an-algorithm-for-one-factorisations}}

Consider how to find a one-factorisation of the complete graph. Here the
problem instance is simply the even integer \(n\) and vertex set \(V\).\\
Recall:

\begin{itemize}
\item
  A one-factor of \(K_n\) is a set of \(n/2\) edges (hence \(n\) is even)
  which partition \(V\).
\item
  A one-factorisation of \(K_n\) is a set of \(n-1\) one-factors which
  partitions the edge set of \(K_n\).
\end{itemize}

Suppose we choose to represent a one-factorisation by a set of
\(\frac{n}{2}(n-1)=(n^2-n)/2\) pairs each of the form \((f_i,\{x,y\})\),
where \(x \neq y, i=1...n-1\), and the following two conditions hold.

\begin{enumerate}
\def\labelenumi{\arabic{enumi}.}
\item
  Every \(\{x,y\}\) occurs in a unique pair \((f_i,\{x,y\})\).
\item
  For every one-factor \(f_i\) and every vertex \(x\), there is a unique
  pair of the form \((f_i,\{x,y\})\).
\end{enumerate}

where \(f_i\)s are one-factors.

Then we consider a feasible solution to be a partial one-factorisation,
again represented by pairs having the same form but this time,

\begin{enumerate}
\def\labelenumi{\arabic{enumi}.}
\item
  Every \(\{x,y\}\) occurs in at most one pair \(f_i,\{x,y\})\).
\item
  For every one-factor \(f_i\) and every vertex \(x\) there is at most one
  pair of the form \((f_i,\{x,y\})\).
\end{enumerate}

Where the \(f_i\)s are \textbf{partial one-factors}.

Which enables a definition for the cost of a feasible solution \(F\) to be
given by: \[c(F) = (n^2-n)/2-|F|\] So that \(F\) is a one-factorisation if
and only if \(c(F)=0\), i.e.~\(|F|=(n^2-n)/2\)

Now suppose that we can implement some procedure \(X\), say, which either
reduces the cost or leaves it unaffected (i.e.~it never increases the
cost) then the following ``hill-climbing'' algorithm, provided it
terminates, will find a one-factorisation.

\begin{longtable}[]{@{}ll@{}}
\toprule
\endhead
While \(c(F) \neq 0\) &\tabularnewline
Do \(X\) &\tabularnewline
\bottomrule
\end{longtable}

A procedure such as \(X\) is called a heuristic. The following two
heuristics (due to Dinitz \& Stinson {[}9{]}) when used together are suitable
for finding a one-factorisation.

Let \(F\) be a partial one-factorisation of \(K_n\):

Heuristic \(H_1\) {[}8{]}

\begin{enumerate}
\def\labelenumi{\arabic{enumi}.}
\item
  Choose any vertex \(x\) such that \(x\) does not occur in every partial
  one-factor of \(F\) (such a vertex is said to be a \textbf{live point}).
\item
  Choose any partial one-factor \(f_i\) such that \(x\) does not occur in
  \(f_i\).
\item
  Choose any \(y \neq x\) such that there is no partial one-factor \(f_j\)
  for which \((f_j,\{x,y\}) \in F\) (we say that \(x\) and \(y\) \emph{do not
  occur together}).
\item
  \textbf{if} \(y\) does not occur in \(f_i\), \textbf{then}
\item
  \(\hspace{1cm}\) Replace \(F\) with \(F \cup \{(f_i,\{x,y\})\}\).
\item
  \textbf{Else} there is a pair in \(F\) of the form
  \((f_i,\{z,y\}) \hspace{0.5cm} (z \neq x)\)
\item
  \(\hspace{1cm}\) Replace \(F\) with \(F \cup \{(f_i,\{x,y\})\} \backslash  \{(f_i,\{z,y\})\}\).
\end{enumerate}

Heutistic \(H_2\) {[}8{]}

\begin{enumerate}
\def\labelenumi{\arabic{enumi}.}
\item
  Choose any partial one-factor \(f_i\) which does not occur in exactly
  \(n/2\) pairs in \(F\) (such a partial one-factor is said to be
  \textbf{\emph{live}}).
\item
  Choose any \(x\) and \(y\) such that \(x\) and \(y\) do not occur together
  in \(f_i\).
\item
  \textbf{if} \(x\) and \(y\) do not occur together, \textbf{then}
\item
  \(\hspace{1cm}\) Replace \(F\) with \(F \cup \{(f_i,\{x,y\})\}\).
\item
  \textbf{Else} there is a pair in \(F\) of the form
  \((f_j,\{x,y\}) \hspace{0.5cm} (j \neq i)\)
\item
  \(\hspace{1cm}\) Replace \(F\) with \(F \cup \{(f_i,\{x,y\})\} \backslash  \{(f_j,\{x,y\})\}\).
\end{enumerate}

\hypertarget{example-1}{%
\subsubsection{Example}\label{example-1}}

Suppose we are in the process of trying to find a one-factorisation for
\(K_6\), and have generated a partial one-factorisation represented by the
set \(F\).
\[F=\{(f_1,\{4,6\}),(f_1,\{3,5\}),(f_2,\{5,6\}),(f_3\{1,6\}), (f_3\{3,4\}),(f_4,\{2,3\}),(f_4,\{4,5\})\}\]
Now apply \(H_1\):

\begin{enumerate}
\def\labelenumi{\arabic{enumi}.}
\item
  Choose \(x=2\). Live, because it doesn't appear in \(f_1,f_2,f_3\) or
  \(f_5\).
\item
  Of these four partial one factors, choose \(f_1\).
\item
  2 only occurs together with 3 (in \(f_4\)), so pick \(y=5\).
\item
  5 already appears in \(f_1\) so \(\{z,y\}=\{3,5\}\). So replace \(F\) by
  \(F \cup \{(f_1,\{2,5\}) \backslash (f_1,\{3,5\})\}\)
\end{enumerate}

So we have extracted one edge from the one-factorisation and replaced it
with another edge, leaving the cost unchanged.
If in 3. we had picked 1 then according to the heuristic we should
replace \(F\) with \(F \cup (f_1,\{2,1\})\), increasing \(|F|\) by one, and so
decreasing the cost by the same. Because the cost cannot increase \(H_1\)
is a suitable heuristic for use in a hill-climbing algorithm.

Now apply \(H_2\) to the new one-factorisation \(F_1=F \cup (f_1,\{2,1\})\)

\begin{enumerate}
\def\labelenumi{\arabic{enumi}.}
\item
  We can pick any of \(f_2, f_3, f_4, f_5\), because all are live.
  Choose \(f_2\).
\item
  Choose \(x=2, y=3\), because neither appear in \(f_2\).
\item
  2 and 3 occur together in \(f_4\). So replace \(F_1\) with
  \(F_1 \cup \{(f_2,\{2,3\}) \backslash (f_4,\{2,3\})\}\)
\end{enumerate}

Again the cost remains unchanged by this procedure, and if in 2. we had
chosen \(x=1,y=4\) instead then we would have replaced \(F_1\) with
\(F_1 \cup \{(f_2,\{1,4\})\}\) decreasing the cost by one. As with \(H_1\),
the cost cannot increase, which makes \(H_2\) a suitable heuristic.
The hill-climbing algorithm for constructing one-factorisations which
was first given in {[}9{]} has a very simple form.

\begin{enumerate}
\def\labelenumi{\arabic{enumi}.}
\item
  \textbf{While} \(c(F) \neq 0\), \textbf{do}
\item
  choose \(r=1\) or \(r=2\) with equal probability
\item
  perform \(H_r\)
\end{enumerate}

\hypertarget{an-algorithm-for-room-squares}{%
\subsection{An Algorithm for Room Squares}\label{an-algorithm-for-room-squares}}

To generate a Room square all that remains is to produce another
one-factorisation \(G\), say, which is orthogonal to \(F\). This will
inevitably require slight modifications to be made to \(H_1\) and \(H_2\).
Now if an array \(R\) is constructed in which the rows are labelled with
the one-factors of \(F(f_1,f_2,...,f_{n-1})\), and the columns are
labelled with the partial one-factors of \(G(g_1,g_2,...,g_{n-1})\). Then
\(R\) will be a Room square if the \((f_i,g_j)\) cell contains \(\{x,y\}\), if
and only if \((f_i,\{x,y\}) \in F\) and \((g_j,\{x,y\}) \in G\) and is empty
otherwise.

Again these two heuristics are due to Dinitz \& Stinson and originally
presented in {[}9{]}. Although a necessary correction has been made as will
become apparent.

\(OH_1\)

\begin{enumerate}
\def\labelenumi{\arabic{enumi}.}
\item
  Choose any live point \(x\).
\item
  Choose any partial one-factor \(g_i\) such that \(x\) does not occur in
  \(g_i\).
\item
  Choose any \(y \neq x\) such that \(x\) and \(y\) do not occur together in
  \(G\).
\item
  Let \(f_j\) be the one-factor of \(F\) which contains the edge
  \(\{x,y\}\).
\item
  \textbf{if} \(R(f_j,g_i)\) is not empty \textbf{then}
\item
  \(OH_1\) fails.
\item
  \textbf{Else if} \(y\) does not occur in \(g_i\), \textbf{then}
\item
  \(\hspace{1cm}\) Replace \(G\) by \(G \cup (g_i,\{x,y\})\).
\item
  \(\hspace{1cm}\) Define \(R(f_j,g_i)=\{x,y\}\).
\item
  \textbf{Else} there is a pair in \(G\) of the form
  \((g_i,\{z,y\}) \hspace{0.5cm} z \neq x\).
\item
  \(\hspace{1cm}\) Replace \(G\) by
  \(G \cup (g_i,\{x,y\}) \backslash (g_i,\{z,y\})\).
\item
  \(\hspace{1cm}\) Define \(R(f_k,g_i)\), to be empty\(^i\), where
  \((f_k,\{z,y\}) \in F\).
\end{enumerate}

\(OH_2\)

\begin{enumerate}
\def\labelenumi{\arabic{enumi}.}
\item
  Choose any live partial one-factor \(g_i\).
\item
  Choose any \(x\) and \(y \neq x\) such that \(x\) and \(y\) do not occur
  together in \(g_i\).
\item
  Let \(f_j\) be the one-factor of \(F\) which contains the edge
  \(\{x,y\}\).
\item
  \textbf{if} \(R(f_j,g_i)\) is not empty \textbf{then}
\item
  \(OH_2\) fails.
\item
  \textbf{Else if} \(x\) and \(y\) do not occur together, \textbf{then}
\item
  \(\hspace{1cm}\) Replace \(G\) by \(G \cup (g_i,\{x,y\})\).
\item
  \(\hspace{1cm}\) Define \(R(f_j,g_i)=\{x,y\}\).
\item
  \textbf{Else} there is a pair in \(G\) of the form
  \((g_k,\{x,y\}) \hspace{0.5cm} (k \neq i\))
\item
  \(\hspace{1cm}\) Replace \(G\) by
  \(G \cup (g_i,\{x,y\}) \backslash (g_k,\{x,y\})\)
\item
  \(\hspace{1cm}\) Define \(R(f_j,g_i)=\{x,y\}\)
\item
  \(\hspace{1cm}\) Define \(R(f_j,g_k)\) to be empty
\end{enumerate}

\hypertarget{example-2.2.1}{%
\subsubsection{\texorpdfstring{\emph{Example 2.2.1}}{Example 2.2.1}}\label{example-2.2.1}}

Suppose the factorisation \(F\) from the earlier example has been
completed and is represented by the set:

\$\$ M=\{

\begin{longtable}[]{@{}l@{}}
\toprule
\endhead
\((f_1\{1,2\}),(f_1\{3,5\}),(f_1\{4,6\}),(f_2\{1,4\}),(f_2\{2,3\}),(f_2\{5,6\}),(f_3\{1,6\})\)\tabularnewline
\((f_3\{2,5\}),(f_3\{3,4\}),(f_4\{1,3\}),(f_4\{2,6\}),(f_4\{4,5\}),(f_5\{1,5\}),(f_5\{3,6\})\)\tabularnewline
\bottomrule
\end{longtable}

\}

Notice that this is precisely the one-factorisation of \(K_6\) given
on page 9.

Now suppose we have established the following one-factors in \(G\):
\[G=\{(g_1,\{1,4\}),(g_2,\{1,6\}),(g_3,\{3,6\}),(g_5\{5,6\}),(g_5\{1,2\})\}\]
At this state \(R\) looks like

\$\$R=

\begin{longtable}[]{@{}lccccc@{}}
\toprule
& \(g_1\) & \(g_2\) & \(g_3\) & \(g_4\) & \(g_5\)\tabularnewline
\midrule
\endhead
\(f_1\) & & & & & \(1,2\)\tabularnewline
\(f_2\) & \(1,4\) & & & & \(5,6\)\tabularnewline
\(f_3\) & & \(1,6\) & & &\tabularnewline
\(f_4\) & & & & &\tabularnewline
\(f_5\) & & & \(3,6\) & &\tabularnewline
\bottomrule
\end{longtable}

\$\$

\textbf{Figure 13}

Now apply \(OH_1\):

\begin{enumerate}
\def\labelenumi{\arabic{enumi}.}
\item
  Choose \(x=5\), suitably live.
\item
  Choose \(g_3\), in which \(5\) does not occur.
\item
  \(5\) does not occur together with \(2\) in \(G\), so we are free to
  choose \(y=2\).
\item
  In \(F\), \(\{2.5\} \in f_3\).
\item
  \(f_3,g_3\) is empty in \(R\), also \(y=2 \notin g_3\).
\item
  Replace \(G\) with \(G \cup (g_3,\{5,2\})\).
\item
  Define \(R(f_3,g_3)=\{5,2\}\).
\end{enumerate}

This decreases the cost by one, alternatively we might have chosen, at
stage 3. \(y=3\), in that case.

\begin{enumerate}
\def\labelenumi{\arabic{enumi}.}
\setcounter{enumi}{3}
\item
  \(\{3,5\} \in f_1\).
\item
  \(f_1,g_3\) is empty in \(R\), also \(y \in g_3\), occurring in the pair
  \((g_3,\{3,6\}), z=6\).
\item
  Replace \(G\) with \(G \cup (g_3,\{3,5\}) \backslash (g_3,\{3,6\})\).
\item
  Define \(R(f_1,g_3)=\{3,5\}\).
\item
  Define \(R(f_5,g_3)\) to be empty.
\end{enumerate}

Which leaves the cost unaffected. Suppose now that \(R\) is the array
after this second version of the application of \(OH_1\):

\$\$R=

\begin{longtable}[]{@{}lccccc@{}}
\toprule
& \(g_1\) & \(g_2\) & \(g_3\) & \(g_4\) & \(g_5\)\tabularnewline
\midrule
\endhead
\(f_1\) & & & \(3,5\) & & \(1,2\)\tabularnewline
\(f_2\) & \(1,4\) & & & & \(5,6\)\tabularnewline
\(f_3\) & & \(1,6\) & & &\tabularnewline
\(f_4\) & & & & &\tabularnewline
\(f_5\) & & & & &\tabularnewline
\bottomrule
\end{longtable}

\$\$

\textbf{Figure 14}

Now if we apply \(OH_2\):

\begin{enumerate}
\def\labelenumi{\arabic{enumi}.}
\item
  Choose \(g_4\), a live partial one-factor.
\item
  Choose \(x=1, y=2\), neither of which occur in \(g_4\).
\item
  \((f_1,\{1,2\}) \in F\).
\item
  \(f_1,g_4\) is empty in \(R\), also \(x\) and \(y\) do occur together,
  \((g_5,\{1,2\}) \in G\).
\item
  Replace \(G\) with \(G \cup (g_4,\{1,2\}) \backslash (g_5,\{1,2\})\)
\item
  Define \(R(f_1,g_4)=\{1,2\}\)
\item
  Define \(R(f_1,g_5)\) to be empty.
\end{enumerate}

This procedure leaves the cost unaffected and if instead we had chosen
at \(2\). \(x=3,y=4\), then would have been required to replace \(G\) with
\(G \cup (g_4,\{3,4\})\), and put \(\{3,4\}\) in cell \((f_3,g_4)\) of \(R\), an
action which reduces the cost by one. However, we know that two
orthogonal one-factorisations of \(K_6\) are equivalent to a Room square
of side 5, which has been shown not to exist. Hence it would be futile
to continue with this method in this particular case. Nevertheless the
example shows how the heuristics work.

There is no guarantee of success with repeated use of these heuristics,
although Dinitz \& Stinson are quick to point out that the algorithm
involving \(H_1\) and \(H_2\) has never\footnote{In over ten-million attempts, they claim} failed to produce the desired
one-factorisation. If we hope to use the \(OH_1\) and \(OH_2\) in a similar
algorithm then the possibility of failure becomes a real possibility.
Two possibilities exist, either both heuristics fail or successive use
of them leads to an infinite loop. In order to avoid both we introduce a
\textbf{\emph{threshold}} function, which simply arrests the progress of the
algorithm after a certain number of iterations of the heuristics. Dinitz
\& Stinson found after experimentation that the following function is
suitable. \[T(n)=100n\] Then the hill-climbing algorithm for finding a
Room square is as follows {[}8{]}:

\begin{enumerate}
\def\labelenumi{\arabic{enumi}.}
\item
  Use the previous hill climbing algorithm to construct \(F\), a
  one-factorisation of \(K_n\).
\item
  Number of iterations initialised to be \(0\)
\item
  While (number of iterations \(<T(n)\)) and \(c(G) \neq 0\), do
\item
  \(\hspace{1cm}\) Choose \(r=1\) or \(r=2\) at random with equal
  probability
\item
  \(\hspace{1cm}\) Perform \(OH_r\)
\item
  \(\hspace{1cm}\) Increment number of iterations
\end{enumerate}

\hypertarget{the-room-square-generator}{%
\subsection{The Room Square Generator}\label{the-room-square-generator}}

Dinitz and Stinson choose to implement the above algorithm in Pascal,
and ran in on an Amdahl 5850 workstation. It was very successful,
finding many Room squares with sides ranging from 11 to 101. For each
successful trial they had 9 or 10 failures (the program being stopped by
the threshold function) and timings ranged from 0.09 seconds for an
11x11 Room square, to 7.3 on average for the 25 different 101x101 Room
squares they found.

I chose to implement the hill-climbing algorithms in Visual Basic 6.0 on
a Pentium III-450/Win 98 Desktop. Needless to say, it was slightly less
successful -- exhibiting a similar probability of success but
unfortunately becoming very slow for Room squares bigger than 21. It
found square of side 21 after an all-night search, but after 48 hours
looking for one of 23x23 I decided to call the search off.

Despite the failures at higher order, the Room square generator was very
successful in finding smaller squares. It found 7x7 Room squares in as
little as 4 seconds, and even 15x15 squares only took a few minutes.

Annotated code for the Room square generator can be found along with
some of the larger squares in Appendix I and below is a screen shot of
the application having successfully located a 9x9 Room square in a
little over one minute after 507 iterations of the heuristics \(OH_1\) and
\(OH_2\). The uppermost panel represents some of the one-factorisation
generated by the algorithm involving \(H_1\) and \(H_2\), while the second
panel shows part of the orthogonal one-factorisation generated by \(OH_1\)
and \(OH_2\). The lower panel is a Room square of side 9.

\textbf{Figure 15 Screenshot of the Room Square Generator}

\hypertarget{proving-the-existence-of-room-squares}{%
\chapter{Proving the Existence of Room Squares}\label{proving-the-existence-of-room-squares}}

The theorem which will ultimately be established in section 4 relies
upon a fundamental theorem in number theory -- in fact \textbf{the}
fundamental theorem. The Fundamental Theorem of Arithmetic states that
every positive integer, except 1, can be expressed uniquely as a product
of primes.

Proof of this theorem can be found in {[}11, section 2.10{]}.

The proof which established the existence of Room squares will rely upon
various other theorems which collectively establish the existence of all
Room squares with prime side, except 3 and 5. Then multiplication
theorems will be developed to establish the existence of composite Room
squares (those whose side is the product of two or more primes). Clearly
if the prime Room squares can be proven to exist, and hence composite
Room squares, the fundamental theorem will allow us to state that all
Room squares exist with odd positive integer side. Apart from a few
exceptional cases, this is basically what we wil be able to do.

\hypertarget{starters-adders-and-cyclic-room-squares}{%
\section{Starters, adders and cyclic Room squares}\label{starters-adders-and-cyclic-room-squares}}

\begin{longtable}[]{@{}cccccccc@{}}
\toprule
\begin{minipage}[b]{0.10\columnwidth}\centering
~\strut
\end{minipage} & \begin{minipage}[b]{0.05\columnwidth}\centering
0\strut
\end{minipage} & \begin{minipage}[b]{0.05\columnwidth}\centering
1\strut
\end{minipage} & \begin{minipage}[b]{0.05\columnwidth}\centering
2\strut
\end{minipage} & \begin{minipage}[b]{0.05\columnwidth}\centering
3\strut
\end{minipage} & \begin{minipage}[b]{0.05\columnwidth}\centering
4\strut
\end{minipage} & \begin{minipage}[b]{0.05\columnwidth}\centering
5\strut
\end{minipage} & \begin{minipage}[b]{0.05\columnwidth}\centering
6\strut
\end{minipage}\tabularnewline
\midrule
\endhead
\begin{minipage}[t]{0.10\columnwidth}\centering
\textbf{0}\strut
\end{minipage} & \begin{minipage}[t]{0.05\columnwidth}\centering
∞0\strut
\end{minipage} & \begin{minipage}[t]{0.05\columnwidth}\centering
\strut
\end{minipage} & \begin{minipage}[t]{0.05\columnwidth}\centering
\strut
\end{minipage} & \begin{minipage}[t]{0.05\columnwidth}\centering
25\strut
\end{minipage} & \begin{minipage}[t]{0.05\columnwidth}\centering
\strut
\end{minipage} & \begin{minipage}[t]{0.05\columnwidth}\centering
16\strut
\end{minipage} & \begin{minipage}[t]{0.05\columnwidth}\centering
34\strut
\end{minipage}\tabularnewline
\begin{minipage}[t]{0.10\columnwidth}\centering
\textbf{1}\strut
\end{minipage} & \begin{minipage}[t]{0.05\columnwidth}\centering
45\strut
\end{minipage} & \begin{minipage}[t]{0.05\columnwidth}\centering
∞1\strut
\end{minipage} & \begin{minipage}[t]{0.05\columnwidth}\centering
\strut
\end{minipage} & \begin{minipage}[t]{0.05\columnwidth}\centering
\strut
\end{minipage} & \begin{minipage}[t]{0.05\columnwidth}\centering
36\strut
\end{minipage} & \begin{minipage}[t]{0.05\columnwidth}\centering
\strut
\end{minipage} & \begin{minipage}[t]{0.05\columnwidth}\centering
20\strut
\end{minipage}\tabularnewline
\begin{minipage}[t]{0.10\columnwidth}\centering
\textbf{2}\strut
\end{minipage} & \begin{minipage}[t]{0.05\columnwidth}\centering
31\strut
\end{minipage} & \begin{minipage}[t]{0.05\columnwidth}\centering
56\strut
\end{minipage} & \begin{minipage}[t]{0.05\columnwidth}\centering
∞2\strut
\end{minipage} & \begin{minipage}[t]{0.05\columnwidth}\centering
\strut
\end{minipage} & \begin{minipage}[t]{0.05\columnwidth}\centering
\strut
\end{minipage} & \begin{minipage}[t]{0.05\columnwidth}\centering
40\strut
\end{minipage} & \begin{minipage}[t]{0.05\columnwidth}\centering
\strut
\end{minipage}\tabularnewline
\begin{minipage}[t]{0.10\columnwidth}\centering
\textbf{3}\strut
\end{minipage} & \begin{minipage}[t]{0.05\columnwidth}\centering
\strut
\end{minipage} & \begin{minipage}[t]{0.05\columnwidth}\centering
42\strut
\end{minipage} & \begin{minipage}[t]{0.05\columnwidth}\centering
60\strut
\end{minipage} & \begin{minipage}[t]{0.05\columnwidth}\centering
∞3\strut
\end{minipage} & \begin{minipage}[t]{0.05\columnwidth}\centering
\strut
\end{minipage} & \begin{minipage}[t]{0.05\columnwidth}\centering
\strut
\end{minipage} & \begin{minipage}[t]{0.05\columnwidth}\centering
51\strut
\end{minipage}\tabularnewline
\begin{minipage}[t]{0.10\columnwidth}\centering
\textbf{4}\strut
\end{minipage} & \begin{minipage}[t]{0.05\columnwidth}\centering
62\strut
\end{minipage} & \begin{minipage}[t]{0.05\columnwidth}\centering
\strut
\end{minipage} & \begin{minipage}[t]{0.05\columnwidth}\centering
53\strut
\end{minipage} & \begin{minipage}[t]{0.05\columnwidth}\centering
01\strut
\end{minipage} & \begin{minipage}[t]{0.05\columnwidth}\centering
∞4\strut
\end{minipage} & \begin{minipage}[t]{0.05\columnwidth}\centering
\strut
\end{minipage} & \begin{minipage}[t]{0.05\columnwidth}\centering
\strut
\end{minipage}\tabularnewline
\begin{minipage}[t]{0.10\columnwidth}\centering
\textbf{5}\strut
\end{minipage} & \begin{minipage}[t]{0.05\columnwidth}\centering
\strut
\end{minipage} & \begin{minipage}[t]{0.05\columnwidth}\centering
03\strut
\end{minipage} & \begin{minipage}[t]{0.05\columnwidth}\centering
\strut
\end{minipage} & \begin{minipage}[t]{0.05\columnwidth}\centering
64\strut
\end{minipage} & \begin{minipage}[t]{0.05\columnwidth}\centering
12\strut
\end{minipage} & \begin{minipage}[t]{0.05\columnwidth}\centering
∞5\strut
\end{minipage} & \begin{minipage}[t]{0.05\columnwidth}\centering
\strut
\end{minipage}\tabularnewline
\begin{minipage}[t]{0.10\columnwidth}\centering
\textbf{6}\strut
\end{minipage} & \begin{minipage}[t]{0.05\columnwidth}\centering
\strut
\end{minipage} & \begin{minipage}[t]{0.05\columnwidth}\centering
\strut
\end{minipage} & \begin{minipage}[t]{0.05\columnwidth}\centering
14\strut
\end{minipage} & \begin{minipage}[t]{0.05\columnwidth}\centering
\strut
\end{minipage} & \begin{minipage}[t]{0.05\columnwidth}\centering
05\strut
\end{minipage} & \begin{minipage}[t]{0.05\columnwidth}\centering
23\strut
\end{minipage} & \begin{minipage}[t]{0.05\columnwidth}\centering
∞6\strut
\end{minipage}\tabularnewline
\bottomrule
\end{longtable}

\textbf{Figure 16 Cyclic Room square}

The Room square in Figure 16 has a special property. The pairs in any
element of the array are obtained by simply adding 1 (mod 7) to the pair
in the element immediately above and to the left; along with the
condition that \[\infty+1 = \infty\] This special property means that
the entire square can be determined by the pairs in the first row, with
successive rows being developed in a cyclical manner according the
simple addition rule. We call squares like the one in Figure 16 \emph{cyclic}
Room squares.\\
~\\
Also notice that \(\{\infty,i\}\) occurs in position \((i,i)\). A square
with this property is said to be \emph{standardised}. It is important to
realise that any Room square can be standardised. As mentioned
previously neither interchanging the rows or columns nor permuting the
symbol-set on which the Room square is based has any effect of the
``Room"-ness of that square.\\
~\\
The significance of cyclic Room squares is that the problem of
constructing a Room square is (potentially) reduced to that of finding
an appropriate first row. These rows cannot be chosen arbitrarily, both
the pairs used and the positions in which they appear need to satisfy
certain criteria, but when they do exist a corresponding Room square
always exists. So proving the existence of this subclass of Room squares
is a matter only of proving the existence of these special first rows.

\hypertarget{finding-a-starter}{%
\subsection{Finding a starter}\label{finding-a-starter}}

Suppose we wish to construct another Room square of the same size as
Figure 16 based on the same symbols. This new square will also be
standardized so we need only determine the three pairs that accompany
\(\{0,\infty\}\) in the first row (the starter), and the positions they
occupy.\\
~\\
The set we will use to build our starter will be \{1,2,\ldots,6\}.\\
~\\
Each member of this set must occur exactly once in the pairs of the
starter -- in order to satisfy the row condition for a Room square.
Because of the cyclical construction the condition is automatically true
for successive rows if true for the first.\\
~\\
Consider the existence in Figure 16 of an arbitrary pair \(\{a,b\}\). We
know one of the following must be true.\\
~\\
Either:
\[\{2+i,5+i\}=\{a,b\} \hspace{0.5cm} \mathrm{or} \hspace{0.5cm} \{1+i,6+i\}=\{a,b\} \hspace{0.5cm} 
\mathrm{or} \hspace{0.5cm} \{3+i,4+i\}=\{a,b\} \hspace{0.5cm} \mathrm{for} \hspace{0.1cm} i=0,1,2,...,6\]
Say \(a-b=1\). Then \(\{2+i,5+i\}=\{a,b\}\) could never be true because
\((2+i)-(5+i)=-3(\)mod \$ 7)=4\$ and \((5+i)-(2+i)=3\). Similarly, the
differences in \(\{1,6\}\) are \(\pm5\) so \(\{a,b\}\) couldn't be generated
from \(\{1,6\}\).\\
~\\
However, \((4+i)-(3+i)=1\) so \(\{a,b\}\) will inevitably be generated by
\(\{3,4\}\) for some value of \(i=0,1,...,6\).\\
e.g.~\(\{2,3\}=\{3+6,4+6\}\)\\
~\\
Because \(a\) and \(b\) separately take on all values from
\(\{0,1,2,...,6\}\), their differences will similarly take on all these
values (except 0 because there are no pairs of the form \(\{a,a\}\)) and
so an essential property for the starter must be that the six
differences generated by its three pairs contain all of
\(\{1,2,...,6\}\).\\
~\\
When a starter satisfies this property, and the condition that the pairs
contain in their union all of \(\{1,2,...,6\}\), it is clear that it will
inevitably generate the correct pairs which populate a 7x7 Room square.
There are three pairs in the starter, each generates seven unique pairs
under cyclical construction, which along with the seven pairs generated
by \(\{0,\infty\}\) counts for all the 28 unordered pairs from
\(\{\infty,0,1,...,6\}\).\\
A starter for larger Room squares of course has to obey the same
criterion. We include a general definition based on {[}8{]}:\\
(1,0){450}\\
\emph{Definition:} If \(G\) is an additive Abelian group of order \(g\), then a
\emph{starter} in \(G\) is a set of unordered pairs:
\[S=\{\{s_i,t_i\}:1 \leq i \leq (g-1)/2\}\] which satisfies these
properties:

\begin{enumerate}
\def\labelenumi{\arabic{enumi}.}
\item
  \(\{s_i:1 \leq i \leq (g-1)/2\} \cup \{t_i : 1 \leq i \leq (g-1)/2\} = G \backslash \{0\}\)
\item
  \(\{\pm (s_i - t_i ) : 1 \leq i \leq (g-1)/2 \} = G \backslash \{0\}\)
\end{enumerate}

(1,0){450}\\
~\\
Whenever we have any \(t\) sets \(D_1,...,D_t\) each of size \(k\) in which
each non-zero member of an additive abelian group can be represented as
a difference between members of the \(D_i \lambda\) times, we say those
sets form a \textbf{\emph{difference system}}.\\
Much use will be made of difference systems throughout this work.\\
Notice that the definition of a starter presumes standardization, and
therefore that \(\{\infty,i\}\) is in position \((i,i)\).\\
The following pairs form a starter in \(G=\{0,1,2,...,6\}\) (an additive
abelian group with order \(g=7\).)
\[\{1,3\} \hspace{1cm} \{2,6\} \hspace{1cm} \{4,5\}\] Property 1 is
satisfied because
\(\{1,3\} \cup \{2,6\} \cup \{4,5\} = G \backslash \{0\}\)\\
Property 2 is also satisfied because
\[\{1-3=5,3-1=2,2-6=3,6-2=4,4-5=6,5-4=1\}=\{1,2,3,4,5,6\}=G\backslash \{0\}\]
Hence

\begin{longtable}[]{@{}cccc@{}}
\toprule
\(\infty 0\) & 13 & 26 & 45\tabularnewline
\midrule
\endhead
\(\infty 1\) & 24 & 30 & 56\tabularnewline
\(\infty 2\) & 35 & 41 & 60\tabularnewline
\(\infty 3\) & 46 & 52 & 01\tabularnewline
\(\infty 4\) & 50 & 63 & 12\tabularnewline
\(\infty 5\) & 61 & 04 & 23\tabularnewline
\(\infty 6\) & 02 & 15 & 34\tabularnewline
\bottomrule
\end{longtable}

\textbf{Table 1}

are all the unordered pairs from \(\{\infty,0,1,...6\}\) sorted into seven
rows that contain each of \(\{\infty,0,1,...,6\}\) exactly once. ALl that
remains is to determine the columns.

\hypertarget{finding-an-adder}{%
\subsection{Finding an adder}\label{finding-an-adder}}

In constructing the starter we made use of the fact that each row has to
contain each symbol exactly once and all unordered pairs from the symbol
set have to occur exactly once in the whole array. The remaining
condition -- namely, that each symbol must occur once in each column -- is
now employed to finish the construction.\\
~\\
Again, because of the cyclical nature of Room squares generated from
starters we can be sure that if one column contains each member of the
symbol set, all columns will.\\
~\\
Also, because we have decided to construct a standardized Room Square we
know that column \(i\) contains \(\{\infty,i\}\). So the final column
(column 6) contains \(\{\infty,6\}\), and depending on where we place the
starter pairs it will also include:
\[\{1,3\}+x \hspace{1cm} \{2,6\}+y \hspace{1cm} \{4,5\}+z\] For some
distinct values of \(x,y\) and \(z\) (only one pair allowed per box).\\
Considering that the new pairs to form column 6 must contain in their
union each of \(\{0,1,2,...,5\}\) we build the following table.

\begin{longtable}[]{@{}cccccc@{}}
\toprule
\(x\) & \(13+x\) & \(y\) & \(26+y\) & \(z\) & \(45+z\)\tabularnewline
\midrule
\endhead
\(0\) & 13 & 0 & 26 & 0 & 45\tabularnewline
\(1\) & 24 & 1 & 30 & 1 & 56\tabularnewline
\(2\) & 35 & 2 & 41 & 2 & 60\tabularnewline
\(3\) & 46 & 3 & 52 & 3 & 01\tabularnewline
\(4\) & 50 & 4 & 63 & 4 & 12\tabularnewline
\(5\) & 61 & 5 & 04 & 5 & 23\tabularnewline
\bottomrule
\end{longtable}

\textbf{Table 2}

Our task is simply to determine three unique values for \(x,y\) and \(z\)
such that \(13+x,26+y\) and \(45+z\) contain in their union each of
\(\{0,1,2,...,5\}\). These values will then determine the positions to
place 13, 26 and 45 in row 1.\\
~\\
Choosing 4 from the first column corresponds to having 50 appear in the
final column of the Room Square and forces the selection of \(y=2\) from
the next column of the table, (41 being the only pair not containing any
of the already used 5,6 or 0). 23 is the only possible choice from the
final column, accompanied by a value of \(z=5\). These three numbers are
known as an \emph{adder} corresponding to the starter 13,26,45. This is not
necessarily the only adder.\\
~\\
If 50 is to be generated in the final column of the Room square by the
pair 13 in the first row, then 13 must go in column \(7-4=3\). Similarly
26 has to be put in column \(7-2=5\) and 45 in \(7-5=2\).\\
We can now construct our cyclic room square.

\begin{longtable}[]{@{}cccccccl@{}}
\toprule
\(\infty 0\) & 45 & 13 & - & 26 & - & - &\tabularnewline
\midrule
\endhead
- & \(\infty 1\) & 56 & 24 & - & 30 & - &\tabularnewline
- & - & \(\infty 2\) & 60 & 35 & - & 41 &\tabularnewline
52 & - & - & \(\infty 3\) & 01 & 46 & - &\tabularnewline
- & 63 & - & - & \(\infty 4\) & 12 & 50 &\tabularnewline
61 & - & 04 & - & - & \(\infty 5\) & 23 &\tabularnewline
34 & 02 & - & 15 & - & - & \(\infty 6\) &\tabularnewline
\bottomrule
\end{longtable}

\textbf{Figure 17}

In general, we define an adder by considering the elements which must
accompany \(\{\infty,0\}\) in column 0. Therefore an adder is defined in
the following way:\\
(1,0){450}\\
An \emph{adder} for a starter \(S=\{\{s_i,t_i\}: 1 \leq i \leq (g-1)/2 \}\) is
a set of \((g-1)/2\) distinct non-zero elements \(a_1,a_2,...,a_{(g-1)/2}\)
of \(G\) such that:
\(s_1 + a_1,t_1 + a_1,s_2 + a_2,...,s_{(g-1)/2} + a_{(g-1)/2}, t_{(g-1)/2}+a_{(g-1)/2}\)
are precisely all the non-zero elements of \(G\).\\
(1,0){450}\\
The \emph{starter-adder} method employed in the above example was introduced
in 1968\footnote{Both Howell and Whitfield had previously found starters and
  adders, but the precise method used here due to Stanton \& Mullin} by Stanton and Mullin {[}23{]}, who used it to construct Room
squares of side 11. They also went on to apply the method to larger
squares and gave the first real suggestions that the number of Room
squares is infinite.\\
~\\
Two simple Lemmas given by Stanton and Mullin demonstrated that the
problem of finding starters for larger Room squares was straightforward.
In fact they can be guaranteed always to exist, and the only difficulty
comes from finding a corresponding adder, which is not guaranteed to
exist.

\hypertarget{lemma-3.2.1}{%
\subsubsection{Lemma 3.2.1}\label{lemma-3.2.1}}

In an additive abelian group \(G\) of order \(g=2n-1\), then pairs
\[\{n-1,n\},\{n-2,n+1\},\{n-3,n+2\},\{n-4,n+3\},...\{1,2n-2\}\] are a
starter for a Room square of side \(2n-1\).\\
~\\
\emph{Example 3.2}\\
A Room square of side \(2n-1=19\).\\
\(n=10, G=Z_{19}\)\\
~\\
The set of pairs\\
\(S_{19} = \{\{9,10\},\{8,11\}, \{7,12\}, \{6,13\}, \{5,14\}, \{4,15\}, \{3,16\}, \{2,17\}, \{1,18\}\}\)\\
is a starter.\\
~\\
Indeed, the differences are\\
\(\{\pm(10-9),\pm(11-8),\pm(12-7),\pm(13-6),\pm(14-5),\pm(3-16),\pm(2-17),\pm(18-1)\}\)\\
\(=\{1,18,3,16,5,14,7,12,9,10,8,11,6,13,4,15,2,17\}=G \backslash \{0\}\)

\hypertarget{lemma-3.2.2}{%
\subsubsection{Lemma 3.2.2}\label{lemma-3.2.2}}

In the Galois field of order \(k-1\), with primitive root \(a\) the
following pairs form a starter for a Room square of side \(k\).
\[\{a,a^n\},\{a^2,a^{n+1}\},\{a^3,a^{n+2}\},...,\{a^{n-1},a^{2n-2}\}\]
\emph{Example 3.3}\\
A Room square of side \(2n-1=23\). \(n=12\). \(a=5\).\\
The set of pairs
\[S_{23}=\{\{5,5^{12}\},\{5^2,5^{13}\},\{5^3,5^{14}\},...,\{5^{11},5^{22}\}\}\]
\[=\{\{5,18\},\{2,21\},\{10,13\},\{4,19\},\{20,3\},\{8,15\},\{17,6\},\{16,7\},\{11,12\},\{9,14\},\{22,1\}\}\]
is a starter.\\
~\\
On closer inspection the two types of starters are identical\footnote{The starter in then first Lemma has pairs whose elements always
  sum to \(2n-1\), while the second Lemma, because \(a^{n-1}=-1\), has
  pairs which can be written in the form \((a^x,-a^x)\).} , with
a general element being of the form \[\{j,-j\}\] Starters of this form
are called \emph{patterned} starters.\\
~\\
Stanton and Mullin went on to show that using the method outlined in
Example 2.1 they could find adders corresponding to the patterned
starters for \(k=7,11,13,15,17\). They had problems with 9 (but were able
to construct one using a different method) and finding it too laborious
for \(k>19\) they developed an algorithm which, when implemented in
Fortran, was able to find patterned starters with adders for all odd \(k\)
up to 49, with no further gaps. Suggesting the possibility (which they
conjectured) that there are Room squares for all odd side greater than
5.\\
~\\
They also found an interesting result regarding the number of Room
Squares which could be obtained from patterned starters, summarised in
Table 3.

\begin{longtable}[]{@{}cc@{}}
\toprule
Value of \(k\) & Number of PRS\tabularnewline
\midrule
\endhead
7 & 2\tabularnewline
9 & 0\tabularnewline
11 & 4\tabularnewline
13 & 8\tabularnewline
15 & 44\tabularnewline
17 & 416\tabularnewline
19 & The programme was turned off after the production of 967 PRS\tabularnewline
\bottomrule
\end{longtable}

\textbf{Table 3}

Stanton \& Mullin's results suggest that the number of PRS (patterned
Room squares) increases very rapidly. Which, bearing in mind that the
PRS are a sub-class of CRS (cyclic Room squares), which are in turn a
sub-class of Room squares, implies that there are vast numbers of Room
squares of large order.\\
~\\
Before introducing a class of starters for which the existence of a
corresponding adder is guaranteed we quickly confirm that when a starter
and adder exist then a Room square will always result. This seems
obvious from the method outlined in the previous section, but now we
prove it explicitly.

\hypertarget{theorem-3.2.1}{%
\subsubsection{Theorem 3.2.1}\label{theorem-3.2.1}}

{[}17{]}\\
If an Abelian group \(G\) of odd order \(2n-1\) admits a starter and an
adder, then there exists a Room square of order \(2n\).\\
~\\
\emph{Proof}\\
A square is constructed on the set \(G \cup \{\infty\}\), where \(G\) is an
additive Abelian group of order \(2n-1\).
\[G=\{g_0=0,g_1,g_2,...,g_{2n-2}\}\] The columns and rows of the square
are labelled as follows

\begin{longtable}[]{@{}lccccc@{}}
\toprule
& \(g_0\) & \(g_1\) & \(g_2\) & \ldots{} & \(g_{2n-1}\)\tabularnewline
\midrule
\endhead
\(g_0\) & & & & &\tabularnewline
\(g_1\) & & & & &\tabularnewline
\(g_2\) & & & & &\tabularnewline
\bottomrule
\end{longtable}

\(g_0\)

\textbf{Figure 18}

If a starter \(\{\{s_1,t_1\},\{s_2,t_2\}...\{s_{n-1},t_{n-1}\}\}\) and an
adder \(\{a_1,a_2,...,a_{n-1}\}\) can be ontained from \(G\) and if the
square is populated by pairs of elements from \(G\) according to the
following rules:

\begin{enumerate}
\def\labelenumi{\arabic{enumi}.}
\item
  \(\{\infty,g_i\}\) goes in \((g_i,g_i)\)
\item
  While \(\{s_i+g_i,t_i+g_i\}\) goes in \((g_i,g_i-a_i)\)
\end{enumerate}

for all \(g_i \in G\). The remaining square will be a Room square on
\(G \cup \{\infty\}\).\\
\emph{Proof}

\begin{enumerate}
\def\labelenumi{\arabic{enumi}.}
\item
  Row \(g_0=0\), contains the pairs \(\{s_i,t_i\}:1 \leq i \leq n-1\),
  which are the elements of the starter, hence all of
  \(G \backslash \{0\}\). These pairs are accompanied by \(\{\infty,0\}\),
  so row 0 contains all of \(G \cup \{\infty\}\). Subsequent rows simple
  contain a permutation of the same elements, hence the \emph{row property}
  of Room squares is satisfied for all rows.
\item
  As mentioned before, the starter forms a difference system in
  \(G \backslash \{0\}\), so all unordered pairs of this set occur along
  with all unordered pairs of the form
  \(\{\infty,g_i\}: 1 \leq i \leq n-1\), hence \emph{all unordered pairs
  from} \(G \cup \{\infty\}\) \emph{occur} in the square exactly once.
\item
  All pairs of the form \(\{s_i+a_i,t_i+a_i\}\) go in \((a_i,0)\), i.e.
  column 0. According to the definition of a starter these pairs are
  all of \(G \backslash \{0\}\), and we know that \(\{\infty,0\}\) is also
  in column 0. So the first column, and hence all others, contains all
  of \(G \cup \{\infty\}\), thus satisfying the \emph{column property} of
  Room squares.
\end{enumerate}

\hypertarget{strong-starters}{%
\section{Strong starters}\label{strong-starters}}

The next state in proving the existence of Room squares came about, not
by continuing to try to find adders for starters that were already known
(the patterned starters, for example), but when Mullin \& Nemeth in {[}17{]},
discovered a class of starters that generated their own adders.

\hypertarget{theorem-3.3.1}{%
\subsubsection{Theorem 3.3.1}\label{theorem-3.3.1}}

{[}17{]}\\
\emph{Suppose a starter}
\(\{\{s_1,t_1\},\{s_2,t_2\},...,\{s_{(g-1)/2},t_{(g-1)/2}\}\}\) \emph{exists,
such that the sums of each pair}\\
\((s_1+t_1,s_2+t_2, etc...)\) \emph{are all distinct and non-zero, then that
starter is said to be strong, and}
\[A(S)=\{a_i = -(s_i+t_i):1 \leq i \leq (g-1)/2\}\] \emph{is an adder for a
starter.\\
~\\
Proof}

\begin{enumerate}
\def\labelenumi{(\roman{enumi})}
\item
  \emph{The \(a_i\) are all distinct and non-zero}\\
  All the \((s_i+t_i)\) are, by definition, distinct and non-zero.
  Therefore all the \(a_i=-(s_i+t_i)\) are distinct and non-zero.
\item
  \(s_1 + a_1, t_1 + a_1, s_2 + a_2,..., s_{(g-1)/2}, t_{(g-1)/2}+a_{(g-1)/2}\)
  \emph{are precisely all the non-zero elements of} \(G\).
\end{enumerate}

\begin{verbatim}
<span>l</span> $s_1+a_1=s_1-(s_1+t_1)=-t_1=t_{(g-1)/2}$\
$t_1+a_1=t_1-(s_1+t_1)=-s_1=s_{(g-1)/2}$\
\
$s_{(g-1)/2} + a_{(g-1)/2} = -t_{(g-1)/2} = t_1$\
$t_{(g-1)/2} + a_{(g-1)/2} = -s_{(g-1)/2} = s_1$\

Are all the non-zero elements of $G$ in reverse order.
\end{verbatim}

(Notice that the patterned starter is not strong, on the contrary, the
sums of its pairs are all identical.)\\
~\\
\emph{Example 3.3.1}\\
The pairs \((5,7)(11,6)(2,8)(9,12)(10,1)(3,4)\), constitute a strong
starter for a Room square of side 13, based on \(G=Z_{13}\)\\
~\\
\emph{Proof}\\
Firstly, the pairs satisfy the conditions for being a starter, as the
union of all pairs is equal to \(G \backslash \{0\}\), and similarly the
differences are all of \(G\backslash \{0\}\).\\
Secondly the sums of the pairs, respectively 12,4,10,8,11,7, are all
distinct and non-zero.\\
Therefore an adder is \(\{-12,-4,-10,-8,-11,-7\}=\{1,9,3,5,2,6\}\)\\
So the following is a legitimate first row for a cyclic Room square of
order 14.

\(\infty,0\) -- -- -- 11,6 -- -- 3,4 9,12 -- 2,8 1,10 5,7
------------ --- --- --- ------ --- --- ----- ------ --- ----- ------ -----

Mullin and Nemeth originally discovered strong starters for Room squares
embedded within another type of combinatorial design, known as a Steiner
triple system. With these they were able to prove that Room squares
exist for all sides \(v=1\) mod 6. Rather than examine this approach we
move on to a type of starter which provides its own adder.

\hypertarget{mullin-nemeth-starters}{%
\section{Mullin-Nemeth starters}\label{mullin-nemeth-starters}}

If \(x\) is a primitive element in \(G=GF(p^n)\), then the elements
\(x^1,x^2,...,x^{p^n-1}=1\) are, by definition, all of
\(G \backslash \{0\}\). Alternatively, we can write \(G \backslash \{0\}=\{x^0=1,x^1,..., x^{p^n-2}\}\).\\
~\\
\emph{Example 3.4.1}\\
The field \(GF(23)\) has a primitive root \(x=5\), because\\
\(5^0=1,\) \(5^1=5,\) \(5^2=2,\) \(5^3=10,\) \(5^4=4,\) \(5^5=20,\) \(5^6=8,\)
\(5^7=17,\) \(5^8=16,\) \(5^9=11,\) \(5^{10}=9,\) \(5^{11}=22,\) \(5^{12}=18,\)
\(5^{13}=21,\) \(5^{14}=13,\) \(5^{15}=19,\) \(5^{16}=3,\) \(5^{17}=15,\)
\(5^{18}=6,\) \(5^{19}=7,\) \(5^{20}=12,\) \(5^{21}=14\) are all the non-zero
elements of \(GF(23)\).\\
~\\
Mullin \& Nemeth in {[}17{]} used the theory of primitive elements to create
strong starters in the additive group of (nearly) any Galois Field of
prime power order. Which, because Theorems 3.2.1 and 3.3.1 were already
known, was equivalent to proving the existence of Room squares for
(nearly) all orders \(p^n+1\). Before introducing the general construction
for these starters, we illustrate the basic method with a couple of
examples of particular cases.\\
~\\
\emph{Example 3.4.2}\\
We can create a strong starter from Example 2.5 simply by pairing the
elements in the order in which they were generated.\\
i.e.
\(S=\{\{1,5\},\{2,10\},\{4,20\},\{8,17\},\{16,11\},\{9,22\},\{18,21\},\{13,19\},\{3,15\},\{6,7\},\{12,14\}\}\)\\
is a strong starter.\\
\emph{Proof}\\
Obviously each member of \(GF(23)\) occurs once, because of the definition
of a primitive root.\\
The differences
\[\{\pm 4, \pm 8, \pm 7, \pm 9, \pm 10, \pm 5, \pm 3, \pm 6, \pm 11, \pm 1, \pm 2\}\]
are similarly all of \(GF(23)\), so \(S\) is a starter. The sums
\[\{6,12,1,2,4,8,16,9,18,13,3\}\] are all unique, and therefore \(S\) is
strong and \[A=\{17,11,22,21,19,15,7,14,5,10,20\}\] is an adder for
\(S\).\\
So, the following row will generate a Room square of order 24 under
cyclic construction.

{-1.66cm}

\(\infty,0\) 4,20 8,17 12,14 16,11 - 1,5 - 9,22 13,19 - - 2,10 6,7 - - 18,21 - 3,15 - - - -
------------ ------ ------ ------- ------- --- ----- --- ------ ------- --- --- ------ ----- --- --- ------- --- ------ --- --- --- ---

This is an example of the simplest case of the general theorem of
Mullith \& Nemeth, where the Galois field is \(Z_p\) (the integers mod
\(p\)), with \(p=23=3(\) mod \(4)\) a prime.\\

\hypertarget{theorem-3.4.1-1}{%
\subsubsection{Theorem 3.4.1 {[}1{]}}\label{theorem-3.4.1-1}}

\emph{If \(p=4m+3\) is prime, \(m \geq 1\), then}
\[S=\{\{x^0,x^1\},\{x^2,x^3\},...,\{x^{4m},x^{4m+1}\}\}\] \emph{is a strong
starter in \(Z_p\), and hence a Room square of order \(p+1\) exists.}\\
~\\
Example 3.4.2 took \(m=5\) and \(x=5\).\\
~\\
A slightly more general version of Theorem 3.4.1, which we prove
instead, involves any field of prime power order where \(p^n=2t+1\), with
\(t>1\) and odd. Of course, when \(p^n\) is not prime, the field will no
longer be the integers mod \(p\), instead the primitive element will be an
irreducible polynomial whose coefficients belong to \(Z_p\).

\hypertarget{theorem-3.4.2-16}{%
\subsubsection{Theorem 3.4.2 {[}16{]}}\label{theorem-3.4.2-16}}

If \(p^n=2t+1=3(\)mod \(4)\) then
\[S=\{\{x^0,x^1\},\{x^2,x^3\},...,\{x^{2t-2},x^{2t-1}\}\}\] is a strong
starter in \(GF(p^n),(p^n \neq 3)\)\\
~\\
\emph{Proof}\\
\(x\) is a primitive element, so the elements in the starter are all the
non-zero members of \(GF(P^n)\).\\
The differences are, respectively
\[\pm x^0(1-x), \pm x^2 (1-x), ... , \pm x^{2t-2}(1-x)\] \((1-x)\) is a
non-zero (\(x=1\) is not primitive) member of \(GF(p^n)\). So in order to
show that these differences are all the \(2t\) non-zero members of
\(GF(p^n)\) we merely need to prove that the \(2t\) differences are all
distinct and non-zero.\\
All the differences can be written \(\pm x^{2i}(1-x), 0 \leq i \leq t-1\)\\
\((1-x) \neq 0\)\\
\(x^{2i}(1-x)=x^{2j}(1-x)\)\\
\(\Rightarrow x^{2i}=x^{2j} \Rightarrow i=j\),\\
because \(0 \leq 2i,2j \leq 2t-2 < p^{n-1}\), and the primitive element,
by definition, produces each element of \(GF(p^n)\) exactly once as the
indices range from 0 to \(p^{n-1}\).\\
Similarly \(-x^{2i}(1-x)=-x^{2j}(1-x)\) only when \(i=j\).\\
So all the positive differences are unique, similarly the negative.\\
However, there remains a possibility for repetition when the signs are
opposite: \[x^{2i}(1-x)=-x^{2j}(1-x) \hspace{1cm}...(1)\] Either \(i=j\)
or \(i \neq j\)\\
Let \(i=j,\) \((1)\) becomes \(x^{2i}+x^{2i}=0, \Rightarrow 2x^{2i}=0\), bit
\(i\) takes values \(0...t-1\), so \(x^2=0\) when \(i=1\), contradicting the
order of the primitive element.\\
In the \(i \neq j\) case, we assume (without loss of generality) that
\(i<j\) and write

\(x^{2i} = -x^{2j}\)\\
as \(x^{2i}(1+x^{2j-2i})=0\)\\
\(\Rightarrow x^{2j-2i}=-1\)

but in \(GF(2t+1)\), \(x^{\frac{1}{2}(q-1)}=x^t=-1\) \[2j-2i=t\] but this is
a contradiction as we insisted that \(t\) be odd. So \(S\) is a starter.\\
~\\
To prove that \(S\) is strong we simply note that the sums can be written:

\(x^0(1+x)\), \(x^2(1+x)\), \ldots, \(x^{2t-2}(1+x)\)

\(1+x=0 \Rightarrow x = -1\) is only true when \(p^n=3\). So
\((1+x) \neq 0\).\\
So \(x^{2i}(1+x)=x^{2j}(1+x) \Rightarrow x^{2i}=x^{2j}\)\\
We have already shown that \[x^{2i}=x^{2j}\] is only true for \(i=j\). So
all the sums are unique, and the starter is strong.\\
~\\
Hence, by theorems 2.1 and 2.2, Room squares exist for all \(p^n=3(\)mod
\(4)\), and in the case when \(p^n=3(\)mod \(4)\) is prime, these Room squares
are based on \(Z_p\).\\
~\\
The most generalised case of Mullin \& Nemeth's theorem proves the
existence of Room squares for all prime powers \(p^n=2^kt+1\) where \(k>1\)
and \(t>1\) is odd (\(k\) and \(t\), both positive integers), and reduces to
Theorem 3.4.2 when \(k=1\).

\hypertarget{theorem-3.4.3-16}{%
\subsubsection{Theorem 3.4.3 {[}16{]}}\label{theorem-3.4.3-16}}

A strong starter exists in \(GF(p^n)\), where \(p^n=2^kt+1\) (with \(k>1\) and
\(t>1\) is odd).\\
~\\
\emph{Proof}\\
Let \(d=2^{k-1}\)\\
Then the strong starter in question looks like this: \$\$ S=\{

\begin{longtable}[]{@{}cccc@{}}
\toprule
\endhead
\((x^0,x^d)\) & \((x^{2d},x^{3d})\) & \$ \hspace{0.5cm} \ldots{} \hspace{0.5cm} \$ & \((x^{(2t-2)d},x^{(2t-1)d})\)\tabularnewline
\((x^1,x^{d+1})\) & \(\hspace{0.5cm}(x^{2d+1},x^{3d+1})\) & \$ \hspace{0.5cm} \ldots{} \hspace{0.5cm} \$ & \((x^{(2t-2)d+1},x^{(2t-1)d+1})\)\tabularnewline
& & \$ \hspace{0.5cm} \ldots{} \hspace{0.5cm} \$ &\tabularnewline
\((x^{d-1},x^{2d-1})\) & \(\hspace{0.5cm}(x^{3d-1},x^{4d-1})\) & \$ \hspace{0.5cm} \ldots{} \hspace{0.5cm} \$ & \((x^{(2t-1)d-1},x^{2td-1})\)\tabularnewline
\bottomrule
\end{longtable}

\} \$\$ Where the pairs have been placed in an array to emphasise that,
when read vertically, this is an exhaustive list of all the non-zero
elements of \(GH(p^n)\), ordered according to powers. Of course, in the
\(k=1,d=1\) case this starter reduces to the one quoted in Theorem 3.4.2.\\
To prove that \(S\) is a starter we need also to show, as usual, that the
differences between pairs are all of \(GF(p^n)\), and to show that the
starter is strong we need to show that the sums of pairs are all
distinct and non-zero.\\
The differences can be written in the following scheme:

\begin{longtable}[]{@{}cccc@{}}
\toprule
\endhead
\(x^0(1-x^d)\), & \(x^{2d}(1-x^d)\), & \ldots, & \(x^{(2t-2)d}(1-x^d)\),\tabularnewline
\(x^1(1-x^d)\), & \(x^{2d+1}(1-x^d)\), & \ldots, & \(x^{(2t-2)d+1}(1-x^d)\),\tabularnewline
& & \ldots{} &\tabularnewline
\(x^{d-1}(1-x^d)\), & \(x^{3d-1}(1-x^d)\), & \ldots, & \(x^{(2t-1)d-1}(1-x^d)\)\tabularnewline
\bottomrule
\end{longtable}

The order of \(x\) is \(p^n-1=2^kt=2^{k-1}2t=2td > d\), (meaning \(x^{2td}=1\)
and \(x^\alpha \neq 1\) when \(1 \leq \alpha < 2dt\)) and so \(x^d \neq 1\),
so \((1-x^d) \neq 0\).\\
We can write the differences in a general form:

\(\pm x^{2id+j}(1-x^d) \hspace{0.5cm}\) where
\(\hspace{0.5cm} 0 \leq i \leq t-1, \hspace{0.5cm} 0 \leq j \leq d-1\)

{: l :}\(\rightarrow\) If there were repetition, either of the
form \(D=D\) or \(-D=-D\), where \(D=x^{2id+j}(1-x^d)\),\\
then the following must hold:\\
~\\
Cancelling by \((1-x^d)\), legitimate because \((1-x^d) \neq 0\) gives:\\
~\\
dividing through by \(x^{2Id+j}\) leaves\\
~\\
~\\
But if \(i \neq I\), then the LHS has an index which is an integer
multiple of \(d\). The index in the RHS,\\
however, can never be an integer multiple of \(d\) because \(J\) and \(j\)
range over the integers \(0...d-1\).\\
So the only possibility for equality is when both indices are zero, i.e.
\(i=I\) and \(j=J\).\\

As in the previous proof we have to deal with the possibility of
repetition for differences of opposite sign. For coincidence we require:
\[x^{2id+j}=-x^{2ID+J}\] \[x^{2id+j}+x^{2Id+J}=0\] We assume that
\(2id+j<2Id+J\) and rewrite this expression as:
\[x^{2id+j}(1+x^{(2I-2i)d+(J-j)})=0\] Which implies that
\(x^{(2I-2i)d+(J-j)}=-1\)\\
But in \(GF(q),x^{\frac{1}{2}(q-1)}=-1\). Where, in this case \(q-1=2^kt\),
so \(\frac{1}{2}(q-1)=2^{k-1}t=dt\). \[\therefore x^{dt} = -1\]
\[\Rightarrow (2I-2i)d+(J-j)=dt\]

\(\Rightarrow (J-j)\) is an integer multiple of \(d\) or zero.

But \(J\) and \(j\) both take only the values \(0...d-1\), so \((J-j)\) is in
the interval \([1-d,d-1]\) and hence must be zero, leaving \[(2I-2i)d=dt\]
\[2I-2i=t\] But \(t\) is strictly odd, and so we have reached a
contradiction, hence the differences are all unique, belong to \(GF(p^n)\)
and there are \(2td\) of them, hence each member of \(GF(p^n)\) occurs
exactly once as a difference. So \(S\) is a starter.\\
To prove that the starter is strong we write the sums as

\begin{longtable}[]{@{}cccc@{}}
\toprule
\endhead
\(x^0(1-x^d)\), & \(x^{2d}(1-x^d)\), & \ldots, & \(x^{(2t-2)d}(1-x^d)\),\tabularnewline
\(x^1(1-x^d)\), & \(x^{2d+1}(1-x^d)\), & \ldots, & \(x^{(2t-2)d+1}(1-x^d)\),\tabularnewline
& & \ldots{} &\tabularnewline
\(x^{d-1}(1-x^d)\), & \(x^{3d-1}(1-x^d)\), & \ldots, & \(x^{(2t-1)d-1}(1-x^d)\)\tabularnewline
\bottomrule
\end{longtable}

and notice that \(x^d=-1 \Rightarrow d=dt\) (because
\(x^{dt}=-1) \Rightarrow t=1\), but instead we insisted that \(t\) be
strictly greater than one (this being the reason why). So
\((1+x^d) \neq 0\) and the above argument (denoted by \(\rightarrow\))
involving \((1-x^d)\) can be invoked, replacing \((1-x^d)\) by \((1+x^d)\). So
\(S\) is a strong starter, and the general theorem of Mullin \& Nemeth is
proven, guaranteeing the existence of a vast class of Room Squares.

\hypertarget{the-trouble-with-fermat-numbers}{%
\section{The trouble with Fermat numbers}\label{the-trouble-with-fermat-numbers}}

Unfortunately, in establishing the Mullin \& Nemeth starters we were
forced to exclude a similarly vast, potentially infinite, class of Room
squares by insisting that \(t\) be strictly greater than one. These
exceptional Room squares have side \(2^k+1\).\\
~\\
Rectifying this problem is essential if we are to prove the existence of
Room squares. As mentioned previously, the proof relies on a
multiplication theorem, so proving that all the 'prime' Room squares
exist is vital. Although the theorem of Mullin \& Nemeth will take care
of all squares with prime power side, the multiplication theorem is
necessary for proving the existence of those whose side can be
decomposed into prime factors different from each other. In fact, the
multiplication theorem means that we can ignore the Mullin \& Nemeth
construction except in the prime case, resorting to multiplication to
recover the prime power squares. Similarly we are only concerned with
recovering the exceptional squares with side \(2^k+1\), when \(2^k+1\) is
prime.\\
~\\
Primes of this form are known as \textbf{Fermat Numbers} or \textbf{Fermat
Primes}, after Pierre de Fermat who, 360 years ago conjectured that
numbers of the form \(2^k+1\) are always prime when \(k\) is a power of two.

\begin{longtable}[]{@{}l@{}}
\toprule
\endhead
\(F_m=2^{2^m}+1\)\tabularnewline
\(F_0=2^1+1=3\)\tabularnewline
\(F_1=2^2+1=5\)\tabularnewline
\(F_2=2^4+1=17\)\tabularnewline
\(F_3=2^8+1=257\)\tabularnewline
\(F_4=2^{16}+1=65537\)\tabularnewline
\bottomrule
\end{longtable}

After the first four of Fermat's numbers, all of which were known to him
to be prime. Nearly one hundred years later Euler calculated the
following, \[F_5 = 2^{32}+1=4294967297=641\times 6700417\] and in doing
so disproved Fermat's conjecture.\\
~\\
Since Euler's time, \(F_6\), \(F_7\) and \(F_8\) have all been factorised\footnote{In 1880 F.Landry showed
  \(F_6=2^{64}+1=274177 \times 67280421310721\). In 1975 Brillhart and
  Morrison showed
  \(F_7=2^{128}+1=59649589127497217 \times 5704689200685129054721\). In
  1981, Brent and Pollard found that\\
  \(2^{256}+1=1238926361552897 \times 93461639715357977769163558199606896584051237541638188580280321\)}
. It is also known, although most of the factorisations remain unknown,
that \(F_m\) is composite for \(m=[9...23]\). \(F_{24}\), a number with over 5
million digits, remains in doubt.\\
~\\
Whether there be an infinite number of Fermat primes or whether, as
empirically seems to be the case, there are only finitely many (possibly
just five) such primes, in order for the proof of the existence of Room
squares for all odd side greater than 7 to be complete these Fermat
prime Room squares must be included.\\
~\\
When the problem of Fermat Room squares was tackled first in the early
1970s, W.D.Wallis used a Theorem of J.D.Horton's which adapted a famous
result of E.H.Moore's from the theory of Steiner triple systems.\\
~\\
Moore, in 1893, was able to prove that if Steiner triple systems of
orders \(v_1\), \(v_2\) and \(v_3\) exist, where the \(v_2\) system is a
sub-system of the \(v_3\) system, then an STS of order \(v_1(v_2-v_3)+v_3\)
also exists. Horton{[}12{]} adapted this result to other combinatorial
objects including Room squares and Wallis{[}25{]} was able to use this
Moore-type construction method to include all of the Fermat primes,
except \(F_3=257\)\footnote{Wallis presented a Room square of side 257 a conference in 1973,
  completing his proof (different from the one presented here) of the
  existence of Room squares.} .\\
~\\
\emph{Example 3.5.1}\\
If Room squares with side \(v_1\), \(v_2\) and \(v_3\) exist, where the square
of side \(v_2\) is a subsquare of the square with side \(v_3\), then a Room
square of side \(F_4=65537\) exists. Room squares of side 7 and 11 exist,
according to the theory of Mullin \& Nemeth. Applying Horton's theorem
once, with \(v_3=0\) gives a new square of side \(v_1v_2=77\) (note that
Horton's theorem reduces to the multiplication theorem when \(v_3=0\)).\\
~\\
The trivial Room square of side one exists, and the Mullin \& Nemeth
starters will provide a Room square of size 13. So we can apply Horton's
theorem once again to gain a Room square of side 989 because:
\[989=13(77-1)+1\] Finally we can use Mullin \& Nemeth to produce a Room
square of side 67, and a final application of Horton's theorem gives:
\[65537=67(989-11)+11\] The proof of Horton's theorem and also an
explanation of Wallis's application of that theorem to solving the
Fermat prime problem is excluded because another solution was
subsequently found. A year after Wallis had published his solution to
the Fermat problem, Chong and Chan published their (independent)
discovery of the strong starters which are known as the Mullin \& Nemeth
starters. Also included in their paper was an alternative solution to
the same problem, but their solution continued to involve the
starter-adder method. This theorem we prove instead.

\hypertarget{theorem-3.5.1-6}{%
\subsubsection{Theorem 3.5.1 {[}6{]}}\label{theorem-3.5.1-6}}

For every Galois field of order \((2^{2^m}+1)\), where \(m \geq 2\), there
exists a Room square of order \((2^{2^m}+2)\).\\
~\\
\emph{Proof}\\
The following pairs in \(Z_p\) (where \(p=2^{2^d}+1\) and \(d=2^{m-1})\)
constitute a strong starter.

\begin{enumerate}
\def\labelenumi{\arabic{enumi}.}
\item
  \(\{i+(r-1)2^d,i2^d-(r-1\}\)
\item
  \(\{(2^d-i)2^d+r,(2^{d-1}-r)2^d+2^{d-1}-i+1\}\)
\item
  \(\{2^{d-1}+r-1)2^d+2^{d-1}+i,(2^{d-1}+i-1)2^d+2^{d-1}-(r-1)\}\)
\item
  \(\{(2^{d-1}-i)2^d+2^{d-1}+r,(2^d-r+1)2^d-i+1\}\)
\end{enumerate}

Where \(1 \leq r \leq 2^{d-2}\) and \(1 \leq i \leq 2^{d-1}\), so rather
than just 4 pairs there are \(4 \cdot 2^{d-2} \cdot 2^{d-1}=2^{2d-1}\)
pairs arranged in four different classes. Before completing the proof we
pause for an example just to illustrate the real simplicity of these
apparently complicated pairs.\\
~\\
\emph{Example 3.5.2}\\
Suppose \(p=2^{2\cdot 2}+1=17=F_2\), then \(d=2^1 \Rightarrow m=2\)\\
\(r=1,1 \leq i \leq 2\) and the following pairs should be a strong
starter.\\

{-1.15cm}

\begin{longtable}[]{@{}lll@{}}
\toprule
& \(i=1,r=1\) & \(i=2,r=1\)\tabularnewline
\midrule
\endhead
1 & \(\{1+0 \cdot 2^2,1 \cdot 2^2 - 0\} = \{1,4\}\) & \(\{2+0 \cdot 2^2,2 \cdot 2^2 -0\} = \{2,8\}\)\tabularnewline
2 & \(\{(2^2-1)2^2+1,(2^1-1)2^2+2^1-1+1\}=\{13,6\}\) & \(\{(2^2-2)2^2+1,(2^1-1)2^2+2^1-2+1\}=\{9,5\}\)\tabularnewline
3 & \(\{(2^1+1-1)2^2+2^1+1,(2^1+1-1)2^2+2^1-(1-1)\}\) & \(\{(2^1+1-1)2^2+2^1+2,(2^1+2-1)2^2+2^1-(1-1)\}\)\tabularnewline
& \(=\{11,10\}\) & \(=\{12,14\}\)\tabularnewline
4 & \(\{(2^1-1)2^2+2^1+1,(2^2-1+1)2^2-1+1\}\) & \(\{(2^1-2)2^2+2^1+1,(2^2-1+1)2^2-2+1\}\)\tabularnewline
& \(=\{7,16\}\) & \(=\{3,15\}\)\tabularnewline
\bottomrule
\end{longtable}

\textbf{Table 4}

The pairs generated by this method contain each non-zero member of
\(Z_{17}\) exactly once in their union satisfying the first property of a
starter.\\
The differences are
\(\{\pm 3,\pm 7,\pm 1,\pm 8, \pm 6, \pm 4, \pm 2, \pm 5\}=Z_{17} \backslash \{0\}\),
satisfying the other necessary property of a starter.\\
The sums 5,2,4,6,10,14,9,1 are all unique, hence the starter is strong
and the set\\
\(\{-5,-2,-4,-6,-10,-14,-9,-1\}=\{12,15,13,11,3,8,16\}\) is an adder. So
the following first row will generate a Room square under cyclic
construction:

\(\infty,0\) 3,15 13,6 - 11,10 1,4 7,16 - - 12,14 2,8 - - - 9,5 - -
------------ ------ ------ --- ------- ----- ------ --- --- ------- ----- --- --- --- ----- --- ---

In order to prove that the pairs \(1...4\) are a strong starter from any
\(Z_p\) we need to prove the following:

\begin{enumerate}
\def\labelenumi{\arabic{enumi}.}
\item
  The union of all the pairs contains each non-zero member of \(Z_p\)
  exactly once.
\item
  The differences are all the non-zero members of \(Z_p\) exactly once.
\item
  The sums are all distinct and non-zero.
\end{enumerate}

This is a formidable task, one that would take many pages to prove in
full detail. So instead we sketch an outline of the proof, explicitly
proving a few specific cases.\\
~\\
First we prove (a) completely.\\
The non-zero members of \(Z_P\), namely \(\{1...2^{2d}\}\), can be
represented uniquely by:
\[C(u,v) = u2^d+v \hspace{0.5cm} \mathrm{where} \hspace{0.5cm} 1 \leq v \leq 2^d
\hspace{0.25cm} \mathrm{and} \hspace{0.25cm} 0 \leq u \leq 2^d-1\]
\emph{Proof}\\
Indeed if \[u_12^d+v_1=u_22^d+v_2\] then \[(u_1-u_2)2^d=(v_2-v_1)\] The
RHS takes integer values in the interval \([-(2^d-1),2^d-1]\), which is
symmetric about the origin and smaller than \(2^d\) on both sides. Whereas
the LHS takes integer multiple steps of size \(2^d\), so the equality can
only hold in the case when both sides equal zero. Which implies
\(u_1=u_2\), \(v_1=v_2\) and \(C(u,v)\) is unique representation the non-zero
members of \(Z_p\). \(u\) takes \(2^d\) values and \(v\) takes \(2^d\) values so
there are \(2^{2d}\) unique non-zero members of \(Z_p\) represented in this
way, so each member of \(Z_p\) is represented.\\
The left and right hand members of each pair can be characterised by a
range of values of \(u\) and \(v\) in the following manner.\\
Take, for instance, the left hand member of pair 1. \[i+(r-1)2^d\] Here
\(v=i\) and so \(1 \leq v \leq 2^{d-1}\), while \(u=(r-1)\), so
\(0 \leq u \leq 2^{d-2}-1\). The full list of intervals for each member of
each pair is tabulated below.

\begin{longtable}[]{@{}cccc@{}}
\toprule
\endhead
Pair & Member & \(u\) & \(V\)\tabularnewline
1. & L & \([0,2^{d-2}-1]\) & \([1,2^{d-1}]\)\tabularnewline
& R & \([0,2^{d-1}-1]\) & \([3 \cdot 2^{d-2}+1,2^{d}]\)\tabularnewline
2. & L & \([2^{d-1},2^{d}-1]\) & \([1,2^{d-2}]\)\tabularnewline
& R & \([2^{d-2},2^{d-1}-1]\) & \([1,2^{d-1}]\)\tabularnewline
3. & L & \([2^{d-1},3 \cdot 2^{d-2}-1]\) & \([1+2^{d-1},2^{d}]\)\tabularnewline
& R & \([2^{d-1},2^{d}-1]\) & \([1+ 2^{d-2},2^{d-1}]\)\tabularnewline
4. & L & \([0,2^{d-1}-1]\) & \([1+ 2^{d-1},3 \cdot 2^{d-2}]\)\tabularnewline
& R & \([3 \cdot 2^{d-2},2^{d}-1]\) & \([1+2^{d-1},2^{d}]\)\tabularnewline
\bottomrule
\end{longtable}

\textbf{Table 5}

It was mentioned earlier that there were \(2^{2d-1}\) pairs, each of which
has two members, so there are \(2^{2d}\) elements altogether in the pairs
of the starter, which is the same as the number of elements in \(Z_p\).
Because \(C(u,v)\) is a unique representation for each member of \(Z_p\),
for an element of \(Z_p\) to occur more than once in the starter requires
repetition of both \(u\) and \(v\). This cannot happen because when two
intervals overlap (as they do in the values of \(v\) for 1L and 2R).\\
~\\
To prove (b) we need to show that the differences between two pairs of
type 1 are all unique, similarly between two pairs of types 2,3 and 4.
Moreover we need to show that there can be no repetition in differences
between a pair of type 1 and a pair of type 2, also type 1 with types 3
in 4. Similarly for 2,3 and 4. All together there are ten cases to
prove, tabulated below, where a pair of numbers represents the two types
of pairs from the starter.

MISSING TABLE HERE

To illustrate, we prove (v), in other words that differences ebtween two
different pairs, both of type 2 are always unique.\\
Type 2 have the form: \(\{(2^d-i)2^d+r, (2^{d-1}-r)2^d+2^{d-1}-i+1\}\).
Therefore, a difference between the elements of a pair of type 2 has the
form: \[\pm \{(2^d-i)2^d+r-(2^{d-1}-r)2^d-2^{d-1}+i-1\}\] If two
different pairs had the same difference we could write:
\[(2^d-i)2^d+r-(2^{d-1}-r)2^d-2^{d-1}+i-1
\equiv \pm \{(2^d-j)2^d +s-(2^{d-1}-s)2^d-2^{d-1}+j-1\}(\mathrm{mod} \hspace{0.1cm} p)\]
for some \(i \neq j, r \neq s\). There are two cases to prove, firstly
consider the one involving the + sign.
\[(2^d-i)2^d+r-(2^{d-1}-r)2^d-2^{d-1}+i-1 \equiv (2^d-j)2^d + s -(2^{d-1}-s)2^d - 2^{d-1} + j -1 (\mathrm{mod} \hspace{0.1cm} p)\]

MISSING EQUATION HERE

In this case we have been helped out with some very convenient
cancelling, leaving just:
\[-i2^d+r+r \cdot 2^d + i \equiv -j \cdot 2^d + s + s \cdot 2^d + j \hspace{0.3cm} (\mathrm{mod} \hspace{0.1cm} p)\]
\[(-i+j)2^d + i -j \equiv (s-r)(1+2^d) \hspace{0.3cm} (\mathrm{mod} \hspace{0.1cm} p)\]
\[(j-i)(2^d -1) \equiv (s-r)(1+2^d) \hspace{0.3cm} (\mathrm{mod} \hspace{0.1cm} p)\]
\(\times (2^d+1)\)
\[(j-i)(2^{2d} -1) \equiv (s-r)(1+2 \cdot 2^d + 2^{2d}) \hspace{0.3cm} (\mathrm{mod} \hspace{0.1cm} p)\]
\(2^{2d}+1=p\) \(\hspace{0.5cm}\therefore\) \(2^{2d}-1 \equiv -2\) (mod \(p\))
\[-2(j-i)\equiv (s-r)2 \cdot 2^d \hspace{0.1cm} (\mathrm{mod} \hspace{0.1cm} p)\]
\[(i-j)\equiv (s-r) 2^d \hspace{0.1cm} (\mathrm{mod} \hspace{0.1cm} p) \hspace{2cm} ...(\mathrm{A})\]
\(1 \leq i, j \leq 2^{d-1}\)\\
\(\therefore (i-j)\) lies in the interval \([-(2^{d-1}-1),2^{d-1}-1]\),
which is symmetric about the origin, with length
\(2 \cdot 2^{d-1}-2=2^d-2 < p = 2^{2d}+1\).\\
\(1 \leq s,r \leq 2^{d-2}\)\\
\(\therefore(s-r)2^d\) lies in the interval
\([-(2^{d-2}-1)2^d,(2^{d-2}-1)2^d]\), again symmetric about the origin
with length \(2^{2d}-2^{d+1} < p\).\\
So for A to hold requires that \[(i-j)=(s-r)2^d\] But the LHS has an
interval with length \(2^d-2 < 2^d\) whereas the RHS is some positive or
negative integer multiple of \(2^d\), so the two could only be equal when
\(i=j,r=s\) contradicting the original hypothesis.\\
~\\
There is still the negative case to deal with:
\[(2^d-i)2^d+r-(2^{d-1}-r)2^d-2^{d-1}+i-1 \equiv
-(2^d-j)2^d-s+(2^{d-1}-s)2^d+2^{d-1}-j+1\hspace{0.1cm} (\mathrm{mod} \hspace{0.1cm} p)\]
\[(2^d-i)2^d-(2^{d}-j)2^d+i+j \equiv
-s+(2^{d-1}-s)2^d + (2^{d-1} -r)2^d -r + 2 \cdot 2^{d-1} + 2 \hspace{0.1cm} (\mathrm{mod} \hspace{0.1cm} p)\]
\[2 \cdot 2^{2d} + (1-2^{d})(i+j) \equiv
-(s+r)(1+2^d)+2^{2d}+1+2^d+1 \hspace{0.1cm} (\mathrm{mod} \hspace{0.1cm} p)\]
\(2^{2d}+1=p \therefore 2^{2d} \equiv -1 \hspace{0.1cm} (\mathrm{mod} \hspace{0.1cm} p)\)
\[-2+(1-2^d)(i+j) \equiv -(s+r)(1+2^d)+1+2^d \hspace{0.1cm} (\mathrm{mod} \hspace{0.1cm} p)\]
\[(1-2^d)(i+j) \equiv -(s+r)(1+2^d)+3+2^d \hspace{0.1cm} (\mathrm{mod} \hspace{0.1cm} p)\]
Now multiply throughout by \((1+2^d)\), noting that:\\
\((1+2^d)(1-2^d) \equiv 2(\hspace{0.1cm} (\mathrm{mod} \hspace{0.1cm} p),(1+2^d)(1+2^d) =1+2 \cdot 2^d + 2^{2d} \equiv 2^{d+1}\hspace{0.1cm} (\mathrm{mod} \hspace{0.1cm} p)\)\\
and
\((1+2^d)(3+2^d)=3+4 \cdot 2^d+2^{2d} \equiv 2^{d+2}+2 \hspace{0.1cm} (\mathrm{mod} \hspace{0.1cm} p)\)\\
\[2(i+j)\equiv -2^{d+1}(s+r)+2^{d+2}+2\hspace{0.1cm} (\mathrm{mod} \hspace{0.1cm} p)\]
\[(i+j)\equiv -2^{d}(s+r)+2^{d+1}+1\hspace{0.1cm} (\mathrm{mod} \hspace{0.1cm} p)\]
\[2^d(s+r)\equiv 2 \cdot 2^{d}-(i+j)+1\hspace{0.1cm} (\mathrm{mod} \hspace{0.1cm} p) \hspace{1cm}... (\mathrm{B})\]
\(1 \leq s, r \leq 2^{d-2}\)\\
\(\therefore 2 \cdot 2^d \leq 2^d(s+r) \leq 2^d2^{d-1}\)\\
The LHS lies in the interval \([2^{d+1},2^{2d-1}]\), which itself is
located somewhere in the interval \([0,p]\).\\
\(1 \leq i, j \leq 2^{d-1}\)\\
\(\therefore 2 \leq i + j \leq 2^d\)\\
\(2^d \leq 2 \cdot 2^d -(i+j) \leq 2^{d+1} - 2\)\\
So the RHS lies in the interval \([2^d+1,2^{d+1}-1]\), again this is
located with \([0,p]\).\\
So for B to be satisfied requires that \[2^d(s+r)=2 \cdot 2^d-(i+j)+1\]
But the RHS and LHS intervals are disjoint so this can never happen.\\
So the absence of repetition in the differences of two different pairs
both of type 2 is proven. All cases involving different pairs of the
same type are proven in this way \{cases (i),(v),(viii),(x)\}.\\
~\\
Finally we demonstrate how the other 6 cases are proven, those involving
pairs of different types.\\
Inevitably the approach is very similar.\\
~\\
Consider a pair of type 1 and another pair of type 4 (case iv). If there
were a repetition of differences between pairs of this type we could
write:
\[i+(r-1)2^d-i2^d+(r-1) \equiv \pm \{(2^{d-1}-j)2^d+2^{d-1}+s-(2^d-s+1)2^d+j-1\}
\hspace{0.3cm} (\mathrm{mod} \hspace{0.1cm} p)\] Consider the + sign,
then
\[i-i2^d-(2^{d-1}-j)2^d-j \equiv -(r-1)2^d-(r-1)+s-(2^d-s+1)2^d+2^{d-1}-1
\hspace{0.3cm} (\mathrm{mod} \hspace{0.1cm} p)\]
\[(1-2^d)(i-j)-2^{2d-1} \equiv -(2^d+1)(r-s)-2^{2d}+2^{d-1}
\hspace{0.3cm} (\mathrm{mod} \hspace{0.1cm} p)\]
\(2^{2d} \equiv -1\hspace{0.3cm} (\mathrm{mod} \hspace{0.1cm} p)\)\\
\[(1-2d)(i-j) \equiv -(2^d+1)(r-s)+2^{d-1} + 2^{2d-1} + 1
\hspace{0.3cm} (\mathrm{mod} \hspace{0.1cm} p)\] \(\times 2\)\\
\[2(1-2d)(i-j) \equiv -2(2^d+1)(r-s)+2^{d} + 2^{2d} + 2
\hspace{0.3cm} (\mathrm{mod} \hspace{0.1cm} p)\]
\[2(1-2d)(i-j) \equiv -2(2^d+1)(r-s)+2^{d}+1
\hspace{0.3cm} (\mathrm{mod} \hspace{0.1cm} p)\] \(\times(2^d+1)\)\\
\[4(i-j) \equiv -2 \cdot 2^{d+1}(r-s)+2^{d+1}
\hspace{0.3cm} (\mathrm{mod} \hspace{0.1cm} p)\]
\[(i-j) \equiv -2^{d}(r-s)+2^{d-1}
\hspace{0.3cm} (\mathrm{mod} \hspace{0.1cm} p)\]
\[2^d(r-s) \equiv 2^{d-1}-(i-j)
\hspace{0.3cm} (\mathrm{mod} \hspace{0.1cm} p)\] In this instance the
LHS has interval \([2^d-2^{2d-2},2^{2d-2}-2^d]\). while the interval of
the RHS is \([1,2^d-1]\). Both are smaller than \(p\) in length, so for
equality requires \[2^d(r-s) = 2^{d-1}-(i-j)\] But this can never be
true because the left side is always either zero or an integer multiple
of \(2^d\), whereas the interval of the right is \([1,2^d-1]\).\\
~\\
All other cases are dealt with in a very similar manner, and the proof
of (c), namely that all sums are unique, is not very different.

\hypertarget{a-multiplication-theorem}{%
\section{A multiplication theorem}\label{a-multiplication-theorem}}

Having a theorem which enables new Room squares to be composed from old
Room squares is of vital importance to the proof of the existence of
Room squares. With such a theorem, in conjunction with the Mullin-Nemeth
starters, we will be able to construct Room squares of almost any order.
The exceptions will be due to the non-existence of orders 4 and 6. The
multiplication theorem that will be proven is:

\hypertarget{theorem-3.6.1}{%
\subsubsection{Theorem 3.6.1}\label{theorem-3.6.1}}

\emph{If Room squares of side m and side n exist then a Room square of side
mn also exists.}\\
~\\
This theorem was proposed initially in {[}5{]} but later a counter-example
to this method was found {[}15{]}. The proof here is based upon {[}1{]}, which
in turn is based upon the proof in {[}22{]}.\\
~\\
\emph{Proof}\\
\(M\) and \(N\) are two Room squares. \(M\) is of side \(m\) and based on
\(\{0,1,2,...,m\}\), while \(N\) is of side \(n\) and based on
\(\{0,1,2,...,n\}\).\\
~\\
The join of two Latin squares \(A\) and \(B\) is the array whose
\((i,j)^{\mathrm{th}}\) entry contains the ordered pair formed from the
\((i,j)^{\mathrm{th}}\) entry of \(A\) taking the left position and the
\((i,j)^{\mathrm{th}}\) entry of \(B\) taking the right. If the join of two
Latin squares contains \(n^2\) unique ordered pairs, the two Latin squares
\(A\) and \(B\) are said to be orthogonal.\\
~\\
\(L_1\) and \(L_2\) are two mutually Latin squares (MOLS) based on
\(\{1,2,3,...,n\}\).\\
We construct the new Room square \(R=MN\) by replacing each element of \(M\)
by an \(n \times n\) array according to the following flow diagram where
\((i,j)\) is a pair from \(M\).

\textbf{Figure 19}

This procedure has replaced each pair in \(m\) by an \(n \times n\) array,
resulting in an \(mn \times mn\) array. This array is based upon
\(\{0,n+1,n+2,...,n+mn\}\), and we now prove that it has the properties of
a Room square, namely:

\begin{enumerate}
\def\labelenumi{\arabic{enumi}.}
\item
  Each element of the array is either empty or contains an unordered
  pair.
\item
  Each row and column contains each of \(\{0,n+1,n+2,...,n+mn\}\)
  exactly once.
\item
  Each pair from \(\{0,n+1,n+2,...,n+mn\}\) occurs exactly once in the
  array.
\end{enumerate}

The first property is easily satisfied. The procedure followed did
nothing but replace empty elements and unordered pairs with arrays
containing nothing more than empty elements or pairs.\\
~\\
The second is similarly straightforward. Consider an arbitrary row of
the new square \(R\), call it \(i\). This row arose from applying
prescriptions (i),(ii), and (iii) to some row of \(M\). This row of \(M\)
contained the elements \(0...m\) exactly once. One of these elements, call
it \(a\), was paired with 0. So in \(i\) from (ii) occur the numbers
\((0;1+an...m+an)\) exactly once. In the join of two MOLS the numbers
\(1...n\) occur twice per row, once in \(L_1\) once in \(L_2\). These are
replaced by \((1+un...n+un)\) and \((1+vn...n+vn)\) as \(u\) and \(v\) take on
all values \(1,2,...,m\) excluding\\
Together these two prescriptions produce the elements
\[\{0;1+n,2+n,...,n+mn\}\] exactly once per row and column.\\
~\\
To prove condition 3 is true we show that \(R\) contains the correct
number of pairs and that\\
these pairs are distinct. Because we have shown 2 to be correct these
pairs must be the right\\
Any Room square, of side \(n\), contains \(\frac{1}{2}(n+1)\) pairs per row,
therefore \(\frac{1}{2}n(n+1)\) pairs over\\
Room square of side \(mn\) ought to contain \(\frac{1}{2}mn(mn+1)\) pairs.\\
In \(M\) there were \(m\) instances of \(\{0,k\}\), each of these was replaced
by a Room square of side\\
\(\frac{1}{2}n(n+1) \cdot m\) pairs were contributed by (ii) to \(R\).\\
In \(M\) there were \(\frac{1}{2}(m+1)-1 = \frac{1}{2} (m-1)\) pairs per row
of the form \(\{u,v\}\), therefore \(\frac{1}{2}m(m\)\\
of these pairs throughout \(M\). These were replaced by MOLS of side \(n\),
containing \(n^2\) pairs\\
each. So \(\frac{1}{2}m(m-1) \cdot n^2\) pairs were contributed to \(R\)
from (iii).\\
(i) contributed no pairs to \(R\).
\[\frac{1}{2} mn (n+1) + \frac{1}{2}m(m-1)n^2\]
\[=\frac{1}{2}mn\{(n+1)+n(m-1)\}\] \[=\frac{1}{2}mn(mn+1)\] So the
number of pairs in \(R\) is correct.\\
~\\
To show that all the pairs are distinct consider \(P(i,j)\) which
represents those pairs generated\\
from the element \((i,j)\) of \(M\). The pairs within \(P(i,j)\) are always
distinct because they are\\
the pairs in a Room square or a join of 2 MOLS.\\
~\\
However we also need to show that \(P(i,j)\) and \(P(h,k)\) have no pairs in
common when \(i,j \neq h,k\). There are 3 cases to consider. Both sets of
pairs are chosen from the join of 2 M\\
both sets are from Room squares, or one set from each.\\
~\\
If both sets of pairs were generated from the join of two MOLS then the
pairs have the form\[\{un+l_1,vn+l_2\}\] If this pair occurs in both
\(P(i,j)\) and \(P(h,k)\) then \((l_1,l_2)\) occurs in two different places
in\\
joins of two MOLS which is a contradiction.\\
~\\
The case where both sets of pairs are generated by Room squares is
easily dealt with, because\\
the Room squares used to construct \(R\) are based on different sets, so
two could never contain\\
same pair.

\hypertarget{summary}{%
\section{Summary}\label{summary}}

So far we have shown that all Room squares whose side can be expressed
as a prime power\\
\(p^n=2^kt+1\) can be constructed by using the Mullin-Nemeth starters. The
Fermat Primes\\
shown to be an exception to the Mullin-Nemeth construction, but this was
overcome by\\
introducing the theorem of Chong and Chan which provides a strong
starter form all Room squares of side \((2^{2^m}+1)\), encompassing the
Fermat primes. So we have proven that all Room squares exist whose side
is a prime number, other than 3 or 5. The multiplication theorem\\
enables us to state that all Room squares exist whose side can be
factored as \(p_1p_2p_3...p_n\)\\
\(p_i \geq 7\).\\
~\\
The non-existence of Room squares with sides 3 and 5, prevents us from
constructing those\\
squares whose sides have a factor of 3 or 5. Within this class of exempt
Room squares the Mullin \& Nemeth starters will take care of the prime
power sides. But for those whose side is not a prime power a final
theorem, due to W.D. Wallis is needed to complete the proof.

\hypertarget{n-tuplication-of-room-squares}{%
\section{n-tuplication of Room squares}\label{n-tuplication-of-room-squares}}

\hypertarget{theorem-3.8.125}{%
\subsubsection{Theorem 3.8.1{[}25{]}}\label{theorem-3.8.125}}

If \(r\) and \(n\) are odd integers such that \(r \geq n\), and if there is a
Room square \(R\) of side \(r\), then there is a Room square of side \(rn\).\\
~\\
\emph{Example 3.8.1}\\
Before proving this theorem, we look at an example of triplication in
order to introduce thisfairly complicated construction.\\
The approach taken is to take a Room square of side 7, create 9 arrays
very similar in structure tothis Room square, and then arrange these 9
arrays into a 21x21 side array which is very\\
Room square.\\
Suppose we wished to triplicate the following Room square.

\begin{longtable}[]{@{}ccccccc@{}}
\toprule
\(\infty 0\) & - & - & 25 & - & 16 & 34\tabularnewline
\midrule
\endhead
45 & \(\infty 1\) & - & - & 36 & - & 20\tabularnewline
31 & 56 & \(\infty 2\) & - & - & 40 & -\tabularnewline
- & 42 & 60 & \(\infty 3\) & - & - & 51\tabularnewline
62 & - & 53 & 01 & \(\infty 4\) & - & -\tabularnewline
- & 03 & - & 64 & 12 & \(\infty 5\) & -\tabularnewline
- & - & 14 & - & 05 & 23 & \(\infty 6\)\tabularnewline
\bottomrule
\end{longtable}

\textbf{Figure 20}

Unfortunately, this row has simply too many elements. There should be
only 11 pairs, not\\
and the new Room-ish square would be based on a set of 24 elements not
22 as we require.Wallis's original idea had been to somehow merge the
three \(\infty _\mathrm{i}\) into one element, but he eventually decided
instead to go back to the original square and strip out the diagonal
elements.Building a Room square is then a matter of arranging the
following arrays, sometimes called frames,

\$R\_\{ij\} = \$

\begin{longtable}[]{@{}lllclcc@{}}
\toprule
& & & \(2_\mathrm{i}5_\mathrm{j}\) & & \(1_\mathrm{i}6_\mathrm{j}\) & \(3_\mathrm{i}4_\mathrm{j}\)\tabularnewline
\midrule
\endhead
\(4_\mathrm{i}5_\mathrm{j}\) & & & & \(3_\mathrm{i}6_\mathrm{j}\) & & \(2_\mathrm{i}0_\mathrm{j}\)\tabularnewline
\(3_\mathrm{i}1_\mathrm{j}\) & \(5_\mathrm{i}6_\mathrm{j}\) & & & & \(4_\mathrm{i}0_\mathrm{j}\) &\tabularnewline
& \(4_\mathrm{i}2_\mathrm{j}\) & \(6_\mathrm{i}0_\mathrm{j}\) & & & & \(5_\mathrm{i}1_\mathrm{j}\)\tabularnewline
\(6_\mathrm{i}2_\mathrm{j}\) & & \(5_\mathrm{i}3_\mathrm{j}\) & \(0_\mathrm{i}1_\mathrm{j}\) & & &\tabularnewline
& \(0_\mathrm{i}3_\mathrm{j}\) & & \(6_\mathrm{i}4_\mathrm{j}\) & \(1_\mathrm{i}2_\mathrm{j}\) & &\tabularnewline
& & \(1_\mathrm{i}4_\mathrm{j}\) & & \(0_\mathrm{i}5_\mathrm{j}\) & \(2_\mathrm{i}3_\mathrm{j}\) &\tabularnewline
\bottomrule
\end{longtable}

\textbf{Figure 21}

into a 21x21 array, and subsequently finding some way to fill in the
missing two pairs from\\
row of the new square, with the aim of producing a Room square based on
\[S = \{\infty,0_1,1_1,...,6_1,0_2,1_2,...,6_2,0_3,1_3,...,6_3\}\]
Inevitably this approach leads to new problems.\\
~\\
Firstly consider how to arrange the frames appropriately. Suppose we put
\(R_{12}\) next to \(R_{13}\),\\
the left hand members of pairs in each row of \(R_{12}\) will be repeated
in the same row of the\\
21x21 square due to the placing of \(R_{13}\). The same would be true for
any \(R_{ij}\) next to any \(R_{ik}\),\\
next to \(R_{kj}\). So for that reason, in the super-array of \(R_{ij}\)s we
require in each super-row that\\
each value 1,2 and 3 and similarly that \(j\) takes on all these values.
To satisfy the column\\
for our new Room square we also require that no \(R_{ij}\) occurs above or
below an \(R_{ik}\) or \(R_{kj}\), and\\
that reason we must also insist that for any super-column of \(R_{ij}\)s,
\(i\) and \(j\) independently\\
values 1..3, so that each member of \(S \backslash \{\infty\}\) occurs
once in the corresponding 7 columns of\\
finished Room square -- (except for the missing
\(\{x_i: 0 \leq x \leq 6\}\) from all columns \(x_i\)).\\
~\\
Furthermore, as we are aiming for an array in which all the unordered
pairs from \(S\) occur\\
once if we also insist that each value of \(i\) is paired with each value
of \(j\) exactly once in\\
super-array then we should obtain most of these pairs. In fact, because
\(R_{ij} \cup R_{ji}\) contains\\
unordered pairs from \(\{9_i,o_j,1_i,1_j,...,6_i,6_j\}\), except those of
the form \(\{x_i,x_j\}\) \(1 \leq i,j \leq 3\)\\
shall obtain all the unordered pairs of \(S\) except those of the form
\[\{\infty,x_1\},\{\infty,x_2\},\{\infty,x_3\},\{x_1,x_2\},\{x_1,x_3\},\{x_2,x_3\}, 0 \leq x \leq 6\]\\
The ideal solution to this problem (because it solves the problem of
missing pairs as well as completing rows/columns) would be to place
pairs of the form \(\{\infty, x_j\}\) and \(\{x_i, x_j\}\) at the
intersection of rows \(x_j\) and column \(x_j\), but of course this
intersection is a single box, and we don't want two pairs in one box.
Wallis's solution to this problem was to permute the columns of some of
the \(R_{ij}\)s from one super-column of the array of frames with the
intention of arranging it so that the elements \(x_i\) \(x_j\) would be
vacant from some column \(y \neq x_j\). This enables us to put the
\(\{\infty, x_j\}\) in column \(x_j\) and the \(\{x_i,x_j\}\) in column \(y\).\\
~\\
\emph{Example,}\\
The elements missing from row \(0_1,\) \$ \infty, 0\_1, 0\_2, 0\_3\$, are also
missing from column \(0_1\) (due to the removal of the pair \(\{\infty,0\}\)
from the original Room square to create the frame). There is no problem
in putting \(\{\infty,0_1\}\) in position \((0_1,0_1)\), but if we want to
put \(\{0_2,0_3\}\) in row \(0_1\) it must go in some other column of block
\(R_{11}\), while remaining in column \(1_1\) of blocks \(R_{23}\) and
\(R_{32}\), this can be achieved through a column permutation applied only
to \(R_{23}\) and \(R_{32}\).\\
~\\
Notice that it would be of little use to swap columns \(0_1\) and \(4_1\) of
blocks \(R_{23}\) and \(R_{32}\), as the fourth column is occupied already
in the first row of block \(R_{11}\). But we could swap \(0_1\) with any of
\(1_1,2_1,3_1\) or \(5_1\) because all these columns are empty in row \(0_1\).
Clearly the essential property we require of any column permutation that
we decide to use, call it \(\theta\), is that \((x,x\theta)\) is unoccupied
in the original Room square.

\hypertarget{lemma-3.8.1}{%
\subsubsection{Lemma 3.8.1}\label{lemma-3.8.1}}

Given a Room square \(R\) of side \(r\), where \(r=2s+1\), there are \(s\)
permutations \(\phi_1,\phi_2,...,\phi_s\) of \(\{1,2,...,r\}\) with the
properties that \(k\phi_i=k\phi_j\) never occurs unless \(i=j\), and that
cell \((k,k\phi_i)\) is empty for \(1 \leq k \leq r, 1 \leq i\leq s\).\\
~\\
\emph{Proof}\\
We define a matrix \(M\) in the following manner:\\
If position \((k,l)\) is \emph{empty} in \(R\) then the \((k,l)\) position of \(M\)
is 1, otherwise it is 0.\\
Because \(M\) is a matrix of 0s and 1s, whose every row and column sum is
equal to \(s\), it can be decomposed into \(s\) matrices, each of which
having exactly one 1 in each row and column\footnote{This is a theorem, not proven here}. \[M=P_1+P_2+...+P_s\]
These matrices ,when interpreted in the following or similar manner, are
known as\\
\textbf{permutation matrices}:\\
Define \(\phi_i\) as the permutation corresponding to matrix \(P_i\) such
that if \((k,l)\) is 1 in \(P_i\) then \(k\phi _i=l\).\\
The definition of \(M\) ensures that the \((k,k\phi _i)\), so \(M\) would have
an entry equal to 2 or more, contradicting the definition.\\
~\\
\emph{Example} cont\ldots,\\
The matrix \(M\) associated with the square from Figure 19 is:

\[\begin{gathered}
M=
  \begin{bmatrix}
  0 & 1 & 1 & 0 & 1 & 0 & 0\\
  0 & 0 & 1 & 1 & 0 & 1 & 0\\
  0 & 0 & 0 & 1 & 1 & 0 & 1\\
  1 & 0 & 0 & 0 & 1 & 1 & 0\\
  0 & 1 & 0 & 0 & 0 & 1 & 1\\
  1 & 0 & 1 & 0 & 0 & 0 & 1\\
  1 & 1 & 0 & 1 & 0 & 0 & 0\\
  \end{bmatrix}\end{gathered}\]

Which can be decomposed (not uniquely) into these permutation matrices:

\[\begin{gathered}
M= P_1 + P_2 + P_3 = 
  \begin{bmatrix}
  0 & 1 & 0 & 0 & 0 & 0 & 0\\
  0 & 0 & 1 & 0 & 0 & 0 & 0\\
  0 & 0 & 0 & 0 & 1 & 0 & 0\\
  1 & 0 & 0 & 0 & 0 & 0 & 0\\
  0 & 0 & 0 & 0 & 0 & 1 & 0\\
  0 & 0 & 0 & 0 & 0 & 0 & 1\\
  0 & 0 & 0 & 1 & 0 & 0 & 0\\
  \end{bmatrix}
  +
  \begin{bmatrix}
  0 & 0 & 0 & 0 & 1 & 0 & 0\\
  0 & 0 & 0 & 1 & 0 & 0 & 0\\
  0 & 0 & 0 & 0 & 0 & 0 & 1\\
  0 & 0 & 0 & 0 & 0 & 1 & 0\\
  0 & 1 & 0 & 0 & 0 & 0 & 0\\
  0 & 0 & 1 & 0 & 0 & 0 & 0\\
  1 & 0 & 0 & 0 & 0 & 0 & 0\\
  \end{bmatrix}
  +
  \begin{bmatrix}
  0 & 0 & 1 & 0 & 0 & 0 & 0\\
  0 & 0 & 0 & 0 & 0 & 1 & 0\\
  0 & 0 & 0 & 1 & 0 & 0 & 0\\
  0 & 0 & 0 & 0 & 1 & 0 & 0\\
  0 & 0 & 0 & 0 & 0 & 0 & 1\\
  1 & 0 & 0 & 0 & 0 & 0 & 0\\
  0 & 1 & 0 & 0 & 0 & 0 & 0\\
  \end{bmatrix}\end{gathered}\]

The permutations associated with these matrices are, in cycle notation:
\[\phi _1 = (1235674), \phi _2 = (1524637), \phi _3 = (1345726)\] If we
choose to apply \(\phi _1\) to the columns of blocks \(R_{23}\) and \(R{32}\)
we get:

{\textbar c\textbar c\textbar c\textbar c\textbar c\textbar c\textbar c\textbar c\textbar{}}

\begin{longtable}[]{@{}c@{}}
\toprule
\endhead
Col/\tabularnewline
Row\tabularnewline
\bottomrule
\end{longtable}

\& \(0_1\) \& \(1_1\) \& \(2_1\) \& \(3_1\) \& \(4_1\) \& \(5_1\) \& \(6_1\)\\
\(0_1\) \& \& \& \& \(2_15_1\) \& \& \(1_16_1\) \& \(3_14_1\)\\
\(1_1\) \& \(4_15_1\) \& \& \& \& \(3_16_1\) \& \& \(2_10_1\)\\
\(2_1\) \& \(3_11_1\) \& \(5_16_1\) \& \& \& \& \(4_10_1\) \&\\
\(3_1\) \& \& \(4_12_1\) \& \(6_10_1\) \& \& \& \& \(5_11_1\)\\
\(4_1\) \& \(6_12_1\) \& \& \(5_13_1\) \& \(0_11_1\) \& \& \&\\
\(5_1\) \& \& \(0_13_1\) \& \& \(6_14_1\) \& \(1_12_1\) \& \&\\
\(6_1\) \& \& \& \(1_14_1\) \& \& \(0_15_1\) \& \(2_13_1\) \&\\
\(0_2\) \& \(2_35_2\) \& \& \& \(3_34_2\) \& \& \& \(1_36_2\)\\
\(1_2\) \& \& \(4_35_2\) \& \& \(2_30_2\) \& \& \(3_36_2\) \&\\
\(2_2\) \& \& \(3_31_2\) \& \(5_36_2\) \& \& \& \& \(4_30_2\)\\
\(3_2\) \& \& \& \(4_32_2\) \& \(5_31_2\) \& \(6_30_2\)\& \&\\
\(4_2\) \& \(0_31_2\) \& \(6_32_2\) \& \& \& \(5_33_2\) \& \&\\
\(5_2\) \& \(6_34_2\) \& \& \(0_33_2\) \& \& \& \(1_32_2\) \&\\
\(6_2\) \& \& \& \& \& \(1_34_2\) \& \(0_35_2\) \& \(2_33_2\)\\
\(0_3\) \& \(2_25_3\) \& \& \& \(3_24_3\) \& \& \& \(1_26_3\)\\
\(1_3\) \& \& \(4_25_3\) \& \& \(2_20_3\) \& \& \(3_26_3\) \&\\
\(2_3\) \& \& \(3_21_3\) \& \(5_26_3\) \& \& \& \& \(4_20_3\)\\
\(3_3\) \& \& \& \(4_22_3\) \& \(5_21_3\) \& \(6_20_3\) \& \&\\
\(4_3\) \& \(0_21_3\) \& \(6_22_3\) \& \& \& \(5_23_3\) \& \&\\
\(5_3\) \& \(6_24_3\) \& \& \(0_23_3\) \& \& \& \(1_22_3\) \&\\
\(6_3\) \& \& \& \& \& \(1_24_3\) \& \(0_25_3\) \& \(2_23_3\)\\

\textbf{Figure 24}

Which leaves us free to put \(\{\infty, x_1\}\) into \((x_1,x_1)\) and
\(\{x_2,x_3\}\) into \((x_1,(x\phi _1)_1)\)\\
e.g.~\(\{1_2,1_3\}\) can go into \((2_1,(2\phi _1)_1)=(2_1,3_1)\). The
permutation chosen ensures that cell \((2,3)\) of the original square is
empty.\\
~\\
Filling in the rest of block \(R11\) gives:

{\textbar c\textbar c\textbar c\textbar c\textbar c\textbar c\textbar c\textbar c\textbar{}}

\begin{longtable}[]{@{}c@{}}
\toprule
\endhead
Col/\tabularnewline
Row\tabularnewline
\bottomrule
\end{longtable}

\& \(0_1\) \& \(1_1\) \& \(2_1\) \& \(3_1\) \& \(4_1\) \& \(5_1\) \& \(6_1\)\\
\(0_1\) \& \(\infty,0_1\) \& \(0_20_3\) \& \& \(2_15_1\) \& \& \(1_16_1\) \& \(3_14_1\)\\
\(1_1\) \& \(4_15_1\) \& \(\infty, 1_1\) \& \(1_21_3\) \& \& \(3_16_1\) \& \& \(2_10_1\)\\
\(2_1\) \& \(3_11_1\) \& \(5_16_1\) \& \(\infty,2_1\) \& \& \(2_22_3\) \& \(4_10_1\) \&\\
\(3_1\) \& \(3_23_3\) \& \(4_12_1\) \& \(6_10_1\) \& \(\infty,3_1\) \& \& \& \(5_11_1\)\\
\(4_1\) \& \(6_12_1\) \& \& \(5_13_1\) \& \(0_11_1\) \& \(\infty,4_1\) \& \(4_24_3\) \&\\
\(5_1\) \& \& \(0_13_1\) \& \& \(6_14_1\) \& \(1_12_1\) \& \(\infty,5_1\) \& \(5_25_3\)\\
\(6_1\) \& \& \& \(1_14_1\) \& \(6_26_3\) \& \(0_15_1\) \& \(2_13_1\) \& \(\infty,6_1\)\\

\textbf{Figure 25}

Notice that this satisfies the row and column properties of a Room
square for the first seven rows and columns.\\
~\\
Next we move onto the second diagonal block, because missing from row
and column \(x_2\) are the elements \(\infty, x_1,x_2,x_3\). However this
time we try to find a home for pairs of the form \(\{\infty,x_2\}\) and
\(\{x_1,x_3\}\).\\
~\\
We can put pairs of the form \(\{x_1,x_3\}\) down the diagonal and permute
the columns of block \(R_{22}\) with a permutation from Lemma 3.8.1 to
ensure that column \((x\phi _2)_2\) has no \(x_2\), allowing us to put
\(\{\infty,x_2\}\) in that column.\\
~\\
For instance, using \(\phi _2\) we can complete block \(R_{13}\), and the
corresponding seven rows and columns, by putting:

\(\{x_1,x_3\}\) in \(x_2,x_2\)\\
and \(\{\infty,x_2\}\) in \(x_2,(x \phi _2)_2\)\\

Taking the same approach with the third diagonal block we permute
columns in \(R_{33}\) using \(\phi _3\)\footnote{We could have used the same permutation in each case, but the
  general theorem (as will be introduced) insists they are different,
  as higher order multiplications require different permutations.
  Quintuplication, for example, needs at least two.} and fill-in by putting:

\(\{x_1,x_2\}\) in \(x_3,x_3\)\\
and \(\{\infty,x_3\}\) in \(x_3,(x \phi _3)_3\)\\

Which results in a Room square of side 21 based on:
\[S = \{\infty,0_1,1_1,...,6_1,0_2,1_2,...,6_2,0_3,1_3,...,6_3\}\] This
square is straightforwardly transformed to a Room square based on
\(\{\infty,0,1,...,20\}\), Figure 26, by using: \[x_i=x+7(i-1)\] We could
have done this from the beginning, but it is perhaps simpler to keep
track of the missing elements by maintaining the subscript notation.\\
~\\
The preceding triplication was slightly contrived because the
arrangement of frames at the beginning was so chosen because it
satisfied a property as yet unexplained. This property being that the
\(R_{ij}\) and \(R_{ji}\) occur in the same super-column of the array as
frames. Clearly this must be so in order that permutations, when applied
to both or neither of \(R_{ij}\) and \(R_{ji}\), preserve the contents of
the columns as far as is required.\\
~\\
That an arrangement of frames with this property can be guaranteed to
exist for any odd integer is fundamental to the generalised theorem.

\hypertarget{lemma-3.8.2}{%
\subsubsection{Lemma 3.8.2}\label{lemma-3.8.2}}

For all odd \(n\) there exists an array with these properties:

\begin{enumerate}
\def\labelenumi{\arabic{enumi}.}
\item
  the entries of the array consist of all the ordered pairs of the set
  \(N=\{1,2,...,n\}\) once each.
\item
  the entries of a given row or column contain between them every
  member of \(N\) once as a left member and once as a right member.
\item
  if \((x,y)\) occurs in a given column of the array \((y,x)\) also occurs
  in that column.
\end{enumerate}

\emph{Proof}\\
\(A_n\) is an \(n \times n\) array whose \((i,j)\) entry is the ordered pair
\((j-i+1,i+j-1)\) with both elements being reduced modulo \(n\) to lie on
the interval \([1,n]\).

\begin{enumerate}
\def\labelenumi{\arabic{enumi}.}
\item
  There are clearly \(n^2\) ordered pairs obtainable from \(N\). \(A_n\) has
  \(n^2\) cells, so it is only necessary to show that each cell contains
  a unique pair. For that reason consider any two pairs from different
  cells, \((x_1,y_1)\) and \((x_2,y_2)\), for these to be equal requires
  both,

  \(x_1=x_2\) and \(y_1=y_2\)

  Now, \(x_1 = x_2\) \[\Rightarrow j_1-i_1 + 1 = j_2-i_2+1\] While,
  \(y_1=y_2\) \[\Rightarrow i_1+j_1 - 1 = i_2+j_2-1\] Together these
  imply, \[j_1-i_1=j_2-i_2 \hspace{1cm}(1)\]
  \[i_1+j_1=i_2+j_2 \hspace{1cm}(2)\] (2) gives, \(j_2=i_1+j_1-i_2\),
  which on substitution in (1) gives, \[j_1-i_1=i_1+j_1-2i_2\]
  \[\Rightarrow 2i_1-2i_2=0\] \[\therefore i_1=i_2\] Substituting this
  into either expression gives \(j_1 = j_2\).\\
  Thereby contradicting the assumption that the pairs occurred in
  different cells. Hence every cell contains a unique pair, so all the
  ordered pairs from \(N\) occur exactly once in \(A_n\).
\item
  Consider row \(i\) of \(A_n\):

  \begin{longtable}[]{@{}ccccc@{}}
  \toprule
  \(j=1\) & \(j=2\) & \ldots{} & \(j=-1\) & \(j=0\)\tabularnewline
  \midrule
  \endhead
  \((2-i,i)\) & \((3-i,i+1)\) & \ldots{} & \((-i,i-2)\) & \((1-i,i-1)\)\tabularnewline
  \bottomrule
  \end{longtable}

  Left hand members are \(\{2-i,3-i,...,1-i\}\), while the right hand
  members are \(\{i,i+1,...,i-1\}\). Both sets contain \(n\) unique
  integers on the interval \([1,n]\) and hence both sets must be \(N\),
  Similarly, consider column \(j\), the left hand positions are occupied
  by \(\{j,j-1,...,j+2,j+1\}\), while the right contain
  \(\{j,j+1,...,j-2,j-1\}\). For the same reasons both these sets are
  equal to \(N\).
\item
  Consider a pair \((x,y)\), from the definition of \(A_n\),\\
  \(x=j-i+1\)\\
  \(y=i+j-1\)\\
  \(x+y=j-i+1+i+j-1=2j\)\\
  \(\therefore j= \frac{1}{2} (x+y)\)\\
  So if \((x,y)\) is in column \(j\) of \(A_n\) then so is \((y,x)\), since
  \(\frac{1}{2}(x+y)=\frac{1}{2}(y+x)\).
\end{enumerate}

We can now present the general result.

\hypertarget{theorem-3.8.1-25}{%
\subsubsection{Theorem 3.8.1 {[}25{]}}\label{theorem-3.8.1-25}}

If \(r\) and \(n\) are odd integers such that \(r \geq n\), and if there is a
Room square \(R\) of side \(r\), then there is a Room square of side \(rn\).\\
~\\
\emph{Proof}\\
Let \(r=2d+1\) and \(n=2t+1\).\\
For a given \(i\) select \(n\) permutations as follows:

\begin{enumerate}
\def\labelenumi{\arabic{enumi}.}
\item
  \(\phi _{jk} = \phi _{jl}\) if, and only if, \((k,l)\), and \((l,k)\)
  appear in column \(j\) of \(A_n\).
\item
  If cell \((j,j)\) of \(A_n\) contains \((x,y)\) then
  \(\phi _{jx}=\phi _{jy}=id\) (the identity permutation).
\item
  All the \(\phi _{jk} (\neq id)\) are selected from the permutations
  associated with \(R\), according to Lemma 3.8.1
\end{enumerate}

Now a Room square of side \(rn\) is constructed by replacing each entry
\((k,l)\) of \(A_n\) by \(R_{kl} \phi _{jk}\) (the array \(R_{kl}\) under the
column permutation \(\phi _{jk}\)), where \((k,l)\) is in column \(j\) of
\(A_n\).\\
~\\
The resulting array has each element of,
\[S = \{0_1,1_1,...,(r-1)_1,0_2,1_2,...,(r-1)_2,0_n,1_n,...,(r-1)_n\}\]
appearing exactly once in each row and column, except that \(x_j\) is
missing from row and column \(x_j,\)
\(1 \leq j \leq n \hspace{0.4cm} 0 \leq x \leq (r-1)\), and \(x_k\) and
\(x_l\) are missing from column \((x \phi _{jk})_j\) for each entry \((k,l)\)
in column \(j\) of \(A_n\).\\
~\\
The array also contains every unordered pair from \(S\) exactly once,
except those of the form \(\{x_k,x_l\}\).\\
Now, for each \(k\), if \((k,l)\) is an entry of column \(j\) of \(A_n\) put
\(\{x_k,x_l\}\) in \((x_j,(x \phi _{jk})_j)\), using \(\{\infty, x_j\}\)
instead of \(\{x_j,x_j\}\) in every case.\\
~\\
The completed array contains each of \(\{\infty\} \cup S\) exactly once
per row and column and every unordered pair from the same set exactly
once.\\
~\\
Finally, map \(\{\infty\} \cup S\) onto \(\{\infty \} \cup Z_{rn}\) by
replacing every \(x_i\) by \(x+r(i-1)\). The finished array is a Room square
of side \(rn\).\\
~\\
\emph{Example}\\
Looking back at the previous example of triplication.

\$A\_3 = \$

\begin{longtable}[]{@{}lccc@{}}
\toprule
& 1 & 2 & 3\tabularnewline
\midrule
\endhead
1 & 1,1 & 2,2 & 3,3\tabularnewline
2 & 3,2 & 1,3 & 2,1\tabularnewline
3 & 2,3 & 3,1 & 1,2\tabularnewline
\bottomrule
\end{longtable}

Now, in column 1 (3,2) and (2,3) both appear, so
\[\phi _{13} = \phi _{12}\] While in column 2 occur both (1,3) and
(3,1), so \[\phi _{21} = \phi _{23}\] Furthermore, both (2,1) and (1,2)
appear in the third column, so \[\phi _{32} = \phi _{31}\] The diagonal
pairs are (1,1),(1,3),(1,2) so
\(\phi _{11}=\phi _{21}=\phi _{23}=\phi _{31}=\phi _{32}=id\)\\
The remaining permutations,
\(\phi _{12},\phi _{13},\phi _{22},\phi _{33}\) are chosen according to
the Lemma 3.8.1 and the following array is the Room square in Fig 26
once the missing pairs have been placed and the transformation to
\(\{\infty, 0, 1, ..., 20\}\) made.

\begin{longtable}[]{@{}ccc@{}}
\toprule
\(R_{11}\phi _{11}\) & \(R_{22}\phi _{22}\) & \(R_{33}\phi _{33}\)\tabularnewline
\midrule
\endhead
\(R_{32}\phi _{13}\) & \(R_{13}\phi _{21}\) & \(R_{21}\phi _{32}\)\tabularnewline
\(R_{23}\phi _{12}\) & \(R_{31}\phi _{23}\) & \(R_{12}\phi _{31}\)\tabularnewline
\bottomrule
\end{longtable}

\(=\)

\begin{longtable}[]{@{}ccc@{}}
\toprule
\(R_{11}id\) & \(R_{22}\phi _{2}\) & \(R_{33}\phi _{3}\)\tabularnewline
\midrule
\endhead
\(R_{32}\phi _{1}\) & \(R_{13}id\) & \(R_{21}id\)\tabularnewline
\(R_{23}\phi _{1}\) & \(R_{31}id\) & \(R_{12}id\)\tabularnewline
\bottomrule
\end{longtable}

\hypertarget{an-existence-theorem-for-room-squares}{%
\chapter{An Existence Theorem for Room Squares}\label{an-existence-theorem-for-room-squares}}

Apparently missing.

\hypertarget{balanced-room-squares}{%
\chapter{Balanced Room Squares}\label{balanced-room-squares}}

\hypertarget{bibds-and-brbs}{%
\section{BIBDs and BRBs}\label{bibds-and-brbs}}

A \textbf{\emph{balanced incomplete block design (BIBD)}} is an arrangement of
elements (varieties) from a set \(S\) of size \(v\), into \(b\) subsets
(blocks) each of size \(k\) such that each variety occurs in \(r\) blocks
and any particular pair of distinct varieties occurs in \(\lambda\)
blocks.

\hypertarget{example-5.1.1}{%
\subsubsection{Example 5.1.1}\label{example-5.1.1}}

\[\{a,b,c\},\{a,b,d\},\{a,c,e\},\{a,d,f\},\{a,e,f\},\{b,c,f\},\{b,d,e\},\{b,e,f\},\{c,d,e\},\{c,d,f\}\]
This is a BIBD with parameters: \(v = 6\), \(b = 10\), \(k = 3\), \(r = 5\),
\(\lambda = 2\).

In any \(BIBD(v, b, r, k, \lambda)\) there are \(b\) blocks each containing \(k\)
elements, so \(bk\) elements in total. Also, each of the \(v\) elements
occurs \(r\) times, so there are \(rv\) elements in total. Therefore:
\[bk = vr\] Also in a particular block, any element makes a pair with
\(k - 1\) other elements, and each element occurs in \(r\) blocks, so there
are \(r(k - 1)\) pairs involving that element. Also, by definition, that
element is paired with each of the other \(v - 1\) members of \(S\) \(\lambda\)
times, so: \[\lambda (v - 1) = r(k - 1)\] The second expression rearranges
to give \(r = \lambda (v - 1)/(k - 1)\), which can be substituted into the first
expression to give \(b=v\lambda (v - 1)/k(k - 1)\). So \(b\) amd \(r\) are both
determined by the values of \(v\), \(k\) and \(\lambda\). For this reason we
need only quote those three parameters when referring to a particular
block design. The example was a \(BIBD(6, 3, 2)\) (or \((6, 3, 2)\)design).

In any Room square, the pairs \(\{x,y\}\) are unordered. If we replace
these pairs by one of the ordered pairs \((x,y)\) or \((y,x)\) we call the
resulting array an \textbf{\emph{ordered Room square (ORS)}}.

Consider the following ordered Room square:

\begin{longtable}[]{@{}ccclcll@{}}
\toprule
\(\infty 0\) & 62 & 54 & & 31 & &\tabularnewline
\midrule
\endhead
& \(\infty 1\) & 03 & 65 & & 42 &\tabularnewline
& & \(\infty 2\) & 14 & 06 & & 53\tabularnewline
64 & & & \(\infty 3\) & 25 & 10 &\tabularnewline
& 05 & & & \(\infty 4\) & 36 & 21\tabularnewline
32 & & 16 & & & \(\infty 5\) & 40\tabularnewline
51 & 43 & & 20 & & & \(\infty 6\)\tabularnewline
\bottomrule
\end{longtable}

\textbf{Figure 27}

Suppose we extract the blocks of a design (not necessarily a BIBD) from
this square in one of the two following ways:

\begin{enumerate}
\def\labelenumi{\arabic{enumi}.}
\item
  As \emph{half-columns}. This means taking all left hand members of pairs
  from a particular column as the members of one block, and all right
  hand members as another block.
\item
  By \emph{half-rows}. Take all left members from a particular row as one
  block, and all right members as another block.
\end{enumerate}

Notice that for this example, both methods generate exactly the same
blocks.
\[\{\infty,6,5,3\},\{0,2,4,1\},\{\infty,0,6,4\},\{1,3,5,2\},\{\infty,1,0,5\},\{2,4,6,3\},\{6, \infty,2,1\},\]
\[\{4,3,5,0\},\{0,\infty,3,2\},\{5,4,6,1\},\{3,1,\infty,4\},\{2,6,5,0\},\{5,4,2,\infty\},\{1, 3,0,6\}\]
Certainly this design seems to have the appropriate parameters to
qualify as a BIBD. The original Room square was based on 8 elements,
hence \(v=8\). Also, the Room square's seven rows each produced 2 blocks
of 4 elements, hence \(b=14\) and \(k=4\). Finally, each element occurred
once in each row of the Room square, therefore in half the total number
of blocks, hence \(r=7\). \[bk=4\cdot 14 = 7 \cdot 8 = vr\] So this design
is a BIBD, provided: \[\lambda = \frac{r(k-1)}{v-1} = \frac{7(3)}{7}=3\]
In other words, provided each pair of elements occurs in precisely 3
blocks.

By definition a \textbf{\emph{balanced Room square {[}\(BRS(n)\){]}}} is an ordered Room
square based on \(n\) elements whose corresponding block design, derived
as above, is a BIBD.

\hypertarget{complete-balanced-howell-rotations}{%
\section{Complete balanced Howell rotations}\label{complete-balanced-howell-rotations}}

Edwin C. Howell is an enigmatic figure in the history of mathematics.
The rotations named after him are designs for the scheduling of Bridge
tournaments.

In a duplicate bridge tournament players compete in partnerships, two
partnerships at a table. At the beginning of a round each table is given
one from a certain number of duplicate boards each of which contains a
pack of cards dealt evenly into four pockets. The boards are labelled
north-south-east-west, and are aligned on the table so that one
partnership plays north-south and the other east-west. After the game is
finished the cards are returned to the pockets as they were dealt and
the duplicate board is returned, to be used again in subsequent
matches.

If one partnership plays directly against another at the same table, the
two partnerships are said to \emph{oppose} each other. If two partnerships
play in the same direction on the same board in different rounds, they
are said to \emph{compete}.

In scoring, not only is the performance of teams in opposition
considered, but also the performance of partnerships which compete.

A good tournament design would have the properties that each partnership
opposed each other partnership at the same table exactly once and also
that each partnership competed against each other partnership at the
same number of times.

A ****complete balanced Howell rotation {[}\(CBHR(n)\){]}****\footnote{This definition is extracted in its entirety from {[}8{]}} is an array
(based on \(n\) elements) of side \(s\), where \(s=n\) for \(n\) odd, and
\(s=n-1\) for \(n\) even, which satisfies the following properties:

\begin{enumerate}
\def\labelenumi{\arabic{enumi}.}
\item
  Each of the \(s^2\) cells is empty or contains an ordered pair of
  distinct elements.
\item
  Each of the \(n\) elements appears exactly once in each row and each
  column. (If \(n\) is odd, then one row and one column is excepted)
\item
  Each unordered pair of distinct elements occurs in exactly one cell
  of the array.
\item
  Each pair of distinct elements appears together in a block\footnote{Where the blocks are derived in the same way as for a BRS}
  exactly \(\left \lfloor{n/2}\right \rfloor -1\) times, where
  \(\left \lfloor{x}\right \rfloor\) means the integral part of \(x\).
\end{enumerate}

To interpret a CBHR as a duplicate bridge tournament schedule, we
represent the partnerships by elements, the boards ny rows and the
rounds of the tournament by columns.

Then an ordered pair \((x,y)\) in position \((i,j)\) of the array
corresponds to partnership \(x\) opposing partnership \(y\) on board \(i\) in
round \(j\). We adopt the convention that \(x\) plays NS, \(y\) EW. The third
property in the definition of a CBHR ensures that all partnerships are
in opposition exactly once, while the fourth (with blocks of the
associated BIBD representing partnerships in competition) ensures that
each partnership competes against each other partnership the same number
of times.

Notice that when \(n\) is even the definition of a CBHR is precisely that
of a balanced Room square. So the BRS above is also a \(CBHR(8)\). The
reason for maintaining both definitions is that a CBHR is, according to
conditions 2 and 4, allowed to have odd order. Also, an important
construction for \(BRS(4n)\), which will be introduced, involves two
\(CBHR(2n-1)\)s.

Unlike Room squares the question of the existence of balanced Room
squares is far from resolved, but various proofs have shown there to be
many infinite classes of these designs. The details of some of these
proofs are given below.

Interestingly the existence problem for balanced Room squares (then
known exclusively as CBHR) was brought to the attention of
mathematicians, in 1955 by Parker and Mood, the very same year as T.G.
Room published his original article. Either this problem has proved to
be very much more difficult, or has aroused significantly less interest.
The first general result, that of Hwang (1970), coming only a few years
before the corresponding problem for Room squares was settled (1973).

In order to discuss these results the first step is to make the
necessary adaptation of the starter-adder approach.

\hypertarget{starters-and-adders-for-brs-and-cbhr}{%
\section{Starters and adders for BRS and CBHR}\label{starters-and-adders-for-brs-and-cbhr}}

In deriving the definitions of a starter and adder for a Room square,
the terminology of difference systems was introduced. We saw that a
requirement of the starter was that each non-zero member of the relevant
Galois field occurred exactly once as a difference between the members
of some pair. This was so that each pair occurred exactly once in the
Room square\\
~\\
BIBDs can be constructed from difference systems in much the same way,
but with the difference that each element in the BIBD occurs occurs with
each other element \(\lambda\) times, not necessarily just once.\\
~\\
\emph{Example}\\
A BIBD on \(GF(7)\) can be constructed from the sets
\(\{6,5,3\}\{2,4,1\}\).\\
~\\
\(\{0,6,4\},\{1,0,5\},\{2,1,6\},\{3,2,0\},\{4,3,1\},\{5,4,2\}\) are the
blocks obtained from the left hand set.\\
\(\{3,5,2\},\{4,6,3\},\{5,0,4\},\{6,1,5\},\{0,2,6\},\{1,3,0\}\) are the
blocks obtained from the right hand set.\\
~\\
Because the two sets are both triples, under cyclic construction the
block design obtained will necessarily have \(k=r=3\).\\
~\\
The left hand set has each non-zero member of \(GF(7)\) occurring exactly
once as a difference between its members. So, taken with its translates
(those blocks obtained from it under cyclic development), this set
should form a BIBD with \(r=3\) and \(\lambda=1\). Similarly the right hand
block forms a BIBD with \(r=3\) and \(\lambda=1\), and the two sets taken
together as difference system therefore generate a BIBD with \(r=6\) and
\(\lambda =2\).\\
~\\
If we add the ideal element \(\infty\) to the left hand set and all its
translates, and 0 to the right hand set (developing it cyclically along
with the other members) then we obtain a new BIBD with \(r=7\) and
\(\lambda = 3\), whose blocks are:
\[\{\infty,6,5,3\},\{\infty,0,6,4\},\{\infty,1,0,5\},\{\infty,2,1,6\},\{\infty,3,2,0\},\{\infty,4,3,1\},\{\infty,5,4,2\}\]
\[\{0,2,4,1\},\{1,3,5,2\},\{2,4,6,3\},\{3,5,0,4\},\{4,6,1,5\},\{5,0,2,6\},\{6,1,3,0\}\]
Which is precisely the BIBD obtained from the Room square at the
beginning of this chapter. If the block design associated with a BRS, or
a CBHR, is obtained by taking the left hand member of the pairs in the
first row as one block and the left hand members in subsequent rows as
the translates, and all right members of pairs as the remaining blocks
of a BIBD, then clearly those two blocks, the left hand and the right
hand, which belong to the starter must form a difference system.\\
~\\
The set of pairs \((x_1,y_1),(x_2,y_2),...,(x_{n-1},y_{n-1})\) is a
\textbf{\emph{balanced starter}} in \(G=GF(2n-1)\) if

\begin{enumerate}
\def\labelenumi{\arabic{enumi}.}
\item
  the unordered pairs \(\{x_i,y_i\}\) form a starter in \(G\), and
\item
  the blocks \(\{x_1,x_2,...,x_{n-1}\}\) and \(\{y_1,y_2,...,y_{n-1}\}\)
  form a difference system.
\end{enumerate}

A balanced starter is \textbf{\emph{strong}} if
\(x_1+y_1,x_2+y_2,...,x_{n-1}+y_{n-1}\) are all distinct mod\((2n-1)\).

\hypertarget{theorem-5.3.1}{%
\subsubsection{Theorem 5.3.1}\label{theorem-5.3.1}}

Given a strong balanced starter on \(G=GF(2n-1)\), then a \(CBHR(2n-1)\)
exists.\\
\emph{Proof}\\
Assign the pair \((x_i+g,y_i+g)\) to cell \((g,x_i+y_i+g)\) for all
\(g \in G\). Condition (i) along with the strong-ness of the starter
ensures that every \(g \in G\) occurs in every row and column exactly once
except \(g\) does not occur in row or column \(g\) for all \(g \in G\) {[}due to
the absence in cell (0,0) of the pair \(\{\infty,0\}\){]}. Condition (i)
also ensures that each unordered pair of \(G\) occurs exactly once in the
array, replace the unordered pairs with ordered pairs.\\
That the block design obtained is balanced follows from condition (ii).\\
Notice that each of the 2 blocks generates \((n-1)(n-2)\) differences and
these represent each of \(2n-2\) members of \(G \backslash \{0\}\),
\(\lambda\) times. Therefore \(2(n-1)(n-2) = \lambda (2n-2),\)
\$ \Rightarrow \lambda = n-2\$.

\hypertarget{theorem-5.3.2}{%
\subsubsection{Theorem 5.3.2}\label{theorem-5.3.2}}

Given a strong balanced starter on \(G=GF(2n-1)\), then a \(BRS(2n)\)
exists.\\
\emph{Proof}\\
Assign the pair \((x_i+g,y_i+g)\) to cell \((g,x_i+y_i+g)\) for all
\(g \in G\), and (provided \(n\) is even), also assign the pair \((\infty,g)\)
to cell \((g,g)\) for all \(g \in G\).\\
Again, condition (i) and the strong-ness of the starter ensure that an
ORS based on \(G \cup \{\infty\}\) is obtained. The block design
associated with this array has initial blocks,
\[\{\infty\} \cup \{x_1,x_2,...,x_{n-1}\}\]
\[\{0\} \cup \{y_1,y_2,...,y_{n-1}\}\] Adjoining 0 to
\(\{y_1,y_2,...,y_{n-1}\}\) creates each non-zero member of \(G\) as a
difference once more, either as \(y_i-0\) or \(0-y_i\).\\
Adjoining \(\infty\) to \(\{x_1,x_2,..,x_{n-1}\}\) creates \(n-1\) pairs
involving \(\infty\) in this block, hence \(\infty\) makes a pair with each
member of \(G\) \(n-1\) times.\\
So the design is balanced with a concurrence number of \(n-1\).\\
~\\
It is a simple matter to derive the remaining parameters of the BIBD
associated with either array.

\begin{longtable}[]{@{}lccccc@{}}
\toprule
& \(v\) & \(b\) & \(r\) & \(k\) & \(\lambda\)\tabularnewline
\midrule
\endhead
\(CBHR(2n-1)\) & \(2n-1\) & \(2(2n-1)\) & \(2(n-1)\) & \(n-1\) & \(n-2\)\tabularnewline
\(CBHR(2n)/BRS(2n)\) & \(2n\) & \(2(2n-1)\) & \(2n-1)\) & \(n\) & \(n-1\)\tabularnewline
\bottomrule
\end{longtable}

\textbf{Table 7 BIBD parameters for BRS and CBHR}

The block design of a balanced Room square is \emph{self-complementary}. This
means that if a block \(D\) belongs to the design, then its complement
\(\overline{D}\) also appears. If the left hand pairs in row \(x\) form one
block then the right hand pairs in the same row are also a block, and
the two blocks are complementary. Alternatively we can say that, in a
self-complementary BIBD, the complementary design (obtained by replacing
all blocks with their complements) is identical to the design itself.\\
Schellenberg proved an interesting result regarding self-complementary
BIBDs.

\hypertarget{theorem-5.3.3}{%
\subsubsection{Theorem 5.3.3}\label{theorem-5.3.3}}

{[}21{]} In a self-complementary BIBD with parameters of the form,\\
\((2n,2(2n-1)t,(2n-1)t,n,(n-1)t)\), every triple of elements is contained
in \(t(n-2)/2\) blocks.\\
~\\
\emph{Proof}\\
Suppose \(B\) is a self-complementary BIBD with parameters of the form
above, based on a set of elements \(V\). Denote by \(S_{i...j}\) the set of
blocks which contain \(u\) but not \(v\) or \(w\). Because the design is
self-complementary to each block of this set there corresponds a unique
block which contains \(v\) and \(w\) but not \(u\). The set of these blocks is
\(S_{vw}-S_u\) and clearly: \[|S_u - \{S_v \cup S_w\}| = |S_{vw} - S_u|\]
Now,\\
\(|S_u-\{S_v \cup S_w\}| = |S_u| - |S_{uv}| -|S_{uw}| + |S_{uvw}|\) and,\\
\(|S_{vw} - S_u| = |S_{vw}| - |S_{uvw}|\)
\[\therefore |S_u| - |S_{uv}| -|S_{uw}| + |S_{uvw}| = |S_{vw}| - |S_{uvw}|\]
\[2|S_{uvw}| = |S_{vw}| + |S_{uv}| + |S_{uw}| - |S_{U}|\] Because \(B\) is
a \(BIBD\), each pair of elements occur together in \(\lambda = (n-1)t\)
blocks and each element occurs in \(r=(2n-1)t\) blocks. So,
\[|S_{vw}| = |S_{uv}| = |S_{uw}| = (n-1)t\] and \[|S_u| = (2n-1)t\]
Therefore, \[2|S_{uvw}|=3(n-1)t-(2n-1)t=(n-2)t\] and so,
\[|S_{uvw}| = \frac{(n-2)t}{2}\] Since \(u\), \(v\) and \(w\) are arbitrary
this implies that each triple occurs in \((n-2)t/2\) blocks. \(\square\)\\
~\\
The implication of this result for balanced Room squares is that,
because \(t=1\) for a \(BRS\)\footnote{The \(t=1\) case was proven by Stanton and Sprott (1964) and Parker
  (1963) prior to Schellenberg.} and we require the LHS of this expression
to be an integer, \(n\) is necessarily even. So writing \(n=2m\), we know
the order of a \(BRS\) is always of the form \(4m\).

\hypertarget{corollary-5.3.1}{%
\subsubsection{Corollary 5.3.1}\label{corollary-5.3.1}}

A \(BRS(n)\) can only exist for \(n \equiv 0\) mod 4.\\
~\\
We now present Hwang's starter-adder construction for balanced Room
squares.

\hypertarget{theorem-5.3.4}{%
\subsubsection{Theorem 5.3.4}\label{theorem-5.3.4}}

{[}14{]} There exists a \(BRS\) of order \(q+1\), where \$q=p\^{}r \equiv \$ 3(mod 4)
is a prime power strictly greater than 3.\\
~\\
\emph{Proof}\\
We show that the pairs
\[X = \left \{(x^{2i+1},x^{2i}): 0 \leq i \leq \frac{q-3}{2} \right \}\]
form a balanced starter, where \(x\) is a primitive element in \(GF(q)\),
and the set
\[A(X) = \left \{-x^{2i}(1+x): 0 \leq i \leq \frac{q-3}{2} \right \}\]
is a corresponding adder.\\
Looking back at Theorem 2.4(?), we have already shown that the unordered
pairs
\[\left \{\{x^{2i},x^{2i+1}\}: 0 \leq i \leq \frac{q-3}{2} \right \}\]
are those of a starter with adder \(A(X)\). So it remains to show that the
starter is balanced which involves proving that the two blocks
\[B_1 = \{0,x,x^3,...,x^{q-2}\} \hspace{0.5cm} \mathrm{and} \hspace{0.5cm} B_2 = \{\infty,x^0,x^2,...,x^{q-3}\}\]
generate a \(BIBD\). This result is due to R.C. Bose and makes use of the
properties of the squares and non-squares of \(GF(q)\) where \(q\) is an odd
prime.\\
Let \(R\) and \(N\) denote the sets of non-zero square and non-squares in
\(GF(q)\), respectively. \[R=\{x^2,x^4,...,x^{q-1}\}\]
\[N=\{x^1,x^3,...,x^{q-2}\}\] These sets both contain precisely
\(\frac{1}{2}(q-1)\) elements.\\
So, \[B_1 = \{0\} \cup N,\] and because \(x^{q-1} = 1 = x^0\),\\
\[B_2 = \{\infty\} \cup R\] Also, if \(a\) is square then \(-a\) is a
non-square\footnote{because \(q \equiv 3\)(mod 4)}, which implies that \(R=-N\).\\
~\\
To show that the blocks with their translates form a \(BIBD\) it is
necessary to show that the differences between members of \(B_1\) and \(R\)
generate all the non-zero members of \(GF(q)\) some number of times, then
by adjoining the element \(\infty\) to \(R\) and its translates will
generate a \(BIBD\).\\
Firstly, suppose that 1, which belongs to \(R\), can be expressed as a
difference between members of \(R\) in a certain number of different ways,
\[1 = x^{2a_1}-x^{2b_1}=...x^{2a_r}-x^{2b_r}\] Were this true, any
member of \(R\) could be written as a difference in the same number of
ways.\\
Multiply through by any \(x^{2s} \in R\).
\[x^{2s} = x^{2(a_1+s)}-x^{2(b_1+s)}=...=x^{2(a_r+s)}-x^{2(b_r+s)}\] But
now suppose that any element of \(R\) can be expressed as a difference
between members of \(R\) in a certain number of ways, i.e.~assume this
second expression holds. Then by dividing through by \(x^{2s}\), we
recover the first expression and hence for each representation of 1
there is a corresponding representation of \(x^{2s}\), and vice versa. So
there are an equal number of representations of every member of \(R\) as a
difference of members of \(R\). The remaining non-zero members of \(GF(q)\),
are all those \(q \notin R\) where \(-q \in R\), but each representation of
\(-r\) gives a corresponding representation of \(r\): \[-r=x^{2a}-x^{2b}\]
\[r = x^{2b} - x^{2a}\] So every element of \(-r\) has the same number of
representations as a difference between members of \(R\) as \(R\) does.\\
~\\
We know that \(R\) has \(\frac{1}{2}(q-1)\) elements, therefore there are\\
\(\frac{1}{2}(q-1) \cdot \frac{1}{2} (q-3) = \frac{1}{4}(q-1)(q-3)\)
differences between those elements, and if each of these differences
generates each of the \(q-1\) non-zero members of \(GF(q)\), \(\lambda\) times
then: \[\lambda(q-1)=\frac{1}{4}(q-1)(q-3)\]
\[\lambda=\frac{1}{4}(q-3)\] Also, because \(N=-R\), we can say that all
the non-zero members of \(GF(q)\) arise as differences between members of
\(N\). Further, \(B_1\) also gives the differences
\(0-n,n-0 \hspace{0.5cm} n\in N\), which are all the non-zero members of
\(GF(q)\), once again. So in total every non-zero member of \(GF(q)\) occurs
as a difference between elements of \(B_1\) and \(R\),
\(2 \cdot \frac{1}{4} (q-3) + 1 = \frac{1}{2} (q-1)\) times.\\
Because \(R\) and its translates each contain \(\frac{1}{2}(q-1)\)
elements, each member of \(GF(q)\) occurs \(\frac{1}{2}(q-1)\) times in all
those blocks. Therefore, adjoining \(\infty\) to each block generates
blocks of size \(\frac{1}{2}(q-1)\), with \(\infty\) making a pair with
each member of \(GF(q)\), \(\frac{1}{2}(q-1)\) times.\\
So we can say that \(B_1\), \(B_2\) and their translates for a
\(BIBD(q+1,\frac{1}{2}(q+1), \frac{1}{2}(q-1))\).\\
Therefore Hwang's starter is balanced. \(\square\)\\
~\\
The \(BRS\) in Figure 27 was obtained from Hwang's starter, hence its
block design was a \(BIBD\).\\
~\\
\emph{Example 5.3.1}\\
A balanced starter in \(GF(19)\) is:
\[X = \{(2,1),(8,4),(13,16),(14,7),(18,9),(15,17),(3,11),(12,6),(10,5)\}\]
Which has a corresponding adder: \[A(X) = \{16,7,9,17,11,6,5,1,4\}\] So
the following row is an appropriate choice for a \(BRS(20)\):

\(\infty,0\) - 14,7 2,1 15,7 - - - 18,9 - 13,16 - 8,4 - 3,11 10,5 - - 12,6
------------ --- ------ ----- ------ --- --- --- ------ --- ------- --- ----- --- ------ ------ --- --- ------

\hypertarget{a-multiplicative-construction-for-brs}{%
\section{A multiplicative construction for BRS}\label{a-multiplicative-construction-for-brs}}

Although the Hwang starter-adder construction establishes the existence
of an infinite class of \(BRS\), like Mullin \& Nemeth's construction for
Room squares, exceptions still remain.\\
~\\
Particularly all those \(BRS\) of order \(q+1\) where q is not a prime
power, yet is congruent 3 modulo 4. For example, the existence of
\(BRS(16)\) is not established yet, because 15 is not a prime power. One
approach to resolving this problem would be to establish a
multiplication theorem like those established for Room squares. The
following doubling construction for \(BRS\) (due to Schellenberg), for
example, enables to construction of a \(BRS(16)\) from two \(BRS(8)\).\\
~\\
Significantly, used along with Hwang's starter-adder construction, it
establishes the existence of another infinite class of \(BRS\), those of
order \(2(q+1)\), where \(q\) is subject to the conditions of Theorem 3.5.\\
~\\
Before presenting the doubling construction a few definitions need to be
made, and another result regarding self-complementary block designs
established.\\
~\\
The \textbf{\emph{reduced}} Room square \(\hat{R}\), is obtained by taking a
standardised Room square \(R\) and removing the pairs involving \(\infty\),
i.e.~the diagonal pairs.\\
~\\
Two \(ORS\), \(R\) and \(S\) (both of side \(a\)) are said to be a \textbf{\emph{Latin
pair}} if on forming the join of \(\hat{R}\) and \(\hat{S}\), and placing
the pair \((i,i)\) in cell \((i,i)\) for \(0 \leq i \leq q-1\) the join of two
\(MOLS\) is obtained, denoted by \(R \odot S\).\\
~\\
A \textbf{\emph{common traversal}} of \(R \odot S\) is a set of \(q\) cells, one from
each row and each column, whose \(n\) ordered pairs have \(n\) distinct
first elements and \(n\) distinct second elements.\\
~\\
Suppose we have a set of varieties \(V = \{1,2,3,...\}\), then by a set
\(V'\), we mean \(V'=\{1',2',3',...\}\).

\hypertarget{lemma-5.4.1}{%
\subsubsection{Lemma 5.4.1}\label{lemma-5.4.1}}

{[}21{]} If
\[\bigcup\limits_{i=1}^{2k-1} \left \{C_i,\bar{C_i} \right \} \hspace{0.5cm} \mathrm{and} \hspace{0.5cm} 
\bigcup\limits_{i=1}^{2k-1} \left \{D_i,\bar{D_i} \right \}\] are the
blocks of two self-complementary \(BIBD\)s, both defined on a set of
elements \(V\), where \(\bar{C}\) denotes the complement of \(C\), with
parameters \((2k,2(2k-1),2k-1,k,k-1)\) then.
\[\bigcup\limits_{i=1}^{2k-1} \left \{C_i \cup D'_i, \bar{C_i} \cup \bar{D'_i}, C_i \cup \bar{D'_i},
\bar{C_i} \cup D'_i  \right \} \cup \{V,V'\}\] is the set of blocks of a
self-complementary \(BIBD\), defined on the set of elements \(V \cup V'\),
with parameters \((4k,2(4k-1),4k-1,2k,2k-1)\).\\
~\\
\emph{Proof}\\
Consider an arbitrary pair \(\{a,b\}\) in the new \(BIBD\) where
\(a,b \in V\). The concurrence number of the original \(BIBD\) was \(k-1\), so
this pair occurred \(k-1\) times in the blocks
\(\bigcup\limits_{i=1}^{2k-1} \left \{C_i,\bar{C_i} \right \}\), but these
blocks clearly appear twice each in the new \(BIBD\). Also the pair
\(\{a,b\}\) occurs once in the block \(V\), so the pair \(\{a,b\}\) occurs
\(2(k-1)+1 = 2k-1\) times. The same would be true for any pair \(\{a,b\}\)
when \(a,b \in V'\).\\
~\\
Now consider an arbitrary pair \(\{a,b\}\) when \(a \in V\), and \(b \in V'\).
From the definition of complementary blocks \(a\) occurs either \(C\) or
\(\bar{C}\) and \(b\) in either \(D'\) or \(\bar{D'}\), for some value of \(i\),
and because all pair combinations between \(C\)-blocks and \(D'\)-blocks are
formed in the new design and because \(i\) takes \(2k-1\) values in the new
design so the pair makes \(2k-1\) appearances.\\
~\\
Hence the new block design is a \(BIBD\) with \(\lambda =2k-1\). \(\square\)

\hypertarget{theorem-5.4.1}{%
\subsubsection{Theorem 5.4.1}\label{theorem-5.4.1}}

Suppose we have two \(BRS(q+1)\), \(R\) and \(S\) based on \(G = GF(q)\), with
the following properties:

\begin{enumerate}
\def\labelenumi{\arabic{enumi}.}
\item
  \(R\) and \(S\) are a \emph{Latin pair} such that,
\item
  \(R \odot S\) has a pair of disjoint \emph{common transversals} \(T_1\) and
  \(T_2\) (with \(T_2\) in \(R\)), which do not intersect the main diagonal,
  and
\item
  the block designs obtained from \(R\) and \(S\), call them \(D(R)\) and
  \(D(S)\) respectively, have the property that if \(\{\infty\} \cup B_i\)
  and \(\{i\} \cup C_i\) are the blocks of \(D(R)\) obtained from row \(i\),
  then \(\{i\} \cup B_i\) and \(\{\infty\} \cup C_i\) are the blocks of
  \(D(S)\) obtained from row \(i\).
\end{enumerate}

Then a \(BRS(2(q+1))\) exists.\\
~\\
\emph{Proof}\\
Define \(A\) to be the array obtained from the superposition of \(\hat{R}\)
and \(\hat{S'}\). Where \(\hat{S'}\) is obtained by replacing the elements
of \(G\) in \(\hat{S}\) (the reduced RS) by the corresponding elements of
\(G'\) and replacing \(\infty\) by \(\infty'\). \(R\) and \(S\) were \(BRS\) so each
column and row of \(A\) contains each member of
\(G \cup G' \cup \{\infty,\infty'\}\), except those elements \(i,i',\infty\)
and \(\infty'\) which are missing from row and column \(i\) due to the
reduction process. Further, \(A\) contains every pair
\(\{i,j\},\{i',j'\} \hspace{0.25cm} i,j \in G,i \neq j\) exactly once.\\
Define \(B\) as the array obtained from \(R \odot S\) when each pair \((i,j)\)
is replaced by \((i,j')\).\\
~\\
Now, \(D\) is defined as the array obtained from \(B\) according to this
procedure:\\
If cell \((m,n), m\neq n\) of \(R\) is not empty, replace \((i,j')\) in cell
\((m,n)\) of \(B\) by \((j',i)\). Because \(R \odot S\) is a pair of superposed
orthogonal latin squares \(D\) contains each member of \(G \cup G'\) exactly
once in each row and column. Also, the way in which we have repalced
elements ensures that every unordered pair \(\{i,j'\}\) for all
\(i,j \in G\) occurs exactly once in \(D\).\\
~\\
Next, construct \(C\) by arranging the arrays \(A\) and \(D\) according to the
following layout, in which \(\phi\) is the \(q \times q\) array with every
cell empty, \(\theta\) the \(1 \times q\) array of empty cells and
\(\theta ^T\) the transpose of \(\theta\):

\(C=\)

\begin{longtable}[]{@{}ccc@{}}
\toprule
\(A\) & \(\phi\) & \(\theta ^T\)\tabularnewline
\midrule
\endhead
\(\phi\) & \(D\) & \(\theta ^T\)\tabularnewline
\(\theta\) & \(\theta\) & \(\infty',\infty\)\tabularnewline
\bottomrule
\end{longtable}

The rows and columns of this new array are labelled
\(0,1,2,...,q-1,0',1',2',...,(q-1)',l\). So that the pair
\((\infty ', \infty)\) is in cell \((l,l)\).\\
~\\
Let

\(T'_p = \{ (i'_m,j'_m) | (i_m,j_m)\) \emph{is a cell transversal} \(T_p \}\)

be the set of cells of \(C\), corresponding to transversal
\(T_p \hspace{0.25cm}(p=1,2)\) of \(R \odot S\).\\
~\\
Construct a new array \(F\), based on \(C\) according to the following
prescription:\\
~\\
Consider cell \((i'_m,j'_m) \in T'_1\). If cell \((i_m,j_m)\) is not empty
in \(R\) then \((i'_m,j'_m)\) of \(C\) contains a pair \((k',n)\), otherwise it
contains a pair \((n,k')\). {[}In either case
\(k,nm \in G \hspace{0.25cm} k \neq n\) (the transversal does not
intersect the diagonal){]}. Remove whichever pair appears in cell
\((i'_m,j'_m)\) and put it in cell \((i'_m,l)\). Also place \((\infty',k')\),
\((\infty,n)\) in cells \((k,j'_m),(n,j'_m)\) respectively.\\
If \((i'_m,j'_m) \in T'_2\) then a pair \((k',n)\) appears in cell
\((i'_m,j'_m)\). Again remove it, but this time put it in \((l,j'_m)\). In
addition pairs \((k',\infty),(\infty ',n)\) go in cells
\((i'_m,k),(i'_m,n)\) respectively.\\
~\\
\(F\) is a \(BRS\) because,

\begin{itemize}
\item
  \(A\) and \(D\) between them intially contain all the unordered pairs
  \(\{i,j\},\{i',j'\}\) \(i,j \in G, i\neq j\) and \(\{i,j'\}\) for all
  \(i,j \in G\). So \(F\) has an ordered pair corresponding to each of
  these.
\item
  The procedure in the preceding paragraph contributes the remaining
  ordered pairs\\
  \((\infty ',k'),(\infty,n),(k',\infty),(\infty ',n)\) for all
  \(k,n \in G \hspace{0.25cm} k \neq n\) in such a way that the elements
  missing from the first \(q\) rows and columns are suitably placed, and
  \(\infty\) and \(\infty '\) are placed in each row and column of \(F\).
  Hence each row and column of \(F\) contains every member of
  \(\{\infty, \infty '\} \cup G \cup G'\) exactly once.
\end{itemize}

The block design obtained from the rows of \(F\) has blocks:

\[
\left.
\begin{tabular}{c}
${} B~i~ {’} C’~i~$\\
${i} C~i~ {i’} B’~i~$\\
\end{tabular}
\right \}
0 \leq i \leq q-1
\]

from rows 0,1,\ldots,\(q-1\) blocks

\[
\left.
\begin{tabular}{c}
${} B~i~ {i’} B’~i~$\\
${i} C~i~ {} C’~i~$\\
\end{tabular}
\right \}
0 \leq i \leq q-1
\]

from rows \(0',1',...,(q-1)'\), and the blocks
\[\{\infty\} \cup F, \{\infty '\} \cup G'\] obtained from row \(l\).\\
According to condition 3 of Theorem 3.6 and Lemma 3.1, these blocks for
a \(BIBD\). \(\square\)\\
~\\
\emph{Example 5.4.1}\\
Consider the following \(BRS(8)\)s

\(R=\)

\begin{longtable}[]{@{}ccclcll@{}}
\toprule
\(\infty 0\) & 26 & 45 & & 13 & &\tabularnewline
\midrule
\endhead
& \(\infty 1\) & 30 & 56 & & 24 &\tabularnewline
& & \(\infty 2\) & 41 & 60 & & 35\tabularnewline
46 & & & \(\infty 3\) & 52 & 01 &\tabularnewline
& 50 & & & \(\infty 4\) & 63 & 12\tabularnewline
23 & & 61 & & & \(\infty 5\) & 04\tabularnewline
15 & 34 & & 02 & & & \(\infty 6\)\tabularnewline
\bottomrule
\end{longtable}

\(\hspace{0.5cm} S=\)

\begin{longtable}[]{@{}cllclcc@{}}
\toprule
\(\infty 0\) & & & 64 & & 32 & 51\tabularnewline
\midrule
\endhead
62 & \(\infty 1\) & & & 05 & & 43\tabularnewline
54 & 03 & \(\infty 2\) & & & 16 &\tabularnewline
& 65 & 14 & \(\infty 3\) & & & 20\tabularnewline
31 & & 06 & 25 & \(\infty 4\) & &\tabularnewline
& 42 & & 10 & 36 & \(\infty 5\) &\tabularnewline
& & 53 & & 21 & 40 & \(\infty 6\)\tabularnewline
\bottomrule
\end{longtable}

\(\hspace{3.5cm}\) \textbf{Figure 28} \(\hspace{6cm}\) \textbf{Figure 29}\\
~\\
These \(BRS\) satisfy all three properties required by Theorem 3.6. (We
shall see why later).\\
Suppose we take the following two disjoint common transversals as \(T_1\)
(lighter grey shading) and \(T_2\) (darker grey shading).

\(R \odot S=\)

\begin{longtable}[]{@{}ccccccc@{}}
\toprule
00 & 26 & 45 & 64 & 13 & 32 & 51\tabularnewline
\midrule
\endhead
62 & 11 & 30 & 56 & 05 & 24 & 43\tabularnewline
54 & 03 & 22 & 41 & 60 & 16 & 35\tabularnewline
46 & 65 & 14 & 33 & 52 & 01 & 20\tabularnewline
31 & 50 & 06 & 25 & 44 & 63 & 12\tabularnewline
23 & 42 & 61 & 10 & 36 & 55 & 04\tabularnewline
15 & 34 & 53 & 02 & 21 & 40 & 66\tabularnewline
\bottomrule
\end{longtable}

\textbf{Figure 30}

Next we obtain \(A\) from the superposition of \(\hat{R}\) and \(\hat{S'}\):

\$A = \$

\begin{longtable}[]{@{}lcccccc@{}}
\toprule
& 26 & 45 & \(6'4'\) & 13 & \(3'2'\) & \(5'1'\)\tabularnewline
\midrule
\endhead
\(6'2'\) & & 30 & 56 & \(0'5'\) & 24 & \(4'3'\)\tabularnewline
\(5'4'\) & \(0'3'\) & & 41 & 60 & \(1'6'\) & 35\tabularnewline
46 & \(6'5'\) & \(1'4'\) & & 52 & 01 & \(2'0'\)\tabularnewline
\(3'1'\) & 50 & \(0'6'\) & \(2'5'\) & & 63 & 12\tabularnewline
23 & \(4'2'\) & 61 & \(1'0'\) & \(3'6'\) & & 04\tabularnewline
15 & 34 & \(5'3'\) & 02 & \(2'1'\) & \(4'0'\) &\tabularnewline
\bottomrule
\end{longtable}

\textbf{Figure 31}

Then we construct \(B\), and after swapping the order of certain pairs,
obtain \(D\):

\(B=\)

\begin{longtable}[]{@{}ccccccc@{}}
\toprule
\(00'\) & \(26'\) & \(45'\) & \(64'\) & \(13'\) & \(32'\) & \(51'\)\tabularnewline
\midrule
\endhead
\(62'\) & \(11'\) & \(30'\) & \(56'\) & \(05'\) & \(24'\) & \(43'\)\tabularnewline
\(54'\) & \(03'\) & \(22'\) & \(41'\) & \(60'\) & \(16'\) & \(35'\)\tabularnewline
\(46'\) & \(65'\) & \(14'\) & \(33'\) & \(52'\) & \(01'\) & \(20'\)\tabularnewline
\(31'\) & \(50'\) & \(06'\) & \(25'\) & \(44'\) & \(63'\) & \(12'\)\tabularnewline
\(23'\) & \(42'\) & \(61'\) & \(10'\) & \(36'\) & \(55'\) & \(04'\)\tabularnewline
\(15'\) & \(34'\) & \(53'\) & \(02'\) & \(21'\) & \(40'\) & \(66'\)\tabularnewline
\bottomrule
\end{longtable}

\(\hspace{0.5cm} \Rightarrow D=\)

\begin{longtable}[]{@{}ccccccc@{}}
\toprule
\(00'\) & \(6'2\) & \(5'4\) & \(64'\) & \(3'1\) & \(32'\) & \(51'\)\tabularnewline
\midrule
\endhead
\(62'\) & \(11'\) & \(0'3\) & \(6'5\) & \(05'\) & \(4'2\) & \(43'\)\tabularnewline
\(54'\) & \(03'\) & \(22'\) & \(1'4\) & \(0'6\) & \(16'\) & \(5'3\)\tabularnewline
\(6'4\) & \(65'\) & \(14'\) & \(33'\) & \(2'5\) & \(1'0\) & \(20'\)\tabularnewline
\(31'\) & \(0'5\) & \(06'\) & \(25'\) & \(44'\) & \(3'6\) & \(2'1\)\tabularnewline
\(3'2\) & \(42'\) & \(1'6\) & \(10'\) & \(36'\) & \(55'\) & \(4'0\)\tabularnewline
\(5'1\) & \(4'3\) & \(53'\) & \(2'0\) & \(21'\) & \(40'\) & \(66'\)\tabularnewline
\bottomrule
\end{longtable}

\(\hspace{3.5cm}\) \textbf{Figure 32} \(\hspace{6cm}\) \textbf{Figure 33}\\
~\\
\(C\) is obtained by arranging \(A\) and \(D\) thus:

\begin{longtable}[]{@{}lccccccccccccccc@{}}
\toprule
& 0 & 1 & 2 & 3 & 4 & 5 & 6 & \(0'\) & \(1'\) & \(2'\) & \(3'\) & \(4'\) & \(5'\) & \(6'\) & \(l\)\tabularnewline
\midrule
\endhead
0 & & 26 & 45 & \(6'4'\) & 13 & \(3'2'\) & \(5'1'\) & & & & & & & &\tabularnewline
1 & \(6'2'\) & & 30 & 56 & \(0'5'\) & 24 & \(4'3'\) & & & & & & & &\tabularnewline
2 & \(5'4'\) & \(0'3'\) & & 41 & 60 & \(1'6'\) & 35 & & & & & & & &\tabularnewline
3 & 46 & \(6'5'\) & \(1'4'\) & & 52 & 01 & \(2'0'\) & & & & & & & &\tabularnewline
4 & \(3'1'\) & 50 & \(0'6'\) & \(2'5'\) & & 63 & 12 & & & & & & & &\tabularnewline
5 & 23 & \(4'2'\) & 61 & \(1'0'\) & \(3'6'\) & & 04 & & & & & & & &\tabularnewline
6 & 15 & 34 & \(5'3'\) & 02 & \(2'1'\) & \(4'0'\) & & & & & & & & &\tabularnewline
\(0'\) & & & & & & & & \(00'\) & \(6'2\) & \(5'4\) & \(64'\) & \(3'1\) & \(32'\) & \(51'\) &\tabularnewline
\(1'\) & & & & & & & & \(62'\) & \(11'\) & \(0'3\) & \(6'5\) & \(05'\) & \(4'2\) & \(43'\) &\tabularnewline
\(2'\) & & & & & & & & \(54'\) & \(03'\) & \(22'\) & \(1'4\) & \(0'6\) & \(16'\) & \(5'3\) &\tabularnewline
\(3'\) & & & & & & & & \(6'4\) & \(65'\) & \(14'\) & \(33'\) & \(2'5\) & \(1'0\) & \(20'\) &\tabularnewline
\(4'\) & & & & & & & & \(31'\) & \(0'5\) & \(06'\) & \(25'\) & \(44'\) & \(3'6\) & \(2'1\) &\tabularnewline
\(5'\) & & & & & & & & \(3'2\) & \(42'\) & \(1'6\) & \(10'\) & \(36'\) & \(55'\) & \(4'0\) &\tabularnewline
\(6'\) & & & & & & & & \(5'1\) & \(4'3\) & \(53'\) & \(2'0\) & \(21'\) & \(40'\) & \(66'\) &\tabularnewline
\(l\) & & & & & & & & & & & & & & & \(\infty ' \infty\)\tabularnewline
\bottomrule
\end{longtable}

\textbf{Figure 34}

Finally we construct \(F\) according to the 2 presciptions in the final
part of the construction. Which involves first replacing some of the
pairs in those cells of \(C\) corresponding to \(D\).\\
Consider
\[T'_1 = \{(2',6'),(3',0'),(4',1'),(5',2'),(6',3'),(0',4'),(1',5')\}\]
In \(R\), the cells \((2,6),(3,0),(4,1),(5,2),(6,3),(0,4)\) and \((1,5)\) were
all non-empty, so \(C\) contains pairs of the form \((k',n)\) in all cells
\((2',6'),(3',0')\), \ldots{} All of these are removed and replaced in the
final column of the same row. And, for each pair \((k',n)\), corresponding
pairs of the form \((\infty ', k'), (\infty,n)\) are put in cells
\((k,j'_m),(n,j'_m)\), respectively.\\
e.g.\\
\((5',3)\) appears in position \((2',6')\) of \(C\). So in \(F\), \((5',3)\)
appears in \((2',l)\) and the additional pairs \((\infty,3)\) and
\((\infty ',5')\) appear in column \(6'\), in rows 3 and 5 respectively.\\
~\\
All of which makes the final columns appear like so:

\begin{longtable}[]{@{}lcccccccc@{}}
\toprule
& \(0'\) & \(1'\) & \(2'\) & \(3'\) & \(4'\) & \(5'\) & \(6'\) & \(l\)\tabularnewline
\midrule
\endhead
0 & & \(\infty ' 0'\) & & \(\infty 0\) & & & &\tabularnewline
1 & & & \(\infty ' 1'\) & & \(\infty 1\) & & &\tabularnewline
2 & & & & \(\infty ' 2'\) & & \(\infty 2\) & &\tabularnewline
3 & & & & & \(\infty ' 3 '\) & & \(\infty 3\) &\tabularnewline
4 & \$\infty  4 \$ & & & & & \(\infty ' 4 '\) & &\tabularnewline
5 & & \$\infty  5 \$ & & & & & \(\infty ' 5 '\) &\tabularnewline
6 & \(\infty ' 6 '\) & & \$\infty 6 \$ & & & & &\tabularnewline
\(0'\) & \(00'\) & \(6'2\) & \(5'4\) & \(64'\) & \(3'1\) & \(32'\) & \(51'\) & \(3'1\)\tabularnewline
\(1'\) & \(62'\) & \(11'\) & \(0'3\) & \(6'5\) & \(05'\) & \(4'2\) & \(43'\) & \(4'2\)\tabularnewline
\(2'\) & \(54'\) & \(03'\) & \(22'\) & \(1'4\) & \(0'6\) & \(16'\) & \(5'3\) & \(5'3\)\tabularnewline
\(3'\) & \(6'4\) & \(65'\) & \(14'\) & \(33'\) & \(2'5\) & \(1'0\) & \(20'\) & \(6'4\)\tabularnewline
\(4'\) & \(31'\) & \(0'5\) & \(06'\) & \(25'\) & \(44'\) & \(3'6\) & \(2'1\) & \(0'5\)\tabularnewline
\(5'\) & \(3'2\) & \(42'\) & \(1'6\) & \(10'\) & \(36'\) & \(55'\) & \(4'0\) & \(1'6\)\tabularnewline
\(6'\) & \(5'1\) & \(4'3\) & \(53'\) & \(2'0\) & \(21'\) & \(40'\) & \(66'\) & \(2'0\)\tabularnewline
\(l\) & & & & & & & & \(\infty ' \infty\)\tabularnewline
\bottomrule
\end{longtable}

\textbf{Figure 35}

Now consider:
\[T'_2 = \{(4',5'),(5',6'),(6',0'),(0',1'),(1',2'),(2',3'),(3',4')\}\]
In each cell \((i'_m,j'_m)\) of \(C\), where \((i'_m,j'_m) \in T'_2\), occurs
a pair \((k',n)\), we remove this and place it in cell \((l,j'_m)\), also
putting pairs \((k',\infty),(\infty ',n)\) in cells \((i'_m,k),(i'_m,n)\).\\
e.g.\\
\((3',6)\) appears in \((4',5')\), so \((3',6)\) goes in \((l,5')\) and
\((3',\infty), (\infty ', 6)\) go in \((4',3),(4',6)\) respectively.

\begin{longtable}[]{@{}lccccccccccccccc@{}}
\toprule
& 0 & 1 & 2 & 3 & 4 & 5 & 6 & \(0'\) & \(1'\) & \(2'\) & \(3'\) & \(4'\) & \(5'\) & \(6'\) & \(l\)\tabularnewline
\midrule
\endhead
\(0'\) & & & \(\infty ' 2\) & & & & \(6' \infty\) & \(00'\) & & \(5'4\) & \(64'\) & & \(32'\) & \(51'\) & \(3'1\)\tabularnewline
\(1'\) & \(0' \infty\) & & & \(\infty ' 3\) & & & & \(62'\) & \(11'\) & & \(6'5\) & \(05'\) & & \(43'\) & \(4'2\)\tabularnewline
\(2'\) & & \(1' \infty\) & & & \(\infty ' 4\) & & & \(54'\) & \(03'\) & \(22'\) & & \(0'6\) & \(16'\) & & \(5'3\)\tabularnewline
\(3'\) & & & \(2' \infty\) & & & \(\infty ' 5\) & & & \(65'\) & \(14'\) & \(33'\) & & \(1'0\) & \(20'\) & \(6'4\)\tabularnewline
\(4'\) & & & & \(3' \infty\) & & & \(\infty ' 6\) & \(31'\) & & \(06'\) & \(25'\) & \(44'\) & & \(2'1\) & \(0'5\)\tabularnewline
\(5'\) & \(\infty ' 0\) & & & & \(4' \infty\) & & & \(3'2\) & \(42'\) & & \(10'\) & \(36'\) & \(55'\) & & \(1'6\)\tabularnewline
\(6'\) & & \(\infty ' 1\) & & & & \(5' \infty\) & & & \(4'3\) & \(53'\) & & \(21'\) & \(40'\) & \(66'\) & \(2'0\)\tabularnewline
\(l\) & & & & & & & & \(5'1\) & \(6'2\) & \(0'3\) & \(1'4\) & \(2'5\) & \(3'6\) & \(4'0\) & \(\infty ' \infty\)\tabularnewline
\bottomrule
\end{longtable}

\textbf{Figure 36}

So the following array, the completed \(F\), is a \(BRS(16)\):

\begin{longtable}[]{@{}lcccccclclcllll@{}}
\toprule
& 26 & 45 & \(6'4'\) & 13 & \(3'2'\) & \(5'1'\) & & \(\infty ' 0'\) & & \(\infty 0\) & & & &\tabularnewline
\midrule
\endhead
\(6'2'\) & & 30 & 56 & \(0'5'\) & 24 & \(4'3'\) & & & \(\infty ' 1'\) & & \(\infty 1\) & & &\tabularnewline
\(5'4'\) & \(0'3'\) & & 41 & 60 & \(1'6'\) & 35 & & & & \(\infty ' 2'\) & & \(\infty 2\) & &\tabularnewline
46 & \(6'5'\) & \(1'4'\) & & 52 & 01 & \(2'0'\) & & & & & \(\infty ' 3'\) & & \(\infty 3\) &\tabularnewline
\(3'1'\) & 50 & \(0'6'\) & \(2'5'\) & & 63 & 12 & \(\infty 4\) & & & & & \(\infty ' 4'\) & &\tabularnewline
23 & \(4'2'\) & 61 & \(1'0'\) & \(3'6'\) & & 04 & & \(\infty 5\) & & & & & \(\infty ' 5'\) &\tabularnewline
15 & 34 & \(5'3'\) & 02 & \(2'1'\) & \(4'0'\) & & \(\infty' 6'\) & & \(\infty 6\) & & & & &\tabularnewline
& & \(\infty ' 2\) & & & & \(6' \infty\) & \(00'\) & & \(5'4\) & \(64'\) & & \(32'\) & \(51'\) & \(3'1\)\tabularnewline
\(0' \infty\) & & & \(\infty ' 3\) & & & & \(62'\) & \(11'\) & & \(6'5\) & \(05'\) & & \(43'\) & \(4'2\)\tabularnewline
& \(1' \infty\) & & & \(\infty ' 4\) & & & \(54'\) & \(03'\) & \(22'\) & & \(0'6\) & \(16'\) & & \(5'3\)\tabularnewline
& & \(2' \infty\) & & & \(\infty ' 5\) & & & \(65'\) & \(14'\) & \(33'\) & & \(1'0\) & \(20'\) & \(6'4\)\tabularnewline
& & & \(3' \infty\) & & & \(\infty ' 6\) & \(31'\) & & \(06'\) & \(25'\) & \(44'\) & & \(2'1\) & \(0'5\)\tabularnewline
\(\infty ' 0\) & & & & \(4' \infty\) & & & \(3'2\) & \(42'\) & & \(10'\) & \(36'\) & \(55'\) & & \(1'6\)\tabularnewline
& \(\infty ' 1\) & & & & \(5' \infty\) & & & \(4'3\) & \(53'\) & & \(21'\) & \(40'\) & \(66'\) & \(2'0\)\tabularnewline
& & & & & & & \(5'1\) & \(6'2\) & \(0'3\) & \(1'4\) & \(2'5\) & \(3'6\) & \(4'0\) & \(\infty ' \infty\)\tabularnewline
\bottomrule
\end{longtable}

\textbf{Figure 37}

Schellenberg applied his multiplicative construction to the \(BRS\)
generated by Hwang's starter-adder construction, and in doing so was
able to establish the following Theorem:

\hypertarget{theorem-5.4.2}{%
\subsubsection{Theorem 5.4.2}\label{theorem-5.4.2}}

{[}21{]} For \(q \equiv 3\)(mod 4), with \(q\) a prime power strictly greater
than 3, there exists a \(BRS(2(q+1))\).\\
~\\
\emph{Proof}\\
Construct \(R\) from the following balanced starter,
\[X = \left \{ (x^{2i},x^{2i+1}):0 \leq i \leq \frac{q-3}{2} \right \}\]
with adder,
\[A(X) = \left \{ -x^{2i}(1+x):0 \leq i \leq \frac{q-3}{2} \right \}\]
Clearly this is just the Hwang starter-adder of Theorem 5.3.4, although
the elements in the starter pairs have swapped order. Now, construct \(S\)
from the balanced starter,
\[Y = \left \{ (x^{2i-1},x^{2i}):0 \leq i \leq \frac{q-3}{2} \right \}\]
\[A(Y) = \left \{ -x^{2i-1}(1+x):0 \leq i \leq \frac{q-3}{2} \right \}\]
\(Y\) has been obtained simply by swapping the pairs of \(X\) and
multiplying them by \(x^{-1}\). Therefore, that \(Y\) is a balanced starter
follows from the fact that \(X\) is balanced, as was established in
Theorem 3.5.\\
It remains to show that \(R\) and \(S\) satisfy conditions 1,2 and 3 of
Theorem 3.6:

\begin{enumerate}
\def\labelenumi{\arabic{enumi}.}
\item
  \(R\) and \(S\) are a Latin pair: The positions of the starter pais in
  the first row of each square are determined by the adder, so
  \(A(X) \cap A(Y) = \varnothing\)\footnote{\(A(X) \cap A(Y) \neq \varnothing\) implies, as \((1+x) \neq 0\),
    that \(-x^{2i} = -x^{2i-1}\) for some \(i\), i.e.~\(1=x^{-1}\),
    \(\therefore A(X) \cap A(Y) = \varnothing\)} ensures that each cell of the
  first row, hence subsequent rows, of \(R \odot S\) contains only one
  pair.\\
  ~\\
  That \(R \odot S\) is a pair of superposed Latin squares requires that
  for the pairs in any row, the left hand members take all values
  0\ldots{}\(q-1\), and the right members also take values 0\ldots{}\(q-1\).\\
  Clearly this property holds for all rows if it holds for the first.
  The pairs in the first row of \(R \odot S\) are:
  \[\{(0,0)\} \cup \left \{ (x^{2i-1},x^{2i}),(x^{2i},x^{2i+1}) \middle| 0 \leq i \leq \frac{q-3}{2} \right \}\]
  So all \((q-1)/2\) non-squares and all \((q-1)/2\) squares occur as left
  hand members of pairs, and similarly as right hand members. So
  \(R \odot S\) is a pair of superposed Latin squares.\\
  For these Latin squares to be orthogonal requires that there is no
  repetition of ordered pairs. For a repetition to occur would require
  some repetition of ordered differences among the starter pairs.\\
  Either:
  \[x^{2i-1} - x^{2i} = x^{2i} - x^{2i+1} \hspace{0.2cm} \mathrm{or} \hspace{0.2cm}
  x^{2i} - x^{2i-1} = x^{2i+1} - x^{2i} \hspace{0.2cm} \mathrm{i.e.}\]
  \[\pm (x^{2i-1} + x^{2i+1}) = \pm (x^{2i} + x^{2i})\]
  \[x^{-1} + x = 2 \Rightarrow x = 1\] Clearly false, so ordered
  differences are unique.
\item
  \(R \odot S\) has a pair of disjoint \emph{common transversals}:\\
  For any fixed non-zero \(j\) the cells \((i,i+j)\), \(0 \leq i \leq q-1\)
  form a common transversal which does not intersect the diagonal.
\item
  The block designs for \(R\) and \(S\) are balanced:\\
  The starters \(X\) and \(Y\) are balanced starters hence the block
  designs for \(R\) and \(S\) are \(BIBD\)s. \(\square\)
\end{enumerate}

So the Schellenberg multiplication construction establishes the
existence of some new orders \(q+1\) where \(q\) is not necessarily a prime
power.\\
e.g.\\
the orders in bold are now established which had not been under Hwang's
construction:

{\textbar c\textbar c\textbar c\textbar c\textbar c\textbar c\textbar c\textbar c\textbar c\textbar c\textbar c\textbar c\textbar c\textbar{}}

\begin{longtable}[]{@{}c@{}}
\toprule
\endhead
Hwang \(BRS\)\tabularnewline
\(p^r + 1: p^r \equiv 3\)(mod 4)\tabularnewline
\bottomrule
\end{longtable}

\& 8 \& 12 \& \& 20 \& 24 \& 28 \& 32 \& \& \& 44 \& 48 \& \ldots{}\\

\begin{longtable}[]{@{}c@{}}
\toprule
\endhead
Schellenberg\tabularnewline
\(2(p^r + 1)\)\tabularnewline
\bottomrule
\end{longtable}

\& \textbf{16} \& 24 \& 32 \& \textbf{40} \& 48 \& \textbf{56} \& \textbf{64} \& 72 \& 80 \& \textbf{88} \&
\textbf{96} \& \ldots{}\\

\textbf{Table 8}

Even with both these theorems, there are obviously missing orders. The
first exception, 36 occurs, for example, because 35 is not a prime power
and because 17 is not congruent to 3 mod 4.\\
~\\
Schellenberg applied the multiplicative construction to the case when
\(p^r \equiv 1\)(mod 4) {[}then, of course, \(2(p^r+1) \equiv 0\)(mod 4) as is
required{]} and was able to obtain further results. Although, as an
interesting parallel to the Mullin-Nemeth construction, he also had
problems with the Fermat primes and again they were treated separately.
Rather than look at his approach we consider the later approach by Du,
Yu, Hwang and Kang amongst others.

\hypertarget{symmetric-skew-balanced-starters}{%
\section{Symmetric skew balanced starters}\label{symmetric-skew-balanced-starters}}

If \((x_1,y_1),(x_2,y_2), ...,(x_{(q-1)/2},y_{(q-1)/2})\) are the pairs of
a balanced starter in \(GF(q)\) then we say that starter is \textbf{\emph{skew}} if
\(\pm(x_1+y_1), \pm(x_2+y_2), ..., \pm(x_{(q-1)/2},y_{(q-1)/2})\) are all
distinct mod \(q\).\\
Notice that a skew starter is necessarily a strong starter (all sums are
distinct).\\
A balanced starter is \textbf{\emph{symmetric}} if
\(\{x_1,x_2,...,x_{(q-1)/2}\}=\{-x_1,-x_2,...,x_{(q-1)/2}\}\)\\
~\\
In {[}13{]} it was claimed that Schellenberg's multiplication theorem could
be applied to two \(CBHR(2n-1)\)s, both obtained from \(SSBS\), to construct
a \(BRS(4n)\). It was then shown that \(SSBS\) exist for all prime powers
\(2n-1=8k+5>5\), and so the existence of \(BRS(4n)\) for these values was
believed to be established.

\begin{longtable}[]{@{}cccccccccccc@{}}
\toprule
k & 1 & 3 & 4 & 6 & 7 & 12 & 13 & 16 & 18 & 19 & \ldots{}\tabularnewline
\midrule
\endhead
\(4n\) & 28 & 60 & 76 & 108 & 124 & 204 & 220 & 268 & 300 & 316 & \ldots{}\tabularnewline
\bottomrule
\end{longtable}

\textbf{Table 9}

Recently, Anderson{[}2{]}, has noticed a flaw in this approach and corrected
it.\\
Suppose we wished to apply Schellenberg's construction to two
\(CBHR(2n-1)\)s, call them \(P\) and \(Q\), then it would be necessary to make
a slight alteration to Theorem 3.6 to accomodate the missing diagonal
pairs, i.e.~\(\{\infty,i\}\) from row \(i\).\\
~\\
\textbf{Theorem 5.5.1} Suppose we have two \(CBHR(2n-1)\), \(P\) and \(Q\) based on
\(G=GF(2n-1)\), with the following properties:

\begin{enumerate}
\def\labelenumi{\arabic{enumi}.}
\item
  \(P\) and \(Q\) are a \emph{Latin pair} such that,
\item
  \(P \odot Q\) has a pair of disjoint \emph{common transversals} \(T_1\) and
  \(T_2\) (with \(T_2\) in \(P\)), which do not intersect that main
  diagonal, and
\item
  the block designs obtained from \(P\) and \(Q\), call them \(D(P)\) and
  \(D(Q)\) respectively, have the property that if \(B_i\) and \(C_i\) are
  the blocks from \(D(P)\) obtained from row \(i\), then \(B_i\) and \(C_i\)
  are also the blocks of \(D(Q)\) obtained from row \(i\).
\end{enumerate}

Then a \(BRS(4n)\) exists.\\
~\\
\emph{Proof}\\
As for Theorem 3.6\\
It was then claimed that the two \(CBHR(2n-1)\)s obtained from a \(SSBS\)
and its transpose satisfy the three conditions of Theorem 3.8. Hence
implying the existence of a \(BRS(4n)\).\\
However, recently a counter-example has been found.

\hypertarget{lemma-5.5.1}{%
\subsubsection{Lemma 5.5.1}\label{lemma-5.5.1}}

{[}2{]} If \(p=8k+5\) is a prime, \(p > 5\), then the pairs
\[(x^{4i},-x{4i+1}),(x^{4i+2},x^{4i+1})\hspace{0.5cm} 0 \leq i \leq 2k\]
form a \(SSBS\) in \(GF(p)\), provided \(x^2-1\) is a square mod \(p\).\\
~\\
\emph{Proof}

\begin{enumerate}
\def\labelenumi{\arabic{enumi}.}
\item
  The elements in the starter pairs are a complete replication of
  \(GF(p) \backslash \{0\}\).\\
  \(x^{\frac{1}{2} (q-1)} = x^{4k+2} = -1\), therefore we can write the
  pairs as,
  \[(X^{4i},x^{4(k+i)+3}),(x^{4i+2},x^{4i+1}) \hspace{0.5cm} 0 \leq i \leq 2k\]
  All the members of these pairs are clearly unique and there are
  \(4(2k+1)=8k+4\) of them, hence they must all of
  \((GF(p) \backslash \{0\}\).
\item
  The differences are similarly a complete replication of
  \(GF(p) \backslash \{0\}\).\\
  The differences are \(\pm x^{4i}(x+1), \pm x^{4i+1}(x-1)\), which can
  be written as
  \[x^{4i}(x+1), x^{4(i+k)+2}(x+1),x^{4i+1}(x-1),x^{4(i+k)+3}(x-1)\]
  Which are all unique provided \((x+1)\) and \((x-1)\) have the same
  quadratic character (meaning either both of neither are squares).
  This is true because \((x+1)(1-x) = x^2-1\) is a square.
\item
  The starter is \textbf{\emph{skew}}, (the positive and negative sums are all
  distinct).\\
  The sums are \(\pm x^{4i}(1-x), \pm x^{4i+1}(x+1)\) which again are
  all unique, provided \((1-x)\) and \((x+1)\) have the same quadratic
  character. True, because \((1-x)(x+1) = -(x^2-1) = x^{4k+2}(x^2-1)\),
  is a square.
\item
  The starter is \textbf{\emph{symmetric}}.\\
  Indeed, \(-x^{4i} = x^{4(i+k)+2}\), generates the same elements as
  \(x^{4i+2}\) shifted by \(k\) places.
\item
  The associated block design is balanced.\\
  The blocks in the first row are,\\
  \(\{\infty\} \cup R\), due to the left hand members (where \(R\) is the
  set of squares).\\
  \(\{0\} \cup N\), due to the right hand members (where \(N\) is the set
  of non-squares).\\
  We can show that these blocks along with their translates form a
  \(BIBD\).\\
  ~\\
  Using an argument from Theorem 5.2.4, we can say that all elements
  of \(R\) are generated as differences of \(R\) from the same number of
  times (say \(\lambda _1\) times). Also, a similar argument allows us
  to say that all the elements of \(N\) are generated as differences of
  \(R\) the same number of times (say \(\lambda _2\)). Now, of course
  \(N = xR\). So we can say that to each difference of members of \(R\)
  which generates a square there is a difference of \(N\) which
  generates a non-square. Therefore every \(r \in R\) occurs as a
  difference in \(N, \lambda _1\) times. For the same reason every
  \(n \in N\) occurs as a difference between members of \(N, \lambda _2\)
  times. Therefore, each element of \(GF(p) \backslash \{0\}\) occurs
  \(\lambda _1 + \lambda _2\) times as a difference in \(R\) or \(N\).\\
  ~\\
  Now, \(|R| = |N| = 4k+2\). So each of the two sets, \(R\) and \(N\), gives
  \((4k+2)(4k+1)\) differences. Furthermore there are \(8k+4\) elements of
  \(GF(p) \backslash \{0\}\), and each occurs as a difference
  \(\lambda _1 + \lambda _2\) times. So,
  \[(2(4k+2)(4k+1) = (\lambda _1 + \lambda _2)(8k+4)\]
  \[\therefore \lambda _1 + \lambda _2 = 4k+1\] Adjoining 0 to \(N\)
  creates each member of \(GF(p) \backslash \{0\}\) once again, as
  either \(0-n\) or \(n-0\).\\
  Adjoining \(\infty\) to \(R\), makes \(|R| = 4k+2\) pairs involving
  \(\infty\). Therefore, \(\{\infty\} \cup R\) and \(\{0\} \cup N\) and
  their translates have between them each pair from
  \(\{\infty\{ \cup GF(p)\) occurring \(4k+2\) times. Hence the block
  design is balanced.
\end{enumerate}

\emph{Example 5.5.1}\\
Let \(P\) be the \(CBHR(2n-1)\) obtained from the \(SSBS\) in Lemma 3.2 when
\(p=13\) \[(1,11),(3,7),(4,2),(9,8),(10,5),(12,6)\] By the \textbf{\emph{transpose}}
\(X^T\) of a starter \(X\) we mean those pairs in the first column of the
square generated by \(X\).\\
In this case, for example, \((1,11)\) goes in cell (0,12) therefore will
be a pair in the first column \((1+i,11+i)\) in position \((0+i,12+i)\),
such that \(12+i = 0\) mod 13,

(2,12) goes in (1,0) hence belongs to the transpose of this starter

In general a starter-pair \((a,b)\) goes in \((0,a+b)\), so the
corresponding transpose-pair will be \((a-a-b,b-a-b) = (-b,-a)\). So the
transpose of the above starter is:
\[(2,12),(6,10),(11,9),(5,4),(8,3),(7,1)\] If we now apply the
Schellenberg construction as suggested in {[}13{]} we obtain \(P \odot Q\):

{-0.45cm}{-0.45cm}

\$P \odot Q = \$

\begin{longtable}[]{@{}ccccccccccccc@{}}
\toprule
0,0 & 2,12 & 10,5 & 6,10 & 9,8 & 12,6 & 4,2 & 11,9 & 7,1 & 5,4 & 3,7 & 8,3 & 1,11\tabularnewline
\midrule
\endhead
2,12 & 1,1 & 3,0 & 11,6 & 7,11 & 10,9 & 0,7 & 5,3 & 12,10 & 8,2 & 6,5 & 4,8 & 9,4\tabularnewline
10,5 & 3,0 & 2,2 & 4,1 & 12,7 & 8,12 & 11,10 & 1,8 & 6,4 & 0,11 & 9,3 & 7,6 & 5,9\tabularnewline
6,10 & 11,6 & 4,1 & 3,3 & 5,2 & 0,8 & 9,0 & 12,11 & 2,9 & 7,5 & 1,12 & 10,4 & 8,7\tabularnewline
9,8 & 7,11 & 12,7 & 5,2 & 4,4 & 6,3 & 1,9 & 10,1 & 0,12 & 3,10 & 8,6 & 2,0 & 11,5\tabularnewline
12,6 & 10,9 & 8,12 & 0,8 & 6,3 & 5,5 & 7,4 & 2,10 & 11,2 & 1,0 & 4,11 & 9,7 & 3,1\tabularnewline
4,2 & 0,7 & 11,10 & 9,0 & 1,9 & 7,4 & 6,6 & 8,5 & 3,11 & 12,3 & 2,1 & 5,12 & 10,8\tabularnewline
11,9 & 5,3 & 1,8 & 12,11 & 10,1 & 2,10 & 8,5 & 7,7 & 9,6 & 4,12 & 0,4 & 3,2 & 6,0\tabularnewline
7,1 & 12,10 & 6,4 & 2,9 & 0,12 & 11,2 & 3,11 & 9,6 & 8,8 & 10,7 & 5,0 & 1,5 & 4,3\tabularnewline
5,4 & 8,2 & 0,11 & 7,5 & 3,10 & 1,0 & 12,3 & 4,12 & 10,7 & 9,9 & 11,8 & 6,1 & 2,6\tabularnewline
3,7 & 6,5 & 9,3 & 1,12 & 8,6 & 4,11 & 2,1 & 0,4 & 5,0 & 11,8 & 10,10 & 12,9 & 7,2\tabularnewline
8,3 & 4,8 & 7,6 & 10,4 & 2,0 & 9,7 & 5,12 & 3,2 & 1,5 & 6,1 & 12,9 & 11,11 & 0,10\tabularnewline
1,11 & 9,4 & 5,9 & 8,7 & 11,5 & 3,1 & 10,8 & 6,0 & 4,3 & 2,6 & 7,2 & 0,10 & 12,12\tabularnewline
\bottomrule
\end{longtable}

\textbf{Figure 38}

Although this is the join of two Latin squares, the repetition of
ordered pairs means that they are not orthogonal. An immediate solution
might be to reverse the order of half the repeated pairs, but this
approach ruins the join of Latin squares property.\\
Anderson, has developed a theorem to overcome this property. We present
a slight adaptation of this theorem:

\hypertarget{theorem-5.5.2}{%
\subsubsection{Theorem 5.5.2}\label{theorem-5.5.2}}

{[}2{]} Let \(p=8k+5\) be a prime, \(p>5\). Then a \(BRS\) of side \(2p+1\) exists.\\
~\\
\emph{Proof}\\
Suppose we take \(P \odot Q\), and label elements of the pairs from \(P\)
with subscript 1 and all those from \(Q\) with subscript 2. Then remove
the diagonal pairs \((a,a)\), and call this new array \(A\).\\
\(A\) contains all the unordered pairs \(\{a_1,b_1\},\{a_2,b_2\}\) exactly
once.\\
The first row of \(A\) contains the ordered pairs,

\((x_1^{4i},-x_1^{4i+1}),(x_1^{4i+2},-x_1^{4i+1})\) due to \(P\), and\\
\((x_2^{4i+1},-x_2^{4i}),(x_2^{4i+1},-x_2^{4i+2})\) due to \(Q\)\\

So all the squares subscript 1 appear as left hand members of pairs as
do all the non-squares subscript 2. Also, all the non-squares appear
with subscript 1 in the right hand positions and all the squares with
subscript 2.\\
~\\
If we denote the set of squares with subscript 1 by \(R_1\), and with
subscript 2 by \(R_2\). And similarly denote the sets of non-squares as
\(N_1\) and \(N_2\). Then in \(A\),

\(R_1 \cup N_2\) occupies the left hand positions\\
while \(N_1 \cup R_2\) occupies the right\\

Now define \(B\) as the array of side \(p\) whose first row contains the
pairs:
\[(x_2^{4i},-x_1^{4i+1}),(x_1^{4i+2},x_2^{4i+1}),(-x_2^{4i},x_1^{4i+1}),(-x_1^{4i+2},-x_2^{4i+1})\]
So in \(B\) the left hand positions are occupied by \(R_2 \cup R_1\), while
the right are occupied by \(N_1 \cup N_2\).\\
In order that all unordered pairs \(\{a_1,b_2\}\) occur once in \(B\)
requires that in the first row all the members of \(GF(p)\) occur once as
a difference in one of two ways:

Either as \(a_1-b_2\) which are called \((1,2)\) \textbf{\emph{mixed differences}}\\
Or as \(b_2 -a_1\), called (2,1) mixed differences.\\

Consider \((2,1)\) mixed differences in the first row, they are:\\
\(x^{4i} + x^{4i+1} = x^{4i}(1+x)\)\\
\(-x^{4i}-x^{4i+1}=-x^{4i}(1+x)\)\\
\(x^{4i+1} - x^{4i+2} = x^{4i+1}(1-x)\)\\
\(-x^{4i+1} + x^{4i+2} = -x^{4i+1}(1-x)\)\\
i.e.~\(\pm x^{4i}(1+x), \pm x^{4i+1}(1-x)\)\\
Again, these are all of \(GF(p)\) provided that \(x^2- 1\) is a square.\\
~\\
Now arrange \(A\) and \(B\) in the following familiar manner to construct an
array of side \(2p+1\):

\begin{longtable}[]{@{}ccc@{}}
\toprule
\(A\) & \(\phi\) & \(\theta ^T\)\tabularnewline
\midrule
\endhead
\(\phi\) & \(B\) & \(\theta ^T\)\tabularnewline
\(\theta\) & \(\theta\) & \(\infty _1, \infty _2\)\tabularnewline
\bottomrule
\end{longtable}

Next place the missing pairs in a manner similar to the original
construction of Schellenberg. Take a pair \((a_1,b_2\) from the first row
of \(B\) in position \((0_2,h_2)\) and consider the cells in transversal
\(T_1\) of \(B\), where
\[T_1 = \left \{((0+g)_2,(h+g)_2) \hspace{0.2cm} \middle| \hspace{0.2cm} g \in Z_p \right \}\]
If cell \((i_2,j_2)\) in \(T_1\) contains \((n_1,k_2)\),

\begin{itemize}
\item
  put \((n_1,k_2)\) in cell \((i_2,\infty)\)
\item
  also put \((\infty _1,k_2)\) in \((k_1,j_2)\) and \((\infty _2,n_1)\) in
  \((n_1,j_2)\)
\end{itemize}

Now consider the transversal \(T_2\) of \(B\), where \((u_1,v_2)\) is a pair
in \((0_2,l_2)\) and,
\[T_2 = \left \{((0+g)_2,(l+g)_2) \hspace{0.2cm} \middle| \hspace{0.2cm} g \in Z_p \right \}\]
If cell \((i_2,j_2)\) in \(T_2\) contains \((h_1,m_2)\),

\begin{itemize}
\item
  put \((h_1,m_2)\) in cell \((\infty,j_2)\)
\item
  also put \((\infty _2,m_2)\) in \((0_2,m_1)\) and \((h_1,\infty _1)\) in
  \((0_2,h_1)\)
\end{itemize}

Finally put \((i_1,i_2)\) in \((i_2,i_2)\) for \(0 \leq i \leq p-1\), and put
\((\infty _1, \infty _2)\) in cell \((\infty, \infty )\).\\
~\\
\emph{Example}\\
Continuing from before,\\
~\\

{-1.5cm}{-1.5cm}

\(A=\) \(\begin{array}{|c|c|c|c|c|c|c|c|c|c|c|c|c|c|} \hline  & 0_1 & 1_1 & 2_1 & 3_1 & 4_1 & 5_1 & 6_1 & 7_1 & 8_1 & 9_1 & 10_1 & 11_1 & 12_1\\\hline 0_1 & & 2_2 12_2 & 10_1 5_1 & 6_2 10_2 & 9_1 8_1 & 12_1 6_1 & 4_1 2_1 & 11_2 9_2 & 7_2 1_2 & 5_2 4_2 & 3_1 7_1 & 8_2 3_2 & 1_1 11_1\\\hline 1_1 & 2_1 12_1 & & 3_2 0_2 & 11_1 6_1 & 7_2 11_2 & 10_1 9_1 & 0_1 7_1 & 5_1 3_1 & 12_2 10_2 & 8_2 2_2 & 6_2 5_2 & 4_1 8_1 & 9_2 4_2\\\hline 2_1 & 10_2 5_2 & 3_1 0_1 & & 4_2 1_2 & 12_1 7_1 & 8_2 12_2 & 11_1 10_1 & 1_1 8_1 & 6_1 4_1 & 0_2 11_2 & 9_2 3_2 & 7_2 6_2 & 5_1 9_1\\\hline 3_1 & 6_1 10_1 & 11_2 6_2 & 4_1 1_1 & & 5_2 2_2 & 0_1 8_1 & 9_2 0_2 & 12_1 11_1 & 2_1 9_1 & 7_1 5_1 & 1_1 12_2 & 10_2 4_2 & 8_2 7_2\\\hline 4_1 & 9_2 8_2 & 7_1 11_1 & 12_2 7_2 & 5_1 2_1 & & 6_2 3_2 & 1_1 9_1 & 10_2 1_2 & 0_1 12_1 & 3_1 10_1 & 8_1 6_1 & 2_2 0_2 & 11_2 5_2\\\hline 5_1 & 12_2 6_2 & 10_2 9_2 & 8_1 12_1 & 0_2 8_2 & 6_1 3_1 & & 7_1 4_1 & 2_1 10_1 & 11_2 2_2 & 1_1 0_1 & 4_1 11_1 & 9_1 7_1 & 3_2 1_2\\\hline 6_1 & 4_2 2_2 & 0_2 7_2 & 11_2 10_2 & 9_1 0_1 & 1_2 9_2 & 7_1 4_1 & & 8_2 5_2 & 3_1 11_1 & 12_2 3_2 & 2_1 1_1 & 5_1 12_1 & 10_1 8_1\\\hline 7_1 & 11_1 9_1 & 5_2 3_2 & 1_2 8_2 & 12_2 11_2 & 10_1 1_1 & 2_2 10_2 & 8_1 5_1 & & 9_2 6_2 & 4_1 12_1 & 0_2 4_2 & 3_1 2_1 & 6_1 0_1\\\hline 8_1 & 7_1 1_1 & 12_1 10_1 & 6_2 4_2 & 2_2 9_2 & 0_1 12_1 & 11_2 2_2 & 3_1 11_1 & 9_1 6_1 & & 10_2 7_2 & 5_1 0_1 & 1_2 5_2 & 4_1 3_1\\\hline 9_1 & 5_1 4_1 & 8_1 2_1 & 0_1 11_1 & 7_2 5_2 & 3_2 10_2 & 1_2 0_2 & 12_1 3_1 & 4_2 12_2 & 10_1 7_1 & & 11_2 8_2 & 6_1 1_1 & 2_2 6_2\\\hline 10_1 & 3_2 7_2 & 6_1 5_1 & 9_1 3_1 & 1_1 12_1 & 8_2 6_2 & 4_2 11_2 & 2_2 1_2 & 0_1 4_1 & 5_2 0_2 & 11_1 8_1 & & 12_2 9_2 & 7_1 2_1\\\hline 11_1 & 8_1 3_1 & 4_2 8_2 & 7_1 6_1 & 10_1 4_1 & 2_1 0_1 & 9_2 7_2 & 5_2 12_2 & 3_2 2_2 & 1_1 5_1 & 6_2 1_2 & 12_1 9_1 & & 0_2 10_2\\\hline 12_1 & 1_2 11_2 & 9_1 4_1 & 5_2 9_2 & 8_1 7_1 & 11_1 5_1 & 3_1 1_1 & 10_2 8_2 & 6_2 0_2 & 4_2 3_2 & 2_1 6_1 & 7_2 2_2 & 0_1 10_1 & \\\hline \end{array}\)

\textbf{Figure 39}

{-1.5cm}{-1.5cm}

\(B=\) \(\begin{array}{|c|c|c|c|c|c|c|c|c|c|c|c|c|c|} \hline  & 0_2 & 1_2 & 2_2 & 3_2 & 4_2 & 5_2 & 6_2 & 7_2 & 8_2 & 9_2 & 10_2& 11_ 2& 12_2\\\hline 0_2 & & 12_2 2_1 & 10_1 5_2 & 10_2 6_1 & 9_2 8_1 & 12_1 6_2 & 4_1 2_2 & 9_1 11_2 & 1_1 7_2 & 4_2 5_1 & 3_2 7_1 & 3_1 8_2 & 1_2 11_1\\\hline 1_2 & 2_2 12_1 & & 0_2 3_1 & 11_1 6_2 & 11_2 7_1 & 10_2 9_1 & 0_1 7_2 & 5_1 3_2 & 10_1 12_2 & 2_1 8_2 & 5_2 6_1 & 4_2 8_1 & 4_1 9_2 \\\hline 2_2 & 5_1 10_2 & 3_2 0_1 & & 1_2 4_1 & 12_1 7_2 & 12_2 8_1 & 11_2 10_1 & 1_1 8_2 & 6_1 4_2 & 11_1 0_2 & 3_1 9_2 & 6_2 7_1 & 5_2 9_1\\\hline 3_2 & 6_2 10_1 & 6_1 11_2 & 4_2 1_1 & & 2_2 5_1 & 0_1 8_2 & 0_2 9_1 & 12_2 11_1 & 2_1 9_2 & 7_1 5_2 & 12_1 1_2& 4_1 10_2 & 7_2 8_1 \\\hline 4_2 & 8_2 9_1 & 7_2 11_1 & 7_1 12_2 & 5_2 2_1 & & 3_2 6_1 & 1_1 9_2 & 1_2 10_1 & 0_2 12_1 & 3_1 10_2 & 8_1 6_2 & 0_1 2_2 & 5_1 11_2 \\\hline 5_2 & 6_1 12_2 & 9_2 10_1 & 8_2 12_1 & 8_1 0_2 & 6_2 3_1 & & 4_2 7_1 & 2_1 10_2 & 2_2 11_1 & 1_2 0_1 & 4_1 11_2 & 9_1 7_2 & 1_1 3_2 \\\hline 6_2 & 2_1 4_2 & 7_1 0_2 & 10_2 11_1 & 9_2 0_1 & 9_1 1_2 & 7_2 4_1 & & 5_2 8_1 & 3_1 11_2 & 3_2 12_1 & 2_2 1_1 & 5_1 12_2 & 10_1 8_2\\\hline 7_2 & 11_1 9_2 & 3_1 5_2 & 8_1 1_2 & 11_2 12_1 & 10_2 1_1 & 10_1 2_2 & 8_2 5_1 & & 6_2 9_1 & 4_1 12_2 & 4_2 0_1 & 3_2 2_1 & 6_1 0_1\\\hline 8_2 & 7_1 1_2 & 12_1 10_2 & 4_1 6_2 & 9_1 2_2 &12_2 0_1 & 11_2 2_1 & 11_1 3_2 & 9_2 6_1 & & 7_2 10_1 & 5_1 0_2 & 5_2 1_1 & 4_2 3_1\\\hline 9_2 & 5_2 4_1 & 8_1 2_2 & 0_1 11_2 & 5_1 7_2 & 10_1 3_2 & 0_2 1_1 & 12_2 3_1 & 12_1 4_2 & 10_2 7_1 & & 8_2 11_1 & 6_1 1_2 & 6_2 2_1\\\hline 10_2 & 7_2 3_1 & 6_2 5_1 & 9_1 3_2 & 1_1 12_2 & 6_1 8_2 & 11_1 4_2 & 1_2 2_1 & 0_2 4_1 & 0_1 5_2 & 11_2 8_1 & & 9_2 12_1 & 7_1 2_2\\\hline 11_2 & 8_1 3_2 & 8_2 4_1 & 7_2 6_1 & 10_1 4_2 & 2_1 0_2 & 7_1 9_2 & 12_1 5_2 & 2_2 3_1 & 1_2 5_1 & 1_1 6_2 & 12_2 9_1 & & 10_2 0_1\\\hline 12_2 & 11_2 1_1 & 9_1 4_2 & 9_2 5_1 & 8_2 7_1 & 11_1 5_2 & 3_1 1_2 & 8_1 10_2 & 0_1 6_2 & 3_2 4_1 & 2_2 6_1 & 2_1 7_2 & 0_2 10_1 & \\\hline \end{array}\)

\textbf{Figure 40}

Now, assemble the large array and apply the construction using the
following transversals,
\[T_1 = \{((0+g)_2,(5+g)_2)\hspace{0.2cm}|\hspace{0.2cm}g \in Z_{13}\}\]
\[T_2 = \{((0+g)_2,(11+g)_2)\hspace{0.2cm}|\hspace{0.2cm}g \in Z_{13}\}\]
The obtained array is the \(BRS(28)\) in Figure 41.\\
~\\
The finished array (Figure 41) is an \(ORS\) because it contains one
ordered pair corresponding to each unordered pair from the set
\(\{\infty _1, \infty _2, 0_1,...,p-1_1,0_2,...,p-1_2\}\), with each
member of the same set occurring exactly once in each row and column. To
prove that this \(ORS\) is a \(BRS\) we need to show that the block design
obtained from the rows in the usual way is balanced.\\
~\\
Consider the blocks obtained from the first \(p\) rows. We have shown
already that the first row of \(A\) conssted of two blocks, left and right
where\\
\(\hspace{1cm}\) Left block: \(R_1 \cup N_2\)\\
\(\hspace{1cm}\) Right block: \(N_1 \cup R_2\)\\
The last stage of the construction put the pairs \(\{\infty _2, 0_1\}\)
and \(\{\infty _1,0_2\}\) in row \(0_1\) of the finished array. So the
blocks obtained from the first \(p\) rows are:
\[R_1 \cup N_2 \cup \{\infty _2, \infty _1 \}\]
\[N_1 \cup R_2 \cup \{0 _1, 0 _2 \}\] and their translates.\\
By similar reasoning the blocks obtained from the next \(p\) rows are:
\[R_2 \cup R_1 \cup \{0 _1, \infty _2 \}\]
\[N_1 \cup N_2 \cup \{0 _2, \infty _1 \}\] and their translates. Finally
the row labelled \(\infty\) contributes a further two blocks to the
design: \(\hspace{1cm} \{0_1,\infty _1\} \cup R_1 \cup N_1\) obtained from
the left hand members of pairs, and\\
\(\hspace{1cm} \{0_2,\infty _2\} \cup R_2 \cup N_2\) obtained from the
right hand members.\\
Fortunately, Schellenberg has shown that these blocks do indeed form a
\(BIBD\) with parameters,\\
\((2(p+1_,p+1,p)\).\\
~\\
To show this we consider another result due to Bose. He showed that the
blocks\\
\(\{\infty,x^0,x^2,...,x^{q-3}\}\{0,x^0,x^2,...,x^{q-3}\}\) form a
difference system in \(GF(q)\) with \(\lambda = (q-1)/2\). This result
implies that the blocks
\(\{\infty, x^1,x^3,...,x^{q-2}\}\{0,x^1,x^3,...,x^{q-2}\}\) also form a
difference system with the same concurrence number.

\hypertarget{lemma-5.5.2}{%
\subsubsection{Lemma 5.5.2}\label{lemma-5.5.2}}

{[}4{]} If \(q = p^n \equiv 1\)(mod 4) is a prime power, then if we write down
the non-zero elements of \(GF(q)\) which are squares:
\[x^0,x^2,x^4,...,x^{q-3}\] Every non-square element of \(GF(q)\) occurs
exactly \((q-1)/4\) times as a difference between these elements and every
square occurs \((q-5)/4\) times.\\
~\\
\emph{Proof}\\
We can write the differences in the following way:

\begin{longtable}[]{@{}cccc@{}}
\toprule
\endhead
\((x^2-1)\) & \(x^{2}(x^{2}-1)\) & \$ \hspace{0.5cm} \ldots{} \hspace{0.5cm} \$ & \(x^{q-3}(x^{2}-1)\)\tabularnewline
\((x^4-1)\) & \(x^{2}(x^{4}-1)\) & \$ \hspace{0.5cm} \ldots{} \hspace{0.5cm} \$ & \(x^{q-3}(x^{4}-1)\)\tabularnewline
& & \$ \hspace{0.5cm} \ldots{} \hspace{0.5cm} \$ &\tabularnewline
\((x^{q-3}-1)\hspace{0.25cm}\) & \(\hspace{0.25cm}x^{2}(x^{q-3}-1)\) & \$ \hspace{0.5cm} \ldots{} \hspace{0.5cm} \$ & \(x^{q-3}(x^{q-3}-1)\)\tabularnewline
\bottomrule
\end{longtable}

Where the first column has been obtained by substracting \(x^0\) from
every other square. The second column by taking \(x^2\) from every other
square, the third by taking \(x^4\) and so on.

Because the differences in every column except the first have an even
power of \(x\) multiplied by some member of the first column it is clear
that the quadratic nature of these terms is determined by whether or not
the elements in the first column are square or not.\\
~\\
Therefore to complete the proof we need to show that among the \((q-3)/2\)
elements, \[x^2-1,x^4-1,...,x^{q-3}-1\] every square occurs \((q-5)/4\)
times and every non-square \((q-1)/4\) times. We can further simplify the
problem if we write,
\[x^{2i}-1=(x^i+1)(x^i-1)=\frac{x^i+1}{x^i-1}(x^i-1)^2\] Then \(x^{2i}-1\)
is a square or non-square depending on whether:
\[z_i = \frac{x^i+1}{x^i-1}\] is a square or not. So it remains only to
count the occurrences of squares/non-squares in the set:
\[Z=\{z_1,z_2,...,z_{(q-3)/2}\}\] Suppose we consider instead the set,
\[A = \{z_1,z_2,...,z_{q-2}\}\] which clearly has \(q-2\) members. We can
show that each member is a unique member of \(GF(q)\), because if two
different elements \(z_i\) and \(z_j\) were the same we would have:
\[\frac{x^i+1}{x^i-1} \equiv \frac{x^j+1}{x^j-1}(\mathrm{mod}\hspace{0.1cm} q)\]
\[(x^j-1)(x^i+1) \equiv (x^j+1)(x^i-1)\hspace{0.1cm}(\mathrm{mod}\hspace{0.1cm} q)\]
\[x^jx^i+x^j-x^i-1 \equiv x^jx^i - x^j + x^i -1\hspace{0.1cm}(\mathrm{mod}\hspace{0.1cm} q)\]
\[x^j \equiv x^i \hspace{0.1cm}(\mathrm{mod}\hspace{0.1cm} q)\] but
\(1 \leq i, j \leq q-2\), therefore the above is only satisfied if
\(x^j = x^i, \therefore i=j\), which contradicts the assumption that the
two elements were different.\\
Further, if \(z_i \equiv -1\hspace{0.1cm}(\mathrm{mod}\hspace{0.1cm} q)\)
then \(x^i+1 = -x^i+1\), which implies \(x^i=0\). So \(-1 \notin A\). Recall
that when \(q \equiv 1\hspace{0.1cm}(\mathrm{mod}\hspace{0.1cm} 4)\), \(-1\)
is a square.\\
Also, if \(z_i \equiv -1\hspace{0.1cm}(\mathrm{mod}\hspace{0.1cm} q)\)
then \(x^i+1=x^i-1\), so \(1 \notin A\). 1 is a square.\\
Finally, \(x^{q-1/2}=-1\), therefore \(z_{q-1/2}=0\).\\
Clearly, then \(A=GF(q) \backslash \{1,-1\}\), and
\(A \backslash \{z_{q-1/2}\} = GF(q) \backslash \{0,1,-1\}\), so \(A \backslash \{z_{q-1/2}\}\) contains
\((q-1)/2\) non-squares and \((q-5)/2\) squares.
\[A \backslash \{z_{(q-1)/2}\}=\{z_1,z_2,...,z_{(q-3)/2},z_{(q+1)/2},z_{(q+3)/2},...,z_{q-2}\}\]
But,
\[z_{\{(q+1)/2\}+k} = \frac{x^{\{(q+1)/2\}+k}+1}{x^{\{(q+1)/2\}+k}-1}
= \frac{-x^k+1}{-x^k-1} = \frac{1}{z^k} \hspace{0.5cm} k = 1,2,...,(q-3)/2\]
So \(z_{\{(q+1)/2\}+k}\) and \(z_k\) \([ k = 1,2,...,(q-3)/2]\) are both
squares or both non-squares, hence \(Z=\{z_1,z_2,...,z_{(q-3)/2}\}\)
contains exactly \((q-1)/4\) non-squares and \((q-5)/4\) squares. So the
lemma is proven.

\hypertarget{corollary-5.5.2}{%
\subsubsection{Corollary 5.5.2}\label{corollary-5.5.2}}

If \(q = p^n \equiv 1\)(mod 4) is a prime power, then if we write down the
non-zero elements of \(GF(q)\) which are non-squares:
\[x^1,x^3,x^5,...,x^{q-2}\] Every square element of \(GF(q)\) occurs
exactly \((q-1)/4\) times as a difference between these elements and every
non-square occurs \((q-5)/4\) times.\\
~\\
\emph{Proof}\\
As for the previous theorem except that each difference is multiplied by
\(x\), so squares become non-squares and non-squares, squares.

\hypertarget{theorem-5.5.3}{%
\subsubsection{Theorem 5.5.3}\label{theorem-5.5.3}}

{[}4{]} The blocks
\[\{\infty,x^0,x^2,...,x^{q-3}\}\{0,x^0,x^2,...,x^{q-3}\}\] along with
their translates, form a \(BIBD\) with \(\lambda = (q-1)/2\) in \(GF(q)\),
when \(q = p^n \equiv 1\)(mod 4).\\
~\\
\emph{Proof}\\
From the Lemma we have shown that in the set
\[\{x^0,x^2,x^4,...,x^{q-3}\}\] each square occurs as a difference
\((q-5)/4\) times, and each non-square \((q-1)/4\) times. The right hand
block also contributes the differences
\[x^0-0,0-x^0,1-x^2,x^2-1,...,x^{q-3}-0,0-x^{q-3}\]
\[\pm x^0, \pm x^2, ... ,\pm x^{q-3}\] And because \(-1\) is a square,
each square of \(GF(q)\) occurs a further twice due to these differences.

\begin{longtable}[]{@{}lccc@{}}
\toprule
& \(\{\infty,x^0,x^2,...,x^{q-3}\}\) & \(\{0,x^0,x^2,...,x^{q-3}\) & Total\tabularnewline
\midrule
\endhead
Squares & \((q-5)/4\) & \(\{(q-5)/4\}+2\) & \((q-1)/2\)\tabularnewline
Non-squares & \((q-1)/4\) & \((q-1)/4\) & \((q-1)/2\)\tabularnewline
\bottomrule
\end{longtable}

\textbf{Table 10}

Adjoining \(\infty\) to a block of size \((q-1)/2\) creates \((q-1)/2\) pairs
involving \(\infty\). Therefore in \(\{\infty,x^0,x^2,...,x^{q-3}\}\) and
the translates obtained from this block \(\infty\) makes a pair with each
member of \(GF(q)\), \((q-1)/2\) times.\\
Therefore the block design obtained from these two blocks is balanced,
and the concurrence number is \(\lambda = (q-1)/2\).\\
~\\
Although we don't need the following result, it now follows
straightforwardly from Corollary 5.4.2.

\hypertarget{corollory-5.5.3}{%
\subsubsection{Corollory 5.5.3}\label{corollory-5.5.3}}

{[}4{]} The blocks,
\[\{\infty,x^1,x^3,...,x^{q-2}\}\{0,x^1,x^3,...,x^{q-2}\}\] along with
their translates, form a \(BIBD\) with \(\lambda = (q-1)/2\) in \(GF(q)\),
when \(q = p^n \equiv 1\)(mod 4).\\
~\\
Now returning to the Anderson construction:\\
Let: \[S = R_1 \cup N_2 \cup \{\infty _2, \infty _1\}\]
\[T = N_1 \cup R_2 \cup \{0_1, 0_2\}\]
\[U = N_1 \cup N_2 \cup \{0 _2, \infty _1\}\]
\[V = R_2 \cup R_1 \cup \{0_1, \infty _2\}\] Now consider the pairs
\(\{a_1,b_1\}\) such that \(a,b \in GF(q) \cup \{\infty\}\). We know, due to
the Bose result, that the blocks\\
\(\{\infty _1\} \cup R_1\) and \(0 _1\} \cup R_1\) along with their
translates contain each pair \(\{a_1,b_1\}\), \((q-1)/2\) times,\\
also\\
\(\{\infty _1\} \cup N_1\) and \(0 _1\} \cup N_1\) and translates contain
each pair \(\{a_1,b_1\}\), \((q-1)/2\) times.\\
But,
\[\{\infty _1\} \cup R_1 \subset R_1 \cup N_2 \cup \{\infty _1, \infty _2\}\]
\[\{\infty _1\} \cup N_1 \subset N_1 \cup R_2 \cup \{0_1, 0_2\}\]
Therefore \(S\) and \(T\), along with their translates, contain each pair
\(\{a_1,b_1\}, 2\cdot (q-1)/2 = q-1\), times. Finally each pair
\(\{a_1,b_1\}\) occurs once more in the set
\(\{\infty _1,0_1\} \cup R_1 \cup N_1\). So in the entire block design
each pair \(\{a_1,b_1\}\) occurs \(q\) times.\\
The same reasoning can be applied to show that the pairs \(\{a_2,b_2\}\)
for \(a,b \in GF(q) \cup \{\infty\}\) also occur \(q\) times in the block
design.\\
It remains to show that the mixed pairs of the form \(\{a_1,b_2\}\), where
\(a,b \in GF(q) \cup \{\infty\}\), also occur \(q\) times in the block
design. Again we consider (1,2) mixed differences.\\
For a pair \(\{a_1,a_2\}\) to occur in the block design requires a mixed
difference of 0 in either \(S\), \(T\), \(U\) or \(V\).\\
In \(T\), \(0_1-0_2 = 0\), gives one occurrence, while in \(U\)\\

MISSING TABLE HERE

gives another \((q-1)/2\) occurrences of 0 as a mixed
difference.\\
~\\
Therefore, 0 occurs as mixed difference in \(S,T,U\) and \(V\)
\((2\cdot (q-1)/2)+1=q\) times, therefore each pair of the form
\(\{a_1,a_2\}\), occurs \(q\) times in the block design.\\
~\\
Next, consider pairs of the form \(\{\infty _1, b_2\}\). In \(S\) and \(U\),
\(\infty _1\) makes a pair with each of the \((q-1)/2\) members of \(N_2\).
Therefore in the blocks consisting of \(S\), \(U\) and their translates,
\(\infty _1\) makes a pair with each element \(b_2\) {[}for all \(b \in GF(q)\){]}
\(2 \cdot (q-1)/2 = q-1\) times. Also in \(U\), \(\infty _1\) is paired with
\(0_2\), so \(\infty _1\) makes a pair with each \(b_2\) once more. So there
are \(q\) pairs \(\{\infty _1,b_2\}\) for each element \(b_2\) in the block
design.\\
~\\
By identical reasoning we can say that \(\infty _2\) makes a pair with
each \(b_1\), for all \(b \in GF(q)\), \((q-1)/2\) times in \(S\) and its
translates. Also \(\{\infty _2,b_1\}\) {[}for all \(b \in GF(q)\){]} occurs
\((q+1)/2\) times due to \(V\) and its translates.\\
Therefore the pairs \(\{\infty _2, b_1 \}\) each occur \(q\) times in the
block design, for all \(b \in GF(q)\).\\
~\\
Next we confirm that all the pairs \(\{a_1,b_2\}\) \(a,b \in GF(q)\) each
occur \(q\) times in the block design. In order to do this we need to show
that the non-zero members of \(GF(q)\) each occur \(q\) times as a mixed
difference \(a_1 - b_2\).\\
We have shown that for \(q \equiv 1(\)mod 4)

\begin{enumerate}
\def\labelenumi{\arabic{enumi}.}
\item
  every square, non-square occurs as a difference of two squares of
  \(GF(q)\), \(s-1\), \(s\) times respectively. \(s = (q-1)/4\).
\item
  every square, non-square occurs as a difference of non-two squares
  of \(GF(q)\), \(s\), \(s-1\) times respectively.
\end{enumerate}

Consider first the mixed differences \(a_1-b_2 = c\), where \(c\) is a
square.\\
\(U\) contains \(N_1 \cup N_2\), so according to \(B\) every square occurs as
a (1,2) difference \((q-1)/4\) times in \(U\). Also according to \(B\) each
non-square occurs as a (1,2) difference between members of \(U\),
\((q-5)/4\) times. Finally, \(U\) contains the element \(0_2\), so each
non-square is generated once more.
\(x_1^1-0_2=x^1_1,x_1^3-0_2=x_1^3,...,x_1^{q-2}-0_2=x_1^{q-2}\).\\
Similarly, in \(V\) each non-square occurs as a mixed difference \((q-1)/4\)
times while each square occurs as a difference \((q-5)/4_1\) times.\\
Next, consider blocks \(S\) and \(T\). In \(T\) each square/non-square exists
as a mixed difference once, in the following way: \$\$

\begin{longtable}[]{@{}lll@{}}
\toprule
\endhead
\begin{minipage}[t]{0.18\columnwidth}\raggedright
\(x_1^1\)
\(x_1^3\)\strut
\end{minipage} & \begin{minipage}[t]{0.11\columnwidth}\raggedright
\(-0_2\)
\(-0_2\)\strut
\end{minipage} & \begin{minipage}[t]{0.16\columnwidth}\raggedright
\(=x^1\)
\(=x^3\)\strut
\end{minipage}\tabularnewline
\begin{minipage}[t]{0.18\columnwidth}\raggedright
\(x^{q-2}_1\)\strut
\end{minipage} & \begin{minipage}[t]{0.11\columnwidth}\raggedright
\(-0_2\)\strut
\end{minipage} & \begin{minipage}[t]{0.16\columnwidth}\raggedright
\(=x^{q-2}\)\strut
\end{minipage}\tabularnewline
\bottomrule
\end{longtable}

\begin{longtable}[]{@{}lll@{}}
\toprule
\endhead
\begin{minipage}[t]{0.18\columnwidth}\raggedright
\(x_2^0\)
\(x_2^2\)\strut
\end{minipage} & \begin{minipage}[t]{0.11\columnwidth}\raggedright
\(-0_1\)
\(-0_1\)\strut
\end{minipage} & \begin{minipage}[t]{0.16\columnwidth}\raggedright
\(=x^0\)
\(=x^2\)\strut
\end{minipage}\tabularnewline
\begin{minipage}[t]{0.18\columnwidth}\raggedright
\(x^{q-3}_2\)\strut
\end{minipage} & \begin{minipage}[t]{0.11\columnwidth}\raggedright
\(-0_1\)\strut
\end{minipage} & \begin{minipage}[t]{0.16\columnwidth}\raggedright
\(=x^{q-3}\)\strut
\end{minipage}\tabularnewline
\bottomrule
\end{longtable}

The remaining mixed differences in \(S\) and \(T\) all involve
differences between one square and one non-square (and vice versa). Now,
for any \(c \in GF(q)\), \(a-b=c\) has \(q\) solutions. Two of these,
\[a=0-(-a)\] \[a=a-0\] involve zero, and can be rejected because we have
already considered all differences involving 0 in these blocks. Further,
from \(A\) and \(B\), we know there are \((q-5)/4 + (q-1)/4 = (q-3)/2\)
solutions which involve either both squares or both non-squares, and so
there remain \[(q-2)-[(q-3)/2] = (q-1)/2\] solutions which involve
elements of opposite quadratic character. Hence we can say that each
square, non-square occurs as a difference in \(S\) and \(T\) in \((q-1)/2+1\)
ways.

{\textbar c\textbar c\textbar c\textbar{}} \&\\
\& c=square \& c = nonsquare\\
S \& \&\\
T \& \&\\
U \& \& +1\\
V \& +1 \&\\
\& q \& q\\

Which confirms that each pair \(\{a_1,b_2\}\) for
\(a,b \in GF(a) \cup \{\infty\}\) appears in \(q\) blocks of the block
design.\\
So every pair \(\{a,b\}\),
\(a,b \in GF(q)_1 \cup GF(q)_2 \cup \{\infty _1, \infty _2\}, a \neq b\)
appears in \(q\) blocks of the block design. Hence that design is a
\(BIBD\), with parameters \((2(q+1),q+1,q)\).\\
So theorem 5.5.2 is established.

\hypertarget{closing-remarks}{%
\chapter{Closing Remarks}\label{closing-remarks}}

The results we have established in the previous chapter regarding the
existence of balanced Room squares represent by no means the complete
story. Du and Hwang {[}10{]} have established the existence of \(SSBS\) for
all prime powers \[q = 2^{\alpha}t+1, \alpha \geq 2, t \geq 3, t\] odd.
Further, Anderson has shown that consequently the construction due
originally to Hwang, Kang and Yu but corrected in {[}2{]}, allows us to
state the existence of the corresponding \(BRS(2q+2)\) in one particular
case.

By far the most significant remaining result which has not been included
in the previous chapter is due to B.A. Anderson who proved that
\(BRS(2^n)\) exist for all odd \(n \geq 3\). His construction was based upon
the theory of finite geometry, an area which has also contributed
constructions for Room squares (the non-balanced kind). Other similar
geometrical constructions have been used to establish the existence of
\(BRS(2^n)\), for \(4 \leq n \leq 18\), \(n\) even. The two smallest values of
\(n \equiv 0(\)mod 4) for which the existence of a \(BRS(n)\) remains in
doubt are 36 and 92. The first of these, along with many others, would
be established by the doubling construction if a \(SSBS\) could be found
in \(Z_{17}\). This remains one of the most significant open problems for
\(BRS\), namely to establish the existence of \(SSBS\) in \(Z_n\) when \(n\) is
a Fermat prime.

The link between graph theory and Room squares that was touched upon in
the second chapter has opened many avenues of research. Possibly the
most interesting of which is the existence of perfect Room squares. A
one-factorisation of \(K_n\) is said to be perfect if the union of two of
its one-factors is a hamiltonian cycle of \(K_n\). A perfect Room square
is one of side \(n\) in which both row and column factorisations of
\(K_{n+1}\) are perfect. Very little seems to be known about perfect
one-factorisations. Individual examples of perfect Room squares of side
11 have been constructed but no infinite classes have yet been found.

\hypertarget{references}{%
\chapter*{References}\label{references}}
\addcontentsline{toc}{chapter}{References}

\hypertarget{refs}{}
\leavevmode\hypertarget{ref-anderson_combinatorial_1990}{}%
Anderson, I. 1990. \emph{Combinatorial Designs: Construction Methods}. Ellis Horwood Series in Mathematics and Its Applications. Ellis Horwood. \url{https://books.google.co.uk/books?id=0fzuAAAAMAAJ}.

\leavevmode\hypertarget{ref-anderson_construction_1999}{}%
Anderson, Ian. 1999. ``On the Construction of Balanced Room Squares.'' \emph{Discrete Mathematics} 197: 53--60.

\leavevmode\hypertarget{ref-biggs_discrete_1985}{}%
Biggs, N., C. Biggs, and P. M. L. S. E. N. L. Biggs. 1985. \emph{Discrete Mathematics}. Oxford Science Publications. Clarendon Press. \url{https://books.google.co.uk/books?id=e_zuAAAAMAAJ}.

\leavevmode\hypertarget{ref-bose_resolvable_1947}{}%
Bose, RC. 1947. ``On a Resolvable Series of Balanced Incomplete Block Designs.'' \emph{Sankhyā: The Indian Journal of Statistics}, 249--56.

\leavevmode\hypertarget{ref-bruck_what_1963}{}%
Bruck, RH. 1963. ``What Is a Loop.'' \emph{Studies in Modern Algebra} 2: 59--99.

\leavevmode\hypertarget{ref-chong_existence_1974}{}%
Chong, BC, and KM Chan. 1974. ``On the Existence of Normalized Room Squares.'' \emph{Nanta Math} 7 (1): 8--17.

\leavevmode\hypertarget{ref-conway_book_2012}{}%
Conway, J. H., and R. Guy. 2012. \emph{The Book of Numbers}. Springer New York. \url{https://books.google.co.uk/books?id=rfLSBwAAQBAJ}.

\leavevmode\hypertarget{ref-dinitz_contemporary_1992}{}%
Dinitz, J. H., and D. R. Stinson. 1992. \emph{Contemporary Design Theory: A Collection of Surveys}. Wiley Series in Discrete Mathematics and Optimization. Wiley. \url{https://books.google.co.uk/books?id=nDiEFrSwvGgC}.

\leavevmode\hypertarget{ref-dinitz_hill-climbing_1987}{}%
---------. 1987. ``A Hill-Climbing Algorithm for the Construction of One-Factorizations and Room Squares.'' \emph{SIAM Journal on Algebraic Discrete Methods} 8 (3): 430--38. \url{https://doi.org/10.1137/0608035}.

\leavevmode\hypertarget{ref-hardy_introduction_1979}{}%
Hardy, G. H., and E. M. Wright. 1979. \emph{An Introduction to the Theory of Numbers}. Oxford Science Publications. Clarendon Press. \url{https://books.google.co.uk/books?id=3hTeH5VUheAC}.

\leavevmode\hypertarget{ref-horton_variations_1970}{}%
Horton, J. D. 1970. ``Variations on a Theme by Moore.'' \emph{Proceedings First Louisiana Conference on Combinatorics}, Graph Theory and Computing,, 146--66.

\leavevmode\hypertarget{ref-hwang_more_1970}{}%
Hwang, F. K. 1970. ``Some More Contributions on Constructing Balanced Howell Rotations.'' \emph{Proc. Second Chapel Hill Conf. On Combin. Math. And Its Appl.}, 307--23.

\leavevmode\hypertarget{ref-hwang_complete_1984}{}%
Hwang, F. K., Qin De Kang, and Jia En Yu. 1984. ``Complete Balanced Howell Rotations for 16k + 12 Partnerships.'' \url{https://doi.org/10.1016/0097-3165(84)90078-5}.

\leavevmode\hypertarget{ref-mullin_counterexample_1969}{}%
Mullin, R. C., and E. Nemeth. 1969a. ``A Counterexample to a Direct Product Construction of Room Squares.'' \url{https://doi.org/10.1016/S0021-9800(69)80021-9}.

\leavevmode\hypertarget{ref-mullin_existence_1969}{}%
---------. 1969b. ``An Existence Theorem for Room Squares.'' \emph{Canad. Math. Bull.} 12: 493--97.

\leavevmode\hypertarget{ref-mullin_furnishing_1969}{}%
---------. 1969c. ``On Furnishing Room Squares.'' \url{https://doi.org/10.1016/S0021-9800(69)80022-0}.

\leavevmode\hypertarget{ref-mullin_existence_1975}{}%
Mullin, R. C., and W. D. Wallis. 1975. ``The Existence of Room Squares.'' \emph{Aequationes Mathematicae} 13 (1): 1--7. \url{https://doi.org/10.1007/BF01834113}.

\leavevmode\hypertarget{ref-nemeth_study_1969}{}%
Nemeth, E. 1969. ``Study of Room Squares.'' PhD Thesis, University of Waterloo.

\leavevmode\hypertarget{ref-room_2569_1955}{}%
Room, Thomas G. 1955. ``2569. A New Type of Magic Square.'' \emph{The Mathematical Gazette} 39 (330): 307--7.

\leavevmode\hypertarget{ref-schellenberg_balanced_1973}{}%
Schellenberg, PJ. 1973. ``On Balanced Room Squares and Complete Balanced Howell Rotations.'' \emph{Aequationes Mathematicae} 9 (1): 75--90.

\leavevmode\hypertarget{ref-stanton_multiplication_1972}{}%
Stanton, RG, and Joseph Douglas Horton. 1972. ``A Multiplication Theorem for Room Squares.'' \emph{Journal of Combinatorial Theory, Series A} 12 (3): 322--25.

\leavevmode\hypertarget{ref-wallis_solution_1974}{}%
Wallis, WD. 1974. ``Solution of the Room Square Existence Problem.'' \emph{Journal of Combinatorial Theory, Series A} 17 (3): 379--83.

\leavevmode\hypertarget{ref-wallis_combinatorics_2006}{}%
Wallis, W. D., A. P. Street, and J. S. Wallis. 2006. \emph{Combinatorics: Room Squares, Sum-Free Sets, Hadamard Matrices}. Lecture Notes in Mathematics. Springer Berlin Heidelberg. \url{https://books.google.co.uk/books?id=cTN8CwAAQBAJ}.

\leavevmode\hypertarget{ref-yu_existence_1988}{}%
Yu, J. E., and F. K. Hwang. 1988. ``The Existence of Symmetric Skew Balanced Starters for Odd Prime Powers.'' \url{https://doi.org/10.1016/S0195-6698(88)80040-4}.

\end{document}
